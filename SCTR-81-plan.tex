% generated from JIRA project LVV
% using template at /Users/womullan/LSSTgit/docsteady/docsteady/templates/tpnoresult.latex.jinja2.
% using docsteady version 2.2.9.post4+g0778ddd.d20220916
% Please do not edit -- update information in Jira instead
\documentclass[dm,lsstdraft,STR,toc]{lsstdoc}
\usepackage{geometry}
\usepackage{longtable,booktabs}
\usepackage{enumitem}
\usepackage{arydshln}
\usepackage{attachfile}
\usepackage{array}
\usepackage{dashrule}

\newcolumntype{L}[1]{>{\raggedright\let\newline\\\arraybackslash\hspace{0pt}}p{#1}}

\input meta.tex

\newcommand{\attachmentsUrl}{https://github.com/\gitorg/\lsstDocType-\lsstDocNum/blob/\gitref/attachments}
\providecommand{\tightlist}{
  \setlength{\itemsep}{0pt}\setlength{\parskip}{0pt}}

\setcounter{tocdepth}{4}

\begin{document}

\def\milestoneName{TMA Pointing and Tracking Verification}
\def\milestoneId{}
\def\product{TMA}

\setDocCompact{true}

\title{LVV-P100: TMA Pointing and Tracking Verification Test Plan }
\setDocRef{\lsstDocType-\lsstDocNum}
\date{ 2023-06-29 }
\author{ Chuck Claver }

% Most recent last
\setDocChangeRecord{
\addtohist{}{2022-09-08}{First draft}{Chuck Claver}
}

\setDocCurator{Chuck Claver}
\setDocUpstreamLocation{\url{https://github.com/lsst-dm/\lsstDocType-\lsstDocNum}}
\setDocUpstreamVersion{\vcsRevision}



\setDocAbstract{
This is the test plan for
\textbf{ TMA Pointing and Tracking Verification},
an LSST milestone pertaining to the Data Management Subsystem.\\
This document is based on content automatically extracted from the Jira test database on \docDate.
The most recent change to the document repository was on \vcsDate.
}


\maketitle

\section{Introduction}
\label{sect:intro}


\subsection{Objectives}
\label{sect:objectives}

 The objective of this test plan is to verify the TMA pointing relative
offset, repeatability, and accuracy. We will also verify the TMA
tracking range in azimuth and elevation and the tracking drift. In
addition to these verifications, we will validate the slew and settle
time and the tracking jitter. These validations will be done after UTE
has verified them using the TMA encoders. This test plan will utilize a
StarTracker and DIMM test setup to perform the verification and
validations.



\subsection{System Overview}
\label{sect:systemoverview}

 \begin{itemize}
\tightlist
\item
  The system consists of the StarTracker (two tubes) and the DIMM
  telescope mounted on the TMA.
\item
  The StarTracker uses jumbo frames these need to be relayed from the
  Hardware through a dedicated network using the cable.
\item
  The control of the TMA is done via CSC control.
\item
  The Dome control is either controlled through the CSC or manually.
\end{itemize}


\subsection{Document Overview}
\label{sect:docoverview}

This document was generated from Jira, obtaining the relevant information from the
\href{https://jira.lsstcorp.org/secure/Tests.jspa\#/testPlan/LVV-P100}{LVV-P100}
~Jira Test Plan and related Test Cycles (
\href{https://jira.lsstcorp.org/secure/Tests.jspa\#/testCycle/LVV-C224}{LVV-C224}
\href{https://jira.lsstcorp.org/secure/Tests.jspa\#/testCycle/LVV-C228}{LVV-C228}
\href{https://jira.lsstcorp.org/secure/Tests.jspa\#/testCycle/LVV-C229}{LVV-C229}
\href{https://jira.lsstcorp.org/secure/Tests.jspa\#/testCycle/LVV-C230}{LVV-C230}
\href{https://jira.lsstcorp.org/secure/Tests.jspa\#/testCycle/LVV-C232}{LVV-C232}
\href{https://jira.lsstcorp.org/secure/Tests.jspa\#/testCycle/LVV-C233}{LVV-C233}
).

Section \ref{sect:intro} provides an overview of the test campaign, the system under test (\product{}),
the applicable documentation, and explains how this document is organized.
Section \ref{sect:testplan} provides additional information about the test plan, like for example the configuration
used for this test or related documentation.
Section \ref{sect:personnel} describes the necessary roles and lists the individuals assigned to them.

Section \ref{sect:overview} provides a summary of the test results, including an overview in Table \ref{table:summary},
an overall assessment statement and suggestions for possible improvements.
Section \ref{sect:detailedtestresults} provides detailed results for each step in each test case.

The current status of test plan \href{https://jira.lsstcorp.org/secure/Tests.jspa\#/testPlan/LVV-P100}{LVV-P100} in Jira is \textbf{ Draft }.

\subsection{References}
\label{sect:references}
\renewcommand{\refname}{}
\bibliography{lsst,refs,books,refs_ads,local}


\newpage
\section{Test Plan Details}
\label{sect:testplan}


\subsection{Data Collection}

  Observing is not required for this test campaign.

\subsection{Verification Environment}
\label{sect:hwconf}
  At the summit inside the main dome, using the final TMA hardware.

  \subsection{Entry Criteria}
  The soak test is executed, and the performance is understood.

\begin{enumerate}
\tightlist
\item
  \href{https://jira.lsstcorp.org/secure/Tests.jspa\#/testCase/LVV-T2748}{LVV-T2748
  (1.0)} TMA Pointing and Tracking - Soak Test with Random Steps using
  MTMount
\end{enumerate}

Hardware:

\begin{enumerate}
\tightlist
\item
  StarTracker available
\item
  TMA Interface build and available
\item
  SMR Reference installed
\item
  IT/Electrical connections established
\item
  Dome rotation possible
\end{enumerate}

Software:

\begin{enumerate}
\tightlist
\item
  System Control -- Minimum needed to run the test

  \begin{enumerate}
  \tightlist
  \item
    TMA readiness/functionality
  \item
    Rubin / Tekniker interfaces: Presentation of the TMA control
    interface
  \item
    Pointing
  \item
    Tracking
  \item
    Dome rotation
  \end{enumerate}
\item
  Generic Camera CSC

  \begin{enumerate}
  \tightlist
  \item
    Header services
  \item
    LFA Readiness
  \item
    Image access from LFA
  \item
    Optional Goal: Larger software integration with the TCS~
  \end{enumerate}
\end{enumerate}

  \subsection{Exit Criteria}
  \begin{itemize}
\tightlist
\item
  All necessary data are taken.
\item
  Data are analyzed and compared with FAT.
\item
  All related tickets closed or in the case of FRACAS tickets progressed
  as far as possible.
\item
  All test steps results are filled.
\item
  The test plan report is generated.
\end{itemize}


\subsection{Related Documentation}

Docushare collection where additional relevant documentation can be found:

\begin{itemize}
\item Verification artifacts:

\begin{itemize}
\tightlist
\item
  Star tracker data in the LFA of the EFD
\item
  Dimm data in the LFA of the EFD
\item
  Analysis products

  \begin{itemize}
  \tightlist
  \item
    Plot on the boresight estimation
  \end{itemize}
\end{itemize}

No DocuShare collection for this test plan is foreseen.
\end{itemize}



\subsection{PMCS Activity}

Primavera milestones related to the test campaign:
\begin{itemize}
\item None
\end{itemize}


\newpage
\section{Personnel}
\label{sect:personnel}

The personnel involved in the test campaign is shown in the following table.

{\small
\begin{longtable}{p{3cm}p{3cm}p{3cm}p{6cm}}
\hline
\multicolumn{2}{r}{T. Plan \href{https://jira.lsstcorp.org/secure/Tests.jspa\#/testPlan/LVV-P100}{LVV-P100} owner:} &
\multicolumn{2}{l}{\textbf{ Chuck Claver } }\\\hline
\multicolumn{2}{r}{T. Cycle \href{https://jira.lsstcorp.org/secure/Tests.jspa\#/testCycle/LVV-C224}{LVV-C224} owner:} &
\multicolumn{2}{l}{\textbf{
Chuck Claver }
} \\\hline
\textbf{Test Cases} & \textbf{Assigned to} & \textbf{Executed by} & \textbf{Additional Test Personnel} \\ \hline
\href{https://jira.lsstcorp.org/secure/Tests.jspa#/testCase/LVV-T2707}{LVV-T2707}
& {\small Holger Drass } & {\small JIRAUSER20616 } &
\begin{minipage}[]{6cm}
\smallskip
{\small Telescope and Dome operator(s).\\
For the Tailgate meeting, all personnel involved in observations with
the main telescope during the night. }
\medskip
\end{minipage}
\\ \hline
\href{https://jira.lsstcorp.org/secure/Tests.jspa#/testCase/LVV-T2706}{LVV-T2706}
& {\small Roberto Tighe } & {\small  } &
\begin{minipage}[]{6cm}
\smallskip
{\small Integration Specialist\\
Telescope Operator\\
Systems Engineer }
\medskip
\end{minipage}
\\ \hline
\href{https://jira.lsstcorp.org/secure/Tests.jspa#/testCase/LVV-T2705}{LVV-T2705}
& {\small Chuck Claver } & {\small  } &
\begin{minipage}[]{6cm}
\smallskip
{\small Optics specialist\\
Integration specialist }
\medskip
\end{minipage}
\\ \hline
\multicolumn{2}{r}{T. Cycle \href{https://jira.lsstcorp.org/secure/Tests.jspa\#/testCycle/LVV-C228}{LVV-C228} owner:} &
\multicolumn{2}{l}{\textbf{
Chuck Claver }
} \\\hline
\textbf{Test Cases} & \textbf{Assigned to} & \textbf{Executed by} & \textbf{Additional Test Personnel} \\ \hline
\href{https://jira.lsstcorp.org/secure/Tests.jspa#/testCase/LVV-T2707}{LVV-T2707}
& {\small Chuck Claver } & {\small  } &
\begin{minipage}[]{6cm}
\smallskip
{\small Telescope and Dome operator(s).\\
For the Tailgate meeting, all personnel involved in observations with
the main telescope during the night. }
\medskip
\end{minipage}
\\ \hline
\href{https://jira.lsstcorp.org/secure/Tests.jspa#/testCase/LVV-T2714}{LVV-T2714}
& {\small Chuck Claver } & {\small Holger Drass } &
\begin{minipage}[]{6cm}
\smallskip
{\small Two observing specialists. }
\medskip
\end{minipage}
\\ \hline
\href{https://jira.lsstcorp.org/secure/Tests.jspa#/testCase/LVV-T2730}{LVV-T2730}
& {\small Chuck Claver } & {\small JIRAUSER20616 } &
\begin{minipage}[]{6cm}
\smallskip
{\small Oversight\\
TMA and Dome operator\\
Startracker Test Script executer\\
Data Analyzer\\
SE support }
\medskip
\end{minipage}
\\ \hline
\href{https://jira.lsstcorp.org/secure/Tests.jspa#/testCase/LVV-T2715}{LVV-T2715}
& {\small Chuck Claver } & {\small  } &
\begin{minipage}[]{6cm}
\smallskip
{\small 2x Observing Specialist }
\medskip
\end{minipage}
\\ \hline
\multicolumn{2}{r}{T. Cycle \href{https://jira.lsstcorp.org/secure/Tests.jspa\#/testCycle/LVV-C229}{LVV-C229} owner:} &
\multicolumn{2}{l}{\textbf{
Chuck Claver }
} \\\hline
\textbf{Test Cases} & \textbf{Assigned to} & \textbf{Executed by} & \textbf{Additional Test Personnel} \\ \hline
\href{https://jira.lsstcorp.org/secure/Tests.jspa#/testCase/LVV-T2707}{LVV-T2707}
& {\small Chuck Claver } & {\small JIRAUSER20609 } &
\begin{minipage}[]{6cm}
\smallskip
{\small Telescope and Dome operator(s).\\
For the Tailgate meeting, all personnel involved in observations with
the main telescope during the night. }
\medskip
\end{minipage}
\\ \hline
\href{https://jira.lsstcorp.org/secure/Tests.jspa#/testCase/LVV-T2714}{LVV-T2714}
& {\small Chuck Claver } & {\small  } &
\begin{minipage}[]{6cm}
\smallskip
{\small Two observing specialists. }
\medskip
\end{minipage}
\\ \hline
\href{https://jira.lsstcorp.org/secure/Tests.jspa#/testCase/LVV-T2731}{LVV-T2731}
& {\small Chuck Claver } & {\small Ioana Sotuela } &
\begin{minipage}[]{6cm}
\smallskip
{\small Observing Specialist\\
Systems Engineer }
\medskip
\end{minipage}
\\ \hline
\href{https://jira.lsstcorp.org/secure/Tests.jspa#/testCase/LVV-T2715}{LVV-T2715}
& {\small Chuck Claver } & {\small  } &
\begin{minipage}[]{6cm}
\smallskip
{\small 2x Observing Specialist }
\medskip
\end{minipage}
\\ \hline
\multicolumn{2}{r}{T. Cycle \href{https://jira.lsstcorp.org/secure/Tests.jspa\#/testCycle/LVV-C230}{LVV-C230} owner:} &
\multicolumn{2}{l}{\textbf{
Chuck Claver }
} \\\hline
\textbf{Test Cases} & \textbf{Assigned to} & \textbf{Executed by} & \textbf{Additional Test Personnel} \\ \hline
\href{https://jira.lsstcorp.org/secure/Tests.jspa#/testCase/LVV-T2707}{LVV-T2707}
& {\small Chuck Claver } & {\small  } &
\begin{minipage}[]{6cm}
\smallskip
{\small Telescope and Dome operator(s).\\
For the Tailgate meeting, all personnel involved in observations with
the main telescope during the night. }
\medskip
\end{minipage}
\\ \hline
\href{https://jira.lsstcorp.org/secure/Tests.jspa#/testCase/LVV-T2714}{LVV-T2714}
& {\small Chuck Claver } & {\small  } &
\begin{minipage}[]{6cm}
\smallskip
{\small Two observing specialists. }
\medskip
\end{minipage}
\\ \hline
\href{https://jira.lsstcorp.org/secure/Tests.jspa#/testCase/LVV-T2732}{LVV-T2732}
& {\small Chuck Claver } & {\small Holger Drass } &
\begin{minipage}[]{6cm}
\smallskip
{\small Observing Specialist\\
Systems Engineer }
\medskip
\end{minipage}
\\ \hline
\href{https://jira.lsstcorp.org/secure/Tests.jspa#/testCase/LVV-T2715}{LVV-T2715}
& {\small Chuck Claver } & {\small  } &
\begin{minipage}[]{6cm}
\smallskip
{\small 2x Observing Specialist }
\medskip
\end{minipage}
\\ \hline
\multicolumn{2}{r}{T. Cycle \href{https://jira.lsstcorp.org/secure/Tests.jspa\#/testCycle/LVV-C232}{LVV-C232} owner:} &
\multicolumn{2}{l}{\textbf{
Chuck Claver }
} \\\hline
\textbf{Test Cases} & \textbf{Assigned to} & \textbf{Executed by} & \textbf{Additional Test Personnel} \\ \hline
\href{https://jira.lsstcorp.org/secure/Tests.jspa#/testCase/LVV-T2738}{LVV-T2738}
& {\small Chuck Claver } & {\small Holger Drass } &
\begin{minipage}[]{6cm}
\smallskip
{\small Data analyst\\
Systems Engineer }
\medskip
\end{minipage}
\\ \hline
\href{https://jira.lsstcorp.org/secure/Tests.jspa#/testCase/LVV-T2703}{LVV-T2703}
& {\small Chuck Claver } & {\small  } &
\begin{minipage}[]{6cm}
\smallskip
{\small Data analyst\\
Systems Engineer }
\medskip
\end{minipage}
\\ \hline
\href{https://jira.lsstcorp.org/secure/Tests.jspa#/testCase/LVV-T2749}{LVV-T2749}
& {\small Holger Drass } & {\small  } &
\begin{minipage}[]{6cm}
\smallskip
{\small Data analyst\\
SE specialist }
\medskip
\end{minipage}
\\ \hline
\multicolumn{2}{r}{T. Cycle \href{https://jira.lsstcorp.org/secure/Tests.jspa\#/testCycle/LVV-C233}{LVV-C233} owner:} &
\multicolumn{2}{l}{\textbf{
Chuck Claver }
} \\\hline
\textbf{Test Cases} & \textbf{Assigned to} & \textbf{Executed by} & \textbf{Additional Test Personnel} \\ \hline
\href{https://jira.lsstcorp.org/secure/Tests.jspa#/testCase/LVV-T2707}{LVV-T2707}
& {\small Chuck Claver } & {\small  } &
\begin{minipage}[]{6cm}
\smallskip
{\small Telescope and Dome operator(s).\\
For the Tailgate meeting, all personnel involved in observations with
the main telescope during the night. }
\medskip
\end{minipage}
\\ \hline
\href{https://jira.lsstcorp.org/secure/Tests.jspa#/testCase/LVV-T2714}{LVV-T2714}
& {\small Chuck Claver } & {\small  } &
\begin{minipage}[]{6cm}
\smallskip
{\small Two observing specialists. }
\medskip
\end{minipage}
\\ \hline
\href{https://jira.lsstcorp.org/secure/Tests.jspa#/testCase/LVV-T2740}{LVV-T2740}
& {\small Chuck Claver } & {\small  } &
\begin{minipage}[]{6cm}
\smallskip
{\small Observing Specialist\\
Systems Engineer }
\medskip
\end{minipage}
\\ \hline
\href{https://jira.lsstcorp.org/secure/Tests.jspa#/testCase/LVV-T2715}{LVV-T2715}
& {\small Chuck Claver } & {\small  } &
\begin{minipage}[]{6cm}
\smallskip
{\small 2x Observing Specialist }
\medskip
\end{minipage}
\\ \hline
\end{longtable}
}

\newpage

\section{Test Campaign Overview}
\label{sect:overview}

\subsection{Summary}
\label{sect:summarytable}

{\small
\begin{longtable}{p{2cm}cp{2.3cm}p{8.6cm}p{2.3cm}}
\toprule
\multicolumn{2}{r}{ T. Plan \href{https://jira.lsstcorp.org/secure/Tests.jspa\#/testPlan/LVV-P100}{LVV-P100}:} &
\multicolumn{2}{p{10.9cm}}{\textbf{ TMA Pointing and Tracking Verification }} & Draft \\\hline
\multicolumn{2}{r}{ T. Cycle \href{https://jira.lsstcorp.org/secure/Tests.jspa\#/testCycle/LVV-C224}{LVV-C224}:} &
\multicolumn{2}{p{10.9cm}}{\textbf{ TMA Pointing and Tracking - Positional Calibration and Instrument
Characterisation }} & In Progress \\\hline
\textbf{Test Cases} &  \textbf{Ver.}  \\\toprule
\href{https://jira.lsstcorp.org/secure/Tests.jspa#/testCase/LVV-T2707}{LVV-T2707}
&  2
\\
\href{https://jira.lsstcorp.org/secure/Tests.jspa#/testCase/LVV-T2706}{LVV-T2706}
&  1
\\
\href{https://jira.lsstcorp.org/secure/Tests.jspa#/testCase/LVV-T2705}{LVV-T2705}
&  1
\\
\\\hline
\multicolumn{2}{r}{ T. Cycle \href{https://jira.lsstcorp.org/secure/Tests.jspa\#/testCycle/LVV-C228}{LVV-C228}:} &
\multicolumn{2}{p{10.9cm}}{\textbf{ TMA Pointing and Tracking - Part 1 and last Part - Pointing using
StarTracker 50" - Pointing Repeatability 1" -- StarTracker }} & In Progress \\\hline
\textbf{Test Cases} &  \textbf{Ver.}  \\\toprule
\href{https://jira.lsstcorp.org/secure/Tests.jspa#/testCase/LVV-T2707}{LVV-T2707}
&  2
\\
\href{https://jira.lsstcorp.org/secure/Tests.jspa#/testCase/LVV-T2714}{LVV-T2714}
&  1
\\
\href{https://jira.lsstcorp.org/secure/Tests.jspa#/testCase/LVV-T2730}{LVV-T2730}
&  1
\\
\href{https://jira.lsstcorp.org/secure/Tests.jspa#/testCase/LVV-T2715}{LVV-T2715}
&  1
\\
\\\hline
\multicolumn{2}{r}{ T. Cycle \href{https://jira.lsstcorp.org/secure/Tests.jspa\#/testCycle/LVV-C229}{LVV-C229}:} &
\multicolumn{2}{p{10.9cm}}{\textbf{ TMA Pointing and Tracking - Part 2 - Reverse on the Sky Part 1 - 50"-
StarTracker }} & In Progress \\\hline
\textbf{Test Cases} &  \textbf{Ver.}  \\\toprule
\href{https://jira.lsstcorp.org/secure/Tests.jspa#/testCase/LVV-T2707}{LVV-T2707}
&  2
\\
\href{https://jira.lsstcorp.org/secure/Tests.jspa#/testCase/LVV-T2714}{LVV-T2714}
&  1
\\
\href{https://jira.lsstcorp.org/secure/Tests.jspa#/testCase/LVV-T2731}{LVV-T2731}
&  1
\\
\href{https://jira.lsstcorp.org/secure/Tests.jspa#/testCase/LVV-T2715}{LVV-T2715}
&  1
\\
\\\hline
\multicolumn{2}{r}{ T. Cycle \href{https://jira.lsstcorp.org/secure/Tests.jspa\#/testCycle/LVV-C230}{LVV-C230}:} &
\multicolumn{2}{p{10.9cm}}{\textbf{ TMA Pointing and Tracking - Part 4 - Offset 0.2" + Slew and Settle + TMA
Tracking Jitter -- Using the DIMM. }} & In Progress \\\hline
\textbf{Test Cases} &  \textbf{Ver.}  \\\toprule
\href{https://jira.lsstcorp.org/secure/Tests.jspa#/testCase/LVV-T2707}{LVV-T2707}
&  2
\\
\href{https://jira.lsstcorp.org/secure/Tests.jspa#/testCase/LVV-T2714}{LVV-T2714}
&  1
\\
\href{https://jira.lsstcorp.org/secure/Tests.jspa#/testCase/LVV-T2732}{LVV-T2732}
&  1
\\
\href{https://jira.lsstcorp.org/secure/Tests.jspa#/testCase/LVV-T2715}{LVV-T2715}
&  1
\\
\\\hline
\multicolumn{2}{r}{ T. Cycle \href{https://jira.lsstcorp.org/secure/Tests.jspa\#/testCycle/LVV-C232}{LVV-C232}:} &
\multicolumn{2}{p{10.9cm}}{\textbf{ TMA Pointing and Tracking - Analysis - In Depth }} & In Progress \\\hline
\textbf{Test Cases} &  \textbf{Ver.}  \\\toprule
\href{https://jira.lsstcorp.org/secure/Tests.jspa#/testCase/LVV-T2738}{LVV-T2738}
&  1
\\
\href{https://jira.lsstcorp.org/secure/Tests.jspa#/testCase/LVV-T2703}{LVV-T2703}
&  1
\\
\href{https://jira.lsstcorp.org/secure/Tests.jspa#/testCase/LVV-T2749}{LVV-T2749}
&  1
\\
\\\hline
\multicolumn{2}{r}{ T. Cycle \href{https://jira.lsstcorp.org/secure/Tests.jspa\#/testCycle/LVV-C233}{LVV-C233}:} &
\multicolumn{2}{p{10.9cm}}{\textbf{ TMA Pointing and Tracking - Part 3 - Tracking at Random Positions - 50"
- StarTracker }} & Not Executed \\\hline
\textbf{Test Cases} &  \textbf{Ver.}  \\\toprule
\href{https://jira.lsstcorp.org/secure/Tests.jspa#/testCase/LVV-T2707}{LVV-T2707}
&  2
\\
\href{https://jira.lsstcorp.org/secure/Tests.jspa#/testCase/LVV-T2714}{LVV-T2714}
&  1
\\
\href{https://jira.lsstcorp.org/secure/Tests.jspa#/testCase/LVV-T2740}{LVV-T2740}
&  1
\\
\href{https://jira.lsstcorp.org/secure/Tests.jspa#/testCase/LVV-T2715}{LVV-T2715}
&  1
\\
\\\hline
\caption{Test Campaign Summary}
\label{table:summary}
\end{longtable}
}

\subsection{Overall Assessment}
\label{sect:overallassessment}

Not yet available.

\subsection{Recommended Improvements}
\label{sect:recommendations}

\newpage
\section{Detailed Tests}
\label{sect:detailedtests}

\subsection{Test Cycle LVV-C224 }

Open test cycle {\it \href{https://jira.lsstcorp.org/secure/Tests.jspa#/testrun/LVV-C224}{TMA Pointing and Tracking - Positional Calibration and Instrument
Characterisation}} in Jira.

Test Cycle name: TMA Pointing and Tracking - Positional Calibration and Instrument
Characterisation\\
Status: In Progress

Preparation for the requirements verification for the pointing and
tracking using the Star Tracker and the DIMM on a dedicated mounting
plate connector to the top end of the TMA.\\
This test cycle includes the test cases to prepare the\\

\begin{itemize}
\tightlist
\item
  Metrology of the TMA~
\item
  Calibration of the StarTracker with respect to the TMA~
\item
  DIMM with respect to the StarTracker
\end{itemize}

\subsubsection{Software Version/Baseline}
Star Tracker software version:\\
Dimm software version:\\
CSC software version:\\
Analysis software repository:

\subsubsection{Configuration}
Not provided.

\subsubsection{Test Cases in LVV-C224 Test Cycle}

\paragraph{ LVV-T2707 - Evening Summit Tailgate Meeting - TMA and Dome Testing Safety Assurance }\mbox{}\\

Version \textbf{2}.
Open  \href{https://jira.lsstcorp.org/secure/Tests.jspa#/testCase/LVV-T2707}{\textit{ LVV-T2707 } }
test case in Jira.

Ensure the safety of observation with the main telescope during
nighttime operations.\\
\textbf{Tailgate Meeting:} Hold a tailgate for the upcoming task with
personnel on the summit working during the night. Go over any relevant
procedures, roles, and
responsibilities.\\[2\baselineskip]\textbf{Note:~}Version two is for
tests that do not involve moving or opening the dome.

\textbf{ Preconditions}:\\
All nonessential personnel has vacated the area.

Final comment:\\


Detailed steps :

\begin{tabular}{p{2cm}}
\toprule
Step 1  \\ \hline
\end{tabular}
 Description \\
{\footnotesize
\textbf{Daytime info collection:}

\begin{itemize}
\tightlist
\item
  Revise the
  \href{https://confluence.lsstcorp.org/display/LTS/Summit+Daylogs}{last
  Summit Daylog} about changes that might influence the work during the
  night.
\item
  Confirm that all the workers in the TMA and Dome areas have already
  left. This is best performed during the walk-through at the end of the
  day.~~
\end{itemize}

}
\hdashrule[0.5ex]{\textwidth}{1pt}{3mm}
  Expected Result \\
{\footnotesize
Is the observatory ready to observe?

}

\begin{tabular}{p{2cm}}
\toprule
Step 2  \\ \hline
\end{tabular}
 Description \\
{\footnotesize
\textbf{Alarm system check}\\
Once available:\\

\begin{itemize}
\tightlist
\item
  Confirm that any audio and visual alarms are operating properly. (Need
  details on what, if any, alarms should be checked)
\item
  Confirm that the safety systems for earthquakes and fire are working.
\end{itemize}

}
\hdashrule[0.5ex]{\textwidth}{1pt}{3mm}
  Expected Result \\
{\footnotesize
All alarms are functioning properly.

\begin{itemize}
\tightlist
\item
  Earthquake alert system is working:
\item
  The fire system is working:
\end{itemize}

}

\begin{tabular}{p{2cm}}
\toprule
Step 3  \\ \hline
\end{tabular}
 Description \\
{\footnotesize
\textbf{LOTO status:}\\[2\baselineskip]If LOTO procedures are in use:\\
Set LOTO per (PROCEDURE, attached) at the following locations:

\begin{enumerate}
\tightlist
\item
  LOTO at the Dome
\item
  LOTO of the TMA Drives
\end{enumerate}

}
\hdashrule[0.5ex]{\textwidth}{1pt}{3mm}
  Expected Result \\
{\footnotesize
The appropriate panels have been locked out or released.

}

\begin{tabular}{p{2cm}}
\toprule
Step 4  \\ \hline
\end{tabular}
 Description \\
{\footnotesize
\textbf{Final walkthrough:}\\[2\baselineskip]Perform a final walkthrough
of the dome. Make sure all personnel is cleared out.

}
\hdashrule[0.5ex]{\textwidth}{1pt}{3mm}
  Expected Result \\
{\footnotesize
The dome is clear and safe for TMA movement.\\
The final walkthrough was performed by:

}

\begin{tabular}{p{2cm}}
\toprule
Step 5  \\ \hline
\end{tabular}
 Description \\
{\footnotesize
\textbf{Dome closure:}\\
If the Dome door GIS is available:\\
Exit the Dome, close the door (any details about what specific door)

}
\hdashrule[0.5ex]{\textwidth}{1pt}{3mm}
  Expected Result \\
{\footnotesize
The GIS system is active.

}

\begin{tabular}{p{2cm}}
\toprule
Step 6  \\ \hline
\end{tabular}
 Description \\
{\footnotesize
\textbf{Dome clearance:}\\[2\baselineskip]The Dome clearance is an EIE
task, and they have to sign off.\\
If EIE is not available, perform these steps:

\begin{itemize}
\tightlist
\item
  Make sure that the dome crane is in the parking position. (Hook up)
\item
  Position of the manlifts. Make sure the manlift supports are stored
  and are not on the rotation part of the dome.
\item
  Walkthrough and make sure that there are no obstacles to move.
\end{itemize}

}
\hdashrule[0.5ex]{\textwidth}{1pt}{3mm}
  Example Code \\
{\footnotesize
https://confluence.lsstcorp.org/display/LTS/Dome+Remote+Software+Control+Procedure

}
\hdashrule[0.5ex]{\textwidth}{1pt}{3mm}
  Expected Result \\
{\footnotesize
The Dome is cleared for nightly operations.

}

\begin{tabular}{p{2cm}}
\toprule
Step 7  \\ \hline
\end{tabular}
 Description \\
{\footnotesize
\textbf{PFlow lift}\\
This is part of EIE's safety check.\\
If EIE is not available, perform this step:\\[2\baselineskip]

\begin{itemize}
\tightlist
\item
  The Pflow lift must be stored before moving the dome.
\end{itemize}

}
\hdashrule[0.5ex]{\textwidth}{1pt}{3mm}
  Expected Result \\
{\footnotesize
The PFlow lift is stored properly

}

\begin{tabular}{p{2cm}}
\toprule
Step 8  \\ \hline
\end{tabular}
 Description \\
{\footnotesize
\textbf{Shutter closer}\\
If you have to close the shutter, the Dome must be under LOTO.\\
\textbf{Note:} There is no LOTO available at the moment. Use the
procedure attached to this test case and the information from the
following link:\\
https://confluence.lsstcorp.org/display/LTS/Dome+Remote+Software+Control+Procedure

}
\hdashrule[0.5ex]{\textwidth}{1pt}{3mm}
  Expected Result \\
{\footnotesize
The shutter was closed in a safe way.

}

\begin{tabular}{p{2cm}}
\toprule
Step 9  \\ \hline
\end{tabular}
 Description \\
{\footnotesize
\textbf{GIS activation:}\\
If the GIS for the Dome is available:

\begin{itemize}
\tightlist
\item
  activate the Dome GIS system.
\end{itemize}

}
\hdashrule[0.5ex]{\textwidth}{1pt}{3mm}
  Expected Result \\
{\footnotesize
If possible, the Dome GIS is activated.

}

\begin{tabular}{p{2cm}}
\toprule
Step 10  \\ \hline
\end{tabular}
 Description \\
{\footnotesize
\textbf{Signoff}\\
As a signoff, mark this step as passed

}
\hdashrule[0.5ex]{\textwidth}{1pt}{3mm}
  Expected Result \\
{\footnotesize
Safety Assurance is confirmed to be complete, and testing may proceed.

}

\begin{tabular}{p{2cm}}
\toprule
Step 11  \\ \hline
\end{tabular}
 Description \\
{\footnotesize
\textbf{Night Shift Leader}\\[2\baselineskip]Identify the Night Shift
Leader (first and the second half of the
night).\\[2\baselineskip]\textbf{Note:} This is the person responsible
for deciding when

\begin{itemize}
\tightlist
\item
  the dome is going to be closed.
\item
  to stop observing due to technical issues
\end{itemize}

}
\hdashrule[0.5ex]{\textwidth}{1pt}{3mm}
  Expected Result \\
{\footnotesize
One person is identified as the Night Shift Leader for each shift.

}

\begin{tabular}{p{2cm}}
\toprule
Step 12  \\ \hline
\end{tabular}
 Description \\
{\footnotesize
\textbf{Tailgate Meeting:}\\
Hold a tailgate for the upcoming task with personnel on the summit
working during the night. Go over any relevant procedures, roles, and
responsibilities.\\

\begin{enumerate}
\tightlist
\item
  If the Dome slit doors are not moving automatically, make sure that
  there are three persons with slit closer training available to close
  the slit manually.
\item
  Verify that there are enough persons with driver training available.
\item
  If the StarTracker is going to be used:

  \begin{enumerate}
  \tightlist
  \item
    Clarify who is taking the off the caps in the evening.
  \item
    Take a test image before opening the Dome.
  \item
    Clarify who is installing the caps in the morning.
  \end{enumerate}
\item
  Discuss if surrounding observatories need to be informed. (Necessary
  when light is switched on in the dome during the night.)
\item
  If surrounding observatories need to be informed, clarify who is going
  to inform them and what information should be transmitted.
\item
  Check weather conditions and weather forecasts are within the
  specifications for observations.
\item
  Describe the tasks planned for the night.
\end{enumerate}

}
\hdashrule[0.5ex]{\textwidth}{1pt}{3mm}
  Expected Result \\
{\footnotesize
All involved personnel understands their roles and responsibilities.\\

\begin{itemize}
\tightlist
\item
  If the Dome slit doors are not moving automatically, there are at
  least two persons with slit closer training available to close the
  slit manually.
\item
  There are enough people with driver training available.
\item
  If the StarTracker is used,

  \begin{itemize}
  \tightlist
  \item
    the caps are taken off in the evening by:
  \item
    the test image is taken by:
  \item
    the StarTracker caps are installed in the morning by:
  \end{itemize}
\item
  If surrounding observatories need to be informed,~

  \begin{itemize}
  \tightlist
  \item
    they are informed by:
  \item
    The following information will be transmitted:
  \end{itemize}
\item
  The weather conditions permit us to open the dome and do the planned
  testing.~
\item
  The tasks planned for the night are:
\end{itemize}

}

\begin{tabular}{p{2cm}}
\toprule
Step 13  \\ \hline
\end{tabular}
 Description \\
{\footnotesize
\textbf{Tailgate Meeting -- Part II:}\\
If new personnel is participating in the nightly summit activities:\\

\begin{itemize}
\tightlist
\item
  Clarify that all personnel has PPE.
\item
  Clarify that persons that need to go up into altitude have fall
  protection training.
\item
  Confirm that we have enough personnel to open/close the dome shutter
  if required.
\item
  Remind everybody that the emergency phone numbers are on the control
  room table.
\item
  Do we have anyone else in the building? Confirm their location.~
\end{itemize}

}
\hdashrule[0.5ex]{\textwidth}{1pt}{3mm}
  Expected Result \\
{\footnotesize
\begin{itemize}
\tightlist
\item
  All personnel has the required PPE.
\item
  Persons that need to go up into altitude have the fall protection
  training
\item
  Everybody acknowledges that the emergency phone numbers are on the
  control room table.
\end{itemize}

}

\begin{tabular}{p{2cm}}
\toprule
Step 14  \\ \hline
\end{tabular}
 Description \\
{\footnotesize
\textbf{TMA and Dome contact}\\
Person in charge of the TMA interlocks\\
Dome responsible

}
\hdashrule[0.5ex]{\textwidth}{1pt}{3mm}
  Expected Result \\
{\footnotesize
TMA and Dome contacts are known

}

\begin{tabular}{p{2cm}}
\toprule
Step 15  \\ \hline
\end{tabular}
 Description \\
{\footnotesize
\textbf{Radio Communication}\\

\begin{itemize}
\tightlist
\item
  Make sure one radio is switched to channel 1, and the volume is high

  \begin{itemize}
  \tightlist
  \item
    Paramedics, mountain assistants (in replacement of the paramedics),
    guards, and surrounding observatories are listening to this channel.
  \end{itemize}
\item
  Make sure one radio is switched to channel 3, and the volume is high

  \begin{itemize}
  \tightlist
  \item
    Rubin's internal coordination channel
  \end{itemize}
\end{itemize}

}
\hdashrule[0.5ex]{\textwidth}{1pt}{3mm}
  Expected Result \\
{\footnotesize
The radios are switched on and on high volume.

}

\begin{tabular}{p{2cm}}
\toprule
Step 16  \\ \hline
\end{tabular}
 Description \\
{\footnotesize
\textbf{Cars}

\begin{itemize}
\tightlist
\item
  Make sure enough cars are available to go to the hotel.
\item
  Make sure the keys for the cars are available.
\end{itemize}

}
\hdashrule[0.5ex]{\textwidth}{1pt}{3mm}
  Expected Result \\
{\footnotesize
Sufficient cars and their keys are available.

}

\begin{tabular}{p{2cm}}
\toprule
Step 17  \\ \hline
\end{tabular}
 Description \\
{\footnotesize
\textbf{ComCam safety}\\[2\baselineskip]Put ComCam in a safe state for
moving. This includes:

\begin{enumerate}
\tightlist
\item
  CryoTels are under observation for vibrations (i.e. a microphone or
  webcam is observing them and operating correctly)
\item
  Turbo pumps are off
\end{enumerate}

}
\hdashrule[0.5ex]{\textwidth}{1pt}{3mm}
  Expected Result \\
{\footnotesize
ComCam is in a safe state for TMA movement.

}

\begin{tabular}{p{2cm}}
\toprule
Step 18  \\ \hline
\end{tabular}
 Description \\
{\footnotesize
\textbf{TMA moving space}\\[2\baselineskip]Go to the dome and visually
verify that there is unrestricted space for the TMA movement.

}
\hdashrule[0.5ex]{\textwidth}{1pt}{3mm}
  Expected Result \\
{\footnotesize
The space is clear and no objects will be struck when the TMA moves.

}

\paragraph{ LVV-T2706 - StarTracker Positional Calibration }\mbox{}\\

Version \textbf{1}.
Open  \href{https://jira.lsstcorp.org/secure/Tests.jspa#/testCase/LVV-T2706}{\textit{ LVV-T2706 } }
test case in Jira.

This procedure provides a 2-point calibration between the optical axes
of the TMA and StarTracker.\\
It assumes the TMA azimuth axis is co-aligned with the optical axis when
the TMA is pointed at zenith as indicated by an elevation encoder value
of 90 degrees.\\
For the pointing tests, the azimuth and elevations are the two axes that
matter and should be measured.\\[2\baselineskip]This procedure has two
parts:\\
1) Calibrate the StarTracker's line-of-sight to the reference axis as
defined by the center of TMA azimuth rotation.\\
2) Establish the index reference of azimuth = 0 deg (180?) and
altitude/latitude reference in elevation with the TMA pointing at the
South Celestial Pole (SCP).\\[2\baselineskip]In each instance, the
StarTracker will produce arcing star trails. The arcs will be fitted to
find the center of rotation in StarTracker pixels (X, Y).\\
For part 1 of this procedure, this position will serve as the reference
location for which the World Coordinate (WCS) system is derived using
astrometry.net.\\
For part 2, the circle center represents the SCP. Iteratively the
pointing of the TMA with offset such that SCP (X,Y) coincides with the
Zenith reference (X,Y) determined in Part 1.\\
The encoder values here are noted, and physical markings are placed on
the TMA (position TBD).\\[2\baselineskip]Notes:\\

\begin{itemize}
\tightlist
\item
  This uses the Laser Tracker Metrology.
\item
  Flexure check of the Star Tracker's mechanical support with respect to
  TMA flexure vs. Elevation Angle\ldots{}??

  \begin{itemize}
  \tightlist
  \item
    Determine lat-lon of azimuth track - e.g. rotation centre.
  \end{itemize}
\item
  There will be an offset between the optical axis of the StarTracker
  with the TMA optical axis.
\item
  The internal ``optical'' axis, as defined by the littering the M1M3
  SMRs (see previous calibration procedure), will be calibrated to the
  azimuth-elevation cordite references as part of the final construction
  of the pointing model when the optics are installed in the presence of
  ComCam.
\end{itemize}

\textbf{ Preconditions}:\\
StarTracker is installed on the TMA.

Final comment:\\


Detailed steps :

\begin{tabular}{p{2cm}}
\toprule
Step 1  \\ \hline
\end{tabular}
 Description \\
{\footnotesize
\subsection{Part 1: TMA Line-of-Site
Calibration:}\label{part-1-tma-line-of-site-calibration}

}
\hdashrule[0.5ex]{\textwidth}{1pt}{3mm}
  Expected Result \\
{\footnotesize

}

\begin{tabular}{p{2cm}}
\toprule
Step 2  \\ \hline
\end{tabular}
 Description \\
{\footnotesize
\textbf{Taking StarTracker images - TMA pointing to the SCP}\\

\begin{itemize}
\tightlist
\item
  Keep the TMA stationary - e.g. no tracking
\item
  Obtain a series (e.g. 100 or more) of back-to-back StarTracker images
\item
  10-sec exposures -- stars not saturated.
\end{itemize}

}
\hdashrule[0.5ex]{\textwidth}{1pt}{3mm}
  Expected Result \\
{\footnotesize
Images showing arcs are available.

}

\begin{tabular}{p{2cm}}
\toprule
Step 3  \\ \hline
\end{tabular}
 Description \\
{\footnotesize
\textbf{SCP StarTracker Image analysis}\\

\begin{itemize}
\tightlist
\item
  Run the StarTracker Images through any preprocessing - e.g. ISR (Image
  Signature Removal)
\item
  Run the processed StarTracker images through astrometry.net to obtain
  WCS solutions centered on the reference determined at the zenith
\item
  Calculate the mean, stdev, and time series in Ra-Dec for all solutions
\item
  Calculate the mean, stdev, and time series in Az-El for all solutions
\item
  Calculate the offset between the reference Ra-Dec and the South
  Celestial Pole - e.g., Az = 180.000000, El = Observatory Latitude\\
  We need the precise geophysical (Lat-Lon) coordinates of the azimuth
  track center. Do we have these values in our metrology?
\item
  Compare the offsets from the SMR references with those from the on-sky
  measurements.
\item
  Develop a simple Sin(el), Cos(el) model for TMA+StarTracker flexure\\
  This model will be applied to the on-sky-pointing verification test
  measurements.
\item
  Determine the actual resolution of the long and short focal length
  StarTracker. Astronmetry net will provide the plate scale
\end{itemize}

}
\hdashrule[0.5ex]{\textwidth}{1pt}{3mm}
  Expected Result \\
{\footnotesize
Actual resolution of the long and short focal length StarTracker.

}

\begin{tabular}{p{2cm}}
\toprule
Step 4  \\ \hline
\end{tabular}
 Description \\
{\footnotesize
\textbf{Ra-Dec characterization}

\begin{itemize}
\tightlist
\item
  Calculate the mean, stdev, and time series in Ra-Dec for all solutions
\item
  Calculate the offset between the reference Ra-Dec and the South
  Celestial Pole - e.g., Az = 180.000000, El = Observatory Latitude\\
  We need the precise geophysical (Lat-Lon) coordinates of the azimuth
  track center.
\item
  {Do we have these values in our metrology?}
\end{itemize}

}
\hdashrule[0.5ex]{\textwidth}{1pt}{3mm}
  Expected Result \\
{\footnotesize
\begin{itemize}
\tightlist
\item
  Plots showing the mean and stdev for the observed Ra-Dec positions.
\item
  Values for the difference between the South Celestial Pole and the
  AZ=180 and El= Obs. Latitude
\end{itemize}

}

\begin{tabular}{p{2cm}}
\toprule
Step 5  \\ \hline
\end{tabular}
 Description \\
{\footnotesize
\textbf{Az-El characterization}\\

\begin{itemize}
\tightlist
\item
  Calculate the mean, stdev, and time series in Az-El for all solutions
\end{itemize}

}
\hdashrule[0.5ex]{\textwidth}{1pt}{3mm}
  Expected Result \\
{\footnotesize
Plots for the mean, stdev and time series in Az-EL coordinates.

}

\begin{tabular}{p{2cm}}
\toprule
Step 6  \\ \hline
\end{tabular}
 Description \\
{\footnotesize
\textbf{Laser Tracker vs. on-sky measurement comparison}\\

\begin{itemize}
\tightlist
\item
  Compare the offsets from the SMR references with those from the on-sky
  measurements.
\end{itemize}

}
\hdashrule[0.5ex]{\textwidth}{1pt}{3mm}
  Expected Result \\
{\footnotesize
Plot showing the difference between Laser Tracker and on-sky
measurements.

}

\begin{tabular}{p{2cm}}
\toprule
Step 7  \\ \hline
\end{tabular}
 Description \\
{\footnotesize
\textbf{TMA + StarTracker flexure model}\\
This model will be applied to the on-sky-pointing verification test
measurements.\\

\begin{itemize}
\tightlist
\item
  Develop a simple Sin(el), Cos(el) model for TMA+StarTracker flexure
\end{itemize}

}
\hdashrule[0.5ex]{\textwidth}{1pt}{3mm}
  Expected Result \\
{\footnotesize
The equation describing the flexure of the TMA.\\
Plot the flexure model.

}

\begin{tabular}{p{2cm}}
\toprule
Step 8  \\ \hline
\end{tabular}
 Description \\
{\footnotesize
\textbf{TMA pointing:}\\
Point the TMA to zenith - elevation = 90.000000 degrees\\
Verify zenith orientation with a precision inclinometer.

}
\hdashrule[0.5ex]{\textwidth}{1pt}{3mm}
  Expected Result \\
{\footnotesize
The TMA points into the Zenith.\\
The inclinometer shows 90 deg.

}

\begin{tabular}{p{2cm}}
\toprule
Step 9  \\ \hline
\end{tabular}
 Description \\
{\footnotesize
\textbf{StarTracker plate vs. TMA calibration:}\\
Determine the TEA plane and the StarTracker plate to determine the
orientation with respect to each other using the laser tracker.

\begin{itemize}
\tightlist
\item
  Measure StarTracker and M1M3 SMR references.
\end{itemize}

}
\hdashrule[0.5ex]{\textwidth}{1pt}{3mm}
  Expected Result \\
{\footnotesize
List fo StarTracker and M1M3 SMR positions.

}

\begin{tabular}{p{2cm}}
\toprule
Step 10  \\ \hline
\end{tabular}
 Description \\
{\footnotesize
\textbf{Analyze the laser tracker data:}\\

\begin{itemize}
\tightlist
\item
  M1M3: Fit a circle, determine the centered vector normal
\item
  StarTracker: Fit a plane, determine the centered vector normal
\item
  Calculate 3-space angles between M1M3 and StarTracker vector normal
  references
\item
  Calculate the difference between reference vector normals at zenith
  and those determined in the Initial Metrology Procedure
\end{itemize}

}
\hdashrule[0.5ex]{\textwidth}{1pt}{3mm}
  Expected Result \\
{\footnotesize
The angles between the vectors normals are known.\\[2\baselineskip]

}

\begin{tabular}{p{2cm}}
\toprule
Step 11  \\ \hline
\end{tabular}
 Description \\
{\footnotesize
\textbf{TMA Zenith rotation}\\[2\baselineskip]With TMA pointed at the
zenith

\begin{itemize}
\tightlist
\item
  Command the TMA to rotate about the Azimuth axis only
\item
  While the TMA is in motion, take a series of exposures (EXPTIME TBD)
  with the StarTracker (both optical systems)
\end{itemize}

}
\hdashrule[0.5ex]{\textwidth}{1pt}{3mm}
  Expected Result \\
{\footnotesize
The StarTracker images show circular arcs.

}

\begin{tabular}{p{2cm}}
\toprule
Step 12  \\ \hline
\end{tabular}
 Description \\
{\footnotesize
\textbf{TMA bore-sight determination~}\\

\begin{itemize}
\tightlist
\item
  Stack/combine individual images to generate an image for analysis that
  has at least 180 degrees of arc to it.
\item
  Analyze the combined StarTracker to determine the pixel (X, Y) of the
  circle center.

  \begin{itemize}
  \tightlist
  \item
    Chuck has a simple Python script using the Hough transform that does
    this. It is computationally intensive but works. Other methods are
    welcome.
  \end{itemize}
\item
  Note the coordinates of the cycle center in the StarTracker arc-trail
  image. This will be considered at the bore-sight of the TMA.
\item
  Depending on the confidence level of the vector normal references, an
  angular correction can be applied to the circle center to transfer the
  reference to the M1M3 vector normal.
\item
  The reference pixel (X,Y) will subsequently be used with
  astrometry.net as the WCS reference when estimating the Ra-Dec of the
  StarTracker images during the pointing verification tests.
\end{itemize}

}
\hdashrule[0.5ex]{\textwidth}{1pt}{3mm}
  Expected Result \\
{\footnotesize
Combined images with a least 180 deg arcs.\\
Reference (X,Y)-pixel value for astrometry.net is
known\\[2\baselineskip]

}

\begin{tabular}{p{2cm}}
\toprule
Step 13  \\ \hline
\end{tabular}
 Description \\
{\footnotesize
\subsection{Part 2: Calibration of Elevation
Dependence}\label{part-2-calibration-of-elevation-dependence}

}
\hdashrule[0.5ex]{\textwidth}{1pt}{3mm}
  Expected Result \\
{\footnotesize

}

\begin{tabular}{p{2cm}}
\toprule
Step 14  \\ \hline
\end{tabular}
 Description \\
{\footnotesize
\textbf{Telescope pointing}\\
Point the TMA to the South Celestial Pole (SCP)

}
\hdashrule[0.5ex]{\textwidth}{1pt}{3mm}
  Expected Result \\
{\footnotesize
The TMA points to the SCP.

}

\begin{tabular}{p{2cm}}
\toprule
Step 15  \\ \hline
\end{tabular}
 Description \\
{\footnotesize
\textbf{Measure StarTracker and M1M3 SMR references}\\
This repeats the previous measurement at a different inclination and
considers therefore the TMA's flexure due to gravity.\\

\begin{itemize}
\tightlist
\item
  M1M3: ~Fit a circle, determine the vector normal
\item
  StarTracker: ~Fit a plane, determine the vector normal
\item
  Calculate 3-space angles between M1M3 and StarTracker references
\item
  Calculate the difference between reference vector normals at the
  zenith, and this determined in the Initial Metrology Procedure
\end{itemize}

}
\hdashrule[0.5ex]{\textwidth}{1pt}{3mm}
  Expected Result \\
{\footnotesize
Plots showing the changes of SMR positions depending on elevation.

}

\paragraph{ LVV-T2705 - DIMM Calibration with respect to the StarTracker }\mbox{}\\

Version \textbf{1}.
Open  \href{https://jira.lsstcorp.org/secure/Tests.jspa#/testCase/LVV-T2705}{\textit{ LVV-T2705 } }
test case in Jira.

Provide a 2-point calibration between the optical axes of the TMA and
StarTracker.\\
It assumes the TMA azimuth axis is co-aligned with the optical axis when
the TMA is pointed at the zenith, as indicated by an elevation encoder
value of 90 degrees.\\
For the pointing tests, the azimuth and elevations are the two axes that
matter and should be measured. The internal ``optical'' axis, as defined
by the littering of the M1M3 SMRs (see previous calibration procedure),
will be calibrated to the azimuth-elevation cordite references as part
of the final construction of the pointing model when the optics are
installed in the presence of ComCam (Note a similar but extended
procedure can/will be used with ComCam to re-verify the TMA pointing
performance.\\
Notes:

\begin{itemize}
\tightlist
\item
  Following interaction with Pelayo, it has been realized
  \href{https://confluence.lsstcorp.org/download/attachments/191987725/AF-173402-13_rev01\%20-\%20DWG\%20LTS-213\%20-\%20FINAL\%20ASSEMBLY.pdf?version=1\&modificationDate=1660149762000\&api=v2}{TMA
  Chile Metrology Report} referenced two coordinate systems. Pages 1-4
  are in the external TMA azimuth-elevation coordinates, whereas from
  page 5 on are in the internal TMA coordinates.
\item
  All indications are that there is good alignment between the Zenith
  pointing vector defined by the TMA azimuth track and the optical axis
  defined by the pointing vector from the M1M3 SMR fit.
\item
  There is a substantial body of metrology that is either missing, in
  places unknown but accessible, or inaccessible. As we enter more
  integrated verification must be rectified.
\end{itemize}

Associate the center of the DIMM center to the Star tracker.

\textbf{ Preconditions}:\\
The StarTracker and DIMM must be working.\\
The CSC for TMA and Dome must be working.

Final comment:\\


Detailed steps :

\begin{tabular}{p{2cm}}
\toprule
Step 1  \\ \hline
\end{tabular}
 Description \\
{\footnotesize
\textbf{Make the reference between the TMA interface plane to the plane
where the StarTracker is mounted.}\\[2\baselineskip]Where the U-shaped
bars are mounted, at the vertex, the center, and at one more position,
SMRs should be mounted.

}
\hdashrule[0.5ex]{\textwidth}{1pt}{3mm}
  Expected Result \\
{\footnotesize
Show the normal vector is intersecting the optical axis\\
What is the angular of the StarTracker plane and the optical axis

}

\begin{tabular}{p{2cm}}
\toprule
Step 2  \\ \hline
\end{tabular}
 Description \\
{\footnotesize
Repeat the measurement at Zenith and Horizon.

}
\hdashrule[0.5ex]{\textwidth}{1pt}{3mm}
  Expected Result \\
{\footnotesize
A slight difference due to the gravitations bending of the TMA.

}

\begin{tabular}{p{2cm}}
\toprule
Step 3  \\ \hline
\end{tabular}
 Description \\
{\footnotesize
\textbf{ComCam -- StarTracker -- Interface plane relation}\\
This is a sanity check and will help later to associate ComCam images
with the TMA orientation.\\

\begin{itemize}
\tightlist
\item
  Use the ComCam SMR and redo the measurements to establish the plane
  for ComCam.
\item
  Compare normal vectors for the Star Tracker plane, the Interface
  Plane, and the ComCam Plane.
\end{itemize}

}
\hdashrule[0.5ex]{\textwidth}{1pt}{3mm}
  Expected Result \\
{\footnotesize
Set of values showing the differences in the vectors for ComCam and the
StarTracker

}

\begin{tabular}{p{2cm}}
\toprule
Step 4  \\ \hline
\end{tabular}
 Description \\
{\footnotesize
Take images with the DIMM and the StarTracker at the same time.

}
\hdashrule[0.5ex]{\textwidth}{1pt}{3mm}
  Expected Result \\
{\footnotesize
Both images~

\begin{itemize}
\tightlist
\item
  appear on Rubin TV~
\item
  are stored in the LFA~
\item
  a link stored in the EFD
\item
  the link can be accessed through Chronograph.
\end{itemize}

}

\begin{tabular}{p{2cm}}
\toprule
Step 5  \\ \hline
\end{tabular}
 Description \\
{\footnotesize
\textbf{Pointing outside of the dome}\\
To determine the offset between the StarTracker and the DIMM\\

\begin{itemize}
\tightlist
\item
  Point the TMA at the weather tower/ DIMM tower.~
\item
  Connect the Laser Tracker via a CSC during the testing.

  \begin{itemize}
  \tightlist
  \item
    This needs Dave and IT to get the Laser Tracker into the Network and
    the EFD.
  \end{itemize}
\end{itemize}

}
\hdashrule[0.5ex]{\textwidth}{1pt}{3mm}
  Expected Result \\
{\footnotesize
Images with all three instruments are available.

}

\begin{tabular}{p{2cm}}
\toprule
Step 6  \\ \hline
\end{tabular}
 Description \\
{\footnotesize
\textbf{Instruments center difference in Pixel}\\[2\baselineskip]Find
the center of rotation in the DIMM images.\\
Calculate the difference in Pixel in Az and El of both
instruments.\\[2\baselineskip]

}
\hdashrule[0.5ex]{\textwidth}{1pt}{3mm}
  Expected Result \\
{\footnotesize
The number of pixels in X and Y between the center of the StarTracker
and the center of the DIMM are known.

}

\begin{tabular}{p{2cm}}
\toprule
Step 7  \\ \hline
\end{tabular}
 Description \\
{\footnotesize
\textbf{Determine the dome position with respect to the StarTracker's
FoV.}\\
This ensures that we have enough tracking distance without being
obscured by the dome.

}
\hdashrule[0.5ex]{\textwidth}{1pt}{3mm}
  Expected Result \\
{\footnotesize
The dome position is known and it is confirmed that there is enough time
to track.

}

\subsection{Test Cycle LVV-C228 }

Open test cycle {\it \href{https://jira.lsstcorp.org/secure/Tests.jspa#/testrun/LVV-C228}{TMA Pointing and Tracking - Part 1 and last Part - Pointing using
StarTracker 50" - Pointing Repeatability 1" -- StarTracker}} in Jira.

Test Cycle name: TMA Pointing and Tracking - Part 1 and last Part - Pointing using
StarTracker 50" - Pointing Repeatability 1" -- StarTracker\\
Status: In Progress

Requirements verification for the pointing and tracking using the Star
Tracker and the DIMM on a dedicated mounting plate connector to the top
end of the TMA.\\
Short-focal StarTracker. 7" per pixel resolution. Precision is about
1".\\
Long-focal StarTracker is even better. 375 mm objective. 3.1" arc pixel
resolution.\\[2\baselineskip]

\subsubsection{Software Version/Baseline}
Star Tracker software version:\\
Dimm software version:\\
CSC software version:\\
Analysis software repository:

\subsubsection{Configuration}
Not provided.

\subsubsection{Test Cases in LVV-C228 Test Cycle}

\paragraph{ LVV-T2707 - Evening Summit Tailgate Meeting - TMA and Dome Testing Safety Assurance }\mbox{}\\

Version \textbf{2}.
Open  \href{https://jira.lsstcorp.org/secure/Tests.jspa#/testCase/LVV-T2707}{\textit{ LVV-T2707 } }
test case in Jira.

Ensure the safety of observation with the main telescope during
nighttime operations.\\
\textbf{Tailgate Meeting:} Hold a tailgate for the upcoming task with
personnel on the summit working during the night. Go over any relevant
procedures, roles, and
responsibilities.\\[2\baselineskip]\textbf{Note:~}Version two is for
tests that do not involve moving or opening the dome.

\textbf{ Preconditions}:\\
All nonessential personnel has vacated the area.

Final comment:\\


Detailed steps :

\begin{tabular}{p{2cm}}
\toprule
Step 1  \\ \hline
\end{tabular}
 Description \\
{\footnotesize
\textbf{Daytime info collection:}

\begin{itemize}
\tightlist
\item
  Revise the
  \href{https://confluence.lsstcorp.org/display/LTS/Summit+Daylogs}{last
  Summit Daylog} about changes that might influence the work during the
  night
\item
  Revise at 16.00 CLT to have time for questions.
\end{itemize}

}
\hdashrule[0.5ex]{\textwidth}{1pt}{3mm}
  Expected Result \\
{\footnotesize
Is the observatory ready to observe?

}

\begin{tabular}{p{2cm}}
\toprule
Step 2  \\ \hline
\end{tabular}
 Description \\
{\footnotesize
\textbf{Alarm system check}\\
Once available:\\

\begin{itemize}
\tightlist
\item
  Confirm that any audio and visual alarms are operating properly. (Need
  details on what, if any, alarms should be checked)
\item
  Confirm that the safety systems for earthquakes and fire are working.
\end{itemize}

}
\hdashrule[0.5ex]{\textwidth}{1pt}{3mm}
  Expected Result \\
{\footnotesize
All alarms are functioning properly.

\begin{itemize}
\tightlist
\item
  Earthquake alert system is working:
\item
  The fire system is working:
\end{itemize}

}

\begin{tabular}{p{2cm}}
\toprule
Step 3  \\ \hline
\end{tabular}
 Description \\
{\footnotesize
\textbf{LOTO status:}\\[2\baselineskip]If LOTO procedures are in use:\\
Set LOTO per (PROCEDURE, attached) at the following locations:

\begin{enumerate}
\tightlist
\item
  LOTO at the Dome
\item
  LOTO of the TMA Drives
\end{enumerate}

}
\hdashrule[0.5ex]{\textwidth}{1pt}{3mm}
  Expected Result \\
{\footnotesize
The appropriate panels have been locked out or released.

}

\begin{tabular}{p{2cm}}
\toprule
Step 4  \\ \hline
\end{tabular}
 Description \\
{\footnotesize
\textbf{Final walkthrough:}\\[2\baselineskip]Perform a final walkthrough
of the dome. Make sure all personnel is cleared out.

}
\hdashrule[0.5ex]{\textwidth}{1pt}{3mm}
  Expected Result \\
{\footnotesize
The dome is clear and safe for TMA movement.\\
The final walkthrough was performed by:

}

\begin{tabular}{p{2cm}}
\toprule
Step 5  \\ \hline
\end{tabular}
 Description \\
{\footnotesize
\textbf{Dome closure:}\\
If the Dome door GIS is available:\\
Exit the Dome, close the door (any details about what specific door)

}
\hdashrule[0.5ex]{\textwidth}{1pt}{3mm}
  Expected Result \\
{\footnotesize
The GIS system is active.

}

\begin{tabular}{p{2cm}}
\toprule
Step 6  \\ \hline
\end{tabular}
 Description \\
{\footnotesize
\textbf{Dome clearance:}\\[2\baselineskip]The Dome clearance is an EIE
task, and they have to sign off.\\
If EIE is not available, perform these steps:

\begin{itemize}
\tightlist
\item
  Make sure that the dome crane is in the parking position. (Hook up)
\item
  Position of the manlifts. Make sure the manlift supports are stored
  and are not on the rotation part of the dome.
\item
  Walkthrough and make sure that there are no obstacles to move.
\end{itemize}

}
\hdashrule[0.5ex]{\textwidth}{1pt}{3mm}
  Example Code \\
{\footnotesize
https://confluence.lsstcorp.org/display/LTS/Dome+Remote+Software+Control+Procedure

}
\hdashrule[0.5ex]{\textwidth}{1pt}{3mm}
  Expected Result \\
{\footnotesize
The Dome is cleared for nightly operations.

}

\begin{tabular}{p{2cm}}
\toprule
Step 7  \\ \hline
\end{tabular}
 Description \\
{\footnotesize
\textbf{PFlow lift}\\
This is part of EIE's safety check.\\
If EIE is not available, perform this step:\\[2\baselineskip]

\begin{itemize}
\tightlist
\item
  The Pflow lift must be stored before moving the dome.
\end{itemize}

}
\hdashrule[0.5ex]{\textwidth}{1pt}{3mm}
  Expected Result \\
{\footnotesize
The PFlow lift is stored properly

}

\begin{tabular}{p{2cm}}
\toprule
Step 8  \\ \hline
\end{tabular}
 Description \\
{\footnotesize
\textbf{Shutter closer}\\
If you have to close the shutter, the Dome must be under LOTO.\\
\textbf{Note:} There is no LOTO available at the moment. Use the
procedure attached to this test case and the information from the
following link:\\
https://confluence.lsstcorp.org/display/LTS/Dome+Remote+Software+Control+Procedure

}
\hdashrule[0.5ex]{\textwidth}{1pt}{3mm}
  Expected Result \\
{\footnotesize
The shutter was closed in a safe way.

}

\begin{tabular}{p{2cm}}
\toprule
Step 9  \\ \hline
\end{tabular}
 Description \\
{\footnotesize
\textbf{GIS activation:}\\
If the GIS for the Dome is available:

\begin{itemize}
\tightlist
\item
  activate the Dome GIS system.
\end{itemize}

}
\hdashrule[0.5ex]{\textwidth}{1pt}{3mm}
  Expected Result \\
{\footnotesize
If possible, the Dome GIS is activated.

}

\begin{tabular}{p{2cm}}
\toprule
Step 10  \\ \hline
\end{tabular}
 Description \\
{\footnotesize
\textbf{Signoff}\\
As a signoff, mark this step as passed

}
\hdashrule[0.5ex]{\textwidth}{1pt}{3mm}
  Expected Result \\
{\footnotesize
Safety Assurance is confirmed to be complete, and testing may proceed.

}

\begin{tabular}{p{2cm}}
\toprule
Step 11  \\ \hline
\end{tabular}
 Description \\
{\footnotesize
\textbf{Night Shift Leader}\\[2\baselineskip]Identify the night shift
leader ( first and second half of the
night).\\[2\baselineskip]\textbf{Note:} This is the person responsible
for deciding when

\begin{itemize}
\tightlist
\item
  the dome is going to be closed.
\item
  to stop observing due to technical issues
\end{itemize}

}
\hdashrule[0.5ex]{\textwidth}{1pt}{3mm}
  Expected Result \\
{\footnotesize
One person is identified as the night shift leader.

}

\begin{tabular}{p{2cm}}
\toprule
Step 12  \\ \hline
\end{tabular}
 Description \\
{\footnotesize
\textbf{Tailgate Meeting:}\\
Hold a tailgate for the upcoming task with personnel on the summit
working during the night. Go over any relevant procedures, roles, and
responsibilities.\\

\begin{enumerate}
\tightlist
\item
  If the Dome slit doors are not moving automatically, make sure that
  there are three persons with slit closer training available to close
  the slit manually.
\item
  Verify that there are enough persons with driver training available.
\item
  If the StarTracker is used, clarify who is taking the off the caps in
  the evening.
\item
  If the StarTracker is used, clarify who is installing the caps in the
  morning.
\item
  Discuss if surrounding observatories need to be informed.
\item
  If surrounding observatories need to be informed, clarify who is going
  to inform them and what information should be transmitted.
\item
  Check weather conditions and weather forecasts are within the
  specifications for observations.
\item
  Describe the tasks planned for the night.
\end{enumerate}

}
\hdashrule[0.5ex]{\textwidth}{1pt}{3mm}
  Expected Result \\
{\footnotesize
All involved personnel understands their roles and
responsibilities.\\[2\baselineskip]

\begin{itemize}
\tightlist
\item
  If the Dome slit doors are not moving automatically, there are three
  persons with slit closer training available to close the slit
  manually.
\item
  There are enough people with driver training available.
\item
  If the StarTracker is used, the caps are taken off in the evening by:
\item
  If the StarTracker is used, the StarTracker caps are installed in the
  morning by:
\item
  Discuss if surrounding observatories need to be informed.
\item
  If surrounding observatories need to be informed,~

  \begin{itemize}
  \tightlist
  \item
    they are informed by:
  \item
    The following information will be transmitted:
  \end{itemize}
\item
  The weather conditions permit us to open the dome and do the planned
  testing.~
\item
  The tasks planned for the night are:
\end{itemize}

}

\begin{tabular}{p{2cm}}
\toprule
Step 13  \\ \hline
\end{tabular}
 Description \\
{\footnotesize
\textbf{Tailgate Meeting -- Part II:}\\
If new personal is participating in the nightly summit activities:\\

\begin{itemize}
\tightlist
\item
  Clarify that all personnel has the PPE
\item
  Clarify that persons that need to go up into altitude have the fall
  protection training
\item
  Remind everybody that the emergency phone numbers are on the control
  room table.
\end{itemize}

}
\hdashrule[0.5ex]{\textwidth}{1pt}{3mm}
  Expected Result \\
{\footnotesize
\begin{itemize}
\tightlist
\item
  All personnel has the required PPE.
\item
  Persons that need to go up into altitude have the fall protection
  training
\item
  Everybody acknowledges that the emergency phone numbers are on the
  control room table.
\end{itemize}

}

\begin{tabular}{p{2cm}}
\toprule
Step 14  \\ \hline
\end{tabular}
 Description \\
{\footnotesize
\textbf{TMA and Dome contact}\\
Person in charge of the TMA interlocks\\
Dome responsible

}
\hdashrule[0.5ex]{\textwidth}{1pt}{3mm}
  Expected Result \\
{\footnotesize
TMA and Dome contacts are known

}

\begin{tabular}{p{2cm}}
\toprule
Step 15  \\ \hline
\end{tabular}
 Description \\
{\footnotesize
\textbf{Radio communication}\\

\begin{itemize}
\tightlist
\item
  Make sure one radio is switched to channel 1 and the volume is high

  \begin{itemize}
  \tightlist
  \item
    Paramedics, mountain assistants (in replacement of the paramedics),
    guards, and surrounding observatories are listening to this channel.
  \end{itemize}
\item
  Make sure one radio is switched to channel 3, and the volume is high

  \begin{itemize}
  \tightlist
  \item
    Rubin's internal coordination channel
  \end{itemize}
\end{itemize}

}
\hdashrule[0.5ex]{\textwidth}{1pt}{3mm}
  Expected Result \\
{\footnotesize
The radios are switched on and on high volume.

}

\begin{tabular}{p{2cm}}
\toprule
Step 16  \\ \hline
\end{tabular}
 Description \\
{\footnotesize
\textbf{Cars}

\begin{itemize}
\tightlist
\item
  Make sure enough cars are available to go to the hotel.
\item
  Make sure the keys for the cars are available.
\end{itemize}

}
\hdashrule[0.5ex]{\textwidth}{1pt}{3mm}
  Expected Result \\
{\footnotesize
Sufficient cars and their keys are available.

}

\begin{tabular}{p{2cm}}
\toprule
Step 17  \\ \hline
\end{tabular}
 Description \\
{\footnotesize
\textbf{ComCam safety}\\[2\baselineskip]Put ComCam in a safe state for
moving. This includes:

\begin{enumerate}
\tightlist
\item
  CryoTels are under observation for vibrations (i.e. a microphone or
  webcam is observing them and operating correctly)
\item
  Turbo pumps are off
\end{enumerate}

}
\hdashrule[0.5ex]{\textwidth}{1pt}{3mm}
  Expected Result \\
{\footnotesize
ComCam is in a safe state for TMA movement.

}

\begin{tabular}{p{2cm}}
\toprule
Step 18  \\ \hline
\end{tabular}
 Description \\
{\footnotesize
\textbf{TMA moving space}\\[2\baselineskip]Go to the dome and visually
verify that there is unrestricted space for the TMA movement.

}
\hdashrule[0.5ex]{\textwidth}{1pt}{3mm}
  Expected Result \\
{\footnotesize
The space is clear and no objects will be struck when the TMA moves.

}

\paragraph{ LVV-T2714 - Configure Observatory Environment for Nighttime Operations }\mbox{}\\

Version \textbf{1}.
Open  \href{https://jira.lsstcorp.org/secure/Tests.jspa#/testCase/LVV-T2714}{\textit{ LVV-T2714 } }
test case in Jira.

At the beginning of the night, prepare the observatory for nightly
operations.

\textbf{ Preconditions}:\\
Dome and the TMA, or at least the TMA must available for observations.

Final comment:\\Done as part of the tailgate meeting and daily routines of the
observers. See the night log for details.


Detailed steps :

\begin{tabular}{p{2cm}}
\toprule
Step 1  \\ \hline
\end{tabular}
 Description \\
{\footnotesize
\textbf{Telescope preparation:}

\begin{itemize}
\tightlist
\item
  Remove the caps on top of the StarTracker telescopes and the DIMM.
\item
  Check the instrument's health status by taking a test image.
\end{itemize}

}
\hdashrule[0.5ex]{\textwidth}{1pt}{3mm}
  Expected Result \\
{\footnotesize
The caps are removed.\\
The test image is taken and stored correspondingly.

}

\begin{tabular}{p{2cm}}
\toprule
Step 2  \\ \hline
\end{tabular}
 Description \\
{\footnotesize
\textbf{Calibration images:}\\
Not needed at the moment, to be included if data analysis reveals the
need.

\begin{itemize}
\tightlist
\item
  Take 10 ``darks'' with the StarTracker and the DIMM instruments. Use
  the same exposure time as for the images.
\item
  Take 10 ``sky flats'' with the StarTracker and the DIMM instruments.
\end{itemize}

}
\hdashrule[0.5ex]{\textwidth}{1pt}{3mm}
  Expected Result \\
{\footnotesize
The Darks and flats are stored in the expected location.

}

\begin{tabular}{p{2cm}}
\toprule
Step 3  \\ \hline
\end{tabular}
 Description \\
{\footnotesize
\textbf{Auxillary systems~nighttime~preparations:}

\begin{itemize}
\tightlist
\item
  Switch off the UMA in the afternoon.
\end{itemize}

}
\hdashrule[0.5ex]{\textwidth}{1pt}{3mm}
  Expected Result \\
{\footnotesize
The UMA is switched off.

}

\begin{tabular}{p{2cm}}
\toprule
Step 4  \\ \hline
\end{tabular}
 Description \\
{\footnotesize
\textbf{Night logging page:}\\
Start the night log similar to the AuxTel night
log:\\[2\baselineskip]https://confluence.lsstcorp.org/display/LSSTCOM/Night+Logs\\[2\baselineskip]

}
\hdashrule[0.5ex]{\textwidth}{1pt}{3mm}
  Expected Result \\
{\footnotesize
Page created with template information.

}

\begin{tabular}{p{2cm}}
\toprule
Step 5  \\ \hline
\end{tabular}
 Description \\
{\footnotesize
\textbf{TMA preparation}

\begin{itemize}
\tightlist
\item
  Check the Oil Supply System (OSS) on the EUI
\item
  Follow the attached manual to startup the TMA.
\end{itemize}

}
\hdashrule[0.5ex]{\textwidth}{1pt}{3mm}
  Expected Result \\
{\footnotesize
The OSS is operational:

}

\begin{tabular}{p{2cm}}
\toprule
Step 6  \\ \hline
\end{tabular}
 Description \\
{\footnotesize
\textbf{CSC activation:}\\[2\baselineskip]Use L.O.V.E to bring the CSC
to the enabled state.\\[2\baselineskip]

}
\hdashrule[0.5ex]{\textwidth}{1pt}{3mm}
  Expected Result \\
{\footnotesize
All needed CSCs are in the enabled state.

}

\paragraph{ LVV-T2730 - StarTracker Pointing and Tracking Test - Forward Azimuth Pattern }\mbox{}\\

Version \textbf{1}.
Open  \href{https://jira.lsstcorp.org/secure/Tests.jspa#/testCase/LVV-T2730}{\textit{ LVV-T2730 } }
test case in Jira.

Collect data with the StarTracker following the azimuth pattern -270,
-180, -90, 0, 90, 180, 270 deg. Nominal at four elevation angles 15, 45,
75, 85 deg. Minimum at the three angles: 15, 45, 85
deg.\\[2\baselineskip]This test

\begin{itemize}
\tightlist
\item
  is foreseen the first of four tests
\item
  takes about one-half summer night in the full version. ~
\end{itemize}

The analysis is done with the test case:
\href{https://jira.lsstcorp.org/secure/Tests.jspa\#/testCase/LVV-T2738}{LVV-T2738}\\[2\baselineskip]\textbf{Note:}
Tracking does not start at elevations higher than 85.0 deg. Same for Az,
we can not go up to 270 deg. We had to use 250 deg.

\textbf{ Preconditions}:\\
\href{https://jira.lsstcorp.org/browse/SITCOM-704}{SITCOM-704}
\href{https://jira.lsstcorp.org/browse/SITCOM-704}{First Pointing Model
Generation - Data Acquisition Preparation} must be
completed\\[2\baselineskip]Track for 10 min without the dome following.

Final comment:\\Dome position is not aligned with TMA. CSC shows the different value.


Detailed steps :

\begin{tabular}{p{2cm}}
\toprule
Step 1  \\ \hline
\end{tabular}
 Description \\
{\footnotesize
\textbf{Point~the Dome and the TMA}

\begin{itemize}
\tightlist
\item
  Command the Dome to {Pointing 1}⁠ to {-270}⁠deg
\item
  Command the TMA to {Pointing 1}⁠ at Az= {{-270}⁠~⁠~}deg, El= {{15}⁠}⁠
  deg.
\end{itemize}

}
\hdashrule[0.5ex]{\textwidth}{1pt}{3mm}
  Expected Result \\
{\footnotesize
The dome starts the movement.\\
The TMA starts to move.

}

\begin{tabular}{p{2cm}}
\toprule
Step 2  \\ \hline
\end{tabular}
 Description \\
{\footnotesize
\textbf{Point~the Dome and the TMA}

\begin{itemize}
\tightlist
\item
  Command the Dome to {Pointing 13}⁠ to {0}⁠deg
\item
  Command the TMA to {Pointing 13}⁠ at Az= {{0}⁠~⁠~}deg, El= {{85.0}⁠}⁠
  deg.
\end{itemize}

}
\hdashrule[0.5ex]{\textwidth}{1pt}{3mm}
  Expected Result \\
{\footnotesize
The dome starts the movement.\\
The TMA starts to move.

}

\begin{tabular}{p{2cm}}
\toprule
Step 3  \\ \hline
\end{tabular}
 Description \\
{\footnotesize
Wait for the Dome to reach the commanded position.\\
Wait for the TMA to reach the commanded position.

}
\hdashrule[0.5ex]{\textwidth}{1pt}{3mm}
  Expected Result \\
{\footnotesize
The Dome reaches the commanded position.\\
The \emph{MTDome\_logevent\_azMotion} and
\emph{MTDome\_logevent\_elMotion} inPosition parameter = true.\\
The TMA reaches the commanded position.\\
The \emph{MTMount\_logevent\_azimuthInPosition} and
\emph{MTMount\_logevent\_elevationInPosition} inPosition parameter =
true.

}

\begin{tabular}{p{2cm}}
\toprule
Step 4  \\ \hline
\end{tabular}
 Description \\
{\footnotesize
\textbf{Image preparation}\\
If the preparation to take images takes longer than 10sec, do
repositioning to target Az= {{0}⁠~⁠} deg, El= {{85.0}⁠~⁠}deg.

}
\hdashrule[0.5ex]{\textwidth}{1pt}{3mm}
  Expected Result \\
{\footnotesize
TMA reaches the commanded position.

}

\begin{tabular}{p{2cm}}
\toprule
Step 5  \\ \hline
\end{tabular}
 Description \\
{\footnotesize
\textbf{Stop the Dome}\\
Verify that the Dome is stopped so that it does not move during the
observations.

}
\hdashrule[0.5ex]{\textwidth}{1pt}{3mm}
  Expected Result \\
{\footnotesize
The Dome is stopped.\\[2\baselineskip]

}

\begin{tabular}{p{2cm}}
\toprule
Step 6  \\ \hline
\end{tabular}
 Description \\
{\footnotesize
\textbf{Track position and take images}\\[2\baselineskip]Star tracking
for 20 synchronized exposures with 5,4,6 seconds for wide, narrow, fast
cameras.\\[2\baselineskip]or\\[2\baselineskip]Star tracking for 10 min
and take synchronized exposures with 5,4,6 seconds for wide, narrow,
fast cameras.\\[2\baselineskip]\textbf{Note:~}Taking 20 images is good
for pointing evaluation..\\
Taking images for 10 min is needed for tracking evaluation

}
\hdashrule[0.5ex]{\textwidth}{1pt}{3mm}
  Test Data \\
 {\footnotesize
The Dome should not move during the observations.

}
\hdashrule[0.5ex]{\textwidth}{1pt}{3mm}
  Expected Result \\
{\footnotesize
\begin{itemize}
\tightlist
\item
  The TMA tracks the given position for 10 min, and the cameras take
  images or 20 images ar take per pointing
\item
  A series of images is successfully taken with the StarTracker and can
  be seen via RubinTV.
\end{itemize}

}

\begin{tabular}{p{2cm}}
\toprule
Step 7  \\ \hline
\end{tabular}
 Description \\
{\footnotesize
\textbf{Image verification:}\\
Look at RubinTV and verify that astrometry.net found an astrometric
solution.

}
\hdashrule[0.5ex]{\textwidth}{1pt}{3mm}
  Expected Result \\
{\footnotesize
Astrometry.net finds a solution.

}

\begin{tabular}{p{2cm}}
\toprule
Step 8  \\ \hline
\end{tabular}
 Description \\
{\footnotesize
\textbf{Offline analysis results}\\
Offline analysis in test case
\href{https://jira.lsstcorp.org/secure/Tests.jspa\#/testCase/LVV-T2739}{LVV-T2739}.

}
\hdashrule[0.5ex]{\textwidth}{1pt}{3mm}
  Expected Result \\
{\footnotesize
Image quality is sufficient.

}

\begin{tabular}{p{2cm}}
\toprule
Step 9  \\ \hline
\end{tabular}
 Description \\
{\footnotesize
\textbf{Point~the Dome and the TMA}

\begin{itemize}
\tightlist
\item
  Command the Dome to {Pointing 15}⁠ to {0}⁠deg
\item
  Command the TMA to {Pointing 15}⁠ at Az= {{0}⁠~⁠~}deg, El= {{45}⁠}⁠
  deg.
\end{itemize}

}
\hdashrule[0.5ex]{\textwidth}{1pt}{3mm}
  Expected Result \\
{\footnotesize
The dome starts the movement.\\
The TMA starts to move.

}

\begin{tabular}{p{2cm}}
\toprule
Step 10  \\ \hline
\end{tabular}
 Description \\
{\footnotesize
Wait for the Dome to reach the commanded position.\\
Wait for the TMA to reach the commanded position.

}
\hdashrule[0.5ex]{\textwidth}{1pt}{3mm}
  Expected Result \\
{\footnotesize
The Dome reaches the commanded position.\\
The \emph{MTDome\_logevent\_azMotion} and
\emph{MTDome\_logevent\_elMotion} inPosition parameter = true.\\
The TMA reaches the commanded position.\\
The \emph{MTMount\_logevent\_azimuthInPosition} and
\emph{MTMount\_logevent\_elevationInPosition} inPosition parameter =
true.

}

\begin{tabular}{p{2cm}}
\toprule
Step 11  \\ \hline
\end{tabular}
 Description \\
{\footnotesize
\textbf{Image preparation}\\
If the preparation to take images takes longer than 10sec, do
repositioning to target Az= {{0}⁠~⁠} deg, El= {{45}⁠~⁠}deg.

}
\hdashrule[0.5ex]{\textwidth}{1pt}{3mm}
  Expected Result \\
{\footnotesize
TMA reaches the commanded position.

}

\begin{tabular}{p{2cm}}
\toprule
Step 12  \\ \hline
\end{tabular}
 Description \\
{\footnotesize
\textbf{Stop the Dome}\\
Verify that the Dome is stopped so that it does not move during the
observations.

}
\hdashrule[0.5ex]{\textwidth}{1pt}{3mm}
  Expected Result \\
{\footnotesize
The Dome is stopped.\\[2\baselineskip]

}

\begin{tabular}{p{2cm}}
\toprule
Step 13  \\ \hline
\end{tabular}
 Description \\
{\footnotesize
\textbf{Track position and take images}\\[2\baselineskip]Star tracking
for 20 synchronized exposures with 5,4,6 seconds for wide, narrow, fast
cameras.\\[2\baselineskip]or\\[2\baselineskip]Star tracking for 10 min
and take synchronized exposures with 5,4,6 seconds for wide, narrow,
fast cameras.\\[2\baselineskip]\textbf{Note:~}Taking 20 images is good
for pointing evaluation..\\
Taking images for 10 min is needed for tracking evaluation

}
\hdashrule[0.5ex]{\textwidth}{1pt}{3mm}
  Test Data \\
 {\footnotesize
The Dome should not move during the observations.

}
\hdashrule[0.5ex]{\textwidth}{1pt}{3mm}
  Expected Result \\
{\footnotesize
\begin{itemize}
\tightlist
\item
  The TMA tracks the given position for 10 min, and the cameras take
  images or 20 images ar take per pointing
\item
  A series of images is successfully taken with the StarTracker and can
  be seen via RubinTV.
\end{itemize}

}

\begin{tabular}{p{2cm}}
\toprule
Step 14  \\ \hline
\end{tabular}
 Description \\
{\footnotesize
\textbf{Image verification:}\\
Look at RubinTV and verify that astrometry.net found an astrometric
solution.

}
\hdashrule[0.5ex]{\textwidth}{1pt}{3mm}
  Expected Result \\
{\footnotesize
Astrometry.net finds a solution.

}

\begin{tabular}{p{2cm}}
\toprule
Step 15  \\ \hline
\end{tabular}
 Description \\
{\footnotesize
\textbf{Offline analysis results}\\
Offline analysis in test case
\href{https://jira.lsstcorp.org/secure/Tests.jspa\#/testCase/LVV-T2739}{LVV-T2739}.

}
\hdashrule[0.5ex]{\textwidth}{1pt}{3mm}
  Expected Result \\
{\footnotesize
Image quality is sufficient.

}

\begin{tabular}{p{2cm}}
\toprule
Step 16  \\ \hline
\end{tabular}
 Description \\
{\footnotesize
\textbf{Point~the Dome and the TMA}

\begin{itemize}
\tightlist
\item
  Command the Dome to {Pointing 16}⁠ to {0}⁠deg
\item
  Command the TMA to {Pointing 16}⁠ at Az= {{0}⁠~⁠~}deg, El= {{15}⁠}⁠
  deg.
\end{itemize}

}
\hdashrule[0.5ex]{\textwidth}{1pt}{3mm}
  Expected Result \\
{\footnotesize
The dome starts the movement.\\
The TMA starts to move.

}

\begin{tabular}{p{2cm}}
\toprule
Step 17  \\ \hline
\end{tabular}
 Description \\
{\footnotesize
Wait for the Dome to reach the commanded position.\\
Wait for the TMA to reach the commanded position.

}
\hdashrule[0.5ex]{\textwidth}{1pt}{3mm}
  Expected Result \\
{\footnotesize
The Dome reaches the commanded position.\\
The \emph{MTDome\_logevent\_azMotion} and
\emph{MTDome\_logevent\_elMotion} inPosition parameter = true.\\
The TMA reaches the commanded position.\\
The \emph{MTMount\_logevent\_azimuthInPosition} and
\emph{MTMount\_logevent\_elevationInPosition} inPosition parameter =
true.

}

\begin{tabular}{p{2cm}}
\toprule
Step 18  \\ \hline
\end{tabular}
 Description \\
{\footnotesize
\textbf{Image preparation}\\
If the preparation to take images takes longer than 10sec, do
repositioning to target Az= {{0}⁠~⁠} deg, El= {{15}⁠~⁠}deg.

}
\hdashrule[0.5ex]{\textwidth}{1pt}{3mm}
  Expected Result \\
{\footnotesize
TMA reaches the commanded position.

}

\begin{tabular}{p{2cm}}
\toprule
Step 19  \\ \hline
\end{tabular}
 Description \\
{\footnotesize
\textbf{Stop the Dome}\\
Verify that the Dome is stopped so that it does not move during the
observations.

}
\hdashrule[0.5ex]{\textwidth}{1pt}{3mm}
  Expected Result \\
{\footnotesize
The Dome is stopped.\\[2\baselineskip]

}

\begin{tabular}{p{2cm}}
\toprule
Step 20  \\ \hline
\end{tabular}
 Description \\
{\footnotesize
\textbf{Track position and take images}\\[2\baselineskip]Star tracking
for 20 synchronized exposures with 5,4,6 seconds for wide, narrow, fast
cameras.\\[2\baselineskip]or\\[2\baselineskip]Star tracking for 10 min
and take synchronized exposures with 5,4,6 seconds for wide, narrow,
fast cameras.\\[2\baselineskip]\textbf{Note:~}Taking 20 images is good
for pointing evaluation..\\
Taking images for 10 min is needed for tracking evaluation

}
\hdashrule[0.5ex]{\textwidth}{1pt}{3mm}
  Test Data \\
 {\footnotesize
The Dome should not move during the observations.

}
\hdashrule[0.5ex]{\textwidth}{1pt}{3mm}
  Expected Result \\
{\footnotesize
\begin{itemize}
\tightlist
\item
  The TMA tracks the given position for 10 min, and the cameras take
  images or 20 images ar take per pointing
\item
  A series of images is successfully taken with the StarTracker and can
  be seen via RubinTV.
\end{itemize}

}

\begin{tabular}{p{2cm}}
\toprule
Step 21  \\ \hline
\end{tabular}
 Description \\
{\footnotesize
\textbf{Image verification:}\\
Look at RubinTV and verify that astrometry.net found an astrometric
solution.

}
\hdashrule[0.5ex]{\textwidth}{1pt}{3mm}
  Expected Result \\
{\footnotesize
Astrometry.net finds a solution.

}

\begin{tabular}{p{2cm}}
\toprule
Step 22  \\ \hline
\end{tabular}
 Description \\
{\footnotesize
\textbf{Offline analysis results}\\
Offline analysis in test case
\href{https://jira.lsstcorp.org/secure/Tests.jspa\#/testCase/LVV-T2739}{LVV-T2739}.

}
\hdashrule[0.5ex]{\textwidth}{1pt}{3mm}
  Expected Result \\
{\footnotesize
Image quality is sufficient.

}

\begin{tabular}{p{2cm}}
\toprule
Step 23  \\ \hline
\end{tabular}
 Description \\
{\footnotesize
\textbf{Point~the Dome and the TMA}

\begin{itemize}
\tightlist
\item
  Command the Dome to {Pointing 17}⁠ to {90}⁠deg
\item
  Command the TMA to {Pointing 17}⁠ at Az= {{90}⁠~⁠~}deg, El= {{15}⁠}⁠
  deg.
\end{itemize}

}
\hdashrule[0.5ex]{\textwidth}{1pt}{3mm}
  Expected Result \\
{\footnotesize
The dome starts the movement.\\
The TMA starts to move.

}

\begin{tabular}{p{2cm}}
\toprule
Step 24  \\ \hline
\end{tabular}
 Description \\
{\footnotesize
Wait for the Dome to reach the commanded position.\\
Wait for the TMA to reach the commanded position.

}
\hdashrule[0.5ex]{\textwidth}{1pt}{3mm}
  Expected Result \\
{\footnotesize
The Dome reaches the commanded position.\\
The \emph{MTDome\_logevent\_azMotion} and
\emph{MTDome\_logevent\_elMotion} inPosition parameter = true.\\
The TMA reaches the commanded position.\\
The \emph{MTMount\_logevent\_azimuthInPosition} and
\emph{MTMount\_logevent\_elevationInPosition} inPosition parameter =
true.

}

\begin{tabular}{p{2cm}}
\toprule
Step 25  \\ \hline
\end{tabular}
 Description \\
{\footnotesize
\textbf{Image preparation}\\
If the preparation to take images takes longer than 10sec, do
repositioning to target Az= {{90}⁠~⁠} deg, El= {{15}⁠~⁠}deg.

}
\hdashrule[0.5ex]{\textwidth}{1pt}{3mm}
  Expected Result \\
{\footnotesize
TMA reaches the commanded position.

}

\begin{tabular}{p{2cm}}
\toprule
Step 26  \\ \hline
\end{tabular}
 Description \\
{\footnotesize
\textbf{Stop the Dome}\\
Verify that the Dome is stopped so that it does not move during the
observations.

}
\hdashrule[0.5ex]{\textwidth}{1pt}{3mm}
  Expected Result \\
{\footnotesize
The Dome is stopped.\\[2\baselineskip]

}

\begin{tabular}{p{2cm}}
\toprule
Step 27  \\ \hline
\end{tabular}
 Description \\
{\footnotesize
\textbf{Track position and take images}\\[2\baselineskip]Star tracking
for 20 synchronized exposures with 5,4,6 seconds for wide, narrow, fast
cameras.\\[2\baselineskip]or\\[2\baselineskip]Star tracking for 10 min
and take synchronized exposures with 5,4,6 seconds for wide, narrow,
fast cameras.\\[2\baselineskip]\textbf{Note:~}Taking 20 images is good
for pointing evaluation..\\
Taking images for 10 min is needed for tracking evaluation

}
\hdashrule[0.5ex]{\textwidth}{1pt}{3mm}
  Test Data \\
 {\footnotesize
The Dome should not move during the observations.

}
\hdashrule[0.5ex]{\textwidth}{1pt}{3mm}
  Expected Result \\
{\footnotesize
\begin{itemize}
\tightlist
\item
  The TMA tracks the given position for 10 min, and the cameras take
  images or 20 images ar take per pointing
\item
  A series of images is successfully taken with the StarTracker and can
  be seen via RubinTV.
\end{itemize}

}

\begin{tabular}{p{2cm}}
\toprule
Step 28  \\ \hline
\end{tabular}
 Description \\
{\footnotesize
\textbf{Image verification:}\\
Look at RubinTV and verify that astrometry.net found an astrometric
solution.

}
\hdashrule[0.5ex]{\textwidth}{1pt}{3mm}
  Expected Result \\
{\footnotesize
Astrometry.net finds a solution.

}

\begin{tabular}{p{2cm}}
\toprule
Step 29  \\ \hline
\end{tabular}
 Description \\
{\footnotesize
\textbf{Offline analysis results}\\
Offline analysis in test case
\href{https://jira.lsstcorp.org/secure/Tests.jspa\#/testCase/LVV-T2739}{LVV-T2739}.

}
\hdashrule[0.5ex]{\textwidth}{1pt}{3mm}
  Expected Result \\
{\footnotesize
Image quality is sufficient.

}

\begin{tabular}{p{2cm}}
\toprule
Step 30  \\ \hline
\end{tabular}
 Description \\
{\footnotesize
\textbf{Point~the Dome and the TMA}

\begin{itemize}
\tightlist
\item
  Command the Dome to {Pointing 18}⁠ to {90}⁠deg
\item
  Command the TMA to {Pointing 18}⁠ at Az= {{90}⁠~⁠~}deg, El= {{45}⁠}⁠
  deg.
\end{itemize}

}
\hdashrule[0.5ex]{\textwidth}{1pt}{3mm}
  Expected Result \\
{\footnotesize
The dome starts the movement.\\
The TMA starts to move.

}

\begin{tabular}{p{2cm}}
\toprule
Step 31  \\ \hline
\end{tabular}
 Description \\
{\footnotesize
Wait for the Dome to reach the commanded position.\\
Wait for the TMA to reach the commanded position.

}
\hdashrule[0.5ex]{\textwidth}{1pt}{3mm}
  Expected Result \\
{\footnotesize
The Dome reaches the commanded position.\\
The \emph{MTDome\_logevent\_azMotion} and
\emph{MTDome\_logevent\_elMotion} inPosition parameter = true.\\
The TMA reaches the commanded position.\\
The \emph{MTMount\_logevent\_azimuthInPosition} and
\emph{MTMount\_logevent\_elevationInPosition} inPosition parameter =
true.

}

\begin{tabular}{p{2cm}}
\toprule
Step 32  \\ \hline
\end{tabular}
 Description \\
{\footnotesize
\textbf{Image preparation}\\
If the preparation to take images takes longer than 10sec, do
repositioning to target Az= {{90}⁠~⁠} deg, El= {{45}⁠~⁠}deg.

}
\hdashrule[0.5ex]{\textwidth}{1pt}{3mm}
  Expected Result \\
{\footnotesize
TMA reaches the commanded position.

}

\begin{tabular}{p{2cm}}
\toprule
Step 33  \\ \hline
\end{tabular}
 Description \\
{\footnotesize
\textbf{Stop the Dome}\\
Verify that the Dome is stopped so that it does not move during the
observations.

}
\hdashrule[0.5ex]{\textwidth}{1pt}{3mm}
  Expected Result \\
{\footnotesize
The Dome is stopped.\\[2\baselineskip]

}

\begin{tabular}{p{2cm}}
\toprule
Step 34  \\ \hline
\end{tabular}
 Description \\
{\footnotesize
\textbf{Track position and take images}\\[2\baselineskip]Star tracking
for 20 synchronized exposures with 5,4,6 seconds for wide, narrow, fast
cameras.\\[2\baselineskip]or\\[2\baselineskip]Star tracking for 10 min
and take synchronized exposures with 5,4,6 seconds for wide, narrow,
fast cameras.\\[2\baselineskip]\textbf{Note:~}Taking 20 images is good
for pointing evaluation..\\
Taking images for 10 min is needed for tracking evaluation

}
\hdashrule[0.5ex]{\textwidth}{1pt}{3mm}
  Test Data \\
 {\footnotesize
The Dome should not move during the observations.

}
\hdashrule[0.5ex]{\textwidth}{1pt}{3mm}
  Expected Result \\
{\footnotesize
\begin{itemize}
\tightlist
\item
  The TMA tracks the given position for 10 min, and the cameras take
  images or 20 images ar take per pointing
\item
  A series of images is successfully taken with the StarTracker and can
  be seen via RubinTV.
\end{itemize}

}

\begin{tabular}{p{2cm}}
\toprule
Step 35  \\ \hline
\end{tabular}
 Description \\
{\footnotesize
\textbf{Image verification:}\\
Look at RubinTV and verify that astrometry.net found an astrometric
solution.

}
\hdashrule[0.5ex]{\textwidth}{1pt}{3mm}
  Expected Result \\
{\footnotesize
Astrometry.net finds a solution.

}

\begin{tabular}{p{2cm}}
\toprule
Step 36  \\ \hline
\end{tabular}
 Description \\
{\footnotesize
\textbf{Offline analysis results}\\
Offline analysis in test case
\href{https://jira.lsstcorp.org/secure/Tests.jspa\#/testCase/LVV-T2739}{LVV-T2739}.

}
\hdashrule[0.5ex]{\textwidth}{1pt}{3mm}
  Expected Result \\
{\footnotesize
Image quality is sufficient.

}

\begin{tabular}{p{2cm}}
\toprule
Step 37  \\ \hline
\end{tabular}
 Description \\
{\footnotesize
\textbf{Point~the Dome and the TMA}

\begin{itemize}
\tightlist
\item
  Command the Dome to {Pointing 20}⁠ to {90}⁠deg
\item
  Command the TMA to {Pointing 20}⁠ at Az= {{90}⁠~⁠~}deg, El= {{85.0}⁠}⁠
  deg.
\end{itemize}

}
\hdashrule[0.5ex]{\textwidth}{1pt}{3mm}
  Expected Result \\
{\footnotesize
The dome starts the movement.\\
The TMA starts to move.

}

\begin{tabular}{p{2cm}}
\toprule
Step 38  \\ \hline
\end{tabular}
 Description \\
{\footnotesize
Wait for the Dome to reach the commanded position.\\
Wait for the TMA to reach the commanded position.

}
\hdashrule[0.5ex]{\textwidth}{1pt}{3mm}
  Expected Result \\
{\footnotesize
The Dome reaches the commanded position.\\
The \emph{MTDome\_logevent\_azMotion} and
\emph{MTDome\_logevent\_elMotion} inPosition parameter = true.\\
The TMA reaches the commanded position.\\
The \emph{MTMount\_logevent\_azimuthInPosition} and
\emph{MTMount\_logevent\_elevationInPosition} inPosition parameter =
true.

}

\begin{tabular}{p{2cm}}
\toprule
Step 39  \\ \hline
\end{tabular}
 Description \\
{\footnotesize
\textbf{Image preparation}\\
If the preparation to take images takes longer than 10sec, do
repositioning to target Az= {{90}⁠~⁠} deg, El= {{85.0}⁠~⁠}deg.

}
\hdashrule[0.5ex]{\textwidth}{1pt}{3mm}
  Expected Result \\
{\footnotesize
TMA reaches the commanded position.

}

\begin{tabular}{p{2cm}}
\toprule
Step 40  \\ \hline
\end{tabular}
 Description \\
{\footnotesize
\textbf{Stop the Dome}\\
Verify that the Dome is stopped so that it does not move during the
observations.

}
\hdashrule[0.5ex]{\textwidth}{1pt}{3mm}
  Expected Result \\
{\footnotesize
The Dome is stopped.\\[2\baselineskip]

}

\begin{tabular}{p{2cm}}
\toprule
Step 41  \\ \hline
\end{tabular}
 Description \\
{\footnotesize
\textbf{Track position and take images}\\[2\baselineskip]Star tracking
for 20 synchronized exposures with 5,4,6 seconds for wide, narrow, fast
cameras.\\[2\baselineskip]or\\[2\baselineskip]Star tracking for 10 min
and take synchronized exposures with 5,4,6 seconds for wide, narrow,
fast cameras.\\[2\baselineskip]\textbf{Note:~}Taking 20 images is good
for pointing evaluation..\\
Taking images for 10 min is needed for tracking evaluation

}
\hdashrule[0.5ex]{\textwidth}{1pt}{3mm}
  Test Data \\
 {\footnotesize
The Dome should not move during the observations.

}
\hdashrule[0.5ex]{\textwidth}{1pt}{3mm}
  Expected Result \\
{\footnotesize
\begin{itemize}
\tightlist
\item
  The TMA tracks the given position for 10 min, and the cameras take
  images or 20 images ar take per pointing
\item
  A series of images is successfully taken with the StarTracker and can
  be seen via RubinTV.
\end{itemize}

}

\begin{tabular}{p{2cm}}
\toprule
Step 42  \\ \hline
\end{tabular}
 Description \\
{\footnotesize
\textbf{Image verification:}\\
Look at RubinTV and verify that astrometry.net found an astrometric
solution.

}
\hdashrule[0.5ex]{\textwidth}{1pt}{3mm}
  Expected Result \\
{\footnotesize
Astrometry.net finds a solution.

}

\begin{tabular}{p{2cm}}
\toprule
Step 43  \\ \hline
\end{tabular}
 Description \\
{\footnotesize
\textbf{Offline analysis results}\\
Offline analysis in test case
\href{https://jira.lsstcorp.org/secure/Tests.jspa\#/testCase/LVV-T2739}{LVV-T2739}.

}
\hdashrule[0.5ex]{\textwidth}{1pt}{3mm}
  Expected Result \\
{\footnotesize
Image quality is sufficient.

}

\begin{tabular}{p{2cm}}
\toprule
Step 44  \\ \hline
\end{tabular}
 Description \\
{\footnotesize
\textbf{Point~the Dome and the TMA}

\begin{itemize}
\tightlist
\item
  Command the Dome to {Pointing 21}⁠ to {180}⁠deg
\item
  Command the TMA to {Pointing 21}⁠ at Az= {{180}⁠~⁠~}deg, El=
  {{85.0}⁠}⁠ deg.
\end{itemize}

}
\hdashrule[0.5ex]{\textwidth}{1pt}{3mm}
  Expected Result \\
{\footnotesize
The dome starts the movement.\\
The TMA starts to move.

}

\begin{tabular}{p{2cm}}
\toprule
Step 45  \\ \hline
\end{tabular}
 Description \\
{\footnotesize
Wait for the Dome to reach the commanded position.\\
Wait for the TMA to reach the commanded position.

}
\hdashrule[0.5ex]{\textwidth}{1pt}{3mm}
  Expected Result \\
{\footnotesize
The Dome reaches the commanded position.\\
The \emph{MTDome\_logevent\_azMotion} and
\emph{MTDome\_logevent\_elMotion} inPosition parameter = true.\\
The TMA reaches the commanded position.\\
The \emph{MTMount\_logevent\_azimuthInPosition} and
\emph{MTMount\_logevent\_elevationInPosition} inPosition parameter =
true.

}

\begin{tabular}{p{2cm}}
\toprule
Step 46  \\ \hline
\end{tabular}
 Description \\
{\footnotesize
\textbf{Image preparation}\\
If the preparation to take images takes longer than 10sec, do
repositioning to target Az= {{180}⁠~⁠} deg, El= {{85.0}⁠~⁠}deg.

}
\hdashrule[0.5ex]{\textwidth}{1pt}{3mm}
  Expected Result \\
{\footnotesize
TMA reaches the commanded position.

}

\begin{tabular}{p{2cm}}
\toprule
Step 47  \\ \hline
\end{tabular}
 Description \\
{\footnotesize
\textbf{Stop the Dome}\\
Verify that the Dome is stopped so that it does not move during the
observations.

}
\hdashrule[0.5ex]{\textwidth}{1pt}{3mm}
  Expected Result \\
{\footnotesize
The Dome is stopped.\\[2\baselineskip]

}

\begin{tabular}{p{2cm}}
\toprule
Step 48  \\ \hline
\end{tabular}
 Description \\
{\footnotesize
\textbf{Track position and take images}\\[2\baselineskip]Star tracking
for 20 synchronized exposures with 5,4,6 seconds for wide, narrow, fast
cameras.\\[2\baselineskip]or\\[2\baselineskip]Star tracking for 10 min
and take synchronized exposures with 5,4,6 seconds for wide, narrow,
fast cameras.\\[2\baselineskip]\textbf{Note:~}Taking 20 images is good
for pointing evaluation..\\
Taking images for 10 min is needed for tracking evaluation

}
\hdashrule[0.5ex]{\textwidth}{1pt}{3mm}
  Test Data \\
 {\footnotesize
The Dome should not move during the observations.

}
\hdashrule[0.5ex]{\textwidth}{1pt}{3mm}
  Expected Result \\
{\footnotesize
\begin{itemize}
\tightlist
\item
  The TMA tracks the given position for 10 min, and the cameras take
  images or 20 images ar take per pointing
\item
  A series of images is successfully taken with the StarTracker and can
  be seen via RubinTV.
\end{itemize}

}

\begin{tabular}{p{2cm}}
\toprule
Step 49  \\ \hline
\end{tabular}
 Description \\
{\footnotesize
\textbf{Image verification:}\\
Look at RubinTV and verify that astrometry.net found an astrometric
solution.

}
\hdashrule[0.5ex]{\textwidth}{1pt}{3mm}
  Expected Result \\
{\footnotesize
Astrometry.net finds a solution.

}

\begin{tabular}{p{2cm}}
\toprule
Step 50  \\ \hline
\end{tabular}
 Description \\
{\footnotesize
\textbf{Offline analysis results}\\
Offline analysis in test case
\href{https://jira.lsstcorp.org/secure/Tests.jspa\#/testCase/LVV-T2739}{LVV-T2739}.

}
\hdashrule[0.5ex]{\textwidth}{1pt}{3mm}
  Expected Result \\
{\footnotesize
Image quality is sufficient.

}

\begin{tabular}{p{2cm}}
\toprule
Step 51  \\ \hline
\end{tabular}
 Description \\
{\footnotesize
\textbf{Point~the Dome and the TMA}

\begin{itemize}
\tightlist
\item
  Command the Dome to {Pointing 23}⁠ to {180}⁠deg
\item
  Command the TMA to {Pointing 23}⁠ at Az= {{180}⁠~⁠~}deg, El= {{45}⁠}⁠
  deg.
\end{itemize}

}
\hdashrule[0.5ex]{\textwidth}{1pt}{3mm}
  Expected Result \\
{\footnotesize
The dome starts the movement.\\
The TMA starts to move.

}

\begin{tabular}{p{2cm}}
\toprule
Step 52  \\ \hline
\end{tabular}
 Description \\
{\footnotesize
Wait for the Dome to reach the commanded position.\\
Wait for the TMA to reach the commanded position.

}
\hdashrule[0.5ex]{\textwidth}{1pt}{3mm}
  Expected Result \\
{\footnotesize
The Dome reaches the commanded position.\\
The \emph{MTDome\_logevent\_azMotion} and
\emph{MTDome\_logevent\_elMotion} inPosition parameter = true.\\
The TMA reaches the commanded position.\\
The \emph{MTMount\_logevent\_azimuthInPosition} and
\emph{MTMount\_logevent\_elevationInPosition} inPosition parameter =
true.

}

\begin{tabular}{p{2cm}}
\toprule
Step 53  \\ \hline
\end{tabular}
 Description \\
{\footnotesize
\textbf{Image preparation}\\
If the preparation to take images takes longer than 10sec, do
repositioning to target Az= {{180}⁠~⁠} deg, El= {{45}⁠~⁠}deg.

}
\hdashrule[0.5ex]{\textwidth}{1pt}{3mm}
  Expected Result \\
{\footnotesize
TMA reaches the commanded position.

}

\begin{tabular}{p{2cm}}
\toprule
Step 54  \\ \hline
\end{tabular}
 Description \\
{\footnotesize
\textbf{Stop the Dome}\\
Verify that the Dome is stopped so that it does not move during the
observations.

}
\hdashrule[0.5ex]{\textwidth}{1pt}{3mm}
  Expected Result \\
{\footnotesize
The Dome is stopped.\\[2\baselineskip]

}

\begin{tabular}{p{2cm}}
\toprule
Step 55  \\ \hline
\end{tabular}
 Description \\
{\footnotesize
\textbf{Track position and take images}\\[2\baselineskip]Star tracking
for 20 synchronized exposures with 5,4,6 seconds for wide, narrow, fast
cameras.\\[2\baselineskip]or\\[2\baselineskip]Star tracking for 10 min
and take synchronized exposures with 5,4,6 seconds for wide, narrow,
fast cameras.\\[2\baselineskip]\textbf{Note:~}Taking 20 images is good
for pointing evaluation..\\
Taking images for 10 min is needed for tracking evaluation

}
\hdashrule[0.5ex]{\textwidth}{1pt}{3mm}
  Test Data \\
 {\footnotesize
The Dome should not move during the observations.

}
\hdashrule[0.5ex]{\textwidth}{1pt}{3mm}
  Expected Result \\
{\footnotesize
\begin{itemize}
\tightlist
\item
  The TMA tracks the given position for 10 min, and the cameras take
  images or 20 images ar take per pointing
\item
  A series of images is successfully taken with the StarTracker and can
  be seen via RubinTV.
\end{itemize}

}

\begin{tabular}{p{2cm}}
\toprule
Step 56  \\ \hline
\end{tabular}
 Description \\
{\footnotesize
\textbf{Image verification:}\\
Look at RubinTV and verify that astrometry.net found an astrometric
solution.

}
\hdashrule[0.5ex]{\textwidth}{1pt}{3mm}
  Expected Result \\
{\footnotesize
Astrometry.net finds a solution.

}

\begin{tabular}{p{2cm}}
\toprule
Step 57  \\ \hline
\end{tabular}
 Description \\
{\footnotesize
\textbf{Offline analysis results}\\
Offline analysis in test case
\href{https://jira.lsstcorp.org/secure/Tests.jspa\#/testCase/LVV-T2739}{LVV-T2739}.

}
\hdashrule[0.5ex]{\textwidth}{1pt}{3mm}
  Expected Result \\
{\footnotesize
Image quality is sufficient.

}

\begin{tabular}{p{2cm}}
\toprule
Step 58  \\ \hline
\end{tabular}
 Description \\
{\footnotesize
\textbf{Point~the Dome and the TMA}

\begin{itemize}
\tightlist
\item
  Command the Dome to {Pointing 24}⁠ to {180}⁠deg
\item
  Command the TMA to {Pointing 24}⁠ at Az= {{180}⁠~⁠~}deg, El= {{15}⁠}⁠
  deg.
\end{itemize}

}
\hdashrule[0.5ex]{\textwidth}{1pt}{3mm}
  Expected Result \\
{\footnotesize
The dome starts the movement.\\
The TMA starts to move.

}

\begin{tabular}{p{2cm}}
\toprule
Step 59  \\ \hline
\end{tabular}
 Description \\
{\footnotesize
Wait for the Dome to reach the commanded position.\\
Wait for the TMA to reach the commanded position.

}
\hdashrule[0.5ex]{\textwidth}{1pt}{3mm}
  Expected Result \\
{\footnotesize
The Dome reaches the commanded position.\\
The \emph{MTDome\_logevent\_azMotion} and
\emph{MTDome\_logevent\_elMotion} inPosition parameter = true.\\
The TMA reaches the commanded position.\\
The \emph{MTMount\_logevent\_azimuthInPosition} and
\emph{MTMount\_logevent\_elevationInPosition} inPosition parameter =
true.

}

\begin{tabular}{p{2cm}}
\toprule
Step 60  \\ \hline
\end{tabular}
 Description \\
{\footnotesize
\textbf{Image preparation}\\
If the preparation to take images takes longer than 10sec, do
repositioning to target Az= {{180}⁠~⁠} deg, El= {{15}⁠~⁠}deg.

}
\hdashrule[0.5ex]{\textwidth}{1pt}{3mm}
  Expected Result \\
{\footnotesize
TMA reaches the commanded position.

}

\begin{tabular}{p{2cm}}
\toprule
Step 61  \\ \hline
\end{tabular}
 Description \\
{\footnotesize
\textbf{Stop the Dome}\\
Verify that the Dome is stopped so that it does not move during the
observations.

}
\hdashrule[0.5ex]{\textwidth}{1pt}{3mm}
  Expected Result \\
{\footnotesize
The Dome is stopped.\\[2\baselineskip]

}

\begin{tabular}{p{2cm}}
\toprule
Step 62  \\ \hline
\end{tabular}
 Description \\
{\footnotesize
\textbf{Track position and take images}\\[2\baselineskip]Star tracking
for 20 synchronized exposures with 5,4,6 seconds for wide, narrow, fast
cameras.\\[2\baselineskip]or\\[2\baselineskip]Star tracking for 10 min
and take synchronized exposures with 5,4,6 seconds for wide, narrow,
fast cameras.\\[2\baselineskip]\textbf{Note:~}Taking 20 images is good
for pointing evaluation..\\
Taking images for 10 min is needed for tracking evaluation

}
\hdashrule[0.5ex]{\textwidth}{1pt}{3mm}
  Test Data \\
 {\footnotesize
The Dome should not move during the observations.

}
\hdashrule[0.5ex]{\textwidth}{1pt}{3mm}
  Expected Result \\
{\footnotesize
\begin{itemize}
\tightlist
\item
  The TMA tracks the given position for 10 min, and the cameras take
  images or 20 images ar take per pointing
\item
  A series of images is successfully taken with the StarTracker and can
  be seen via RubinTV.
\end{itemize}

}

\begin{tabular}{p{2cm}}
\toprule
Step 63  \\ \hline
\end{tabular}
 Description \\
{\footnotesize
\textbf{Image verification:}\\
Look at RubinTV and verify that astrometry.net found an astrometric
solution.

}
\hdashrule[0.5ex]{\textwidth}{1pt}{3mm}
  Expected Result \\
{\footnotesize
Astrometry.net finds a solution.

}

\begin{tabular}{p{2cm}}
\toprule
Step 64  \\ \hline
\end{tabular}
 Description \\
{\footnotesize
\textbf{Offline analysis results}\\
Offline analysis in test case
\href{https://jira.lsstcorp.org/secure/Tests.jspa\#/testCase/LVV-T2739}{LVV-T2739}.

}
\hdashrule[0.5ex]{\textwidth}{1pt}{3mm}
  Expected Result \\
{\footnotesize
Image quality is sufficient.

}

\begin{tabular}{p{2cm}}
\toprule
Step 65  \\ \hline
\end{tabular}
 Description \\
{\footnotesize
\textbf{Point~the Dome and the TMA}

\begin{itemize}
\tightlist
\item
  Command the Dome to {Pointing 25}⁠ to {270}⁠deg
\item
  Command the TMA to {Pointing 25}⁠ at Az= {{270}⁠~⁠~}deg, El= {{15}⁠}⁠
  deg.
\end{itemize}

}
\hdashrule[0.5ex]{\textwidth}{1pt}{3mm}
  Expected Result \\
{\footnotesize
The dome starts the movement.\\
The TMA starts to move.

}

\begin{tabular}{p{2cm}}
\toprule
Step 66  \\ \hline
\end{tabular}
 Description \\
{\footnotesize
Wait for the Dome to reach the commanded position.\\
Wait for the TMA to reach the commanded position.

}
\hdashrule[0.5ex]{\textwidth}{1pt}{3mm}
  Expected Result \\
{\footnotesize
The Dome reaches the commanded position.\\
The \emph{MTDome\_logevent\_azMotion} and
\emph{MTDome\_logevent\_elMotion} inPosition parameter = true.\\
The TMA reaches the commanded position.\\
The \emph{MTMount\_logevent\_azimuthInPosition} and
\emph{MTMount\_logevent\_elevationInPosition} inPosition parameter =
true.

}

\begin{tabular}{p{2cm}}
\toprule
Step 67  \\ \hline
\end{tabular}
 Description \\
{\footnotesize
\textbf{Image preparation}\\
If the preparation to take images takes longer than 10sec, do
repositioning to target Az= {{270}⁠~⁠} deg, El= {{15}⁠~⁠}deg.

}
\hdashrule[0.5ex]{\textwidth}{1pt}{3mm}
  Expected Result \\
{\footnotesize
TMA reaches the commanded position.

}

\begin{tabular}{p{2cm}}
\toprule
Step 68  \\ \hline
\end{tabular}
 Description \\
{\footnotesize
\textbf{Stop the Dome}\\
Verify that the Dome is stopped so that it does not move during the
observations.

}
\hdashrule[0.5ex]{\textwidth}{1pt}{3mm}
  Expected Result \\
{\footnotesize
The Dome is stopped.\\[2\baselineskip]

}

\begin{tabular}{p{2cm}}
\toprule
Step 69  \\ \hline
\end{tabular}
 Description \\
{\footnotesize
\textbf{Track position and take images}\\[2\baselineskip]Star tracking
for 20 synchronized exposures with 5,4,6 seconds for wide, narrow, fast
cameras.\\[2\baselineskip]or\\[2\baselineskip]Star tracking for 10 min
and take synchronized exposures with 5,4,6 seconds for wide, narrow,
fast cameras.\\[2\baselineskip]\textbf{Note:~}Taking 20 images is good
for pointing evaluation..\\
Taking images for 10 min is needed for tracking evaluation

}
\hdashrule[0.5ex]{\textwidth}{1pt}{3mm}
  Test Data \\
 {\footnotesize
The Dome should not move during the observations.

}
\hdashrule[0.5ex]{\textwidth}{1pt}{3mm}
  Expected Result \\
{\footnotesize
\begin{itemize}
\tightlist
\item
  The TMA tracks the given position for 10 min, and the cameras take
  images or 20 images ar take per pointing
\item
  A series of images is successfully taken with the StarTracker and can
  be seen via RubinTV.
\end{itemize}

}

\begin{tabular}{p{2cm}}
\toprule
Step 70  \\ \hline
\end{tabular}
 Description \\
{\footnotesize
\textbf{Image verification:}\\
Look at RubinTV and verify that astrometry.net found an astrometric
solution.

}
\hdashrule[0.5ex]{\textwidth}{1pt}{3mm}
  Expected Result \\
{\footnotesize
Astrometry.net finds a solution.

}

\begin{tabular}{p{2cm}}
\toprule
Step 71  \\ \hline
\end{tabular}
 Description \\
{\footnotesize
\textbf{Offline analysis results}\\
Offline analysis in test case
\href{https://jira.lsstcorp.org/secure/Tests.jspa\#/testCase/LVV-T2739}{LVV-T2739}.

}
\hdashrule[0.5ex]{\textwidth}{1pt}{3mm}
  Expected Result \\
{\footnotesize
Image quality is sufficient.

}

\begin{tabular}{p{2cm}}
\toprule
Step 72  \\ \hline
\end{tabular}
 Description \\
{\footnotesize
Wait for the Dome to reach the commanded position.\\
Wait for the TMA to reach the commanded position.

}
\hdashrule[0.5ex]{\textwidth}{1pt}{3mm}
  Expected Result \\
{\footnotesize
The Dome reaches the commanded position.\\
The \emph{MTDome\_logevent\_azMotion} and
\emph{MTDome\_logevent\_elMotion} inPosition parameter = true.\\
The TMA reaches the commanded position.\\
The \emph{MTMount\_logevent\_azimuthInPosition} and
\emph{MTMount\_logevent\_elevationInPosition} inPosition parameter =
true.

}

\begin{tabular}{p{2cm}}
\toprule
Step 73  \\ \hline
\end{tabular}
 Description \\
{\footnotesize
\textbf{Point~the Dome and the TMA}

\begin{itemize}
\tightlist
\item
  Command the Dome to {Pointing 2}⁠ to {-270}⁠deg
\item
  Command the TMA to {Pointing 2}⁠ at Az= {{-270}⁠~⁠~}deg, El= {{45}⁠}⁠
  deg.
\end{itemize}

}
\hdashrule[0.5ex]{\textwidth}{1pt}{3mm}
  Expected Result \\
{\footnotesize
The dome starts the movement.\\
The TMA starts to move.

}

\begin{tabular}{p{2cm}}
\toprule
Step 74  \\ \hline
\end{tabular}
 Description \\
{\footnotesize
Wait for the Dome to reach the commanded position.\\
Wait for the TMA to reach the commanded position.

}
\hdashrule[0.5ex]{\textwidth}{1pt}{3mm}
  Expected Result \\
{\footnotesize
The Dome reaches the commanded position.\\
The \emph{MTDome\_logevent\_azMotion} and
\emph{MTDome\_logevent\_elMotion} inPosition parameter = true.\\
The TMA reaches the commanded position.\\
The \emph{MTMount\_logevent\_azimuthInPosition} and
\emph{MTMount\_logevent\_elevationInPosition} inPosition parameter =
true.

}

\begin{tabular}{p{2cm}}
\toprule
Step 75  \\ \hline
\end{tabular}
 Description \\
{\footnotesize
\textbf{Image preparation}\\
If the preparation to take images takes longer than 10sec, do
repositioning to target Az= {{-270}⁠~⁠} deg, El= {{45}⁠~⁠}deg.

}
\hdashrule[0.5ex]{\textwidth}{1pt}{3mm}
  Expected Result \\
{\footnotesize
TMA reaches the commanded position.

}

\begin{tabular}{p{2cm}}
\toprule
Step 76  \\ \hline
\end{tabular}
 Description \\
{\footnotesize
\textbf{Stop the Dome}\\
Verify that the Dome is stopped so that it does not move during the
observations.

}
\hdashrule[0.5ex]{\textwidth}{1pt}{3mm}
  Expected Result \\
{\footnotesize
The Dome is stopped.\\[2\baselineskip]

}

\begin{tabular}{p{2cm}}
\toprule
Step 77  \\ \hline
\end{tabular}
 Description \\
{\footnotesize
\textbf{Track position and take images}\\[2\baselineskip]Star tracking
for 20 synchronized exposures with 5,4,6 seconds for wide, narrow, fast
cameras.\\[2\baselineskip]or\\[2\baselineskip]Star tracking for 10 min
and take synchronized exposures with 5,4,6 seconds for wide, narrow,
fast cameras.\\[2\baselineskip]\textbf{Note:~}Taking 20 images is good
for pointing evaluation..\\
Taking images for 10 min is needed for tracking evaluation

}
\hdashrule[0.5ex]{\textwidth}{1pt}{3mm}
  Test Data \\
 {\footnotesize
The Dome should not move during the observations.

}
\hdashrule[0.5ex]{\textwidth}{1pt}{3mm}
  Expected Result \\
{\footnotesize
\begin{itemize}
\tightlist
\item
  The TMA tracks the given position for 10 min, and the cameras take
  images or 20 images ar take per pointing
\item
  A series of images is successfully taken with the StarTracker and can
  be seen via RubinTV.
\end{itemize}

}

\begin{tabular}{p{2cm}}
\toprule
Step 78  \\ \hline
\end{tabular}
 Description \\
{\footnotesize
\textbf{Image verification:}\\
Look at RubinTV and verify that astrometry.net found an astrometric
solution.

}
\hdashrule[0.5ex]{\textwidth}{1pt}{3mm}
  Expected Result \\
{\footnotesize
Astrometry.net finds a solution.

}

\begin{tabular}{p{2cm}}
\toprule
Step 79  \\ \hline
\end{tabular}
 Description \\
{\footnotesize
\textbf{Offline analysis results}\\
Offline analysis in test case
\href{https://jira.lsstcorp.org/secure/Tests.jspa\#/testCase/LVV-T2739}{LVV-T2739}.

}
\hdashrule[0.5ex]{\textwidth}{1pt}{3mm}
  Expected Result \\
{\footnotesize
Image quality is sufficient.

}

\begin{tabular}{p{2cm}}
\toprule
Step 80  \\ \hline
\end{tabular}
 Description \\
{\footnotesize
\textbf{Point~the Dome and the TMA}

\begin{itemize}
\tightlist
\item
  Command the Dome to {Pointing 26}⁠ to {270}⁠deg
\item
  Command the TMA to {Pointing 26}⁠ at Az= {{270}⁠~⁠~}deg, El= {{45}⁠}⁠
  deg.
\end{itemize}

}
\hdashrule[0.5ex]{\textwidth}{1pt}{3mm}
  Expected Result \\
{\footnotesize
The dome starts the movement.\\
The TMA starts to move.

}

\begin{tabular}{p{2cm}}
\toprule
Step 81  \\ \hline
\end{tabular}
 Description \\
{\footnotesize
Wait for the Dome to reach the commanded position.\\
Wait for the TMA to reach the commanded position.

}
\hdashrule[0.5ex]{\textwidth}{1pt}{3mm}
  Expected Result \\
{\footnotesize
The Dome reaches the commanded position.\\
The \emph{MTDome\_logevent\_azMotion} and
\emph{MTDome\_logevent\_elMotion} inPosition parameter = true.\\
The TMA reaches the commanded position.\\
The \emph{MTMount\_logevent\_azimuthInPosition} and
\emph{MTMount\_logevent\_elevationInPosition} inPosition parameter =
true.

}

\begin{tabular}{p{2cm}}
\toprule
Step 82  \\ \hline
\end{tabular}
 Description \\
{\footnotesize
\textbf{Image preparation}\\
If the preparation to take images takes longer than 10sec, do
repositioning to target Az= {{270}⁠~⁠} deg, El= {{45}⁠~⁠}deg.

}
\hdashrule[0.5ex]{\textwidth}{1pt}{3mm}
  Expected Result \\
{\footnotesize
TMA reaches the commanded position.

}

\begin{tabular}{p{2cm}}
\toprule
Step 83  \\ \hline
\end{tabular}
 Description \\
{\footnotesize
\textbf{Stop the Dome}\\
Verify that the Dome is stopped so that it does not move during the
observations.

}
\hdashrule[0.5ex]{\textwidth}{1pt}{3mm}
  Expected Result \\
{\footnotesize
The Dome is stopped.\\[2\baselineskip]

}

\begin{tabular}{p{2cm}}
\toprule
Step 84  \\ \hline
\end{tabular}
 Description \\
{\footnotesize
\textbf{Track position and take images}\\[2\baselineskip]Star tracking
for 20 synchronized exposures with 5,4,6 seconds for wide, narrow, fast
cameras.\\[2\baselineskip]or\\[2\baselineskip]Star tracking for 10 min
and take synchronized exposures with 5,4,6 seconds for wide, narrow,
fast cameras.\\[2\baselineskip]\textbf{Note:~}Taking 20 images is good
for pointing evaluation..\\
Taking images for 10 min is needed for tracking evaluation

}
\hdashrule[0.5ex]{\textwidth}{1pt}{3mm}
  Test Data \\
 {\footnotesize
The Dome should not move during the observations.

}
\hdashrule[0.5ex]{\textwidth}{1pt}{3mm}
  Expected Result \\
{\footnotesize
\begin{itemize}
\tightlist
\item
  The TMA tracks the given position for 10 min, and the cameras take
  images or 20 images ar take per pointing
\item
  A series of images is successfully taken with the StarTracker and can
  be seen via RubinTV.
\end{itemize}

}

\begin{tabular}{p{2cm}}
\toprule
Step 85  \\ \hline
\end{tabular}
 Description \\
{\footnotesize
\textbf{Image verification:}\\
Look at RubinTV and verify that astrometry.net found an astrometric
solution.

}
\hdashrule[0.5ex]{\textwidth}{1pt}{3mm}
  Expected Result \\
{\footnotesize
Astrometry.net finds a solution.

}

\begin{tabular}{p{2cm}}
\toprule
Step 86  \\ \hline
\end{tabular}
 Description \\
{\footnotesize
\textbf{Offline analysis results}\\
Offline analysis in test case
\href{https://jira.lsstcorp.org/secure/Tests.jspa\#/testCase/LVV-T2739}{LVV-T2739}.

}
\hdashrule[0.5ex]{\textwidth}{1pt}{3mm}
  Expected Result \\
{\footnotesize
Image quality is sufficient.

}

\begin{tabular}{p{2cm}}
\toprule
Step 87  \\ \hline
\end{tabular}
 Description \\
{\footnotesize
\textbf{Point~the Dome and the TMA}

\begin{itemize}
\tightlist
\item
  Command the Dome to {Pointing 28}⁠ to {270}⁠deg
\item
  Command the TMA to {Pointing 28}⁠ at Az= {{270}⁠~⁠~}deg, El=
  {{85.0}⁠}⁠ deg.
\end{itemize}

}
\hdashrule[0.5ex]{\textwidth}{1pt}{3mm}
  Expected Result \\
{\footnotesize
The dome starts the movement.\\
The TMA starts to move.

}

\begin{tabular}{p{2cm}}
\toprule
Step 88  \\ \hline
\end{tabular}
 Description \\
{\footnotesize
Wait for the Dome to reach the commanded position.\\
Wait for the TMA to reach the commanded position.

}
\hdashrule[0.5ex]{\textwidth}{1pt}{3mm}
  Expected Result \\
{\footnotesize
The Dome reaches the commanded position.\\
The \emph{MTDome\_logevent\_azMotion} and
\emph{MTDome\_logevent\_elMotion} inPosition parameter = true.\\
The TMA reaches the commanded position.\\
The \emph{MTMount\_logevent\_azimuthInPosition} and
\emph{MTMount\_logevent\_elevationInPosition} inPosition parameter =
true.

}

\begin{tabular}{p{2cm}}
\toprule
Step 89  \\ \hline
\end{tabular}
 Description \\
{\footnotesize
\textbf{Image preparation}\\
If the preparation to take images takes longer than 10sec, do
repositioning to target Az= {{270}⁠~⁠} deg, El= {{85.0}⁠~⁠}deg.

}
\hdashrule[0.5ex]{\textwidth}{1pt}{3mm}
  Expected Result \\
{\footnotesize
TMA reaches the commanded position.

}

\begin{tabular}{p{2cm}}
\toprule
Step 90  \\ \hline
\end{tabular}
 Description \\
{\footnotesize
\textbf{Stop the Dome}\\
Verify that the Dome is stopped so that it does not move during the
observations.

}
\hdashrule[0.5ex]{\textwidth}{1pt}{3mm}
  Expected Result \\
{\footnotesize
The Dome is stopped.\\[2\baselineskip]

}

\begin{tabular}{p{2cm}}
\toprule
Step 91  \\ \hline
\end{tabular}
 Description \\
{\footnotesize
\textbf{Track position and take images}\\[2\baselineskip]Star tracking
for 20 synchronized exposures with 5,4,6 seconds for wide, narrow, fast
cameras.\\[2\baselineskip]or\\[2\baselineskip]Star tracking for 10 min
and take synchronized exposures with 5,4,6 seconds for wide, narrow,
fast cameras.\\[2\baselineskip]\textbf{Note:~}Taking 20 images is good
for pointing evaluation..\\
Taking images for 10 min is needed for tracking evaluation

}
\hdashrule[0.5ex]{\textwidth}{1pt}{3mm}
  Test Data \\
 {\footnotesize
The Dome should not move during the observations.

}
\hdashrule[0.5ex]{\textwidth}{1pt}{3mm}
  Expected Result \\
{\footnotesize
\begin{itemize}
\tightlist
\item
  The TMA tracks the given position for 10 min, and the cameras take
  images or 20 images ar take per pointing
\item
  A series of images is successfully taken with the StarTracker and can
  be seen via RubinTV.
\end{itemize}

}

\begin{tabular}{p{2cm}}
\toprule
Step 92  \\ \hline
\end{tabular}
 Description \\
{\footnotesize
\textbf{Image verification:}\\
Look at RubinTV and verify that astrometry.net found an astrometric
solution.

}
\hdashrule[0.5ex]{\textwidth}{1pt}{3mm}
  Expected Result \\
{\footnotesize
Astrometry.net finds a solution.

}

\begin{tabular}{p{2cm}}
\toprule
Step 93  \\ \hline
\end{tabular}
 Description \\
{\footnotesize
\textbf{Offline analysis results}\\
Offline analysis in test case
\href{https://jira.lsstcorp.org/secure/Tests.jspa\#/testCase/LVV-T2739}{LVV-T2739}.

}
\hdashrule[0.5ex]{\textwidth}{1pt}{3mm}
  Expected Result \\
{\footnotesize
Image quality is sufficient.

}

\begin{tabular}{p{2cm}}
\toprule
Step 94  \\ \hline
\end{tabular}
 Description \\
{\footnotesize
\textbf{Point~the Dome and the TMA}

\begin{itemize}
\tightlist
\item
  Command the Dome to {Pointing 22}⁠ to {-270}⁠deg
\item
  Command the TMA to {Pointing 22}⁠ at Az= {{-270}⁠~⁠~}deg, El= {{75}⁠}⁠
  deg.
\end{itemize}

}
\hdashrule[0.5ex]{\textwidth}{1pt}{3mm}
  Expected Result \\
{\footnotesize
The dome starts the movement.\\
The TMA starts to move.

}

\begin{tabular}{p{2cm}}
\toprule
Step 95  \\ \hline
\end{tabular}
 Description \\
{\footnotesize
Wait for the Dome to reach the commanded position.\\
Wait for the TMA to reach the commanded position.

}
\hdashrule[0.5ex]{\textwidth}{1pt}{3mm}
  Expected Result \\
{\footnotesize
The Dome reaches the commanded position.\\
The \emph{MTDome\_logevent\_azMotion} and
\emph{MTDome\_logevent\_elMotion} inPosition parameter = true.\\
The TMA reaches the commanded position.\\
The \emph{MTMount\_logevent\_azimuthInPosition} and
\emph{MTMount\_logevent\_elevationInPosition} inPosition parameter =
true.

}

\begin{tabular}{p{2cm}}
\toprule
Step 96  \\ \hline
\end{tabular}
 Description \\
{\footnotesize
\textbf{Image preparation}\\
If the preparation to take images takes longer than 10sec, do
repositioning to target Az= {{-270}⁠~⁠} deg, El= {{75}⁠~⁠}deg.

}
\hdashrule[0.5ex]{\textwidth}{1pt}{3mm}
  Expected Result \\
{\footnotesize
TMA reaches the commanded position.

}

\begin{tabular}{p{2cm}}
\toprule
Step 97  \\ \hline
\end{tabular}
 Description \\
{\footnotesize
\textbf{Stop the Dome}\\
Verify that the Dome is stopped so that it does not move during the
observations.

}
\hdashrule[0.5ex]{\textwidth}{1pt}{3mm}
  Expected Result \\
{\footnotesize
The Dome is stopped.\\[2\baselineskip]

}

\begin{tabular}{p{2cm}}
\toprule
Step 98  \\ \hline
\end{tabular}
 Description \\
{\footnotesize
\textbf{Track position and take images}\\[2\baselineskip]Star tracking
for 20 synchronized exposures with 5,4,6 seconds for wide, narrow, fast
cameras.\\[2\baselineskip]or\\[2\baselineskip]Star tracking for 10 min
and take synchronized exposures with 5,4,6 seconds for wide, narrow,
fast cameras.\\[2\baselineskip]\textbf{Note:~}Taking 20 images is good
for pointing evaluation..\\
Taking images for 10 min is needed for tracking evaluation

}
\hdashrule[0.5ex]{\textwidth}{1pt}{3mm}
  Test Data \\
 {\footnotesize
The Dome should not move during the observations.

}
\hdashrule[0.5ex]{\textwidth}{1pt}{3mm}
  Expected Result \\
{\footnotesize
\begin{itemize}
\tightlist
\item
  The TMA tracks the given position for 10 min, and the cameras take
  images or 20 images ar take per pointing
\item
  A series of images is successfully taken with the StarTracker and can
  be seen via RubinTV.
\end{itemize}

}

\begin{tabular}{p{2cm}}
\toprule
Step 99  \\ \hline
\end{tabular}
 Description \\
{\footnotesize
\textbf{Image verification:}\\
Look at RubinTV and verify that astrometry.net found an astrometric
solution.

}
\hdashrule[0.5ex]{\textwidth}{1pt}{3mm}
  Expected Result \\
{\footnotesize
Astrometry.net finds a solution.

}

\begin{tabular}{p{2cm}}
\toprule
Step 100  \\ \hline
\end{tabular}
 Description \\
{\footnotesize
\textbf{Offline analysis results}\\
Offline analysis in test case
\href{https://jira.lsstcorp.org/secure/Tests.jspa\#/testCase/LVV-T2739}{LVV-T2739}.

}
\hdashrule[0.5ex]{\textwidth}{1pt}{3mm}
  Expected Result \\
{\footnotesize
Image quality is sufficient.

}

\begin{tabular}{p{2cm}}
\toprule
Step 101  \\ \hline
\end{tabular}
 Description \\
{\footnotesize
\textbf{Point~the Dome and the TMA}

\begin{itemize}
\tightlist
\item
  Command the Dome to {Pointing 23}⁠ to {-180}⁠deg
\item
  Command the TMA to {Pointing 23}⁠ at Az= {{-180}⁠~⁠~}deg, El= {{75}⁠}⁠
  deg.
\end{itemize}

}
\hdashrule[0.5ex]{\textwidth}{1pt}{3mm}
  Expected Result \\
{\footnotesize
The dome starts the movement.\\
The TMA starts to move.

}

\begin{tabular}{p{2cm}}
\toprule
Step 102  \\ \hline
\end{tabular}
 Description \\
{\footnotesize
Wait for the Dome to reach the commanded position.\\
Wait for the TMA to reach the commanded position.

}
\hdashrule[0.5ex]{\textwidth}{1pt}{3mm}
  Expected Result \\
{\footnotesize
The Dome reaches the commanded position.\\
The \emph{MTDome\_logevent\_azMotion} and
\emph{MTDome\_logevent\_elMotion} inPosition parameter = true.\\
The TMA reaches the commanded position.\\
The \emph{MTMount\_logevent\_azimuthInPosition} and
\emph{MTMount\_logevent\_elevationInPosition} inPosition parameter =
true.

}

\begin{tabular}{p{2cm}}
\toprule
Step 103  \\ \hline
\end{tabular}
 Description \\
{\footnotesize
\textbf{Image preparation}\\
If the preparation to take images takes longer than 10sec, do
repositioning to target Az= {{-180}⁠~⁠} deg, El= {{75}⁠~⁠}deg.

}
\hdashrule[0.5ex]{\textwidth}{1pt}{3mm}
  Expected Result \\
{\footnotesize
TMA reaches the commanded position.

}

\begin{tabular}{p{2cm}}
\toprule
Step 104  \\ \hline
\end{tabular}
 Description \\
{\footnotesize
\textbf{Stop the Dome}\\
Verify that the Dome is stopped so that it does not move during the
observations.

}
\hdashrule[0.5ex]{\textwidth}{1pt}{3mm}
  Expected Result \\
{\footnotesize
The Dome is stopped.\\[2\baselineskip]

}

\begin{tabular}{p{2cm}}
\toprule
Step 105  \\ \hline
\end{tabular}
 Description \\
{\footnotesize
\textbf{Track position and take images}\\[2\baselineskip]Star tracking
for 20 synchronized exposures with 5,4,6 seconds for wide, narrow, fast
cameras.\\[2\baselineskip]or\\[2\baselineskip]Star tracking for 10 min
and take synchronized exposures with 5,4,6 seconds for wide, narrow,
fast cameras.\\[2\baselineskip]\textbf{Note:~}Taking 20 images is good
for pointing evaluation..\\
Taking images for 10 min is needed for tracking evaluation

}
\hdashrule[0.5ex]{\textwidth}{1pt}{3mm}
  Test Data \\
 {\footnotesize
The Dome should not move during the observations.

}
\hdashrule[0.5ex]{\textwidth}{1pt}{3mm}
  Expected Result \\
{\footnotesize
\begin{itemize}
\tightlist
\item
  The TMA tracks the given position for 10 min, and the cameras take
  images or 20 images ar take per pointing
\item
  A series of images is successfully taken with the StarTracker and can
  be seen via RubinTV.
\end{itemize}

}

\begin{tabular}{p{2cm}}
\toprule
Step 106  \\ \hline
\end{tabular}
 Description \\
{\footnotesize
\textbf{Image verification:}\\
Look at RubinTV and verify that astrometry.net found an astrometric
solution.

}
\hdashrule[0.5ex]{\textwidth}{1pt}{3mm}
  Expected Result \\
{\footnotesize
Astrometry.net finds a solution.

}

\begin{tabular}{p{2cm}}
\toprule
Step 107  \\ \hline
\end{tabular}
 Description \\
{\footnotesize
\textbf{Offline analysis results}\\
Offline analysis in test case
\href{https://jira.lsstcorp.org/secure/Tests.jspa\#/testCase/LVV-T2739}{LVV-T2739}.

}
\hdashrule[0.5ex]{\textwidth}{1pt}{3mm}
  Expected Result \\
{\footnotesize
Image quality is sufficient.

}

\begin{tabular}{p{2cm}}
\toprule
Step 108  \\ \hline
\end{tabular}
 Description \\
{\footnotesize
\textbf{Point~the Dome and the TMA}

\begin{itemize}
\tightlist
\item
  Command the Dome to {Pointing 24}⁠ to {-90}⁠deg
\item
  Command the TMA to {Pointing 24}⁠ at Az= {{-90}⁠~⁠~}deg, El= {{75}⁠}⁠
  deg.
\end{itemize}

}
\hdashrule[0.5ex]{\textwidth}{1pt}{3mm}
  Expected Result \\
{\footnotesize
The dome starts the movement.\\
The TMA starts to move.

}

\begin{tabular}{p{2cm}}
\toprule
Step 109  \\ \hline
\end{tabular}
 Description \\
{\footnotesize
Wait for the Dome to reach the commanded position.\\
Wait for the TMA to reach the commanded position.

}
\hdashrule[0.5ex]{\textwidth}{1pt}{3mm}
  Expected Result \\
{\footnotesize
The Dome reaches the commanded position.\\
The \emph{MTDome\_logevent\_azMotion} and
\emph{MTDome\_logevent\_elMotion} inPosition parameter = true.\\
The TMA reaches the commanded position.\\
The \emph{MTMount\_logevent\_azimuthInPosition} and
\emph{MTMount\_logevent\_elevationInPosition} inPosition parameter =
true.

}

\begin{tabular}{p{2cm}}
\toprule
Step 110  \\ \hline
\end{tabular}
 Description \\
{\footnotesize
\textbf{Image preparation}\\
If the preparation to take images takes longer than 10sec, do
repositioning to target Az= {{-90}⁠~⁠} deg, El= {{75}⁠~⁠}deg.

}
\hdashrule[0.5ex]{\textwidth}{1pt}{3mm}
  Expected Result \\
{\footnotesize
TMA reaches the commanded position.

}

\begin{tabular}{p{2cm}}
\toprule
Step 111  \\ \hline
\end{tabular}
 Description \\
{\footnotesize
\textbf{Stop the Dome}\\
Verify that the Dome is stopped so that it does not move during the
observations.

}
\hdashrule[0.5ex]{\textwidth}{1pt}{3mm}
  Expected Result \\
{\footnotesize
The Dome is stopped.\\[2\baselineskip]

}

\begin{tabular}{p{2cm}}
\toprule
Step 112  \\ \hline
\end{tabular}
 Description \\
{\footnotesize
\textbf{Track position and take images}\\[2\baselineskip]Star tracking
for 20 synchronized exposures with 5,4,6 seconds for wide, narrow, fast
cameras.\\[2\baselineskip]or\\[2\baselineskip]Star tracking for 10 min
and take synchronized exposures with 5,4,6 seconds for wide, narrow,
fast cameras.\\[2\baselineskip]\textbf{Note:~}Taking 20 images is good
for pointing evaluation..\\
Taking images for 10 min is needed for tracking evaluation

}
\hdashrule[0.5ex]{\textwidth}{1pt}{3mm}
  Test Data \\
 {\footnotesize
The Dome should not move during the observations.

}
\hdashrule[0.5ex]{\textwidth}{1pt}{3mm}
  Expected Result \\
{\footnotesize
\begin{itemize}
\tightlist
\item
  The TMA tracks the given position for 10 min, and the cameras take
  images or 20 images ar take per pointing
\item
  A series of images is successfully taken with the StarTracker and can
  be seen via RubinTV.
\end{itemize}

}

\begin{tabular}{p{2cm}}
\toprule
Step 113  \\ \hline
\end{tabular}
 Description \\
{\footnotesize
\textbf{Image verification:}\\
Look at RubinTV and verify that astrometry.net found an astrometric
solution.

}
\hdashrule[0.5ex]{\textwidth}{1pt}{3mm}
  Expected Result \\
{\footnotesize
Astrometry.net finds a solution.

}

\begin{tabular}{p{2cm}}
\toprule
Step 114  \\ \hline
\end{tabular}
 Description \\
{\footnotesize
\textbf{Offline analysis results}\\
Offline analysis in test case
\href{https://jira.lsstcorp.org/secure/Tests.jspa\#/testCase/LVV-T2739}{LVV-T2739}.

}
\hdashrule[0.5ex]{\textwidth}{1pt}{3mm}
  Expected Result \\
{\footnotesize
Image quality is sufficient.

}

\begin{tabular}{p{2cm}}
\toprule
Step 115  \\ \hline
\end{tabular}
 Description \\
{\footnotesize
\textbf{Point~the Dome and the TMA}

\begin{itemize}
\tightlist
\item
  Command the Dome to {Pointing 25}⁠ to {0}⁠deg
\item
  Command the TMA to {Pointing 25}⁠ at Az= {{0}⁠~⁠~}deg, El= {{75}⁠}⁠
  deg.
\end{itemize}

}
\hdashrule[0.5ex]{\textwidth}{1pt}{3mm}
  Expected Result \\
{\footnotesize
The dome starts the movement.\\
The TMA starts to move.

}

\begin{tabular}{p{2cm}}
\toprule
Step 116  \\ \hline
\end{tabular}
 Description \\
{\footnotesize
Wait for the Dome to reach the commanded position.\\
Wait for the TMA to reach the commanded position.

}
\hdashrule[0.5ex]{\textwidth}{1pt}{3mm}
  Expected Result \\
{\footnotesize
The Dome reaches the commanded position.\\
The \emph{MTDome\_logevent\_azMotion} and
\emph{MTDome\_logevent\_elMotion} inPosition parameter = true.\\
The TMA reaches the commanded position.\\
The \emph{MTMount\_logevent\_azimuthInPosition} and
\emph{MTMount\_logevent\_elevationInPosition} inPosition parameter =
true.

}

\begin{tabular}{p{2cm}}
\toprule
Step 117  \\ \hline
\end{tabular}
 Description \\
{\footnotesize
\textbf{Image preparation}\\
If the preparation to take images takes longer than 10sec, do
repositioning to target Az= {{0}⁠~⁠} deg, El= {{75}⁠~⁠}deg.

}
\hdashrule[0.5ex]{\textwidth}{1pt}{3mm}
  Expected Result \\
{\footnotesize
TMA reaches the commanded position.

}

\begin{tabular}{p{2cm}}
\toprule
Step 118  \\ \hline
\end{tabular}
 Description \\
{\footnotesize
\textbf{Stop the Dome}\\
Verify that the Dome is stopped so that it does not move during the
observations.

}
\hdashrule[0.5ex]{\textwidth}{1pt}{3mm}
  Expected Result \\
{\footnotesize
The Dome is stopped.\\[2\baselineskip]

}

\begin{tabular}{p{2cm}}
\toprule
Step 119  \\ \hline
\end{tabular}
 Description \\
{\footnotesize
\textbf{Track position and take images}\\[2\baselineskip]Star tracking
for 20 synchronized exposures with 5,4,6 seconds for wide, narrow, fast
cameras.\\[2\baselineskip]or\\[2\baselineskip]Star tracking for 10 min
and take synchronized exposures with 5,4,6 seconds for wide, narrow,
fast cameras.\\[2\baselineskip]\textbf{Note:~}Taking 20 images is good
for pointing evaluation..\\
Taking images for 10 min is needed for tracking evaluation

}
\hdashrule[0.5ex]{\textwidth}{1pt}{3mm}
  Test Data \\
 {\footnotesize
The Dome should not move during the observations.

}
\hdashrule[0.5ex]{\textwidth}{1pt}{3mm}
  Expected Result \\
{\footnotesize
\begin{itemize}
\tightlist
\item
  The TMA tracks the given position for 10 min, and the cameras take
  images or 20 images ar take per pointing
\item
  A series of images is successfully taken with the StarTracker and can
  be seen via RubinTV.
\end{itemize}

}

\begin{tabular}{p{2cm}}
\toprule
Step 120  \\ \hline
\end{tabular}
 Description \\
{\footnotesize
\textbf{Image verification:}\\
Look at RubinTV and verify that astrometry.net found an astrometric
solution.

}
\hdashrule[0.5ex]{\textwidth}{1pt}{3mm}
  Expected Result \\
{\footnotesize
Astrometry.net finds a solution.

}

\begin{tabular}{p{2cm}}
\toprule
Step 121  \\ \hline
\end{tabular}
 Description \\
{\footnotesize
\textbf{Offline analysis results}\\
Offline analysis in test case
\href{https://jira.lsstcorp.org/secure/Tests.jspa\#/testCase/LVV-T2739}{LVV-T2739}.

}
\hdashrule[0.5ex]{\textwidth}{1pt}{3mm}
  Expected Result \\
{\footnotesize
Image quality is sufficient.

}

\begin{tabular}{p{2cm}}
\toprule
Step 122  \\ \hline
\end{tabular}
 Description \\
{\footnotesize
\textbf{Point~the Dome and the TMA}

\begin{itemize}
\tightlist
\item
  Command the Dome to {Pointing 26}⁠ to {90}⁠deg
\item
  Command the TMA to {Pointing 26}⁠ at Az= {{90}⁠~⁠~}deg, El= {{75}⁠}⁠
  deg.
\end{itemize}

}
\hdashrule[0.5ex]{\textwidth}{1pt}{3mm}
  Expected Result \\
{\footnotesize
The dome starts the movement.\\
The TMA starts to move.

}

\begin{tabular}{p{2cm}}
\toprule
Step 123  \\ \hline
\end{tabular}
 Description \\
{\footnotesize
Wait for the Dome to reach the commanded position.\\
Wait for the TMA to reach the commanded position.

}
\hdashrule[0.5ex]{\textwidth}{1pt}{3mm}
  Expected Result \\
{\footnotesize
The Dome reaches the commanded position.\\
The \emph{MTDome\_logevent\_azMotion} and
\emph{MTDome\_logevent\_elMotion} inPosition parameter = true.\\
The TMA reaches the commanded position.\\
The \emph{MTMount\_logevent\_azimuthInPosition} and
\emph{MTMount\_logevent\_elevationInPosition} inPosition parameter =
true.

}

\begin{tabular}{p{2cm}}
\toprule
Step 124  \\ \hline
\end{tabular}
 Description \\
{\footnotesize
\textbf{Image preparation}\\
If the preparation to take images takes longer than 10sec, do
repositioning to target Az= {{90}⁠~⁠} deg, El= {{75}⁠~⁠}deg.

}
\hdashrule[0.5ex]{\textwidth}{1pt}{3mm}
  Expected Result \\
{\footnotesize
TMA reaches the commanded position.

}

\begin{tabular}{p{2cm}}
\toprule
Step 125  \\ \hline
\end{tabular}
 Description \\
{\footnotesize
\textbf{Stop the Dome}\\
Verify that the Dome is stopped so that it does not move during the
observations.

}
\hdashrule[0.5ex]{\textwidth}{1pt}{3mm}
  Expected Result \\
{\footnotesize
The Dome is stopped.\\[2\baselineskip]

}

\begin{tabular}{p{2cm}}
\toprule
Step 126  \\ \hline
\end{tabular}
 Description \\
{\footnotesize
\textbf{Track position and take images}\\[2\baselineskip]Star tracking
for 20 synchronized exposures with 5,4,6 seconds for wide, narrow, fast
cameras.\\[2\baselineskip]or\\[2\baselineskip]Star tracking for 10 min
and take synchronized exposures with 5,4,6 seconds for wide, narrow,
fast cameras.\\[2\baselineskip]\textbf{Note:~}Taking 20 images is good
for pointing evaluation..\\
Taking images for 10 min is needed for tracking evaluation

}
\hdashrule[0.5ex]{\textwidth}{1pt}{3mm}
  Test Data \\
 {\footnotesize
The Dome should not move during the observations.

}
\hdashrule[0.5ex]{\textwidth}{1pt}{3mm}
  Expected Result \\
{\footnotesize
\begin{itemize}
\tightlist
\item
  The TMA tracks the given position for 10 min, and the cameras take
  images or 20 images ar take per pointing
\item
  A series of images is successfully taken with the StarTracker and can
  be seen via RubinTV.
\end{itemize}

}

\begin{tabular}{p{2cm}}
\toprule
Step 127  \\ \hline
\end{tabular}
 Description \\
{\footnotesize
\textbf{Image verification:}\\
Look at RubinTV and verify that astrometry.net found an astrometric
solution.

}
\hdashrule[0.5ex]{\textwidth}{1pt}{3mm}
  Expected Result \\
{\footnotesize
Astrometry.net finds a solution.

}

\begin{tabular}{p{2cm}}
\toprule
Step 128  \\ \hline
\end{tabular}
 Description \\
{\footnotesize
\textbf{Offline analysis results}\\
Offline analysis in test case
\href{https://jira.lsstcorp.org/secure/Tests.jspa\#/testCase/LVV-T2739}{LVV-T2739}.

}
\hdashrule[0.5ex]{\textwidth}{1pt}{3mm}
  Expected Result \\
{\footnotesize
Image quality is sufficient.

}

\begin{tabular}{p{2cm}}
\toprule
Step 129  \\ \hline
\end{tabular}
 Description \\
{\footnotesize
\textbf{Point~the Dome and the TMA}

\begin{itemize}
\tightlist
\item
  Command the Dome to {Pointing 27}⁠ to {180}⁠deg
\item
  Command the TMA to {Pointing 27}⁠ at Az= {{180}⁠~⁠~}deg, El= {{75}⁠}⁠
  deg.
\end{itemize}

}
\hdashrule[0.5ex]{\textwidth}{1pt}{3mm}
  Expected Result \\
{\footnotesize
The dome starts the movement.\\
The TMA starts to move.

}

\begin{tabular}{p{2cm}}
\toprule
Step 130  \\ \hline
\end{tabular}
 Description \\
{\footnotesize
Wait for the Dome to reach the commanded position.\\
Wait for the TMA to reach the commanded position.

}
\hdashrule[0.5ex]{\textwidth}{1pt}{3mm}
  Expected Result \\
{\footnotesize
The Dome reaches the commanded position.\\
The \emph{MTDome\_logevent\_azMotion} and
\emph{MTDome\_logevent\_elMotion} inPosition parameter = true.\\
The TMA reaches the commanded position.\\
The \emph{MTMount\_logevent\_azimuthInPosition} and
\emph{MTMount\_logevent\_elevationInPosition} inPosition parameter =
true.

}

\begin{tabular}{p{2cm}}
\toprule
Step 131  \\ \hline
\end{tabular}
 Description \\
{\footnotesize
\textbf{Image preparation}\\
If the preparation to take images takes longer than 10sec, do
repositioning to target Az= {{180}⁠~⁠} deg, El= {{75}⁠~⁠}deg.

}
\hdashrule[0.5ex]{\textwidth}{1pt}{3mm}
  Expected Result \\
{\footnotesize
TMA reaches the commanded position.

}

\begin{tabular}{p{2cm}}
\toprule
Step 132  \\ \hline
\end{tabular}
 Description \\
{\footnotesize
\textbf{Stop the Dome}\\
Verify that the Dome is stopped so that it does not move during the
observations.

}
\hdashrule[0.5ex]{\textwidth}{1pt}{3mm}
  Expected Result \\
{\footnotesize
The Dome is stopped.\\[2\baselineskip]

}

\begin{tabular}{p{2cm}}
\toprule
Step 133  \\ \hline
\end{tabular}
 Description \\
{\footnotesize
\textbf{Track position and take images}\\[2\baselineskip]Star tracking
for 20 synchronized exposures with 5,4,6 seconds for wide, narrow, fast
cameras.\\[2\baselineskip]or\\[2\baselineskip]Star tracking for 10 min
and take synchronized exposures with 5,4,6 seconds for wide, narrow,
fast cameras.\\[2\baselineskip]\textbf{Note:~}Taking 20 images is good
for pointing evaluation..\\
Taking images for 10 min is needed for tracking evaluation

}
\hdashrule[0.5ex]{\textwidth}{1pt}{3mm}
  Test Data \\
 {\footnotesize
The Dome should not move during the observations.

}
\hdashrule[0.5ex]{\textwidth}{1pt}{3mm}
  Expected Result \\
{\footnotesize
\begin{itemize}
\tightlist
\item
  The TMA tracks the given position for 10 min, and the cameras take
  images or 20 images ar take per pointing
\item
  A series of images is successfully taken with the StarTracker and can
  be seen via RubinTV.
\end{itemize}

}

\begin{tabular}{p{2cm}}
\toprule
Step 134  \\ \hline
\end{tabular}
 Description \\
{\footnotesize
\textbf{Image verification:}\\
Look at RubinTV and verify that astrometry.net found an astrometric
solution.

}
\hdashrule[0.5ex]{\textwidth}{1pt}{3mm}
  Expected Result \\
{\footnotesize
Astrometry.net finds a solution.

}

\begin{tabular}{p{2cm}}
\toprule
Step 135  \\ \hline
\end{tabular}
 Description \\
{\footnotesize
\textbf{Offline analysis results}\\
Offline analysis in test case
\href{https://jira.lsstcorp.org/secure/Tests.jspa\#/testCase/LVV-T2739}{LVV-T2739}.

}
\hdashrule[0.5ex]{\textwidth}{1pt}{3mm}
  Expected Result \\
{\footnotesize
Image quality is sufficient.

}

\begin{tabular}{p{2cm}}
\toprule
Step 136  \\ \hline
\end{tabular}
 Description \\
{\footnotesize
\textbf{Point~the Dome and the TMA}

\begin{itemize}
\tightlist
\item
  Command the Dome to {Pointing 28}⁠ to {270}⁠deg
\item
  Command the TMA to {Pointing 28}⁠ at Az= {{270}⁠~⁠~}deg, El= {{75}⁠}⁠
  deg.
\end{itemize}

}
\hdashrule[0.5ex]{\textwidth}{1pt}{3mm}
  Expected Result \\
{\footnotesize
The dome starts the movement.\\
The TMA starts to move.

}

\begin{tabular}{p{2cm}}
\toprule
Step 137  \\ \hline
\end{tabular}
 Description \\
{\footnotesize
Wait for the Dome to reach the commanded position.\\
Wait for the TMA to reach the commanded position.

}
\hdashrule[0.5ex]{\textwidth}{1pt}{3mm}
  Expected Result \\
{\footnotesize
The Dome reaches the commanded position.\\
The \emph{MTDome\_logevent\_azMotion} and
\emph{MTDome\_logevent\_elMotion} inPosition parameter = true.\\
The TMA reaches the commanded position.\\
The \emph{MTMount\_logevent\_azimuthInPosition} and
\emph{MTMount\_logevent\_elevationInPosition} inPosition parameter =
true.

}

\begin{tabular}{p{2cm}}
\toprule
Step 138  \\ \hline
\end{tabular}
 Description \\
{\footnotesize
\textbf{Image preparation}\\
If the preparation to take images takes longer than 10sec, do
repositioning to target Az= {{270}⁠~⁠} deg, El= {{75}⁠~⁠}deg.

}
\hdashrule[0.5ex]{\textwidth}{1pt}{3mm}
  Expected Result \\
{\footnotesize
TMA reaches the commanded position.

}

\begin{tabular}{p{2cm}}
\toprule
Step 139  \\ \hline
\end{tabular}
 Description \\
{\footnotesize
\textbf{Stop the Dome}\\
Verify that the Dome is stopped so that it does not move during the
observations.

}
\hdashrule[0.5ex]{\textwidth}{1pt}{3mm}
  Expected Result \\
{\footnotesize
The Dome is stopped.\\[2\baselineskip]

}

\begin{tabular}{p{2cm}}
\toprule
Step 140  \\ \hline
\end{tabular}
 Description \\
{\footnotesize
\textbf{Track position and take images}\\[2\baselineskip]Star tracking
for 20 synchronized exposures with 5,4,6 seconds for wide, narrow, fast
cameras.\\[2\baselineskip]or\\[2\baselineskip]Star tracking for 10 min
and take synchronized exposures with 5,4,6 seconds for wide, narrow,
fast cameras.\\[2\baselineskip]\textbf{Note:~}Taking 20 images is good
for pointing evaluation..\\
Taking images for 10 min is needed for tracking evaluation

}
\hdashrule[0.5ex]{\textwidth}{1pt}{3mm}
  Test Data \\
 {\footnotesize
The Dome should not move during the observations.

}
\hdashrule[0.5ex]{\textwidth}{1pt}{3mm}
  Expected Result \\
{\footnotesize
\begin{itemize}
\tightlist
\item
  The TMA tracks the given position for 10 min, and the cameras take
  images or 20 images ar take per pointing
\item
  A series of images is successfully taken with the StarTracker and can
  be seen via RubinTV.
\end{itemize}

}

\begin{tabular}{p{2cm}}
\toprule
Step 141  \\ \hline
\end{tabular}
 Description \\
{\footnotesize
\textbf{Image verification:}\\
Look at RubinTV and verify that astrometry.net found an astrometric
solution.

}
\hdashrule[0.5ex]{\textwidth}{1pt}{3mm}
  Expected Result \\
{\footnotesize
Astrometry.net finds a solution.

}

\begin{tabular}{p{2cm}}
\toprule
Step 142  \\ \hline
\end{tabular}
 Description \\
{\footnotesize
\textbf{Offline analysis results}\\
Offline analysis in test case
\href{https://jira.lsstcorp.org/secure/Tests.jspa\#/testCase/LVV-T2739}{LVV-T2739}.

}
\hdashrule[0.5ex]{\textwidth}{1pt}{3mm}
  Expected Result \\
{\footnotesize
Image quality is sufficient.

}

\begin{tabular}{p{2cm}}
\toprule
Step 143  \\ \hline
\end{tabular}
 Description \\
{\footnotesize
\textbf{Image preparation}\\
If the preparation to take images takes longer than 10sec, do
repositioning to target Az= {{-270}⁠~⁠} deg, El= {{15}⁠~⁠}deg.

}
\hdashrule[0.5ex]{\textwidth}{1pt}{3mm}
  Expected Result \\
{\footnotesize
TMA reaches the commanded position.

}

\begin{tabular}{p{2cm}}
\toprule
Step 144  \\ \hline
\end{tabular}
 Description \\
{\footnotesize
\textbf{Point~the Dome and the TMA}

\begin{itemize}
\tightlist
\item
  Command the Dome to {Pointing 4}⁠ to {-270}⁠deg
\item
  Command the TMA to {Pointing 4}⁠ at Az= {{-270}⁠~⁠~}deg, El=
  {{85.0}⁠}⁠ deg.
\end{itemize}

}
\hdashrule[0.5ex]{\textwidth}{1pt}{3mm}
  Expected Result \\
{\footnotesize
The dome starts the movement.\\
The TMA starts to move.

}

\begin{tabular}{p{2cm}}
\toprule
Step 145  \\ \hline
\end{tabular}
 Description \\
{\footnotesize
Wait for the Dome to reach the commanded position.\\
Wait for the TMA to reach the commanded position.

}
\hdashrule[0.5ex]{\textwidth}{1pt}{3mm}
  Expected Result \\
{\footnotesize
The Dome reaches the commanded position.\\
The \emph{MTDome\_logevent\_azMotion} and
\emph{MTDome\_logevent\_elMotion} inPosition parameter = true.\\
The TMA reaches the commanded position.\\
The \emph{MTMount\_logevent\_azimuthInPosition} and
\emph{MTMount\_logevent\_elevationInPosition} inPosition parameter =
true.

}

\begin{tabular}{p{2cm}}
\toprule
Step 146  \\ \hline
\end{tabular}
 Description \\
{\footnotesize
\textbf{Image preparation}\\
If the preparation to take images takes longer than 10sec, do
repositioning to target Az= {{-270}⁠~⁠} deg, El= {{85.0}⁠~⁠}deg.

}
\hdashrule[0.5ex]{\textwidth}{1pt}{3mm}
  Expected Result \\
{\footnotesize
TMA reaches the commanded position.

}

\begin{tabular}{p{2cm}}
\toprule
Step 147  \\ \hline
\end{tabular}
 Description \\
{\footnotesize
\textbf{Stop the Dome}\\
Verify that the Dome is stopped so that it does not move during the
observations.

}
\hdashrule[0.5ex]{\textwidth}{1pt}{3mm}
  Expected Result \\
{\footnotesize
The Dome is stopped.\\[2\baselineskip]

}

\begin{tabular}{p{2cm}}
\toprule
Step 148  \\ \hline
\end{tabular}
 Description \\
{\footnotesize
\textbf{Track position and take images}\\[2\baselineskip]Star tracking
for 20 synchronized exposures with 5,4,6 seconds for wide, narrow, fast
cameras.\\[2\baselineskip]or\\[2\baselineskip]Star tracking for 10 min
and take synchronized exposures with 5,4,6 seconds for wide, narrow,
fast cameras.\\[2\baselineskip]\textbf{Note:~}Taking 20 images is good
for pointing evaluation..\\
Taking images for 10 min is needed for tracking evaluation

}
\hdashrule[0.5ex]{\textwidth}{1pt}{3mm}
  Test Data \\
 {\footnotesize
The Dome should not move during the observations.

}
\hdashrule[0.5ex]{\textwidth}{1pt}{3mm}
  Expected Result \\
{\footnotesize
\begin{itemize}
\tightlist
\item
  The TMA tracks the given position for 10 min, and the cameras take
  images or 20 images ar take per pointing
\item
  A series of images is successfully taken with the StarTracker and can
  be seen via RubinTV.
\end{itemize}

}

\begin{tabular}{p{2cm}}
\toprule
Step 149  \\ \hline
\end{tabular}
 Description \\
{\footnotesize
\textbf{Image verification:}\\
Look at RubinTV and verify that astrometry.net found an astrometric
solution.

}
\hdashrule[0.5ex]{\textwidth}{1pt}{3mm}
  Expected Result \\
{\footnotesize
Astrometry.net finds a solution.

}

\begin{tabular}{p{2cm}}
\toprule
Step 150  \\ \hline
\end{tabular}
 Description \\
{\footnotesize
\textbf{Offline analysis results}\\
Offline analysis in test case
\href{https://jira.lsstcorp.org/secure/Tests.jspa\#/testCase/LVV-T2739}{LVV-T2739}.

}
\hdashrule[0.5ex]{\textwidth}{1pt}{3mm}
  Expected Result \\
{\footnotesize
Image quality is sufficient.

}

\begin{tabular}{p{2cm}}
\toprule
Step 151  \\ \hline
\end{tabular}
 Description \\
{\footnotesize
\textbf{Stop the Dome}\\
Verify that the Dome is stopped so that it does not move during the
observations.

}
\hdashrule[0.5ex]{\textwidth}{1pt}{3mm}
  Expected Result \\
{\footnotesize
The Dome is stopped.\\[2\baselineskip]

}

\begin{tabular}{p{2cm}}
\toprule
Step 152  \\ \hline
\end{tabular}
 Description \\
{\footnotesize
\textbf{Point~the Dome and the TMA}

\begin{itemize}
\tightlist
\item
  Command the Dome to {Pointing 5}⁠ to {-180}⁠deg
\item
  Command the TMA to {Pointing 5}⁠ at Az= {{-180}⁠~⁠~}deg, El=
  {{85.0}⁠}⁠ deg.
\end{itemize}

}
\hdashrule[0.5ex]{\textwidth}{1pt}{3mm}
  Expected Result \\
{\footnotesize
The dome starts the movement.\\
The TMA starts to move.

}

\begin{tabular}{p{2cm}}
\toprule
Step 153  \\ \hline
\end{tabular}
 Description \\
{\footnotesize
Wait for the Dome to reach the commanded position.\\
Wait for the TMA to reach the commanded position.

}
\hdashrule[0.5ex]{\textwidth}{1pt}{3mm}
  Expected Result \\
{\footnotesize
The Dome reaches the commanded position.\\
The \emph{MTDome\_logevent\_azMotion} and
\emph{MTDome\_logevent\_elMotion} inPosition parameter = true.\\
The TMA reaches the commanded position.\\
The \emph{MTMount\_logevent\_azimuthInPosition} and
\emph{MTMount\_logevent\_elevationInPosition} inPosition parameter =
true.

}

\begin{tabular}{p{2cm}}
\toprule
Step 154  \\ \hline
\end{tabular}
 Description \\
{\footnotesize
\textbf{Image preparation}\\
If the preparation to take images takes longer than 10sec, do
repositioning to target Az= {{-180}⁠~⁠} deg, El= {{85.0}⁠~⁠}deg.

}
\hdashrule[0.5ex]{\textwidth}{1pt}{3mm}
  Expected Result \\
{\footnotesize
TMA reaches the commanded position.

}

\begin{tabular}{p{2cm}}
\toprule
Step 155  \\ \hline
\end{tabular}
 Description \\
{\footnotesize
\textbf{Stop the Dome}\\
Verify that the Dome is stopped so that it does not move during the
observations.

}
\hdashrule[0.5ex]{\textwidth}{1pt}{3mm}
  Expected Result \\
{\footnotesize
The Dome is stopped.\\[2\baselineskip]

}

\begin{tabular}{p{2cm}}
\toprule
Step 156  \\ \hline
\end{tabular}
 Description \\
{\footnotesize
\textbf{Track position and take images}\\[2\baselineskip]Star tracking
for 20 synchronized exposures with 5,4,6 seconds for wide, narrow, fast
cameras.\\[2\baselineskip]or\\[2\baselineskip]Star tracking for 10 min
and take synchronized exposures with 5,4,6 seconds for wide, narrow,
fast cameras.\\[2\baselineskip]\textbf{Note:~}Taking 20 images is good
for pointing evaluation..\\
Taking images for 10 min is needed for tracking evaluation

}
\hdashrule[0.5ex]{\textwidth}{1pt}{3mm}
  Test Data \\
 {\footnotesize
The Dome should not move during the observations.

}
\hdashrule[0.5ex]{\textwidth}{1pt}{3mm}
  Expected Result \\
{\footnotesize
\begin{itemize}
\tightlist
\item
  The TMA tracks the given position for 10 min, and the cameras take
  images or 20 images ar take per pointing
\item
  A series of images is successfully taken with the StarTracker and can
  be seen via RubinTV.
\end{itemize}

}

\begin{tabular}{p{2cm}}
\toprule
Step 157  \\ \hline
\end{tabular}
 Description \\
{\footnotesize
\textbf{Image verification:}\\
Look at RubinTV and verify that astrometry.net found an astrometric
solution.

}
\hdashrule[0.5ex]{\textwidth}{1pt}{3mm}
  Expected Result \\
{\footnotesize
Astrometry.net finds a solution.

}

\begin{tabular}{p{2cm}}
\toprule
Step 158  \\ \hline
\end{tabular}
 Description \\
{\footnotesize
\textbf{Offline analysis results}\\
Offline analysis in test case
\href{https://jira.lsstcorp.org/secure/Tests.jspa\#/testCase/LVV-T2739}{LVV-T2739}.

}
\hdashrule[0.5ex]{\textwidth}{1pt}{3mm}
  Expected Result \\
{\footnotesize
Image quality is sufficient.

}

\begin{tabular}{p{2cm}}
\toprule
Step 159  \\ \hline
\end{tabular}
 Description \\
{\footnotesize
\textbf{Track position and take images}\\[2\baselineskip]Star tracking
for 20 synchronized exposures with 5,4,6 seconds for wide, narrow, fast
cameras.\\[2\baselineskip]or\\[2\baselineskip]Star tracking for 10 min
and take synchronized exposures with 5,4,6 seconds for wide, narrow,
fast cameras.\\[2\baselineskip]\textbf{Note:~}Taking 20 images is good
for pointing evaluation..\\
Taking images for 10 min is needed for tracking evaluation

}
\hdashrule[0.5ex]{\textwidth}{1pt}{3mm}
  Test Data \\
 {\footnotesize
The Dome should not move during the observations.

}
\hdashrule[0.5ex]{\textwidth}{1pt}{3mm}
  Expected Result \\
{\footnotesize
\begin{itemize}
\tightlist
\item
  The TMA tracks the given position for 10 min, and the cameras take
  images or 20 images ar take per pointing
\item
  A series of images is successfully taken with the StarTracker and can
  be seen via RubinTV.
\end{itemize}

}

\begin{tabular}{p{2cm}}
\toprule
Step 160  \\ \hline
\end{tabular}
 Description \\
{\footnotesize
\textbf{Point~the Dome and the TMA}

\begin{itemize}
\tightlist
\item
  Command the Dome to {Pointing 7}⁠ to {-180}⁠deg
\item
  Command the TMA to {Pointing 7}⁠ at Az= {{-180}⁠~⁠~}deg, El= {{45}⁠}⁠
  deg.
\end{itemize}

}
\hdashrule[0.5ex]{\textwidth}{1pt}{3mm}
  Expected Result \\
{\footnotesize
The dome starts the movement.\\
The TMA starts to move.

}

\begin{tabular}{p{2cm}}
\toprule
Step 161  \\ \hline
\end{tabular}
 Description \\
{\footnotesize
Wait for the Dome to reach the commanded position.\\
Wait for the TMA to reach the commanded position.

}
\hdashrule[0.5ex]{\textwidth}{1pt}{3mm}
  Expected Result \\
{\footnotesize
The Dome reaches the commanded position.\\
The \emph{MTDome\_logevent\_azMotion} and
\emph{MTDome\_logevent\_elMotion} inPosition parameter = true.\\
The TMA reaches the commanded position.\\
The \emph{MTMount\_logevent\_azimuthInPosition} and
\emph{MTMount\_logevent\_elevationInPosition} inPosition parameter =
true.

}

\begin{tabular}{p{2cm}}
\toprule
Step 162  \\ \hline
\end{tabular}
 Description \\
{\footnotesize
\textbf{Image preparation}\\
If the preparation to take images takes longer than 10sec, do
repositioning to target Az= {{-180}⁠~⁠} deg, El= {{45}⁠~⁠}deg.

}
\hdashrule[0.5ex]{\textwidth}{1pt}{3mm}
  Expected Result \\
{\footnotesize
TMA reaches the commanded position.

}

\begin{tabular}{p{2cm}}
\toprule
Step 163  \\ \hline
\end{tabular}
 Description \\
{\footnotesize
\textbf{Stop the Dome}\\
Verify that the Dome is stopped so that it does not move during the
observations.

}
\hdashrule[0.5ex]{\textwidth}{1pt}{3mm}
  Expected Result \\
{\footnotesize
The Dome is stopped.\\[2\baselineskip]

}

\begin{tabular}{p{2cm}}
\toprule
Step 164  \\ \hline
\end{tabular}
 Description \\
{\footnotesize
\textbf{Track position and take images}\\[2\baselineskip]Star tracking
for 20 synchronized exposures with 5,4,6 seconds for wide, narrow, fast
cameras.\\[2\baselineskip]or\\[2\baselineskip]Star tracking for 10 min
and take synchronized exposures with 5,4,6 seconds for wide, narrow,
fast cameras.\\[2\baselineskip]\textbf{Note:~}Taking 20 images is good
for pointing evaluation..\\
Taking images for 10 min is needed for tracking evaluation

}
\hdashrule[0.5ex]{\textwidth}{1pt}{3mm}
  Test Data \\
 {\footnotesize
The Dome should not move during the observations.

}
\hdashrule[0.5ex]{\textwidth}{1pt}{3mm}
  Expected Result \\
{\footnotesize
\begin{itemize}
\tightlist
\item
  The TMA tracks the given position for 10 min, and the cameras take
  images or 20 images ar take per pointing
\item
  A series of images is successfully taken with the StarTracker and can
  be seen via RubinTV.
\end{itemize}

}

\begin{tabular}{p{2cm}}
\toprule
Step 165  \\ \hline
\end{tabular}
 Description \\
{\footnotesize
\textbf{Image verification:}\\
Look at RubinTV and verify that astrometry.net found an astrometric
solution.

}
\hdashrule[0.5ex]{\textwidth}{1pt}{3mm}
  Expected Result \\
{\footnotesize
Astrometry.net finds a solution.

}

\begin{tabular}{p{2cm}}
\toprule
Step 166  \\ \hline
\end{tabular}
 Description \\
{\footnotesize
\textbf{Offline analysis results}\\
Offline analysis in test case
\href{https://jira.lsstcorp.org/secure/Tests.jspa\#/testCase/LVV-T2739}{LVV-T2739}.

}
\hdashrule[0.5ex]{\textwidth}{1pt}{3mm}
  Expected Result \\
{\footnotesize
Image quality is sufficient.

}

\begin{tabular}{p{2cm}}
\toprule
Step 167  \\ \hline
\end{tabular}
 Description \\
{\footnotesize
\textbf{Image verification:}\\
Look at RubinTV and verify that astrometry.net found an astrometric
solution.

}
\hdashrule[0.5ex]{\textwidth}{1pt}{3mm}
  Expected Result \\
{\footnotesize
Astrometry.net finds a solution.

}

\begin{tabular}{p{2cm}}
\toprule
Step 168  \\ \hline
\end{tabular}
 Description \\
{\footnotesize
\textbf{Point~the Dome and the TMA}

\begin{itemize}
\tightlist
\item
  Command the Dome to {Pointing 8}⁠ to {-180}⁠deg
\item
  Command the TMA to {Pointing 8}⁠ at Az= {{-180}⁠~⁠~}deg, El= {{15}⁠}⁠
  deg.
\end{itemize}

}
\hdashrule[0.5ex]{\textwidth}{1pt}{3mm}
  Expected Result \\
{\footnotesize
The dome starts the movement.\\
The TMA starts to move.

}

\begin{tabular}{p{2cm}}
\toprule
Step 169  \\ \hline
\end{tabular}
 Description \\
{\footnotesize
Wait for the Dome to reach the commanded position.\\
Wait for the TMA to reach the commanded position.

}
\hdashrule[0.5ex]{\textwidth}{1pt}{3mm}
  Expected Result \\
{\footnotesize
The Dome reaches the commanded position.\\
The \emph{MTDome\_logevent\_azMotion} and
\emph{MTDome\_logevent\_elMotion} inPosition parameter = true.\\
The TMA reaches the commanded position.\\
The \emph{MTMount\_logevent\_azimuthInPosition} and
\emph{MTMount\_logevent\_elevationInPosition} inPosition parameter =
true.

}

\begin{tabular}{p{2cm}}
\toprule
Step 170  \\ \hline
\end{tabular}
 Description \\
{\footnotesize
\textbf{Image preparation}\\
If the preparation to take images takes longer than 10sec, do
repositioning to target Az= {{-180}⁠~⁠} deg, El= {{15}⁠~⁠}deg.

}
\hdashrule[0.5ex]{\textwidth}{1pt}{3mm}
  Expected Result \\
{\footnotesize
TMA reaches the commanded position.

}

\begin{tabular}{p{2cm}}
\toprule
Step 171  \\ \hline
\end{tabular}
 Description \\
{\footnotesize
\textbf{Stop the Dome}\\
Verify that the Dome is stopped so that it does not move during the
observations.

}
\hdashrule[0.5ex]{\textwidth}{1pt}{3mm}
  Expected Result \\
{\footnotesize
The Dome is stopped.\\[2\baselineskip]

}

\begin{tabular}{p{2cm}}
\toprule
Step 172  \\ \hline
\end{tabular}
 Description \\
{\footnotesize
\textbf{Track position and take images}\\[2\baselineskip]Star tracking
for 20 synchronized exposures with 5,4,6 seconds for wide, narrow, fast
cameras.\\[2\baselineskip]or\\[2\baselineskip]Star tracking for 10 min
and take synchronized exposures with 5,4,6 seconds for wide, narrow,
fast cameras.\\[2\baselineskip]\textbf{Note:~}Taking 20 images is good
for pointing evaluation..\\
Taking images for 10 min is needed for tracking evaluation

}
\hdashrule[0.5ex]{\textwidth}{1pt}{3mm}
  Test Data \\
 {\footnotesize
The Dome should not move during the observations.

}
\hdashrule[0.5ex]{\textwidth}{1pt}{3mm}
  Expected Result \\
{\footnotesize
\begin{itemize}
\tightlist
\item
  The TMA tracks the given position for 10 min, and the cameras take
  images or 20 images ar take per pointing
\item
  A series of images is successfully taken with the StarTracker and can
  be seen via RubinTV.
\end{itemize}

}

\begin{tabular}{p{2cm}}
\toprule
Step 173  \\ \hline
\end{tabular}
 Description \\
{\footnotesize
\textbf{Image verification:}\\
Look at RubinTV and verify that astrometry.net found an astrometric
solution.

}
\hdashrule[0.5ex]{\textwidth}{1pt}{3mm}
  Expected Result \\
{\footnotesize
Astrometry.net finds a solution.

}

\begin{tabular}{p{2cm}}
\toprule
Step 174  \\ \hline
\end{tabular}
 Description \\
{\footnotesize
\textbf{Offline analysis results}\\
Offline analysis in test case
\href{https://jira.lsstcorp.org/secure/Tests.jspa\#/testCase/LVV-T2739}{LVV-T2739}.

}
\hdashrule[0.5ex]{\textwidth}{1pt}{3mm}
  Expected Result \\
{\footnotesize
Image quality is sufficient.

}

\begin{tabular}{p{2cm}}
\toprule
Step 175  \\ \hline
\end{tabular}
 Description \\
{\footnotesize
\textbf{Offline analysis results}\\
Offline analysis in test case
\href{https://jira.lsstcorp.org/secure/Tests.jspa\#/testCase/LVV-T2739}{LVV-T2739}.

}
\hdashrule[0.5ex]{\textwidth}{1pt}{3mm}
  Expected Result \\
{\footnotesize
Image quality is sufficient.

}

\begin{tabular}{p{2cm}}
\toprule
Step 176  \\ \hline
\end{tabular}
 Description \\
{\footnotesize
\textbf{Point~the Dome and the TMA}

\begin{itemize}
\tightlist
\item
  Command the Dome to {Pointing 9}⁠ to {-90}⁠deg
\item
  Command the TMA to {Pointing 9}⁠ at Az= {{-90}⁠~⁠~}deg, El= {{15}⁠}⁠
  deg.
\end{itemize}

}
\hdashrule[0.5ex]{\textwidth}{1pt}{3mm}
  Expected Result \\
{\footnotesize
The dome starts the movement.\\
The TMA starts to move.

}

\begin{tabular}{p{2cm}}
\toprule
Step 177  \\ \hline
\end{tabular}
 Description \\
{\footnotesize
Wait for the Dome to reach the commanded position.\\
Wait for the TMA to reach the commanded position.

}
\hdashrule[0.5ex]{\textwidth}{1pt}{3mm}
  Expected Result \\
{\footnotesize
The Dome reaches the commanded position.\\
The \emph{MTDome\_logevent\_azMotion} and
\emph{MTDome\_logevent\_elMotion} inPosition parameter = true.\\
The TMA reaches the commanded position.\\
The \emph{MTMount\_logevent\_azimuthInPosition} and
\emph{MTMount\_logevent\_elevationInPosition} inPosition parameter =
true.

}

\begin{tabular}{p{2cm}}
\toprule
Step 178  \\ \hline
\end{tabular}
 Description \\
{\footnotesize
\textbf{Image preparation}\\
If the preparation to take images takes longer than 10sec, do
repositioning to target Az= {{-90}⁠~⁠} deg, El= {{15}⁠~⁠}deg.

}
\hdashrule[0.5ex]{\textwidth}{1pt}{3mm}
  Expected Result \\
{\footnotesize
TMA reaches the commanded position.

}

\begin{tabular}{p{2cm}}
\toprule
Step 179  \\ \hline
\end{tabular}
 Description \\
{\footnotesize
\textbf{Stop the Dome}\\
Verify that the Dome is stopped so that it does not move during the
observations.

}
\hdashrule[0.5ex]{\textwidth}{1pt}{3mm}
  Expected Result \\
{\footnotesize
The Dome is stopped.\\[2\baselineskip]

}

\begin{tabular}{p{2cm}}
\toprule
Step 180  \\ \hline
\end{tabular}
 Description \\
{\footnotesize
\textbf{Track position and take images}\\[2\baselineskip]Star tracking
for 20 synchronized exposures with 5,4,6 seconds for wide, narrow, fast
cameras.\\[2\baselineskip]or\\[2\baselineskip]Star tracking for 10 min
and take synchronized exposures with 5,4,6 seconds for wide, narrow,
fast cameras.\\[2\baselineskip]\textbf{Note:~}Taking 20 images is good
for pointing evaluation..\\
Taking images for 10 min is needed for tracking evaluation

}
\hdashrule[0.5ex]{\textwidth}{1pt}{3mm}
  Test Data \\
 {\footnotesize
The Dome should not move during the observations.

}
\hdashrule[0.5ex]{\textwidth}{1pt}{3mm}
  Expected Result \\
{\footnotesize
\begin{itemize}
\tightlist
\item
  The TMA tracks the given position for 10 min, and the cameras take
  images or 20 images ar take per pointing
\item
  A series of images is successfully taken with the StarTracker and can
  be seen via RubinTV.
\end{itemize}

}

\begin{tabular}{p{2cm}}
\toprule
Step 181  \\ \hline
\end{tabular}
 Description \\
{\footnotesize
\textbf{Image verification:}\\
Look at RubinTV and verify that astrometry.net found an astrometric
solution.

}
\hdashrule[0.5ex]{\textwidth}{1pt}{3mm}
  Expected Result \\
{\footnotesize
Astrometry.net finds a solution.

}

\begin{tabular}{p{2cm}}
\toprule
Step 182  \\ \hline
\end{tabular}
 Description \\
{\footnotesize
\textbf{Offline analysis results}\\
Offline analysis in test case
\href{https://jira.lsstcorp.org/secure/Tests.jspa\#/testCase/LVV-T2739}{LVV-T2739}.

}
\hdashrule[0.5ex]{\textwidth}{1pt}{3mm}
  Expected Result \\
{\footnotesize
Image quality is sufficient.

}

\begin{tabular}{p{2cm}}
\toprule
Step 183  \\ \hline
\end{tabular}
 Description \\
{\footnotesize
\textbf{Point~the Dome and the TMA}

\begin{itemize}
\tightlist
\item
  Command the Dome to {Pointing 10}⁠ to {-90}⁠deg
\item
  Command the TMA to {Pointing 10}⁠ at Az= {{-90}⁠~⁠~}deg, El= {{45}⁠}⁠
  deg.
\end{itemize}

}
\hdashrule[0.5ex]{\textwidth}{1pt}{3mm}
  Expected Result \\
{\footnotesize
The dome starts the movement.\\
The TMA starts to move.

}

\begin{tabular}{p{2cm}}
\toprule
Step 184  \\ \hline
\end{tabular}
 Description \\
{\footnotesize
Wait for the Dome to reach the commanded position.\\
Wait for the TMA to reach the commanded position.

}
\hdashrule[0.5ex]{\textwidth}{1pt}{3mm}
  Expected Result \\
{\footnotesize
The Dome reaches the commanded position.\\
The \emph{MTDome\_logevent\_azMotion} and
\emph{MTDome\_logevent\_elMotion} inPosition parameter = true.\\
The TMA reaches the commanded position.\\
The \emph{MTMount\_logevent\_azimuthInPosition} and
\emph{MTMount\_logevent\_elevationInPosition} inPosition parameter =
true.

}

\begin{tabular}{p{2cm}}
\toprule
Step 185  \\ \hline
\end{tabular}
 Description \\
{\footnotesize
\textbf{Image preparation}\\
If the preparation to take images takes longer than 10sec, do
repositioning to target Az= {{-90}⁠~⁠} deg, El= {{45}⁠~⁠}deg.

}
\hdashrule[0.5ex]{\textwidth}{1pt}{3mm}
  Expected Result \\
{\footnotesize
TMA reaches the commanded position.

}

\begin{tabular}{p{2cm}}
\toprule
Step 186  \\ \hline
\end{tabular}
 Description \\
{\footnotesize
\textbf{Stop the Dome}\\
Verify that the Dome is stopped so that it does not move during the
observations.

}
\hdashrule[0.5ex]{\textwidth}{1pt}{3mm}
  Expected Result \\
{\footnotesize
The Dome is stopped.\\[2\baselineskip]

}

\begin{tabular}{p{2cm}}
\toprule
Step 187  \\ \hline
\end{tabular}
 Description \\
{\footnotesize
\textbf{Track position and take images}\\[2\baselineskip]Star tracking
for 20 synchronized exposures with 5,4,6 seconds for wide, narrow, fast
cameras.\\[2\baselineskip]or\\[2\baselineskip]Star tracking for 10 min
and take synchronized exposures with 5,4,6 seconds for wide, narrow,
fast cameras.\\[2\baselineskip]\textbf{Note:~}Taking 20 images is good
for pointing evaluation..\\
Taking images for 10 min is needed for tracking evaluation

}
\hdashrule[0.5ex]{\textwidth}{1pt}{3mm}
  Test Data \\
 {\footnotesize
The Dome should not move during the observations.

}
\hdashrule[0.5ex]{\textwidth}{1pt}{3mm}
  Expected Result \\
{\footnotesize
\begin{itemize}
\tightlist
\item
  The TMA tracks the given position for 10 min, and the cameras take
  images or 20 images ar take per pointing
\item
  A series of images is successfully taken with the StarTracker and can
  be seen via RubinTV.
\end{itemize}

}

\begin{tabular}{p{2cm}}
\toprule
Step 188  \\ \hline
\end{tabular}
 Description \\
{\footnotesize
\textbf{Image verification:}\\
Look at RubinTV and verify that astrometry.net found an astrometric
solution.

}
\hdashrule[0.5ex]{\textwidth}{1pt}{3mm}
  Expected Result \\
{\footnotesize
Astrometry.net finds a solution.

}

\begin{tabular}{p{2cm}}
\toprule
Step 189  \\ \hline
\end{tabular}
 Description \\
{\footnotesize
\textbf{Offline analysis results}\\
Offline analysis in test case
\href{https://jira.lsstcorp.org/secure/Tests.jspa\#/testCase/LVV-T2739}{LVV-T2739}.

}
\hdashrule[0.5ex]{\textwidth}{1pt}{3mm}
  Expected Result \\
{\footnotesize
Image quality is sufficient.

}

\begin{tabular}{p{2cm}}
\toprule
Step 190  \\ \hline
\end{tabular}
 Description \\
{\footnotesize
\textbf{Point~the Dome and the TMA}

\begin{itemize}
\tightlist
\item
  Command the Dome to {Pointing 12}⁠ to {-90}⁠deg
\item
  Command the TMA to {Pointing 12}⁠ at Az= {{-90}⁠~⁠~}deg, El=
  {{85.0}⁠}⁠ deg.
\end{itemize}

}
\hdashrule[0.5ex]{\textwidth}{1pt}{3mm}
  Expected Result \\
{\footnotesize
The dome starts the movement.\\
The TMA starts to move.

}

\begin{tabular}{p{2cm}}
\toprule
Step 191  \\ \hline
\end{tabular}
 Description \\
{\footnotesize
Wait for the Dome to reach the commanded position.\\
Wait for the TMA to reach the commanded position.

}
\hdashrule[0.5ex]{\textwidth}{1pt}{3mm}
  Expected Result \\
{\footnotesize
The Dome reaches the commanded position.\\
The \emph{MTDome\_logevent\_azMotion} and
\emph{MTDome\_logevent\_elMotion} inPosition parameter = true.\\
The TMA reaches the commanded position.\\
The \emph{MTMount\_logevent\_azimuthInPosition} and
\emph{MTMount\_logevent\_elevationInPosition} inPosition parameter =
true.

}

\begin{tabular}{p{2cm}}
\toprule
Step 192  \\ \hline
\end{tabular}
 Description \\
{\footnotesize
\textbf{Image preparation}\\
If the preparation to take images takes longer than 10sec, do
repositioning to target Az= {{-90}⁠~⁠} deg, El= {{85.0}⁠~⁠}deg.

}
\hdashrule[0.5ex]{\textwidth}{1pt}{3mm}
  Expected Result \\
{\footnotesize
TMA reaches the commanded position.

}

\begin{tabular}{p{2cm}}
\toprule
Step 193  \\ \hline
\end{tabular}
 Description \\
{\footnotesize
\textbf{Stop the Dome}\\
Verify that the Dome is stopped so that it does not move during the
observations.

}
\hdashrule[0.5ex]{\textwidth}{1pt}{3mm}
  Expected Result \\
{\footnotesize
The Dome is stopped.\\[2\baselineskip]

}

\begin{tabular}{p{2cm}}
\toprule
Step 194  \\ \hline
\end{tabular}
 Description \\
{\footnotesize
\textbf{Track position and take images}\\[2\baselineskip]Star tracking
for 20 synchronized exposures with 5,4,6 seconds for wide, narrow, fast
cameras.\\[2\baselineskip]or\\[2\baselineskip]Star tracking for 10 min
and take synchronized exposures with 5,4,6 seconds for wide, narrow,
fast cameras.\\[2\baselineskip]\textbf{Note:~}Taking 20 images is good
for pointing evaluation..\\
Taking images for 10 min is needed for tracking evaluation

}
\hdashrule[0.5ex]{\textwidth}{1pt}{3mm}
  Test Data \\
 {\footnotesize
The Dome should not move during the observations.

}
\hdashrule[0.5ex]{\textwidth}{1pt}{3mm}
  Expected Result \\
{\footnotesize
\begin{itemize}
\tightlist
\item
  The TMA tracks the given position for 10 min, and the cameras take
  images or 20 images ar take per pointing
\item
  A series of images is successfully taken with the StarTracker and can
  be seen via RubinTV.
\end{itemize}

}

\begin{tabular}{p{2cm}}
\toprule
Step 195  \\ \hline
\end{tabular}
 Description \\
{\footnotesize
\textbf{Image verification:}\\
Look at RubinTV and verify that astrometry.net found an astrometric
solution.

}
\hdashrule[0.5ex]{\textwidth}{1pt}{3mm}
  Expected Result \\
{\footnotesize
Astrometry.net finds a solution.

}

\begin{tabular}{p{2cm}}
\toprule
Step 196  \\ \hline
\end{tabular}
 Description \\
{\footnotesize
\textbf{Offline analysis results}\\
Offline analysis in test case
\href{https://jira.lsstcorp.org/secure/Tests.jspa\#/testCase/LVV-T2739}{LVV-T2739}.

}
\hdashrule[0.5ex]{\textwidth}{1pt}{3mm}
  Expected Result \\
{\footnotesize
Image quality is sufficient.

}

\paragraph{ LVV-T2715 - Configure Observatory Environment for Daytime Operations }\mbox{}\\

Version \textbf{1}.
Open  \href{https://jira.lsstcorp.org/secure/Tests.jspa#/testCase/LVV-T2715}{\textit{ LVV-T2715 } }
test case in Jira.

After using the observatory during the nighttime, prepare the
observatory for daytime operations.

\textbf{ Preconditions}:\\
The observatory was used during nighttime.

Final comment:\\


Detailed steps :

\begin{tabular}{p{2cm}}
\toprule
Step 1  \\ \hline
\end{tabular}
 Description \\
{\footnotesize
\textbf{CSCs}\\

\begin{itemize}
\tightlist
\item
  Transition the CSCs into STANDBY state
\end{itemize}

}
\hdashrule[0.5ex]{\textwidth}{1pt}{3mm}
  Expected Result \\
{\footnotesize
All CSCs are in their standbyState.

}

\begin{tabular}{p{2cm}}
\toprule
Step 2  \\ \hline
\end{tabular}
 Description \\
{\footnotesize
\textbf{Telescope daytime preparations:}

\begin{itemize}
\tightlist
\item
  Switch off or bring into standby the StarTracker and DIMM instruments
\item
  Install the caps on top of the StarTracker telescopes and the DIMM
\end{itemize}

}
\hdashrule[0.5ex]{\textwidth}{1pt}{3mm}
  Expected Result \\
{\footnotesize
The caps are installed.

}

\begin{tabular}{p{2cm}}
\toprule
Step 3  \\ \hline
\end{tabular}
 Description \\
{\footnotesize
\textbf{Dome:}\\

\begin{itemize}
\tightlist
\item
  Bring the dome into the park position
\end{itemize}

Until the dome shutter is motorized:\textbf{\\
}

\begin{itemize}
\tightlist
\item
  Send a message to the site manager :

  \begin{itemize}
  \tightlist
  \item
    confirming that nightly operations have finished~
  \item
    asking for a dome closer before the sun starts to shine on the
    StarTracker and the DIMM.
  \end{itemize}
\end{itemize}

}
\hdashrule[0.5ex]{\textwidth}{1pt}{3mm}
  Expected Result \\
{\footnotesize
Dome closure is organized.

}

\begin{tabular}{p{2cm}}
\toprule
Step 4  \\ \hline
\end{tabular}
 Description \\
{\footnotesize
\textbf{Auxillary systems~daytime preparations:}\\
If needed for daytime operations:\textbf{\\
}

\begin{itemize}
\tightlist
\item
  Switch on the UMA in the morning.
\item
  When available and need to be modified for the day:

  \begin{itemize}
  \tightlist
  \item
    Oil supply system on standby?
  \item
    Dynalyne into standby?
  \end{itemize}
\end{itemize}

}
\hdashrule[0.5ex]{\textwidth}{1pt}{3mm}
  Expected Result \\
{\footnotesize
All auxiliary systems are in the states suitable for daytime operations.

\begin{itemize}
\tightlist
\item
  The UMA is switched on.
\end{itemize}

}

\begin{tabular}{p{2cm}}
\toprule
Step 5  \\ \hline
\end{tabular}
 Description \\
{\footnotesize
\textbf{TMA position in the morning}\\

\begin{itemize}
\tightlist
\item
  Park the TMA in the position needed for the next day.
\end{itemize}

}
\hdashrule[0.5ex]{\textwidth}{1pt}{3mm}
  Expected Result \\
{\footnotesize
TMA parked in the corresponding position.

}

\begin{tabular}{p{2cm}}
\toprule
Step 6  \\ \hline
\end{tabular}
 Description \\
{\footnotesize
\textbf{Night log}\\

\begin{itemize}
\tightlist
\item
  Close the night log by writing a summary of the nightly events
\item
  Send a link with the summary to the site manager.
\end{itemize}

}
\hdashrule[0.5ex]{\textwidth}{1pt}{3mm}
  Expected Result \\
{\footnotesize
The night log is closed.

}

\subsection{Test Cycle LVV-C229 }

Open test cycle {\it \href{https://jira.lsstcorp.org/secure/Tests.jspa#/testrun/LVV-C229}{TMA Pointing and Tracking - Part 2 - Reverse on the Sky Part 1 - 50"-
StarTracker}} in Jira.

Test Cycle name: TMA Pointing and Tracking - Part 2 - Reverse on the Sky Part 1 - 50"-
StarTracker\\
Status: In Progress

Requirements verification for the pointing and tracking using the Star
Tracker and the DIMM on a dedicated mounting plate connector to the top
end of the TMA.

\subsubsection{Software Version/Baseline}
Star Tracker software version:\\
Dimm software version:\\
CSC software version:\\
Analysis software repository:

\subsubsection{Configuration}
Not provided.

\subsubsection{Test Cases in LVV-C229 Test Cycle}

\paragraph{ LVV-T2707 - Evening Summit Tailgate Meeting - TMA and Dome Testing Safety Assurance }\mbox{}\\

Version \textbf{2}.
Open  \href{https://jira.lsstcorp.org/secure/Tests.jspa#/testCase/LVV-T2707}{\textit{ LVV-T2707 } }
test case in Jira.

Ensure the safety of observation with the main telescope during
nighttime operations.\\
\textbf{Tailgate Meeting:} Hold a tailgate for the upcoming task with
personnel on the summit working during the night. Go over any relevant
procedures, roles, and
responsibilities.\\[2\baselineskip]\textbf{Note:~}Version two is for
tests that do not involve moving or opening the dome.

\textbf{ Preconditions}:\\
All nonessential personnel has vacated the area.

Final comment:\\


Detailed steps :

\begin{tabular}{p{2cm}}
\toprule
Step 1  \\ \hline
\end{tabular}
 Description \\
{\footnotesize
\textbf{Daytime info collection:}

\begin{itemize}
\tightlist
\item
  Revise the
  \href{https://confluence.lsstcorp.org/display/LTS/Summit+Daylogs}{last
  Summit Daylog} about changes that might influence the work during the
  night.
\item
  Confirm that all the workers in the TMA and Dome areas have already
  left. This is best performed during the walk-through at the end of the
  day.~~
\end{itemize}

}
\hdashrule[0.5ex]{\textwidth}{1pt}{3mm}
  Expected Result \\
{\footnotesize
Is the observatory ready to observe?

}

\begin{tabular}{p{2cm}}
\toprule
Step 2  \\ \hline
\end{tabular}
 Description \\
{\footnotesize
\textbf{Alarm system check}\\
Once available:\\

\begin{itemize}
\tightlist
\item
  Confirm that any audio and visual alarms are operating properly. (Need
  details on what, if any, alarms should be checked)
\item
  Confirm that the safety systems for earthquakes and fire are working.
\end{itemize}

}
\hdashrule[0.5ex]{\textwidth}{1pt}{3mm}
  Expected Result \\
{\footnotesize
All alarms are functioning properly.

\begin{itemize}
\tightlist
\item
  Earthquake alert system is working:
\item
  The fire system is working:
\end{itemize}

}

\begin{tabular}{p{2cm}}
\toprule
Step 3  \\ \hline
\end{tabular}
 Description \\
{\footnotesize
\textbf{LOTO status:}\\[2\baselineskip]If LOTO procedures are in use:\\
Set LOTO per (PROCEDURE, attached) at the following locations:

\begin{enumerate}
\tightlist
\item
  LOTO at the Dome
\item
  LOTO of the TMA Drives
\end{enumerate}

}
\hdashrule[0.5ex]{\textwidth}{1pt}{3mm}
  Expected Result \\
{\footnotesize
The appropriate panels have been locked out or released.

}

\begin{tabular}{p{2cm}}
\toprule
Step 4  \\ \hline
\end{tabular}
 Description \\
{\footnotesize
\textbf{Final walkthrough:}\\[2\baselineskip]Perform a final walkthrough
of the dome. Make sure all personnel is cleared out.

}
\hdashrule[0.5ex]{\textwidth}{1pt}{3mm}
  Expected Result \\
{\footnotesize
The dome is clear and safe for TMA movement.\\
The final walkthrough was performed by:

}

\begin{tabular}{p{2cm}}
\toprule
Step 5  \\ \hline
\end{tabular}
 Description \\
{\footnotesize
\textbf{Dome closure:}\\
If the Dome door GIS is available:\\
Exit the Dome, close the door (any details about what specific door)

}
\hdashrule[0.5ex]{\textwidth}{1pt}{3mm}
  Expected Result \\
{\footnotesize
The GIS system is active.

}

\begin{tabular}{p{2cm}}
\toprule
Step 6  \\ \hline
\end{tabular}
 Description \\
{\footnotesize
\textbf{Dome clearance:}\\[2\baselineskip]The Dome clearance is an EIE
task, and they have to sign off.\\
If EIE is not available, perform these steps:

\begin{itemize}
\tightlist
\item
  Make sure that the dome crane is in the parking position. (Hook up)
\item
  Position of the manlifts. Make sure the manlift supports are stored
  and are not on the rotation part of the dome.
\item
  Walkthrough and make sure that there are no obstacles to move.
\end{itemize}

}
\hdashrule[0.5ex]{\textwidth}{1pt}{3mm}
  Example Code \\
{\footnotesize
https://confluence.lsstcorp.org/display/LTS/Dome+Remote+Software+Control+Procedure

}
\hdashrule[0.5ex]{\textwidth}{1pt}{3mm}
  Expected Result \\
{\footnotesize
The Dome is cleared for nightly operations.

}

\begin{tabular}{p{2cm}}
\toprule
Step 7  \\ \hline
\end{tabular}
 Description \\
{\footnotesize
\textbf{PFlow lift}\\
This is part of EIE's safety check.\\
If EIE is not available, perform this step:\\[2\baselineskip]

\begin{itemize}
\tightlist
\item
  The Pflow lift must be stored before moving the dome.
\end{itemize}

}
\hdashrule[0.5ex]{\textwidth}{1pt}{3mm}
  Expected Result \\
{\footnotesize
The PFlow lift is stored properly

}

\begin{tabular}{p{2cm}}
\toprule
Step 8  \\ \hline
\end{tabular}
 Description \\
{\footnotesize
\textbf{Shutter closer}\\
If you have to close the shutter, the Dome must be under LOTO.\\
\textbf{Note:} There is no LOTO available at the moment. Use the
procedure attached to this test case and the information from the
following link:\\
https://confluence.lsstcorp.org/display/LTS/Dome+Remote+Software+Control+Procedure

}
\hdashrule[0.5ex]{\textwidth}{1pt}{3mm}
  Expected Result \\
{\footnotesize
The shutter was closed in a safe way.

}

\begin{tabular}{p{2cm}}
\toprule
Step 9  \\ \hline
\end{tabular}
 Description \\
{\footnotesize
\textbf{GIS activation:}\\
If the GIS for the Dome is available:

\begin{itemize}
\tightlist
\item
  activate the Dome GIS system.
\end{itemize}

}
\hdashrule[0.5ex]{\textwidth}{1pt}{3mm}
  Expected Result \\
{\footnotesize
If possible, the Dome GIS is activated.

}

\begin{tabular}{p{2cm}}
\toprule
Step 10  \\ \hline
\end{tabular}
 Description \\
{\footnotesize
\textbf{Signoff}\\
As a signoff, mark this step as passed

}
\hdashrule[0.5ex]{\textwidth}{1pt}{3mm}
  Expected Result \\
{\footnotesize
Safety Assurance is confirmed to be complete, and testing may proceed.

}

\begin{tabular}{p{2cm}}
\toprule
Step 11  \\ \hline
\end{tabular}
 Description \\
{\footnotesize
\textbf{Night Shift Leader}\\[2\baselineskip]Identify the Night Shift
Leader (first and the second half of the
night).\\[2\baselineskip]\textbf{Note:} This is the person responsible
for deciding when

\begin{itemize}
\tightlist
\item
  the dome is going to be closed.
\item
  to stop observing due to technical issues
\end{itemize}

}
\hdashrule[0.5ex]{\textwidth}{1pt}{3mm}
  Expected Result \\
{\footnotesize
One person is identified as the Night Shift Leader for each shift.

}

\begin{tabular}{p{2cm}}
\toprule
Step 12  \\ \hline
\end{tabular}
 Description \\
{\footnotesize
\textbf{Tailgate Meeting:}\\
Hold a tailgate for the upcoming task with personnel on the summit
working during the night. Go over any relevant procedures, roles, and
responsibilities.\\

\begin{enumerate}
\tightlist
\item
  If the Dome slit doors are not moving automatically, make sure that
  there are three persons with slit closer training available to close
  the slit manually.
\item
  Verify that there are enough persons with driver training available.
\item
  If the StarTracker is going to be used:

  \begin{enumerate}
  \tightlist
  \item
    Clarify who is taking the off the caps in the evening.
  \item
    Take a test image before opening the Dome.
  \item
    Clarify who is installing the caps in the morning.
  \end{enumerate}
\item
  Discuss if surrounding observatories need to be informed. (Necessary
  when light is switched on in the dome during the night.)
\item
  If surrounding observatories need to be informed, clarify who is going
  to inform them and what information should be transmitted.
\item
  Check weather conditions and weather forecasts are within the
  specifications for observations.
\item
  Describe the tasks planned for the night.
\end{enumerate}

}
\hdashrule[0.5ex]{\textwidth}{1pt}{3mm}
  Expected Result \\
{\footnotesize
All involved personnel understands their roles and responsibilities.\\

\begin{itemize}
\tightlist
\item
  If the Dome slit doors are not moving automatically, there are at
  least two persons with slit closer training available to close the
  slit manually.
\item
  There are enough people with driver training available.
\item
  If the StarTracker is used,

  \begin{itemize}
  \tightlist
  \item
    the caps are taken off in the evening by:
  \item
    the test image is taken by:
  \item
    the StarTracker caps are installed in the morning by:
  \end{itemize}
\item
  If surrounding observatories need to be informed,~

  \begin{itemize}
  \tightlist
  \item
    they are informed by:
  \item
    The following information will be transmitted:
  \end{itemize}
\item
  The weather conditions permit us to open the dome and do the planned
  testing.~
\item
  The tasks planned for the night are:
\end{itemize}

}

\begin{tabular}{p{2cm}}
\toprule
Step 13  \\ \hline
\end{tabular}
 Description \\
{\footnotesize
\textbf{Tailgate Meeting -- Part II:}\\
If new personnel is participating in the nightly summit activities:\\

\begin{itemize}
\tightlist
\item
  Clarify that all personnel has PPE.
\item
  Clarify that persons that need to go up into altitude have fall
  protection training.
\item
  Confirm that we have enough personnel to open/close the dome shutter
  if required.
\item
  Remind everybody that the emergency phone numbers are on the control
  room table.
\end{itemize}

}
\hdashrule[0.5ex]{\textwidth}{1pt}{3mm}
  Expected Result \\
{\footnotesize
\begin{itemize}
\tightlist
\item
  All personnel has the required PPE.
\item
  Persons that need to go up into altitude have the fall protection
  training
\item
  Everybody acknowledges that the emergency phone numbers are on the
  control room table.
\end{itemize}

}

\begin{tabular}{p{2cm}}
\toprule
Step 14  \\ \hline
\end{tabular}
 Description \\
{\footnotesize
\textbf{TMA and Dome contact}\\
Person in charge of the TMA interlocks\\
Dome responsible

}
\hdashrule[0.5ex]{\textwidth}{1pt}{3mm}
  Expected Result \\
{\footnotesize
TMA and Dome contacts are known

}

\begin{tabular}{p{2cm}}
\toprule
Step 15  \\ \hline
\end{tabular}
 Description \\
{\footnotesize
\textbf{Radio Communication}\\

\begin{itemize}
\tightlist
\item
  Make sure one radio is switched to channel 1, and the volume is high

  \begin{itemize}
  \tightlist
  \item
    Paramedics, mountain assistants (in replacement of the paramedics),
    guards, and surrounding observatories are listening to this channel.
  \end{itemize}
\item
  Make sure one radio is switched to channel 3, and the volume is high

  \begin{itemize}
  \tightlist
  \item
    Rubin's internal coordination channel
  \end{itemize}
\end{itemize}

}
\hdashrule[0.5ex]{\textwidth}{1pt}{3mm}
  Expected Result \\
{\footnotesize
The radios are switched on and on high volume.

}

\begin{tabular}{p{2cm}}
\toprule
Step 16  \\ \hline
\end{tabular}
 Description \\
{\footnotesize
\textbf{Cars}

\begin{itemize}
\tightlist
\item
  Make sure enough cars are available to go to the hotel.
\item
  Make sure the keys for the cars are available.
\end{itemize}

}
\hdashrule[0.5ex]{\textwidth}{1pt}{3mm}
  Expected Result \\
{\footnotesize
Sufficient cars and their keys are available.

}

\begin{tabular}{p{2cm}}
\toprule
Step 17  \\ \hline
\end{tabular}
 Description \\
{\footnotesize
\textbf{ComCam safety}\\[2\baselineskip]Put ComCam in a safe state for
moving. This includes:

\begin{enumerate}
\tightlist
\item
  CryoTels are under observation for vibrations (i.e. a microphone or
  webcam is observing them and operating correctly)
\item
  Turbo pumps are off
\end{enumerate}

}
\hdashrule[0.5ex]{\textwidth}{1pt}{3mm}
  Expected Result \\
{\footnotesize
ComCam is in a safe state for TMA movement.

}

\begin{tabular}{p{2cm}}
\toprule
Step 18  \\ \hline
\end{tabular}
 Description \\
{\footnotesize
\textbf{TMA moving space}\\[2\baselineskip]Go to the dome and visually
verify that there is unrestricted space for the TMA movement.

}
\hdashrule[0.5ex]{\textwidth}{1pt}{3mm}
  Expected Result \\
{\footnotesize
The space is clear and no objects will be struck when the TMA moves.

}

\paragraph{ LVV-T2714 - Configure Observatory Environment for Nighttime Operations }\mbox{}\\

Version \textbf{1}.
Open  \href{https://jira.lsstcorp.org/secure/Tests.jspa#/testCase/LVV-T2714}{\textit{ LVV-T2714 } }
test case in Jira.

At the beginning of the night, prepare the observatory for nightly
operations.

\textbf{ Preconditions}:\\
Dome and the TMA, or at least the TMA must available for observations.

Final comment:\\


Detailed steps :

\begin{tabular}{p{2cm}}
\toprule
Step 1  \\ \hline
\end{tabular}
 Description \\
{\footnotesize
\textbf{Telescope preparation:}

\begin{itemize}
\tightlist
\item
  Remove the caps on top of the StarTracker telescopes and the DIMM.
\item
  Check the instrument's health status by taking a test image.
\end{itemize}

}
\hdashrule[0.5ex]{\textwidth}{1pt}{3mm}
  Expected Result \\
{\footnotesize
The caps are removed.\\
The test image is taken and stored correspondingly.

}

\begin{tabular}{p{2cm}}
\toprule
Step 2  \\ \hline
\end{tabular}
 Description \\
{\footnotesize
\textbf{Calibration images:}\\
Not needed at the moment, to be included if data analysis reveals the
need.

\begin{itemize}
\tightlist
\item
  Take 10 ``darks'' with the StarTracker and the DIMM instruments. Use
  the same exposure time as for the images.
\item
  Take 10 ``sky flats'' with the StarTracker and the DIMM instruments.
\end{itemize}

}
\hdashrule[0.5ex]{\textwidth}{1pt}{3mm}
  Expected Result \\
{\footnotesize
The Darks and flats are stored in the expected location.

}

\begin{tabular}{p{2cm}}
\toprule
Step 3  \\ \hline
\end{tabular}
 Description \\
{\footnotesize
\textbf{Auxillary systems~nighttime~preparations:}

\begin{itemize}
\tightlist
\item
  Switch off the UMA in the afternoon.
\end{itemize}

}
\hdashrule[0.5ex]{\textwidth}{1pt}{3mm}
  Expected Result \\
{\footnotesize
The UMA is switched off.

}

\begin{tabular}{p{2cm}}
\toprule
Step 4  \\ \hline
\end{tabular}
 Description \\
{\footnotesize
\textbf{Night logging page:}\\
Start the night log similar to the AuxTel night
log:\\[2\baselineskip]https://confluence.lsstcorp.org/display/LSSTCOM/Night+Logs\\[2\baselineskip]

}
\hdashrule[0.5ex]{\textwidth}{1pt}{3mm}
  Expected Result \\
{\footnotesize
Page created with template information.

}

\begin{tabular}{p{2cm}}
\toprule
Step 5  \\ \hline
\end{tabular}
 Description \\
{\footnotesize
\textbf{TMA preparation}

\begin{itemize}
\tightlist
\item
  Check the Oil Supply System (OSS) on the EUI
\item
  Follow the attached manual to startup the TMA.
\end{itemize}

}
\hdashrule[0.5ex]{\textwidth}{1pt}{3mm}
  Expected Result \\
{\footnotesize
The OSS is operational:

}

\begin{tabular}{p{2cm}}
\toprule
Step 6  \\ \hline
\end{tabular}
 Description \\
{\footnotesize
\textbf{CSC activation:}\\[2\baselineskip]Use L.O.V.E to bring the CSC
to the enabled state.\\[2\baselineskip]

}
\hdashrule[0.5ex]{\textwidth}{1pt}{3mm}
  Expected Result \\
{\footnotesize
All needed CSCs are in the enabled state.

}

\paragraph{ LVV-T2731 - StarTracker Pointing and Tracking Test - Reverse Azimuth Pattern }\mbox{}\\

Version \textbf{1}.
Open  \href{https://jira.lsstcorp.org/secure/Tests.jspa#/testCase/LVV-T2731}{\textit{ LVV-T2731 } }
test case in Jira.

Collect data with the StarTracker following the reverse azimuth pattern
with respect to the forward Azimuth pattern
(https://jira.lsstcorp.org/secure/Tests.jspa\#/testCase/LVV-T2731)\\
The azimuth here is pattern 270, 180, 90, 0, -90, -180, -270 deg.
Nominal at four elevation angles 15, 45, 70, 85 deg. Minimum at the
three angles: 15, 45, 85 deg.\\[2\baselineskip]This test

\begin{itemize}
\tightlist
\item
  is foreseen the second of four tests
\item
  takes about one-half summer night
\end{itemize}

\textbf{ Preconditions}:\\
The safety test case and nightly operations test case have been
executed.

Final comment:\\


Detailed steps :

\begin{tabular}{p{2cm}}
\toprule
Step 1  \\ \hline
\end{tabular}
 Description \\
{\footnotesize
\textbf{Point} \textbf{the Dome:}\\
Command the Dome to {Pointing 1}⁠ to {270}⁠

}
\hdashrule[0.5ex]{\textwidth}{1pt}{3mm}
  Expected Result \\
{\footnotesize
The Dome starts moving.

}

\begin{tabular}{p{2cm}}
\toprule
Step 2  \\ \hline
\end{tabular}
 Description \\
{\footnotesize
\textbf{Point} \textbf{the Dome:}\\
Command the Dome to {Pointing 13}⁠ to {0}⁠

}
\hdashrule[0.5ex]{\textwidth}{1pt}{3mm}
  Expected Result \\
{\footnotesize
The Dome starts moving.

}

\begin{tabular}{p{2cm}}
\toprule
Step 3  \\ \hline
\end{tabular}
 Description \\
{\footnotesize
Wait for the Dome to reach the commanded position.

}
\hdashrule[0.5ex]{\textwidth}{1pt}{3mm}
  Expected Result \\
{\footnotesize
The \emph{MTDome\_logevent\_azMotion} and
\emph{MTDome\_logevent\_elMotion} inPosition parameter = true.

}

\begin{tabular}{p{2cm}}
\toprule
Step 4  \\ \hline
\end{tabular}
 Description \\
{\footnotesize
\textbf{Point the TMA}\\
Command the TMA to {Pointing 13}⁠ at {0}⁠ , {85}⁠ .

}
\hdashrule[0.5ex]{\textwidth}{1pt}{3mm}
  Expected Result \\
{\footnotesize
The TMA starts moving

}

\begin{tabular}{p{2cm}}
\toprule
Step 5  \\ \hline
\end{tabular}
 Description \\
{\footnotesize
Wait for the TMA to reach the commanded position.

}
\hdashrule[0.5ex]{\textwidth}{1pt}{3mm}
  Expected Result \\
{\footnotesize
The \emph{MTMount\_logevent\_azimuthInPosition} and
\emph{MTMount\_logevent\_elevationInPosition} inPosition parameter =
true.

}

\begin{tabular}{p{2cm}}
\toprule
Step 6  \\ \hline
\end{tabular}
 Description \\
{\footnotesize
\textbf{Image preparation}\\
If the preparation to take images takes longer than 10sec, do
repositioning to target {{{0}⁠}}, {{{85}⁠~}}.

}
\hdashrule[0.5ex]{\textwidth}{1pt}{3mm}
  Expected Result \\
{\footnotesize
TMA reaches the commanded position.

}

\begin{tabular}{p{2cm}}
\toprule
Step 7  \\ \hline
\end{tabular}
 Description \\
{\footnotesize
\textbf{Track position and take images}\\[2\baselineskip]Take a
StarTracker image with 10s exposure time.\\[2\baselineskip]If the time
the available:

\begin{itemize}
\tightlist
\item
  Track a position for 10 min and take StarTracker images.
\end{itemize}

}
\hdashrule[0.5ex]{\textwidth}{1pt}{3mm}
  Expected Result \\
{\footnotesize
\begin{itemize}
\tightlist
\item
  If time is available: The TMA is tracking a given position for 10 min
  and taking images.
\item
  At least one image is successfully taken with the StarTracker.
\end{itemize}

}

\begin{tabular}{p{2cm}}
\toprule
Step 8  \\ \hline
\end{tabular}
 Description \\
{\footnotesize
\textbf{On-the-fly Image Quality Check}\\
While tracking and taking images, check the images on RubinTV for an
astrometric solution.

}
\hdashrule[0.5ex]{\textwidth}{1pt}{3mm}
  Expected Result \\
{\footnotesize
RubinTV is showing an astrometric solution.

}

\begin{tabular}{p{2cm}}
\toprule
Step 9  \\ \hline
\end{tabular}
 Description \\
{\footnotesize
\textbf{Offline analysis results}\\
Offline analysis in Test case
\href{https://jira.lsstcorp.org/secure/Tests.jspa\#/testCase/LVV-T2739}{LVV-T2739}
Says that we do not have sufficient image quality.

}
\hdashrule[0.5ex]{\textwidth}{1pt}{3mm}
  Expected Result \\
{\footnotesize
Image quality is sufficient.

}

\begin{tabular}{p{2cm}}
\toprule
Step 10  \\ \hline
\end{tabular}
 Description \\
{\footnotesize
\textbf{Point} \textbf{the Dome:}\\
Command the Dome to {Pointing 15}⁠ to {0}⁠

}
\hdashrule[0.5ex]{\textwidth}{1pt}{3mm}
  Expected Result \\
{\footnotesize
The Dome starts moving.

}

\begin{tabular}{p{2cm}}
\toprule
Step 11  \\ \hline
\end{tabular}
 Description \\
{\footnotesize
Wait for the Dome to reach the commanded position.

}
\hdashrule[0.5ex]{\textwidth}{1pt}{3mm}
  Expected Result \\
{\footnotesize
The \emph{MTDome\_logevent\_azMotion} and
\emph{MTDome\_logevent\_elMotion} inPosition parameter = true.

}

\begin{tabular}{p{2cm}}
\toprule
Step 12  \\ \hline
\end{tabular}
 Description \\
{\footnotesize
\textbf{Point the TMA}\\
Command the TMA to {Pointing 15}⁠ at {0}⁠ , {45}⁠ .

}
\hdashrule[0.5ex]{\textwidth}{1pt}{3mm}
  Expected Result \\
{\footnotesize
The TMA starts moving

}

\begin{tabular}{p{2cm}}
\toprule
Step 13  \\ \hline
\end{tabular}
 Description \\
{\footnotesize
Wait for the TMA to reach the commanded position.

}
\hdashrule[0.5ex]{\textwidth}{1pt}{3mm}
  Expected Result \\
{\footnotesize
The \emph{MTMount\_logevent\_azimuthInPosition} and
\emph{MTMount\_logevent\_elevationInPosition} inPosition parameter =
true.

}

\begin{tabular}{p{2cm}}
\toprule
Step 14  \\ \hline
\end{tabular}
 Description \\
{\footnotesize
\textbf{Image preparation}\\
If the preparation to take images takes longer than 10sec, do
repositioning to target {{{0}⁠}}, {{{45}⁠~}}.

}
\hdashrule[0.5ex]{\textwidth}{1pt}{3mm}
  Expected Result \\
{\footnotesize
TMA reaches the commanded position.

}

\begin{tabular}{p{2cm}}
\toprule
Step 15  \\ \hline
\end{tabular}
 Description \\
{\footnotesize
\textbf{Track position and take images}\\[2\baselineskip]Take a
StarTracker image with 10s exposure time.\\[2\baselineskip]If the time
the available:

\begin{itemize}
\tightlist
\item
  Track a position for 10 min and take StarTracker images.
\end{itemize}

}
\hdashrule[0.5ex]{\textwidth}{1pt}{3mm}
  Expected Result \\
{\footnotesize
\begin{itemize}
\tightlist
\item
  If time is available: The TMA is tracking a given position for 10 min
  and taking images.
\item
  At least one image is successfully taken with the StarTracker.
\end{itemize}

}

\begin{tabular}{p{2cm}}
\toprule
Step 16  \\ \hline
\end{tabular}
 Description \\
{\footnotesize
\textbf{On-the-fly Image Quality Check}\\
While tracking and taking images, check the images on RubinTV for an
astrometric solution.

}
\hdashrule[0.5ex]{\textwidth}{1pt}{3mm}
  Expected Result \\
{\footnotesize
RubinTV is showing an astrometric solution.

}

\begin{tabular}{p{2cm}}
\toprule
Step 17  \\ \hline
\end{tabular}
 Description \\
{\footnotesize
\textbf{Offline analysis results}\\
Offline analysis in Test case
\href{https://jira.lsstcorp.org/secure/Tests.jspa\#/testCase/LVV-T2739}{LVV-T2739}
Says that we do not have sufficient image quality.

}
\hdashrule[0.5ex]{\textwidth}{1pt}{3mm}
  Expected Result \\
{\footnotesize
Image quality is sufficient.

}

\begin{tabular}{p{2cm}}
\toprule
Step 18  \\ \hline
\end{tabular}
 Description \\
{\footnotesize
\textbf{Point} \textbf{the Dome:}\\
Command the Dome to {Pointing 16}⁠ to {0}⁠

}
\hdashrule[0.5ex]{\textwidth}{1pt}{3mm}
  Expected Result \\
{\footnotesize
The Dome starts moving.

}

\begin{tabular}{p{2cm}}
\toprule
Step 19  \\ \hline
\end{tabular}
 Description \\
{\footnotesize
Wait for the Dome to reach the commanded position.

}
\hdashrule[0.5ex]{\textwidth}{1pt}{3mm}
  Expected Result \\
{\footnotesize
The \emph{MTDome\_logevent\_azMotion} and
\emph{MTDome\_logevent\_elMotion} inPosition parameter = true.

}

\begin{tabular}{p{2cm}}
\toprule
Step 20  \\ \hline
\end{tabular}
 Description \\
{\footnotesize
\textbf{Point the TMA}\\
Command the TMA to {Pointing 16}⁠ at {0}⁠ , {15}⁠ .

}
\hdashrule[0.5ex]{\textwidth}{1pt}{3mm}
  Expected Result \\
{\footnotesize
The TMA starts moving

}

\begin{tabular}{p{2cm}}
\toprule
Step 21  \\ \hline
\end{tabular}
 Description \\
{\footnotesize
Wait for the TMA to reach the commanded position.

}
\hdashrule[0.5ex]{\textwidth}{1pt}{3mm}
  Expected Result \\
{\footnotesize
The \emph{MTMount\_logevent\_azimuthInPosition} and
\emph{MTMount\_logevent\_elevationInPosition} inPosition parameter =
true.

}

\begin{tabular}{p{2cm}}
\toprule
Step 22  \\ \hline
\end{tabular}
 Description \\
{\footnotesize
\textbf{Image preparation}\\
If the preparation to take images takes longer than 10sec, do
repositioning to target {{{0}⁠}}, {{{15}⁠~}}.

}
\hdashrule[0.5ex]{\textwidth}{1pt}{3mm}
  Expected Result \\
{\footnotesize
TMA reaches the commanded position.

}

\begin{tabular}{p{2cm}}
\toprule
Step 23  \\ \hline
\end{tabular}
 Description \\
{\footnotesize
\textbf{Track position and take images}\\[2\baselineskip]Take a
StarTracker image with 10s exposure time.\\[2\baselineskip]If the time
the available:

\begin{itemize}
\tightlist
\item
  Track a position for 10 min and take StarTracker images.
\end{itemize}

}
\hdashrule[0.5ex]{\textwidth}{1pt}{3mm}
  Expected Result \\
{\footnotesize
\begin{itemize}
\tightlist
\item
  If time is available: The TMA is tracking a given position for 10 min
  and taking images.
\item
  At least one image is successfully taken with the StarTracker.
\end{itemize}

}

\begin{tabular}{p{2cm}}
\toprule
Step 24  \\ \hline
\end{tabular}
 Description \\
{\footnotesize
\textbf{On-the-fly Image Quality Check}\\
While tracking and taking images, check the images on RubinTV for an
astrometric solution.

}
\hdashrule[0.5ex]{\textwidth}{1pt}{3mm}
  Expected Result \\
{\footnotesize
RubinTV is showing an astrometric solution.

}

\begin{tabular}{p{2cm}}
\toprule
Step 25  \\ \hline
\end{tabular}
 Description \\
{\footnotesize
\textbf{Offline analysis results}\\
Offline analysis in Test case
\href{https://jira.lsstcorp.org/secure/Tests.jspa\#/testCase/LVV-T2739}{LVV-T2739}
Says that we do not have sufficient image quality.

}
\hdashrule[0.5ex]{\textwidth}{1pt}{3mm}
  Expected Result \\
{\footnotesize
Image quality is sufficient.

}

\begin{tabular}{p{2cm}}
\toprule
Step 26  \\ \hline
\end{tabular}
 Description \\
{\footnotesize
\textbf{Point} \textbf{the Dome:}\\
Command the Dome to {Pointing 17}⁠ to {-90}⁠

}
\hdashrule[0.5ex]{\textwidth}{1pt}{3mm}
  Expected Result \\
{\footnotesize
The Dome starts moving.

}

\begin{tabular}{p{2cm}}
\toprule
Step 27  \\ \hline
\end{tabular}
 Description \\
{\footnotesize
Wait for the Dome to reach the commanded position.

}
\hdashrule[0.5ex]{\textwidth}{1pt}{3mm}
  Expected Result \\
{\footnotesize
The \emph{MTDome\_logevent\_azMotion} and
\emph{MTDome\_logevent\_elMotion} inPosition parameter = true.

}

\begin{tabular}{p{2cm}}
\toprule
Step 28  \\ \hline
\end{tabular}
 Description \\
{\footnotesize
\textbf{Point the TMA}\\
Command the TMA to {Pointing 17}⁠ at {-90}⁠ , {15}⁠ .

}
\hdashrule[0.5ex]{\textwidth}{1pt}{3mm}
  Expected Result \\
{\footnotesize
The TMA starts moving

}

\begin{tabular}{p{2cm}}
\toprule
Step 29  \\ \hline
\end{tabular}
 Description \\
{\footnotesize
Wait for the TMA to reach the commanded position.

}
\hdashrule[0.5ex]{\textwidth}{1pt}{3mm}
  Expected Result \\
{\footnotesize
The \emph{MTMount\_logevent\_azimuthInPosition} and
\emph{MTMount\_logevent\_elevationInPosition} inPosition parameter =
true.

}

\begin{tabular}{p{2cm}}
\toprule
Step 30  \\ \hline
\end{tabular}
 Description \\
{\footnotesize
\textbf{Image preparation}\\
If the preparation to take images takes longer than 10sec, do
repositioning to target {{{-90}⁠}}, {{{15}⁠~}}.

}
\hdashrule[0.5ex]{\textwidth}{1pt}{3mm}
  Expected Result \\
{\footnotesize
TMA reaches the commanded position.

}

\begin{tabular}{p{2cm}}
\toprule
Step 31  \\ \hline
\end{tabular}
 Description \\
{\footnotesize
\textbf{Track position and take images}\\[2\baselineskip]Take a
StarTracker image with 10s exposure time.\\[2\baselineskip]If the time
the available:

\begin{itemize}
\tightlist
\item
  Track a position for 10 min and take StarTracker images.
\end{itemize}

}
\hdashrule[0.5ex]{\textwidth}{1pt}{3mm}
  Expected Result \\
{\footnotesize
\begin{itemize}
\tightlist
\item
  If time is available: The TMA is tracking a given position for 10 min
  and taking images.
\item
  At least one image is successfully taken with the StarTracker.
\end{itemize}

}

\begin{tabular}{p{2cm}}
\toprule
Step 32  \\ \hline
\end{tabular}
 Description \\
{\footnotesize
\textbf{On-the-fly Image Quality Check}\\
While tracking and taking images, check the images on RubinTV for an
astrometric solution.

}
\hdashrule[0.5ex]{\textwidth}{1pt}{3mm}
  Expected Result \\
{\footnotesize
RubinTV is showing an astrometric solution.

}

\begin{tabular}{p{2cm}}
\toprule
Step 33  \\ \hline
\end{tabular}
 Description \\
{\footnotesize
\textbf{Offline analysis results}\\
Offline analysis in Test case
\href{https://jira.lsstcorp.org/secure/Tests.jspa\#/testCase/LVV-T2739}{LVV-T2739}
Says that we do not have sufficient image quality.

}
\hdashrule[0.5ex]{\textwidth}{1pt}{3mm}
  Expected Result \\
{\footnotesize
Image quality is sufficient.

}

\begin{tabular}{p{2cm}}
\toprule
Step 34  \\ \hline
\end{tabular}
 Description \\
{\footnotesize
\textbf{Point} \textbf{the Dome:}\\
Command the Dome to {Pointing 18}⁠ to {-90}⁠

}
\hdashrule[0.5ex]{\textwidth}{1pt}{3mm}
  Expected Result \\
{\footnotesize
The Dome starts moving.

}

\begin{tabular}{p{2cm}}
\toprule
Step 35  \\ \hline
\end{tabular}
 Description \\
{\footnotesize
Wait for the Dome to reach the commanded position.

}
\hdashrule[0.5ex]{\textwidth}{1pt}{3mm}
  Expected Result \\
{\footnotesize
The \emph{MTDome\_logevent\_azMotion} and
\emph{MTDome\_logevent\_elMotion} inPosition parameter = true.

}

\begin{tabular}{p{2cm}}
\toprule
Step 36  \\ \hline
\end{tabular}
 Description \\
{\footnotesize
\textbf{Point the TMA}\\
Command the TMA to {Pointing 18}⁠ at {-90}⁠ , {45}⁠ .

}
\hdashrule[0.5ex]{\textwidth}{1pt}{3mm}
  Expected Result \\
{\footnotesize
The TMA starts moving

}

\begin{tabular}{p{2cm}}
\toprule
Step 37  \\ \hline
\end{tabular}
 Description \\
{\footnotesize
Wait for the TMA to reach the commanded position.

}
\hdashrule[0.5ex]{\textwidth}{1pt}{3mm}
  Expected Result \\
{\footnotesize
The \emph{MTMount\_logevent\_azimuthInPosition} and
\emph{MTMount\_logevent\_elevationInPosition} inPosition parameter =
true.

}

\begin{tabular}{p{2cm}}
\toprule
Step 38  \\ \hline
\end{tabular}
 Description \\
{\footnotesize
\textbf{Image preparation}\\
If the preparation to take images takes longer than 10sec, do
repositioning to target {{{-90}⁠}}, {{{45}⁠~}}.

}
\hdashrule[0.5ex]{\textwidth}{1pt}{3mm}
  Expected Result \\
{\footnotesize
TMA reaches the commanded position.

}

\begin{tabular}{p{2cm}}
\toprule
Step 39  \\ \hline
\end{tabular}
 Description \\
{\footnotesize
\textbf{Track position and take images}\\[2\baselineskip]Take a
StarTracker image with 10s exposure time.\\[2\baselineskip]If the time
the available:

\begin{itemize}
\tightlist
\item
  Track a position for 10 min and take StarTracker images.
\end{itemize}

}
\hdashrule[0.5ex]{\textwidth}{1pt}{3mm}
  Expected Result \\
{\footnotesize
\begin{itemize}
\tightlist
\item
  If time is available: The TMA is tracking a given position for 10 min
  and taking images.
\item
  At least one image is successfully taken with the StarTracker.
\end{itemize}

}

\begin{tabular}{p{2cm}}
\toprule
Step 40  \\ \hline
\end{tabular}
 Description \\
{\footnotesize
\textbf{On-the-fly Image Quality Check}\\
While tracking and taking images, check the images on RubinTV for an
astrometric solution.

}
\hdashrule[0.5ex]{\textwidth}{1pt}{3mm}
  Expected Result \\
{\footnotesize
RubinTV is showing an astrometric solution.

}

\begin{tabular}{p{2cm}}
\toprule
Step 41  \\ \hline
\end{tabular}
 Description \\
{\footnotesize
\textbf{Offline analysis results}\\
Offline analysis in Test case
\href{https://jira.lsstcorp.org/secure/Tests.jspa\#/testCase/LVV-T2739}{LVV-T2739}
Says that we do not have sufficient image quality.

}
\hdashrule[0.5ex]{\textwidth}{1pt}{3mm}
  Expected Result \\
{\footnotesize
Image quality is sufficient.

}

\begin{tabular}{p{2cm}}
\toprule
Step 42  \\ \hline
\end{tabular}
 Description \\
{\footnotesize
\textbf{Point} \textbf{the Dome:}\\
Command the Dome to {Pointing 20}⁠ to {-90}⁠

}
\hdashrule[0.5ex]{\textwidth}{1pt}{3mm}
  Expected Result \\
{\footnotesize
The Dome starts moving.

}

\begin{tabular}{p{2cm}}
\toprule
Step 43  \\ \hline
\end{tabular}
 Description \\
{\footnotesize
Wait for the Dome to reach the commanded position.

}
\hdashrule[0.5ex]{\textwidth}{1pt}{3mm}
  Expected Result \\
{\footnotesize
The \emph{MTDome\_logevent\_azMotion} and
\emph{MTDome\_logevent\_elMotion} inPosition parameter = true.

}

\begin{tabular}{p{2cm}}
\toprule
Step 44  \\ \hline
\end{tabular}
 Description \\
{\footnotesize
\textbf{Point the TMA}\\
Command the TMA to {Pointing 20}⁠ at {-90}⁠ , {85}⁠ .

}
\hdashrule[0.5ex]{\textwidth}{1pt}{3mm}
  Expected Result \\
{\footnotesize
The TMA starts moving

}

\begin{tabular}{p{2cm}}
\toprule
Step 45  \\ \hline
\end{tabular}
 Description \\
{\footnotesize
Wait for the TMA to reach the commanded position.

}
\hdashrule[0.5ex]{\textwidth}{1pt}{3mm}
  Expected Result \\
{\footnotesize
The \emph{MTMount\_logevent\_azimuthInPosition} and
\emph{MTMount\_logevent\_elevationInPosition} inPosition parameter =
true.

}

\begin{tabular}{p{2cm}}
\toprule
Step 46  \\ \hline
\end{tabular}
 Description \\
{\footnotesize
\textbf{Image preparation}\\
If the preparation to take images takes longer than 10sec, do
repositioning to target {{{-90}⁠}}, {{{85}⁠~}}.

}
\hdashrule[0.5ex]{\textwidth}{1pt}{3mm}
  Expected Result \\
{\footnotesize
TMA reaches the commanded position.

}

\begin{tabular}{p{2cm}}
\toprule
Step 47  \\ \hline
\end{tabular}
 Description \\
{\footnotesize
\textbf{Track position and take images}\\[2\baselineskip]Take a
StarTracker image with 10s exposure time.\\[2\baselineskip]If the time
the available:

\begin{itemize}
\tightlist
\item
  Track a position for 10 min and take StarTracker images.
\end{itemize}

}
\hdashrule[0.5ex]{\textwidth}{1pt}{3mm}
  Expected Result \\
{\footnotesize
\begin{itemize}
\tightlist
\item
  If time is available: The TMA is tracking a given position for 10 min
  and taking images.
\item
  At least one image is successfully taken with the StarTracker.
\end{itemize}

}

\begin{tabular}{p{2cm}}
\toprule
Step 48  \\ \hline
\end{tabular}
 Description \\
{\footnotesize
\textbf{On-the-fly Image Quality Check}\\
While tracking and taking images, check the images on RubinTV for an
astrometric solution.

}
\hdashrule[0.5ex]{\textwidth}{1pt}{3mm}
  Expected Result \\
{\footnotesize
RubinTV is showing an astrometric solution.

}

\begin{tabular}{p{2cm}}
\toprule
Step 49  \\ \hline
\end{tabular}
 Description \\
{\footnotesize
\textbf{Offline analysis results}\\
Offline analysis in Test case
\href{https://jira.lsstcorp.org/secure/Tests.jspa\#/testCase/LVV-T2739}{LVV-T2739}
Says that we do not have sufficient image quality.

}
\hdashrule[0.5ex]{\textwidth}{1pt}{3mm}
  Expected Result \\
{\footnotesize
Image quality is sufficient.

}

\begin{tabular}{p{2cm}}
\toprule
Step 50  \\ \hline
\end{tabular}
 Description \\
{\footnotesize
\textbf{Point} \textbf{the Dome:}\\
Command the Dome to {Pointing 21}⁠ to {180}⁠

}
\hdashrule[0.5ex]{\textwidth}{1pt}{3mm}
  Expected Result \\
{\footnotesize
The Dome starts moving.

}

\begin{tabular}{p{2cm}}
\toprule
Step 51  \\ \hline
\end{tabular}
 Description \\
{\footnotesize
Wait for the Dome to reach the commanded position.

}
\hdashrule[0.5ex]{\textwidth}{1pt}{3mm}
  Expected Result \\
{\footnotesize
The \emph{MTDome\_logevent\_azMotion} and
\emph{MTDome\_logevent\_elMotion} inPosition parameter = true.

}

\begin{tabular}{p{2cm}}
\toprule
Step 52  \\ \hline
\end{tabular}
 Description \\
{\footnotesize
\textbf{Point the TMA}\\
Command the TMA to {Pointing 21}⁠ at {-180}⁠ , {85}⁠ .

}
\hdashrule[0.5ex]{\textwidth}{1pt}{3mm}
  Expected Result \\
{\footnotesize
The TMA starts moving

}

\begin{tabular}{p{2cm}}
\toprule
Step 53  \\ \hline
\end{tabular}
 Description \\
{\footnotesize
Wait for the TMA to reach the commanded position.

}
\hdashrule[0.5ex]{\textwidth}{1pt}{3mm}
  Expected Result \\
{\footnotesize
The \emph{MTMount\_logevent\_azimuthInPosition} and
\emph{MTMount\_logevent\_elevationInPosition} inPosition parameter =
true.

}

\begin{tabular}{p{2cm}}
\toprule
Step 54  \\ \hline
\end{tabular}
 Description \\
{\footnotesize
\textbf{Image preparation}\\
If the preparation to take images takes longer than 10sec, do
repositioning to target {{{-180}⁠}}, {{{85}⁠~}}.

}
\hdashrule[0.5ex]{\textwidth}{1pt}{3mm}
  Expected Result \\
{\footnotesize
TMA reaches the commanded position.

}

\begin{tabular}{p{2cm}}
\toprule
Step 55  \\ \hline
\end{tabular}
 Description \\
{\footnotesize
\textbf{Track position and take images}\\[2\baselineskip]Take a
StarTracker image with 10s exposure time.\\[2\baselineskip]If the time
the available:

\begin{itemize}
\tightlist
\item
  Track a position for 10 min and take StarTracker images.
\end{itemize}

}
\hdashrule[0.5ex]{\textwidth}{1pt}{3mm}
  Expected Result \\
{\footnotesize
\begin{itemize}
\tightlist
\item
  If time is available: The TMA is tracking a given position for 10 min
  and taking images.
\item
  At least one image is successfully taken with the StarTracker.
\end{itemize}

}

\begin{tabular}{p{2cm}}
\toprule
Step 56  \\ \hline
\end{tabular}
 Description \\
{\footnotesize
\textbf{On-the-fly Image Quality Check}\\
While tracking and taking images, check the images on RubinTV for an
astrometric solution.

}
\hdashrule[0.5ex]{\textwidth}{1pt}{3mm}
  Expected Result \\
{\footnotesize
RubinTV is showing an astrometric solution.

}

\begin{tabular}{p{2cm}}
\toprule
Step 57  \\ \hline
\end{tabular}
 Description \\
{\footnotesize
\textbf{Offline analysis results}\\
Offline analysis in Test case
\href{https://jira.lsstcorp.org/secure/Tests.jspa\#/testCase/LVV-T2739}{LVV-T2739}
Says that we do not have sufficient image quality.

}
\hdashrule[0.5ex]{\textwidth}{1pt}{3mm}
  Expected Result \\
{\footnotesize
Image quality is sufficient.

}

\begin{tabular}{p{2cm}}
\toprule
Step 58  \\ \hline
\end{tabular}
 Description \\
{\footnotesize
\textbf{Point} \textbf{the Dome:}\\
Command the Dome to {Pointing 23}⁠ to {180}⁠

}
\hdashrule[0.5ex]{\textwidth}{1pt}{3mm}
  Expected Result \\
{\footnotesize
The Dome starts moving.

}

\begin{tabular}{p{2cm}}
\toprule
Step 59  \\ \hline
\end{tabular}
 Description \\
{\footnotesize
Wait for the Dome to reach the commanded position.

}
\hdashrule[0.5ex]{\textwidth}{1pt}{3mm}
  Expected Result \\
{\footnotesize
The \emph{MTDome\_logevent\_azMotion} and
\emph{MTDome\_logevent\_elMotion} inPosition parameter = true.

}

\begin{tabular}{p{2cm}}
\toprule
Step 60  \\ \hline
\end{tabular}
 Description \\
{\footnotesize
\textbf{Point the TMA}\\
Command the TMA to {Pointing 23}⁠ at {-180}⁠ , {45}⁠ .

}
\hdashrule[0.5ex]{\textwidth}{1pt}{3mm}
  Expected Result \\
{\footnotesize
The TMA starts moving

}

\begin{tabular}{p{2cm}}
\toprule
Step 61  \\ \hline
\end{tabular}
 Description \\
{\footnotesize
Wait for the TMA to reach the commanded position.

}
\hdashrule[0.5ex]{\textwidth}{1pt}{3mm}
  Expected Result \\
{\footnotesize
The \emph{MTMount\_logevent\_azimuthInPosition} and
\emph{MTMount\_logevent\_elevationInPosition} inPosition parameter =
true.

}

\begin{tabular}{p{2cm}}
\toprule
Step 62  \\ \hline
\end{tabular}
 Description \\
{\footnotesize
\textbf{Image preparation}\\
If the preparation to take images takes longer than 10sec, do
repositioning to target {{{-180}⁠}}, {{{45}⁠~}}.

}
\hdashrule[0.5ex]{\textwidth}{1pt}{3mm}
  Expected Result \\
{\footnotesize
TMA reaches the commanded position.

}

\begin{tabular}{p{2cm}}
\toprule
Step 63  \\ \hline
\end{tabular}
 Description \\
{\footnotesize
\textbf{Track position and take images}\\[2\baselineskip]Take a
StarTracker image with 10s exposure time.\\[2\baselineskip]If the time
the available:

\begin{itemize}
\tightlist
\item
  Track a position for 10 min and take StarTracker images.
\end{itemize}

}
\hdashrule[0.5ex]{\textwidth}{1pt}{3mm}
  Expected Result \\
{\footnotesize
\begin{itemize}
\tightlist
\item
  If time is available: The TMA is tracking a given position for 10 min
  and taking images.
\item
  At least one image is successfully taken with the StarTracker.
\end{itemize}

}

\begin{tabular}{p{2cm}}
\toprule
Step 64  \\ \hline
\end{tabular}
 Description \\
{\footnotesize
\textbf{On-the-fly Image Quality Check}\\
While tracking and taking images, check the images on RubinTV for an
astrometric solution.

}
\hdashrule[0.5ex]{\textwidth}{1pt}{3mm}
  Expected Result \\
{\footnotesize
RubinTV is showing an astrometric solution.

}

\begin{tabular}{p{2cm}}
\toprule
Step 65  \\ \hline
\end{tabular}
 Description \\
{\footnotesize
\textbf{Offline analysis results}\\
Offline analysis in Test case
\href{https://jira.lsstcorp.org/secure/Tests.jspa\#/testCase/LVV-T2739}{LVV-T2739}
Says that we do not have sufficient image quality.

}
\hdashrule[0.5ex]{\textwidth}{1pt}{3mm}
  Expected Result \\
{\footnotesize
Image quality is sufficient.

}

\begin{tabular}{p{2cm}}
\toprule
Step 66  \\ \hline
\end{tabular}
 Description \\
{\footnotesize
\textbf{Point} \textbf{the Dome:}\\
Command the Dome to {Pointing 24}⁠ to {180}⁠

}
\hdashrule[0.5ex]{\textwidth}{1pt}{3mm}
  Expected Result \\
{\footnotesize
The Dome starts moving.

}

\begin{tabular}{p{2cm}}
\toprule
Step 67  \\ \hline
\end{tabular}
 Description \\
{\footnotesize
Wait for the Dome to reach the commanded position.

}
\hdashrule[0.5ex]{\textwidth}{1pt}{3mm}
  Expected Result \\
{\footnotesize
The \emph{MTDome\_logevent\_azMotion} and
\emph{MTDome\_logevent\_elMotion} inPosition parameter = true.

}

\begin{tabular}{p{2cm}}
\toprule
Step 68  \\ \hline
\end{tabular}
 Description \\
{\footnotesize
\textbf{Point the TMA}\\
Command the TMA to {Pointing 24}⁠ at {-180}⁠ , {15}⁠ .

}
\hdashrule[0.5ex]{\textwidth}{1pt}{3mm}
  Expected Result \\
{\footnotesize
The TMA starts moving

}

\begin{tabular}{p{2cm}}
\toprule
Step 69  \\ \hline
\end{tabular}
 Description \\
{\footnotesize
Wait for the TMA to reach the commanded position.

}
\hdashrule[0.5ex]{\textwidth}{1pt}{3mm}
  Expected Result \\
{\footnotesize
The \emph{MTMount\_logevent\_azimuthInPosition} and
\emph{MTMount\_logevent\_elevationInPosition} inPosition parameter =
true.

}

\begin{tabular}{p{2cm}}
\toprule
Step 70  \\ \hline
\end{tabular}
 Description \\
{\footnotesize
\textbf{Image preparation}\\
If the preparation to take images takes longer than 10sec, do
repositioning to target {{{-180}⁠}}, {{{15}⁠~}}.

}
\hdashrule[0.5ex]{\textwidth}{1pt}{3mm}
  Expected Result \\
{\footnotesize
TMA reaches the commanded position.

}

\begin{tabular}{p{2cm}}
\toprule
Step 71  \\ \hline
\end{tabular}
 Description \\
{\footnotesize
\textbf{Track position and take images}\\[2\baselineskip]Take a
StarTracker image with 10s exposure time.\\[2\baselineskip]If the time
the available:

\begin{itemize}
\tightlist
\item
  Track a position for 10 min and take StarTracker images.
\end{itemize}

}
\hdashrule[0.5ex]{\textwidth}{1pt}{3mm}
  Expected Result \\
{\footnotesize
\begin{itemize}
\tightlist
\item
  If time is available: The TMA is tracking a given position for 10 min
  and taking images.
\item
  At least one image is successfully taken with the StarTracker.
\end{itemize}

}

\begin{tabular}{p{2cm}}
\toprule
Step 72  \\ \hline
\end{tabular}
 Description \\
{\footnotesize
\textbf{On-the-fly Image Quality Check}\\
While tracking and taking images, check the images on RubinTV for an
astrometric solution.

}
\hdashrule[0.5ex]{\textwidth}{1pt}{3mm}
  Expected Result \\
{\footnotesize
RubinTV is showing an astrometric solution.

}

\begin{tabular}{p{2cm}}
\toprule
Step 73  \\ \hline
\end{tabular}
 Description \\
{\footnotesize
\textbf{Offline analysis results}\\
Offline analysis in Test case
\href{https://jira.lsstcorp.org/secure/Tests.jspa\#/testCase/LVV-T2739}{LVV-T2739}
Says that we do not have sufficient image quality.

}
\hdashrule[0.5ex]{\textwidth}{1pt}{3mm}
  Expected Result \\
{\footnotesize
Image quality is sufficient.

}

\begin{tabular}{p{2cm}}
\toprule
Step 74  \\ \hline
\end{tabular}
 Description \\
{\footnotesize
\textbf{Point} \textbf{the Dome:}\\
Command the Dome to {Pointing 25}⁠ to {-270}⁠

}
\hdashrule[0.5ex]{\textwidth}{1pt}{3mm}
  Expected Result \\
{\footnotesize
The Dome starts moving.

}

\begin{tabular}{p{2cm}}
\toprule
Step 75  \\ \hline
\end{tabular}
 Description \\
{\footnotesize
Wait for the Dome to reach the commanded position.

}
\hdashrule[0.5ex]{\textwidth}{1pt}{3mm}
  Expected Result \\
{\footnotesize
The \emph{MTDome\_logevent\_azMotion} and
\emph{MTDome\_logevent\_elMotion} inPosition parameter = true.

}

\begin{tabular}{p{2cm}}
\toprule
Step 76  \\ \hline
\end{tabular}
 Description \\
{\footnotesize
\textbf{Point the TMA}\\
Command the TMA to {Pointing 25}⁠ at {-270}⁠ , {15}⁠ .

}
\hdashrule[0.5ex]{\textwidth}{1pt}{3mm}
  Expected Result \\
{\footnotesize
The TMA starts moving

}

\begin{tabular}{p{2cm}}
\toprule
Step 77  \\ \hline
\end{tabular}
 Description \\
{\footnotesize
Wait for the TMA to reach the commanded position.

}
\hdashrule[0.5ex]{\textwidth}{1pt}{3mm}
  Expected Result \\
{\footnotesize
The \emph{MTMount\_logevent\_azimuthInPosition} and
\emph{MTMount\_logevent\_elevationInPosition} inPosition parameter =
true.

}

\begin{tabular}{p{2cm}}
\toprule
Step 78  \\ \hline
\end{tabular}
 Description \\
{\footnotesize
\textbf{Image preparation}\\
If the preparation to take images takes longer than 10sec, do
repositioning to target {{{-270}⁠}}, {{{15}⁠~}}.

}
\hdashrule[0.5ex]{\textwidth}{1pt}{3mm}
  Expected Result \\
{\footnotesize
TMA reaches the commanded position.

}

\begin{tabular}{p{2cm}}
\toprule
Step 79  \\ \hline
\end{tabular}
 Description \\
{\footnotesize
\textbf{Track position and take images}\\[2\baselineskip]Take a
StarTracker image with 10s exposure time.\\[2\baselineskip]If the time
the available:

\begin{itemize}
\tightlist
\item
  Track a position for 10 min and take StarTracker images.
\end{itemize}

}
\hdashrule[0.5ex]{\textwidth}{1pt}{3mm}
  Expected Result \\
{\footnotesize
\begin{itemize}
\tightlist
\item
  If time is available: The TMA is tracking a given position for 10 min
  and taking images.
\item
  At least one image is successfully taken with the StarTracker.
\end{itemize}

}

\begin{tabular}{p{2cm}}
\toprule
Step 80  \\ \hline
\end{tabular}
 Description \\
{\footnotesize
\textbf{On-the-fly Image Quality Check}\\
While tracking and taking images, check the images on RubinTV for an
astrometric solution.

}
\hdashrule[0.5ex]{\textwidth}{1pt}{3mm}
  Expected Result \\
{\footnotesize
RubinTV is showing an astrometric solution.

}

\begin{tabular}{p{2cm}}
\toprule
Step 81  \\ \hline
\end{tabular}
 Description \\
{\footnotesize
\textbf{Offline analysis results}\\
Offline analysis in Test case
\href{https://jira.lsstcorp.org/secure/Tests.jspa\#/testCase/LVV-T2739}{LVV-T2739}
Says that we do not have sufficient image quality.

}
\hdashrule[0.5ex]{\textwidth}{1pt}{3mm}
  Expected Result \\
{\footnotesize
Image quality is sufficient.

}

\begin{tabular}{p{2cm}}
\toprule
Step 82  \\ \hline
\end{tabular}
 Description \\
{\footnotesize
Wait for the Dome to reach the commanded position.

}
\hdashrule[0.5ex]{\textwidth}{1pt}{3mm}
  Expected Result \\
{\footnotesize
The \emph{MTDome\_logevent\_azMotion} and
\emph{MTDome\_logevent\_elMotion} inPosition parameter = true.

}

\begin{tabular}{p{2cm}}
\toprule
Step 83  \\ \hline
\end{tabular}
 Description \\
{\footnotesize
\textbf{Point} \textbf{the Dome:}\\
Command the Dome to {Pointing 2}⁠ to {270}⁠

}
\hdashrule[0.5ex]{\textwidth}{1pt}{3mm}
  Expected Result \\
{\footnotesize
The Dome starts moving.

}

\begin{tabular}{p{2cm}}
\toprule
Step 84  \\ \hline
\end{tabular}
 Description \\
{\footnotesize
Wait for the Dome to reach the commanded position.

}
\hdashrule[0.5ex]{\textwidth}{1pt}{3mm}
  Expected Result \\
{\footnotesize
The \emph{MTDome\_logevent\_azMotion} and
\emph{MTDome\_logevent\_elMotion} inPosition parameter = true.

}

\begin{tabular}{p{2cm}}
\toprule
Step 85  \\ \hline
\end{tabular}
 Description \\
{\footnotesize
\textbf{Point the TMA}\\
Command the TMA to {Pointing 2}⁠ at {270}⁠ , {45}⁠ .

}
\hdashrule[0.5ex]{\textwidth}{1pt}{3mm}
  Expected Result \\
{\footnotesize
The TMA starts moving

}

\begin{tabular}{p{2cm}}
\toprule
Step 86  \\ \hline
\end{tabular}
 Description \\
{\footnotesize
Wait for the TMA to reach the commanded position.

}
\hdashrule[0.5ex]{\textwidth}{1pt}{3mm}
  Expected Result \\
{\footnotesize
The \emph{MTMount\_logevent\_azimuthInPosition} and
\emph{MTMount\_logevent\_elevationInPosition} inPosition parameter =
true.

}

\begin{tabular}{p{2cm}}
\toprule
Step 87  \\ \hline
\end{tabular}
 Description \\
{\footnotesize
\textbf{Image preparation}\\
If the preparation to take images takes longer than 10sec, do
repositioning to target {{{270}⁠}}, {{{45}⁠~}}.

}
\hdashrule[0.5ex]{\textwidth}{1pt}{3mm}
  Expected Result \\
{\footnotesize
TMA reaches the commanded position.

}

\begin{tabular}{p{2cm}}
\toprule
Step 88  \\ \hline
\end{tabular}
 Description \\
{\footnotesize
\textbf{Track position and take images}\\[2\baselineskip]Take a
StarTracker image with 10s exposure time.\\[2\baselineskip]If the time
the available:

\begin{itemize}
\tightlist
\item
  Track a position for 10 min and take StarTracker images.
\end{itemize}

}
\hdashrule[0.5ex]{\textwidth}{1pt}{3mm}
  Expected Result \\
{\footnotesize
\begin{itemize}
\tightlist
\item
  If time is available: The TMA is tracking a given position for 10 min
  and taking images.
\item
  At least one image is successfully taken with the StarTracker.
\end{itemize}

}

\begin{tabular}{p{2cm}}
\toprule
Step 89  \\ \hline
\end{tabular}
 Description \\
{\footnotesize
\textbf{On-the-fly Image Quality Check}\\
While tracking and taking images, check the images on RubinTV for an
astrometric solution.

}
\hdashrule[0.5ex]{\textwidth}{1pt}{3mm}
  Expected Result \\
{\footnotesize
RubinTV is showing an astrometric solution.

}

\begin{tabular}{p{2cm}}
\toprule
Step 90  \\ \hline
\end{tabular}
 Description \\
{\footnotesize
\textbf{Offline analysis results}\\
Offline analysis in Test case
\href{https://jira.lsstcorp.org/secure/Tests.jspa\#/testCase/LVV-T2739}{LVV-T2739}
Says that we do not have sufficient image quality.

}
\hdashrule[0.5ex]{\textwidth}{1pt}{3mm}
  Expected Result \\
{\footnotesize
Image quality is sufficient.

}

\begin{tabular}{p{2cm}}
\toprule
Step 91  \\ \hline
\end{tabular}
 Description \\
{\footnotesize
\textbf{Point} \textbf{the Dome:}\\
Command the Dome to {Pointing 26}⁠ to {-270}⁠

}
\hdashrule[0.5ex]{\textwidth}{1pt}{3mm}
  Expected Result \\
{\footnotesize
The Dome starts moving.

}

\begin{tabular}{p{2cm}}
\toprule
Step 92  \\ \hline
\end{tabular}
 Description \\
{\footnotesize
Wait for the Dome to reach the commanded position.

}
\hdashrule[0.5ex]{\textwidth}{1pt}{3mm}
  Expected Result \\
{\footnotesize
The \emph{MTDome\_logevent\_azMotion} and
\emph{MTDome\_logevent\_elMotion} inPosition parameter = true.

}

\begin{tabular}{p{2cm}}
\toprule
Step 93  \\ \hline
\end{tabular}
 Description \\
{\footnotesize
\textbf{Point the TMA}\\
Command the TMA to {Pointing 26}⁠ at {-270}⁠ , {45}⁠ .

}
\hdashrule[0.5ex]{\textwidth}{1pt}{3mm}
  Expected Result \\
{\footnotesize
The TMA starts moving

}

\begin{tabular}{p{2cm}}
\toprule
Step 94  \\ \hline
\end{tabular}
 Description \\
{\footnotesize
Wait for the TMA to reach the commanded position.

}
\hdashrule[0.5ex]{\textwidth}{1pt}{3mm}
  Expected Result \\
{\footnotesize
The \emph{MTMount\_logevent\_azimuthInPosition} and
\emph{MTMount\_logevent\_elevationInPosition} inPosition parameter =
true.

}

\begin{tabular}{p{2cm}}
\toprule
Step 95  \\ \hline
\end{tabular}
 Description \\
{\footnotesize
\textbf{Image preparation}\\
If the preparation to take images takes longer than 10sec, do
repositioning to target {{{-270}⁠}}, {{{45}⁠~}}.

}
\hdashrule[0.5ex]{\textwidth}{1pt}{3mm}
  Expected Result \\
{\footnotesize
TMA reaches the commanded position.

}

\begin{tabular}{p{2cm}}
\toprule
Step 96  \\ \hline
\end{tabular}
 Description \\
{\footnotesize
\textbf{Track position and take images}\\[2\baselineskip]Take a
StarTracker image with 10s exposure time.\\[2\baselineskip]If the time
the available:

\begin{itemize}
\tightlist
\item
  Track a position for 10 min and take StarTracker images.
\end{itemize}

}
\hdashrule[0.5ex]{\textwidth}{1pt}{3mm}
  Expected Result \\
{\footnotesize
\begin{itemize}
\tightlist
\item
  If time is available: The TMA is tracking a given position for 10 min
  and taking images.
\item
  At least one image is successfully taken with the StarTracker.
\end{itemize}

}

\begin{tabular}{p{2cm}}
\toprule
Step 97  \\ \hline
\end{tabular}
 Description \\
{\footnotesize
\textbf{On-the-fly Image Quality Check}\\
While tracking and taking images, check the images on RubinTV for an
astrometric solution.

}
\hdashrule[0.5ex]{\textwidth}{1pt}{3mm}
  Expected Result \\
{\footnotesize
RubinTV is showing an astrometric solution.

}

\begin{tabular}{p{2cm}}
\toprule
Step 98  \\ \hline
\end{tabular}
 Description \\
{\footnotesize
\textbf{Offline analysis results}\\
Offline analysis in Test case
\href{https://jira.lsstcorp.org/secure/Tests.jspa\#/testCase/LVV-T2739}{LVV-T2739}
Says that we do not have sufficient image quality.

}
\hdashrule[0.5ex]{\textwidth}{1pt}{3mm}
  Expected Result \\
{\footnotesize
Image quality is sufficient.

}

\begin{tabular}{p{2cm}}
\toprule
Step 99  \\ \hline
\end{tabular}
 Description \\
{\footnotesize
\textbf{Point} \textbf{the Dome:}\\
Command the Dome to {Pointing 28}⁠ to {-270}⁠

}
\hdashrule[0.5ex]{\textwidth}{1pt}{3mm}
  Expected Result \\
{\footnotesize
The Dome starts moving.

}

\begin{tabular}{p{2cm}}
\toprule
Step 100  \\ \hline
\end{tabular}
 Description \\
{\footnotesize
Wait for the Dome to reach the commanded position.

}
\hdashrule[0.5ex]{\textwidth}{1pt}{3mm}
  Expected Result \\
{\footnotesize
The \emph{MTDome\_logevent\_azMotion} and
\emph{MTDome\_logevent\_elMotion} inPosition parameter = true.

}

\begin{tabular}{p{2cm}}
\toprule
Step 101  \\ \hline
\end{tabular}
 Description \\
{\footnotesize
\textbf{Point the TMA}\\
Command the TMA to {Pointing 28}⁠ at {-270}⁠ , {85}⁠ .

}
\hdashrule[0.5ex]{\textwidth}{1pt}{3mm}
  Expected Result \\
{\footnotesize
The TMA starts moving

}

\begin{tabular}{p{2cm}}
\toprule
Step 102  \\ \hline
\end{tabular}
 Description \\
{\footnotesize
Wait for the TMA to reach the commanded position.

}
\hdashrule[0.5ex]{\textwidth}{1pt}{3mm}
  Expected Result \\
{\footnotesize
The \emph{MTMount\_logevent\_azimuthInPosition} and
\emph{MTMount\_logevent\_elevationInPosition} inPosition parameter =
true.

}

\begin{tabular}{p{2cm}}
\toprule
Step 103  \\ \hline
\end{tabular}
 Description \\
{\footnotesize
\textbf{Image preparation}\\
If the preparation to take images takes longer than 10sec, do
repositioning to target {{{-270}⁠}}, {{{85}⁠~}}.

}
\hdashrule[0.5ex]{\textwidth}{1pt}{3mm}
  Expected Result \\
{\footnotesize
TMA reaches the commanded position.

}

\begin{tabular}{p{2cm}}
\toprule
Step 104  \\ \hline
\end{tabular}
 Description \\
{\footnotesize
\textbf{Track position and take images}\\[2\baselineskip]Take a
StarTracker image with 10s exposure time.\\[2\baselineskip]If the time
the available:

\begin{itemize}
\tightlist
\item
  Track a position for 10 min and take StarTracker images.
\end{itemize}

}
\hdashrule[0.5ex]{\textwidth}{1pt}{3mm}
  Expected Result \\
{\footnotesize
\begin{itemize}
\tightlist
\item
  If time is available: The TMA is tracking a given position for 10 min
  and taking images.
\item
  At least one image is successfully taken with the StarTracker.
\end{itemize}

}

\begin{tabular}{p{2cm}}
\toprule
Step 105  \\ \hline
\end{tabular}
 Description \\
{\footnotesize
\textbf{On-the-fly Image Quality Check}\\
While tracking and taking images, check the images on RubinTV for an
astrometric solution.

}
\hdashrule[0.5ex]{\textwidth}{1pt}{3mm}
  Expected Result \\
{\footnotesize
RubinTV is showing an astrometric solution.

}

\begin{tabular}{p{2cm}}
\toprule
Step 106  \\ \hline
\end{tabular}
 Description \\
{\footnotesize
\textbf{Offline analysis results}\\
Offline analysis in Test case
\href{https://jira.lsstcorp.org/secure/Tests.jspa\#/testCase/LVV-T2739}{LVV-T2739}
Says that we do not have sufficient image quality.

}
\hdashrule[0.5ex]{\textwidth}{1pt}{3mm}
  Expected Result \\
{\footnotesize
Image quality is sufficient.

}

\begin{tabular}{p{2cm}}
\toprule
Step 107  \\ \hline
\end{tabular}
 Description \\
{\footnotesize
\textbf{Point} \textbf{the Dome:}\\
Command the Dome to {Pointing 22}⁠ to {270}⁠

}
\hdashrule[0.5ex]{\textwidth}{1pt}{3mm}
  Expected Result \\
{\footnotesize
The Dome starts moving.

}

\begin{tabular}{p{2cm}}
\toprule
Step 108  \\ \hline
\end{tabular}
 Description \\
{\footnotesize
Wait for the Dome to reach the commanded position.

}
\hdashrule[0.5ex]{\textwidth}{1pt}{3mm}
  Expected Result \\
{\footnotesize
The \emph{MTDome\_logevent\_azMotion} and
\emph{MTDome\_logevent\_elMotion} inPosition parameter = true.

}

\begin{tabular}{p{2cm}}
\toprule
Step 109  \\ \hline
\end{tabular}
 Description \\
{\footnotesize
\textbf{Point the TMA}\\
Command the TMA to {Pointing 22}⁠ at {270}⁠ , {75}⁠ .

}
\hdashrule[0.5ex]{\textwidth}{1pt}{3mm}
  Expected Result \\
{\footnotesize
The TMA starts moving

}

\begin{tabular}{p{2cm}}
\toprule
Step 110  \\ \hline
\end{tabular}
 Description \\
{\footnotesize
Wait for the TMA to reach the commanded position.

}
\hdashrule[0.5ex]{\textwidth}{1pt}{3mm}
  Expected Result \\
{\footnotesize
The \emph{MTMount\_logevent\_azimuthInPosition} and
\emph{MTMount\_logevent\_elevationInPosition} inPosition parameter =
true.

}

\begin{tabular}{p{2cm}}
\toprule
Step 111  \\ \hline
\end{tabular}
 Description \\
{\footnotesize
\textbf{Image preparation}\\
If the preparation to take images takes longer than 10sec, do
repositioning to target {{{270}⁠}}, {{{75}⁠~}}.

}
\hdashrule[0.5ex]{\textwidth}{1pt}{3mm}
  Expected Result \\
{\footnotesize
TMA reaches the commanded position.

}

\begin{tabular}{p{2cm}}
\toprule
Step 112  \\ \hline
\end{tabular}
 Description \\
{\footnotesize
\textbf{Track position and take images}\\[2\baselineskip]Take a
StarTracker image with 10s exposure time.\\[2\baselineskip]If the time
the available:

\begin{itemize}
\tightlist
\item
  Track a position for 10 min and take StarTracker images.
\end{itemize}

}
\hdashrule[0.5ex]{\textwidth}{1pt}{3mm}
  Expected Result \\
{\footnotesize
\begin{itemize}
\tightlist
\item
  If time is available: The TMA is tracking a given position for 10 min
  and taking images.
\item
  At least one image is successfully taken with the StarTracker.
\end{itemize}

}

\begin{tabular}{p{2cm}}
\toprule
Step 113  \\ \hline
\end{tabular}
 Description \\
{\footnotesize
\textbf{On-the-fly Image Quality Check}\\
While tracking and taking images, check the images on RubinTV for an
astrometric solution.

}
\hdashrule[0.5ex]{\textwidth}{1pt}{3mm}
  Expected Result \\
{\footnotesize
RubinTV is showing an astrometric solution.

}

\begin{tabular}{p{2cm}}
\toprule
Step 114  \\ \hline
\end{tabular}
 Description \\
{\footnotesize
\textbf{Offline analysis results}\\
Offline analysis in Test case
\href{https://jira.lsstcorp.org/secure/Tests.jspa\#/testCase/LVV-T2739}{LVV-T2739}
Says that we do not have sufficient image quality.

}
\hdashrule[0.5ex]{\textwidth}{1pt}{3mm}
  Expected Result \\
{\footnotesize
Image quality is sufficient.

}

\begin{tabular}{p{2cm}}
\toprule
Step 115  \\ \hline
\end{tabular}
 Description \\
{\footnotesize
\textbf{Point} \textbf{the Dome:}\\
Command the Dome to {Pointing 23}⁠ to {180}⁠

}
\hdashrule[0.5ex]{\textwidth}{1pt}{3mm}
  Expected Result \\
{\footnotesize
The Dome starts moving.

}

\begin{tabular}{p{2cm}}
\toprule
Step 116  \\ \hline
\end{tabular}
 Description \\
{\footnotesize
Wait for the Dome to reach the commanded position.

}
\hdashrule[0.5ex]{\textwidth}{1pt}{3mm}
  Expected Result \\
{\footnotesize
The \emph{MTDome\_logevent\_azMotion} and
\emph{MTDome\_logevent\_elMotion} inPosition parameter = true.

}

\begin{tabular}{p{2cm}}
\toprule
Step 117  \\ \hline
\end{tabular}
 Description \\
{\footnotesize
\textbf{Point the TMA}\\
Command the TMA to {Pointing 23}⁠ at {180}⁠ , {75}⁠ .

}
\hdashrule[0.5ex]{\textwidth}{1pt}{3mm}
  Expected Result \\
{\footnotesize
The TMA starts moving

}

\begin{tabular}{p{2cm}}
\toprule
Step 118  \\ \hline
\end{tabular}
 Description \\
{\footnotesize
Wait for the TMA to reach the commanded position.

}
\hdashrule[0.5ex]{\textwidth}{1pt}{3mm}
  Expected Result \\
{\footnotesize
The \emph{MTMount\_logevent\_azimuthInPosition} and
\emph{MTMount\_logevent\_elevationInPosition} inPosition parameter =
true.

}

\begin{tabular}{p{2cm}}
\toprule
Step 119  \\ \hline
\end{tabular}
 Description \\
{\footnotesize
\textbf{Image preparation}\\
If the preparation to take images takes longer than 10sec, do
repositioning to target {{{180}⁠}}, {{{75}⁠~}}.

}
\hdashrule[0.5ex]{\textwidth}{1pt}{3mm}
  Expected Result \\
{\footnotesize
TMA reaches the commanded position.

}

\begin{tabular}{p{2cm}}
\toprule
Step 120  \\ \hline
\end{tabular}
 Description \\
{\footnotesize
\textbf{Track position and take images}\\[2\baselineskip]Take a
StarTracker image with 10s exposure time.\\[2\baselineskip]If the time
the available:

\begin{itemize}
\tightlist
\item
  Track a position for 10 min and take StarTracker images.
\end{itemize}

}
\hdashrule[0.5ex]{\textwidth}{1pt}{3mm}
  Expected Result \\
{\footnotesize
\begin{itemize}
\tightlist
\item
  If time is available: The TMA is tracking a given position for 10 min
  and taking images.
\item
  At least one image is successfully taken with the StarTracker.
\end{itemize}

}

\begin{tabular}{p{2cm}}
\toprule
Step 121  \\ \hline
\end{tabular}
 Description \\
{\footnotesize
\textbf{On-the-fly Image Quality Check}\\
While tracking and taking images, check the images on RubinTV for an
astrometric solution.

}
\hdashrule[0.5ex]{\textwidth}{1pt}{3mm}
  Expected Result \\
{\footnotesize
RubinTV is showing an astrometric solution.

}

\begin{tabular}{p{2cm}}
\toprule
Step 122  \\ \hline
\end{tabular}
 Description \\
{\footnotesize
\textbf{Offline analysis results}\\
Offline analysis in Test case
\href{https://jira.lsstcorp.org/secure/Tests.jspa\#/testCase/LVV-T2739}{LVV-T2739}
Says that we do not have sufficient image quality.

}
\hdashrule[0.5ex]{\textwidth}{1pt}{3mm}
  Expected Result \\
{\footnotesize
Image quality is sufficient.

}

\begin{tabular}{p{2cm}}
\toprule
Step 123  \\ \hline
\end{tabular}
 Description \\
{\footnotesize
\textbf{Point} \textbf{the Dome:}\\
Command the Dome to {Pointing 24}⁠ to {90}⁠

}
\hdashrule[0.5ex]{\textwidth}{1pt}{3mm}
  Expected Result \\
{\footnotesize
The Dome starts moving.

}

\begin{tabular}{p{2cm}}
\toprule
Step 124  \\ \hline
\end{tabular}
 Description \\
{\footnotesize
Wait for the Dome to reach the commanded position.

}
\hdashrule[0.5ex]{\textwidth}{1pt}{3mm}
  Expected Result \\
{\footnotesize
The \emph{MTDome\_logevent\_azMotion} and
\emph{MTDome\_logevent\_elMotion} inPosition parameter = true.

}

\begin{tabular}{p{2cm}}
\toprule
Step 125  \\ \hline
\end{tabular}
 Description \\
{\footnotesize
\textbf{Point the TMA}\\
Command the TMA to {Pointing 24}⁠ at {90}⁠ , {75}⁠ .

}
\hdashrule[0.5ex]{\textwidth}{1pt}{3mm}
  Expected Result \\
{\footnotesize
The TMA starts moving

}

\begin{tabular}{p{2cm}}
\toprule
Step 126  \\ \hline
\end{tabular}
 Description \\
{\footnotesize
Wait for the TMA to reach the commanded position.

}
\hdashrule[0.5ex]{\textwidth}{1pt}{3mm}
  Expected Result \\
{\footnotesize
The \emph{MTMount\_logevent\_azimuthInPosition} and
\emph{MTMount\_logevent\_elevationInPosition} inPosition parameter =
true.

}

\begin{tabular}{p{2cm}}
\toprule
Step 127  \\ \hline
\end{tabular}
 Description \\
{\footnotesize
\textbf{Image preparation}\\
If the preparation to take images takes longer than 10sec, do
repositioning to target {{{90}⁠}}, {{{75}⁠~}}.

}
\hdashrule[0.5ex]{\textwidth}{1pt}{3mm}
  Expected Result \\
{\footnotesize
TMA reaches the commanded position.

}

\begin{tabular}{p{2cm}}
\toprule
Step 128  \\ \hline
\end{tabular}
 Description \\
{\footnotesize
\textbf{Track position and take images}\\[2\baselineskip]Take a
StarTracker image with 10s exposure time.\\[2\baselineskip]If the time
the available:

\begin{itemize}
\tightlist
\item
  Track a position for 10 min and take StarTracker images.
\end{itemize}

}
\hdashrule[0.5ex]{\textwidth}{1pt}{3mm}
  Expected Result \\
{\footnotesize
\begin{itemize}
\tightlist
\item
  If time is available: The TMA is tracking a given position for 10 min
  and taking images.
\item
  At least one image is successfully taken with the StarTracker.
\end{itemize}

}

\begin{tabular}{p{2cm}}
\toprule
Step 129  \\ \hline
\end{tabular}
 Description \\
{\footnotesize
\textbf{On-the-fly Image Quality Check}\\
While tracking and taking images, check the images on RubinTV for an
astrometric solution.

}
\hdashrule[0.5ex]{\textwidth}{1pt}{3mm}
  Expected Result \\
{\footnotesize
RubinTV is showing an astrometric solution.

}

\begin{tabular}{p{2cm}}
\toprule
Step 130  \\ \hline
\end{tabular}
 Description \\
{\footnotesize
\textbf{Offline analysis results}\\
Offline analysis in Test case
\href{https://jira.lsstcorp.org/secure/Tests.jspa\#/testCase/LVV-T2739}{LVV-T2739}
Says that we do not have sufficient image quality.

}
\hdashrule[0.5ex]{\textwidth}{1pt}{3mm}
  Expected Result \\
{\footnotesize
Image quality is sufficient.

}

\begin{tabular}{p{2cm}}
\toprule
Step 131  \\ \hline
\end{tabular}
 Description \\
{\footnotesize
\textbf{Point} \textbf{the Dome:}\\
Command the Dome to {Pointing 25}⁠ to {0}⁠

}
\hdashrule[0.5ex]{\textwidth}{1pt}{3mm}
  Expected Result \\
{\footnotesize
The Dome starts moving.

}

\begin{tabular}{p{2cm}}
\toprule
Step 132  \\ \hline
\end{tabular}
 Description \\
{\footnotesize
Wait for the Dome to reach the commanded position.

}
\hdashrule[0.5ex]{\textwidth}{1pt}{3mm}
  Expected Result \\
{\footnotesize
The \emph{MTDome\_logevent\_azMotion} and
\emph{MTDome\_logevent\_elMotion} inPosition parameter = true.

}

\begin{tabular}{p{2cm}}
\toprule
Step 133  \\ \hline
\end{tabular}
 Description \\
{\footnotesize
\textbf{Point the TMA}\\
Command the TMA to {Pointing 25}⁠ at {0}⁠ , {75}⁠ .

}
\hdashrule[0.5ex]{\textwidth}{1pt}{3mm}
  Expected Result \\
{\footnotesize
The TMA starts moving

}

\begin{tabular}{p{2cm}}
\toprule
Step 134  \\ \hline
\end{tabular}
 Description \\
{\footnotesize
Wait for the TMA to reach the commanded position.

}
\hdashrule[0.5ex]{\textwidth}{1pt}{3mm}
  Expected Result \\
{\footnotesize
The \emph{MTMount\_logevent\_azimuthInPosition} and
\emph{MTMount\_logevent\_elevationInPosition} inPosition parameter =
true.

}

\begin{tabular}{p{2cm}}
\toprule
Step 135  \\ \hline
\end{tabular}
 Description \\
{\footnotesize
\textbf{Image preparation}\\
If the preparation to take images takes longer than 10sec, do
repositioning to target {{{0}⁠}}, {{{75}⁠~}}.

}
\hdashrule[0.5ex]{\textwidth}{1pt}{3mm}
  Expected Result \\
{\footnotesize
TMA reaches the commanded position.

}

\begin{tabular}{p{2cm}}
\toprule
Step 136  \\ \hline
\end{tabular}
 Description \\
{\footnotesize
\textbf{Track position and take images}\\[2\baselineskip]Take a
StarTracker image with 10s exposure time.\\[2\baselineskip]If the time
the available:

\begin{itemize}
\tightlist
\item
  Track a position for 10 min and take StarTracker images.
\end{itemize}

}
\hdashrule[0.5ex]{\textwidth}{1pt}{3mm}
  Expected Result \\
{\footnotesize
\begin{itemize}
\tightlist
\item
  If time is available: The TMA is tracking a given position for 10 min
  and taking images.
\item
  At least one image is successfully taken with the StarTracker.
\end{itemize}

}

\begin{tabular}{p{2cm}}
\toprule
Step 137  \\ \hline
\end{tabular}
 Description \\
{\footnotesize
\textbf{On-the-fly Image Quality Check}\\
While tracking and taking images, check the images on RubinTV for an
astrometric solution.

}
\hdashrule[0.5ex]{\textwidth}{1pt}{3mm}
  Expected Result \\
{\footnotesize
RubinTV is showing an astrometric solution.

}

\begin{tabular}{p{2cm}}
\toprule
Step 138  \\ \hline
\end{tabular}
 Description \\
{\footnotesize
\textbf{Offline analysis results}\\
Offline analysis in Test case
\href{https://jira.lsstcorp.org/secure/Tests.jspa\#/testCase/LVV-T2739}{LVV-T2739}
Says that we do not have sufficient image quality.

}
\hdashrule[0.5ex]{\textwidth}{1pt}{3mm}
  Expected Result \\
{\footnotesize
Image quality is sufficient.

}

\begin{tabular}{p{2cm}}
\toprule
Step 139  \\ \hline
\end{tabular}
 Description \\
{\footnotesize
\textbf{Point} \textbf{the Dome:}\\
Command the Dome to {Pointing 26}⁠ to {-90}⁠

}
\hdashrule[0.5ex]{\textwidth}{1pt}{3mm}
  Expected Result \\
{\footnotesize
The Dome starts moving.

}

\begin{tabular}{p{2cm}}
\toprule
Step 140  \\ \hline
\end{tabular}
 Description \\
{\footnotesize
Wait for the Dome to reach the commanded position.

}
\hdashrule[0.5ex]{\textwidth}{1pt}{3mm}
  Expected Result \\
{\footnotesize
The \emph{MTDome\_logevent\_azMotion} and
\emph{MTDome\_logevent\_elMotion} inPosition parameter = true.

}

\begin{tabular}{p{2cm}}
\toprule
Step 141  \\ \hline
\end{tabular}
 Description \\
{\footnotesize
\textbf{Point the TMA}\\
Command the TMA to {Pointing 26}⁠ at {-90}⁠ , {75}⁠ .

}
\hdashrule[0.5ex]{\textwidth}{1pt}{3mm}
  Expected Result \\
{\footnotesize
The TMA starts moving

}

\begin{tabular}{p{2cm}}
\toprule
Step 142  \\ \hline
\end{tabular}
 Description \\
{\footnotesize
Wait for the TMA to reach the commanded position.

}
\hdashrule[0.5ex]{\textwidth}{1pt}{3mm}
  Expected Result \\
{\footnotesize
The \emph{MTMount\_logevent\_azimuthInPosition} and
\emph{MTMount\_logevent\_elevationInPosition} inPosition parameter =
true.

}

\begin{tabular}{p{2cm}}
\toprule
Step 143  \\ \hline
\end{tabular}
 Description \\
{\footnotesize
\textbf{Image preparation}\\
If the preparation to take images takes longer than 10sec, do
repositioning to target {{{-90}⁠}}, {{{75}⁠~}}.

}
\hdashrule[0.5ex]{\textwidth}{1pt}{3mm}
  Expected Result \\
{\footnotesize
TMA reaches the commanded position.

}

\begin{tabular}{p{2cm}}
\toprule
Step 144  \\ \hline
\end{tabular}
 Description \\
{\footnotesize
\textbf{Track position and take images}\\[2\baselineskip]Take a
StarTracker image with 10s exposure time.\\[2\baselineskip]If the time
the available:

\begin{itemize}
\tightlist
\item
  Track a position for 10 min and take StarTracker images.
\end{itemize}

}
\hdashrule[0.5ex]{\textwidth}{1pt}{3mm}
  Expected Result \\
{\footnotesize
\begin{itemize}
\tightlist
\item
  If time is available: The TMA is tracking a given position for 10 min
  and taking images.
\item
  At least one image is successfully taken with the StarTracker.
\end{itemize}

}

\begin{tabular}{p{2cm}}
\toprule
Step 145  \\ \hline
\end{tabular}
 Description \\
{\footnotesize
\textbf{On-the-fly Image Quality Check}\\
While tracking and taking images, check the images on RubinTV for an
astrometric solution.

}
\hdashrule[0.5ex]{\textwidth}{1pt}{3mm}
  Expected Result \\
{\footnotesize
RubinTV is showing an astrometric solution.

}

\begin{tabular}{p{2cm}}
\toprule
Step 146  \\ \hline
\end{tabular}
 Description \\
{\footnotesize
\textbf{Offline analysis results}\\
Offline analysis in Test case
\href{https://jira.lsstcorp.org/secure/Tests.jspa\#/testCase/LVV-T2739}{LVV-T2739}
Says that we do not have sufficient image quality.

}
\hdashrule[0.5ex]{\textwidth}{1pt}{3mm}
  Expected Result \\
{\footnotesize
Image quality is sufficient.

}

\begin{tabular}{p{2cm}}
\toprule
Step 147  \\ \hline
\end{tabular}
 Description \\
{\footnotesize
\textbf{Point} \textbf{the Dome:}\\
Command the Dome to {Pointing 27}⁠ to {-180}⁠

}
\hdashrule[0.5ex]{\textwidth}{1pt}{3mm}
  Expected Result \\
{\footnotesize
The Dome starts moving.

}

\begin{tabular}{p{2cm}}
\toprule
Step 148  \\ \hline
\end{tabular}
 Description \\
{\footnotesize
Wait for the Dome to reach the commanded position.

}
\hdashrule[0.5ex]{\textwidth}{1pt}{3mm}
  Expected Result \\
{\footnotesize
The \emph{MTDome\_logevent\_azMotion} and
\emph{MTDome\_logevent\_elMotion} inPosition parameter = true.

}

\begin{tabular}{p{2cm}}
\toprule
Step 149  \\ \hline
\end{tabular}
 Description \\
{\footnotesize
\textbf{Point the TMA}\\
Command the TMA to {Pointing 27}⁠ at {-180}⁠ , {75}⁠ .

}
\hdashrule[0.5ex]{\textwidth}{1pt}{3mm}
  Expected Result \\
{\footnotesize
The TMA starts moving

}

\begin{tabular}{p{2cm}}
\toprule
Step 150  \\ \hline
\end{tabular}
 Description \\
{\footnotesize
Wait for the TMA to reach the commanded position.

}
\hdashrule[0.5ex]{\textwidth}{1pt}{3mm}
  Expected Result \\
{\footnotesize
The \emph{MTMount\_logevent\_azimuthInPosition} and
\emph{MTMount\_logevent\_elevationInPosition} inPosition parameter =
true.

}

\begin{tabular}{p{2cm}}
\toprule
Step 151  \\ \hline
\end{tabular}
 Description \\
{\footnotesize
\textbf{Image preparation}\\
If the preparation to take images takes longer than 10sec, do
repositioning to target {{{-180}⁠}}, {{{75}⁠~}}.

}
\hdashrule[0.5ex]{\textwidth}{1pt}{3mm}
  Expected Result \\
{\footnotesize
TMA reaches the commanded position.

}

\begin{tabular}{p{2cm}}
\toprule
Step 152  \\ \hline
\end{tabular}
 Description \\
{\footnotesize
\textbf{Track position and take images}\\[2\baselineskip]Take a
StarTracker image with 10s exposure time.\\[2\baselineskip]If the time
the available:

\begin{itemize}
\tightlist
\item
  Track a position for 10 min and take StarTracker images.
\end{itemize}

}
\hdashrule[0.5ex]{\textwidth}{1pt}{3mm}
  Expected Result \\
{\footnotesize
\begin{itemize}
\tightlist
\item
  If time is available: The TMA is tracking a given position for 10 min
  and taking images.
\item
  At least one image is successfully taken with the StarTracker.
\end{itemize}

}

\begin{tabular}{p{2cm}}
\toprule
Step 153  \\ \hline
\end{tabular}
 Description \\
{\footnotesize
\textbf{On-the-fly Image Quality Check}\\
While tracking and taking images, check the images on RubinTV for an
astrometric solution.

}
\hdashrule[0.5ex]{\textwidth}{1pt}{3mm}
  Expected Result \\
{\footnotesize
RubinTV is showing an astrometric solution.

}

\begin{tabular}{p{2cm}}
\toprule
Step 154  \\ \hline
\end{tabular}
 Description \\
{\footnotesize
\textbf{Offline analysis results}\\
Offline analysis in Test case
\href{https://jira.lsstcorp.org/secure/Tests.jspa\#/testCase/LVV-T2739}{LVV-T2739}
Says that we do not have sufficient image quality.

}
\hdashrule[0.5ex]{\textwidth}{1pt}{3mm}
  Expected Result \\
{\footnotesize
Image quality is sufficient.

}

\begin{tabular}{p{2cm}}
\toprule
Step 155  \\ \hline
\end{tabular}
 Description \\
{\footnotesize
\textbf{Point} \textbf{the Dome:}\\
Command the Dome to {Pointing 28}⁠ to {-270}⁠

}
\hdashrule[0.5ex]{\textwidth}{1pt}{3mm}
  Expected Result \\
{\footnotesize
The Dome starts moving.

}

\begin{tabular}{p{2cm}}
\toprule
Step 156  \\ \hline
\end{tabular}
 Description \\
{\footnotesize
Wait for the Dome to reach the commanded position.

}
\hdashrule[0.5ex]{\textwidth}{1pt}{3mm}
  Expected Result \\
{\footnotesize
The \emph{MTDome\_logevent\_azMotion} and
\emph{MTDome\_logevent\_elMotion} inPosition parameter = true.

}

\begin{tabular}{p{2cm}}
\toprule
Step 157  \\ \hline
\end{tabular}
 Description \\
{\footnotesize
\textbf{Point the TMA}\\
Command the TMA to {Pointing 28}⁠ at {-270}⁠ , {75}⁠ .

}
\hdashrule[0.5ex]{\textwidth}{1pt}{3mm}
  Expected Result \\
{\footnotesize
The TMA starts moving

}

\begin{tabular}{p{2cm}}
\toprule
Step 158  \\ \hline
\end{tabular}
 Description \\
{\footnotesize
Wait for the TMA to reach the commanded position.

}
\hdashrule[0.5ex]{\textwidth}{1pt}{3mm}
  Expected Result \\
{\footnotesize
The \emph{MTMount\_logevent\_azimuthInPosition} and
\emph{MTMount\_logevent\_elevationInPosition} inPosition parameter =
true.

}

\begin{tabular}{p{2cm}}
\toprule
Step 159  \\ \hline
\end{tabular}
 Description \\
{\footnotesize
\textbf{Image preparation}\\
If the preparation to take images takes longer than 10sec, do
repositioning to target {{{-270}⁠}}, {{{75}⁠~}}.

}
\hdashrule[0.5ex]{\textwidth}{1pt}{3mm}
  Expected Result \\
{\footnotesize
TMA reaches the commanded position.

}

\begin{tabular}{p{2cm}}
\toprule
Step 160  \\ \hline
\end{tabular}
 Description \\
{\footnotesize
\textbf{Track position and take images}\\[2\baselineskip]Take a
StarTracker image with 10s exposure time.\\[2\baselineskip]If the time
the available:

\begin{itemize}
\tightlist
\item
  Track a position for 10 min and take StarTracker images.
\end{itemize}

}
\hdashrule[0.5ex]{\textwidth}{1pt}{3mm}
  Expected Result \\
{\footnotesize
\begin{itemize}
\tightlist
\item
  If time is available: The TMA is tracking a given position for 10 min
  and taking images.
\item
  At least one image is successfully taken with the StarTracker.
\end{itemize}

}

\begin{tabular}{p{2cm}}
\toprule
Step 161  \\ \hline
\end{tabular}
 Description \\
{\footnotesize
\textbf{On-the-fly Image Quality Check}\\
While tracking and taking images, check the images on RubinTV for an
astrometric solution.

}
\hdashrule[0.5ex]{\textwidth}{1pt}{3mm}
  Expected Result \\
{\footnotesize
RubinTV is showing an astrometric solution.

}

\begin{tabular}{p{2cm}}
\toprule
Step 162  \\ \hline
\end{tabular}
 Description \\
{\footnotesize
\textbf{Offline analysis results}\\
Offline analysis in Test case
\href{https://jira.lsstcorp.org/secure/Tests.jspa\#/testCase/LVV-T2739}{LVV-T2739}
Says that we do not have sufficient image quality.

}
\hdashrule[0.5ex]{\textwidth}{1pt}{3mm}
  Expected Result \\
{\footnotesize
Image quality is sufficient.

}

\begin{tabular}{p{2cm}}
\toprule
Step 163  \\ \hline
\end{tabular}
 Description \\
{\footnotesize
\textbf{Point the TMA}\\
Command the TMA to {Pointing 1}⁠ at {270}⁠ , {15}⁠ .

}
\hdashrule[0.5ex]{\textwidth}{1pt}{3mm}
  Expected Result \\
{\footnotesize
The TMA starts moving

}

\begin{tabular}{p{2cm}}
\toprule
Step 164  \\ \hline
\end{tabular}
 Description \\
{\footnotesize
\textbf{Point} \textbf{the Dome:}\\
Command the Dome to {Pointing 4}⁠ to {270}⁠

}
\hdashrule[0.5ex]{\textwidth}{1pt}{3mm}
  Expected Result \\
{\footnotesize
The Dome starts moving.

}

\begin{tabular}{p{2cm}}
\toprule
Step 165  \\ \hline
\end{tabular}
 Description \\
{\footnotesize
Wait for the Dome to reach the commanded position.

}
\hdashrule[0.5ex]{\textwidth}{1pt}{3mm}
  Expected Result \\
{\footnotesize
The \emph{MTDome\_logevent\_azMotion} and
\emph{MTDome\_logevent\_elMotion} inPosition parameter = true.

}

\begin{tabular}{p{2cm}}
\toprule
Step 166  \\ \hline
\end{tabular}
 Description \\
{\footnotesize
\textbf{Point the TMA}\\
Command the TMA to {Pointing 4}⁠ at {270}⁠ , {85}⁠ .

}
\hdashrule[0.5ex]{\textwidth}{1pt}{3mm}
  Expected Result \\
{\footnotesize
The TMA starts moving

}

\begin{tabular}{p{2cm}}
\toprule
Step 167  \\ \hline
\end{tabular}
 Description \\
{\footnotesize
Wait for the TMA to reach the commanded position.

}
\hdashrule[0.5ex]{\textwidth}{1pt}{3mm}
  Expected Result \\
{\footnotesize
The \emph{MTMount\_logevent\_azimuthInPosition} and
\emph{MTMount\_logevent\_elevationInPosition} inPosition parameter =
true.

}

\begin{tabular}{p{2cm}}
\toprule
Step 168  \\ \hline
\end{tabular}
 Description \\
{\footnotesize
\textbf{Image preparation}\\
If the preparation to take images takes longer than 10sec, do
repositioning to target {{{270}⁠}}, {{{85}⁠~}}.

}
\hdashrule[0.5ex]{\textwidth}{1pt}{3mm}
  Expected Result \\
{\footnotesize
TMA reaches the commanded position.

}

\begin{tabular}{p{2cm}}
\toprule
Step 169  \\ \hline
\end{tabular}
 Description \\
{\footnotesize
\textbf{Track position and take images}\\[2\baselineskip]Take a
StarTracker image with 10s exposure time.\\[2\baselineskip]If the time
the available:

\begin{itemize}
\tightlist
\item
  Track a position for 10 min and take StarTracker images.
\end{itemize}

}
\hdashrule[0.5ex]{\textwidth}{1pt}{3mm}
  Expected Result \\
{\footnotesize
\begin{itemize}
\tightlist
\item
  If time is available: The TMA is tracking a given position for 10 min
  and taking images.
\item
  At least one image is successfully taken with the StarTracker.
\end{itemize}

}

\begin{tabular}{p{2cm}}
\toprule
Step 170  \\ \hline
\end{tabular}
 Description \\
{\footnotesize
\textbf{On-the-fly Image Quality Check}\\
While tracking and taking images, check the images on RubinTV for an
astrometric solution.

}
\hdashrule[0.5ex]{\textwidth}{1pt}{3mm}
  Expected Result \\
{\footnotesize
RubinTV is showing an astrometric solution.

}

\begin{tabular}{p{2cm}}
\toprule
Step 171  \\ \hline
\end{tabular}
 Description \\
{\footnotesize
\textbf{Offline analysis results}\\
Offline analysis in Test case
\href{https://jira.lsstcorp.org/secure/Tests.jspa\#/testCase/LVV-T2739}{LVV-T2739}
Says that we do not have sufficient image quality.

}
\hdashrule[0.5ex]{\textwidth}{1pt}{3mm}
  Expected Result \\
{\footnotesize
Image quality is sufficient.

}

\begin{tabular}{p{2cm}}
\toprule
Step 172  \\ \hline
\end{tabular}
 Description \\
{\footnotesize
Wait for the TMA to reach the commanded position.

}
\hdashrule[0.5ex]{\textwidth}{1pt}{3mm}
  Expected Result \\
{\footnotesize
The \emph{MTMount\_logevent\_azimuthInPosition} and
\emph{MTMount\_logevent\_elevationInPosition} inPosition parameter =
true.

}

\begin{tabular}{p{2cm}}
\toprule
Step 173  \\ \hline
\end{tabular}
 Description \\
{\footnotesize
\textbf{Point} \textbf{the Dome:}\\
Command the Dome to {Pointing 5}⁠ to {180}⁠

}
\hdashrule[0.5ex]{\textwidth}{1pt}{3mm}
  Expected Result \\
{\footnotesize
The Dome starts moving.

}

\begin{tabular}{p{2cm}}
\toprule
Step 174  \\ \hline
\end{tabular}
 Description \\
{\footnotesize
Wait for the Dome to reach the commanded position.

}
\hdashrule[0.5ex]{\textwidth}{1pt}{3mm}
  Expected Result \\
{\footnotesize
The \emph{MTDome\_logevent\_azMotion} and
\emph{MTDome\_logevent\_elMotion} inPosition parameter = true.

}

\begin{tabular}{p{2cm}}
\toprule
Step 175  \\ \hline
\end{tabular}
 Description \\
{\footnotesize
\textbf{Point the TMA}\\
Command the TMA to {Pointing 5}⁠ at {180}⁠ , {85}⁠ .

}
\hdashrule[0.5ex]{\textwidth}{1pt}{3mm}
  Expected Result \\
{\footnotesize
The TMA starts moving

}

\begin{tabular}{p{2cm}}
\toprule
Step 176  \\ \hline
\end{tabular}
 Description \\
{\footnotesize
Wait for the TMA to reach the commanded position.

}
\hdashrule[0.5ex]{\textwidth}{1pt}{3mm}
  Expected Result \\
{\footnotesize
The \emph{MTMount\_logevent\_azimuthInPosition} and
\emph{MTMount\_logevent\_elevationInPosition} inPosition parameter =
true.

}

\begin{tabular}{p{2cm}}
\toprule
Step 177  \\ \hline
\end{tabular}
 Description \\
{\footnotesize
\textbf{Image preparation}\\
If the preparation to take images takes longer than 10sec, do
repositioning to target {{{180}⁠}}, {{{85}⁠~}}.

}
\hdashrule[0.5ex]{\textwidth}{1pt}{3mm}
  Expected Result \\
{\footnotesize
TMA reaches the commanded position.

}

\begin{tabular}{p{2cm}}
\toprule
Step 178  \\ \hline
\end{tabular}
 Description \\
{\footnotesize
\textbf{Track position and take images}\\[2\baselineskip]Take a
StarTracker image with 10s exposure time.\\[2\baselineskip]If the time
the available:

\begin{itemize}
\tightlist
\item
  Track a position for 10 min and take StarTracker images.
\end{itemize}

}
\hdashrule[0.5ex]{\textwidth}{1pt}{3mm}
  Expected Result \\
{\footnotesize
\begin{itemize}
\tightlist
\item
  If time is available: The TMA is tracking a given position for 10 min
  and taking images.
\item
  At least one image is successfully taken with the StarTracker.
\end{itemize}

}

\begin{tabular}{p{2cm}}
\toprule
Step 179  \\ \hline
\end{tabular}
 Description \\
{\footnotesize
\textbf{On-the-fly Image Quality Check}\\
While tracking and taking images, check the images on RubinTV for an
astrometric solution.

}
\hdashrule[0.5ex]{\textwidth}{1pt}{3mm}
  Expected Result \\
{\footnotesize
RubinTV is showing an astrometric solution.

}

\begin{tabular}{p{2cm}}
\toprule
Step 180  \\ \hline
\end{tabular}
 Description \\
{\footnotesize
\textbf{Offline analysis results}\\
Offline analysis in Test case
\href{https://jira.lsstcorp.org/secure/Tests.jspa\#/testCase/LVV-T2739}{LVV-T2739}
Says that we do not have sufficient image quality.

}
\hdashrule[0.5ex]{\textwidth}{1pt}{3mm}
  Expected Result \\
{\footnotesize
Image quality is sufficient.

}

\begin{tabular}{p{2cm}}
\toprule
Step 181  \\ \hline
\end{tabular}
 Description \\
{\footnotesize
\textbf{Image preparation}\\
If the preparation to take images takes longer than 10sec, do
repositioning to target {{{270}⁠}}, {{{15}⁠~}}.

}
\hdashrule[0.5ex]{\textwidth}{1pt}{3mm}
  Expected Result \\
{\footnotesize
TMA reaches the commanded position.

}

\begin{tabular}{p{2cm}}
\toprule
Step 182  \\ \hline
\end{tabular}
 Description \\
{\footnotesize
\textbf{Point} \textbf{the Dome:}\\
Command the Dome to {Pointing 7}⁠ to {180}⁠

}
\hdashrule[0.5ex]{\textwidth}{1pt}{3mm}
  Expected Result \\
{\footnotesize
The Dome starts moving.

}

\begin{tabular}{p{2cm}}
\toprule
Step 183  \\ \hline
\end{tabular}
 Description \\
{\footnotesize
Wait for the Dome to reach the commanded position.

}
\hdashrule[0.5ex]{\textwidth}{1pt}{3mm}
  Expected Result \\
{\footnotesize
The \emph{MTDome\_logevent\_azMotion} and
\emph{MTDome\_logevent\_elMotion} inPosition parameter = true.

}

\begin{tabular}{p{2cm}}
\toprule
Step 184  \\ \hline
\end{tabular}
 Description \\
{\footnotesize
\textbf{Point the TMA}\\
Command the TMA to {Pointing 7}⁠ at {180}⁠ , {45}⁠ .

}
\hdashrule[0.5ex]{\textwidth}{1pt}{3mm}
  Expected Result \\
{\footnotesize
The TMA starts moving

}

\begin{tabular}{p{2cm}}
\toprule
Step 185  \\ \hline
\end{tabular}
 Description \\
{\footnotesize
Wait for the TMA to reach the commanded position.

}
\hdashrule[0.5ex]{\textwidth}{1pt}{3mm}
  Expected Result \\
{\footnotesize
The \emph{MTMount\_logevent\_azimuthInPosition} and
\emph{MTMount\_logevent\_elevationInPosition} inPosition parameter =
true.

}

\begin{tabular}{p{2cm}}
\toprule
Step 186  \\ \hline
\end{tabular}
 Description \\
{\footnotesize
\textbf{Image preparation}\\
If the preparation to take images takes longer than 10sec, do
repositioning to target {{{180}⁠}}, {{{45}⁠~}}.

}
\hdashrule[0.5ex]{\textwidth}{1pt}{3mm}
  Expected Result \\
{\footnotesize
TMA reaches the commanded position.

}

\begin{tabular}{p{2cm}}
\toprule
Step 187  \\ \hline
\end{tabular}
 Description \\
{\footnotesize
\textbf{Track position and take images}\\[2\baselineskip]Take a
StarTracker image with 10s exposure time.\\[2\baselineskip]If the time
the available:

\begin{itemize}
\tightlist
\item
  Track a position for 10 min and take StarTracker images.
\end{itemize}

}
\hdashrule[0.5ex]{\textwidth}{1pt}{3mm}
  Expected Result \\
{\footnotesize
\begin{itemize}
\tightlist
\item
  If time is available: The TMA is tracking a given position for 10 min
  and taking images.
\item
  At least one image is successfully taken with the StarTracker.
\end{itemize}

}

\begin{tabular}{p{2cm}}
\toprule
Step 188  \\ \hline
\end{tabular}
 Description \\
{\footnotesize
\textbf{On-the-fly Image Quality Check}\\
While tracking and taking images, check the images on RubinTV for an
astrometric solution.

}
\hdashrule[0.5ex]{\textwidth}{1pt}{3mm}
  Expected Result \\
{\footnotesize
RubinTV is showing an astrometric solution.

}

\begin{tabular}{p{2cm}}
\toprule
Step 189  \\ \hline
\end{tabular}
 Description \\
{\footnotesize
\textbf{Offline analysis results}\\
Offline analysis in Test case
\href{https://jira.lsstcorp.org/secure/Tests.jspa\#/testCase/LVV-T2739}{LVV-T2739}
Says that we do not have sufficient image quality.

}
\hdashrule[0.5ex]{\textwidth}{1pt}{3mm}
  Expected Result \\
{\footnotesize
Image quality is sufficient.

}

\begin{tabular}{p{2cm}}
\toprule
Step 190  \\ \hline
\end{tabular}
 Description \\
{\footnotesize
\textbf{Track position and take images}\\[2\baselineskip]Take a
StarTracker image with 10s exposure time.\\[2\baselineskip]If the time
the available:

\begin{itemize}
\tightlist
\item
  Track a position for 10 min and take StarTracker images.
\end{itemize}

}
\hdashrule[0.5ex]{\textwidth}{1pt}{3mm}
  Expected Result \\
{\footnotesize
\begin{itemize}
\tightlist
\item
  If time is available: The TMA is tracking a given position for 10 min
  and taking images.
\item
  At least one image is successfully taken with the StarTracker.
\end{itemize}

}

\begin{tabular}{p{2cm}}
\toprule
Step 191  \\ \hline
\end{tabular}
 Description \\
{\footnotesize
\textbf{Point} \textbf{the Dome:}\\
Command the Dome to {Pointing 8}⁠ to {180}⁠

}
\hdashrule[0.5ex]{\textwidth}{1pt}{3mm}
  Expected Result \\
{\footnotesize
The Dome starts moving.

}

\begin{tabular}{p{2cm}}
\toprule
Step 192  \\ \hline
\end{tabular}
 Description \\
{\footnotesize
Wait for the Dome to reach the commanded position.

}
\hdashrule[0.5ex]{\textwidth}{1pt}{3mm}
  Expected Result \\
{\footnotesize
The \emph{MTDome\_logevent\_azMotion} and
\emph{MTDome\_logevent\_elMotion} inPosition parameter = true.

}

\begin{tabular}{p{2cm}}
\toprule
Step 193  \\ \hline
\end{tabular}
 Description \\
{\footnotesize
\textbf{Point the TMA}\\
Command the TMA to {Pointing 8}⁠ at {180}⁠ , {15}⁠ .

}
\hdashrule[0.5ex]{\textwidth}{1pt}{3mm}
  Expected Result \\
{\footnotesize
The TMA starts moving

}

\begin{tabular}{p{2cm}}
\toprule
Step 194  \\ \hline
\end{tabular}
 Description \\
{\footnotesize
Wait for the TMA to reach the commanded position.

}
\hdashrule[0.5ex]{\textwidth}{1pt}{3mm}
  Expected Result \\
{\footnotesize
The \emph{MTMount\_logevent\_azimuthInPosition} and
\emph{MTMount\_logevent\_elevationInPosition} inPosition parameter =
true.

}

\begin{tabular}{p{2cm}}
\toprule
Step 195  \\ \hline
\end{tabular}
 Description \\
{\footnotesize
\textbf{Image preparation}\\
If the preparation to take images takes longer than 10sec, do
repositioning to target {{{180}⁠}}, {{{15}⁠~}}.

}
\hdashrule[0.5ex]{\textwidth}{1pt}{3mm}
  Expected Result \\
{\footnotesize
TMA reaches the commanded position.

}

\begin{tabular}{p{2cm}}
\toprule
Step 196  \\ \hline
\end{tabular}
 Description \\
{\footnotesize
\textbf{Track position and take images}\\[2\baselineskip]Take a
StarTracker image with 10s exposure time.\\[2\baselineskip]If the time
the available:

\begin{itemize}
\tightlist
\item
  Track a position for 10 min and take StarTracker images.
\end{itemize}

}
\hdashrule[0.5ex]{\textwidth}{1pt}{3mm}
  Expected Result \\
{\footnotesize
\begin{itemize}
\tightlist
\item
  If time is available: The TMA is tracking a given position for 10 min
  and taking images.
\item
  At least one image is successfully taken with the StarTracker.
\end{itemize}

}

\begin{tabular}{p{2cm}}
\toprule
Step 197  \\ \hline
\end{tabular}
 Description \\
{\footnotesize
\textbf{On-the-fly Image Quality Check}\\
While tracking and taking images, check the images on RubinTV for an
astrometric solution.

}
\hdashrule[0.5ex]{\textwidth}{1pt}{3mm}
  Expected Result \\
{\footnotesize
RubinTV is showing an astrometric solution.

}

\begin{tabular}{p{2cm}}
\toprule
Step 198  \\ \hline
\end{tabular}
 Description \\
{\footnotesize
\textbf{Offline analysis results}\\
Offline analysis in Test case
\href{https://jira.lsstcorp.org/secure/Tests.jspa\#/testCase/LVV-T2739}{LVV-T2739}
Says that we do not have sufficient image quality.

}
\hdashrule[0.5ex]{\textwidth}{1pt}{3mm}
  Expected Result \\
{\footnotesize
Image quality is sufficient.

}

\begin{tabular}{p{2cm}}
\toprule
Step 199  \\ \hline
\end{tabular}
 Description \\
{\footnotesize
\textbf{On-the-fly Image Quality Check}\\
While tracking and taking images, check the images on RubinTV for an
astrometric solution.

}
\hdashrule[0.5ex]{\textwidth}{1pt}{3mm}
  Expected Result \\
{\footnotesize
RubinTV is showing an astrometric solution.

}

\begin{tabular}{p{2cm}}
\toprule
Step 200  \\ \hline
\end{tabular}
 Description \\
{\footnotesize
\textbf{Point} \textbf{the Dome:}\\
Command the Dome to {Pointing 9}⁠ to {90}⁠

}
\hdashrule[0.5ex]{\textwidth}{1pt}{3mm}
  Expected Result \\
{\footnotesize
The Dome starts moving.

}

\begin{tabular}{p{2cm}}
\toprule
Step 201  \\ \hline
\end{tabular}
 Description \\
{\footnotesize
Wait for the Dome to reach the commanded position.

}
\hdashrule[0.5ex]{\textwidth}{1pt}{3mm}
  Expected Result \\
{\footnotesize
The \emph{MTDome\_logevent\_azMotion} and
\emph{MTDome\_logevent\_elMotion} inPosition parameter = true.

}

\begin{tabular}{p{2cm}}
\toprule
Step 202  \\ \hline
\end{tabular}
 Description \\
{\footnotesize
\textbf{Point the TMA}\\
Command the TMA to {Pointing 9}⁠ at {90}⁠ , {15}⁠ .

}
\hdashrule[0.5ex]{\textwidth}{1pt}{3mm}
  Expected Result \\
{\footnotesize
The TMA starts moving

}

\begin{tabular}{p{2cm}}
\toprule
Step 203  \\ \hline
\end{tabular}
 Description \\
{\footnotesize
Wait for the TMA to reach the commanded position.

}
\hdashrule[0.5ex]{\textwidth}{1pt}{3mm}
  Expected Result \\
{\footnotesize
The \emph{MTMount\_logevent\_azimuthInPosition} and
\emph{MTMount\_logevent\_elevationInPosition} inPosition parameter =
true.

}

\begin{tabular}{p{2cm}}
\toprule
Step 204  \\ \hline
\end{tabular}
 Description \\
{\footnotesize
\textbf{Image preparation}\\
If the preparation to take images takes longer than 10sec, do
repositioning to target {{{90}⁠}}, {{{15}⁠~}}.

}
\hdashrule[0.5ex]{\textwidth}{1pt}{3mm}
  Expected Result \\
{\footnotesize
TMA reaches the commanded position.

}

\begin{tabular}{p{2cm}}
\toprule
Step 205  \\ \hline
\end{tabular}
 Description \\
{\footnotesize
\textbf{Track position and take images}\\[2\baselineskip]Take a
StarTracker image with 10s exposure time.\\[2\baselineskip]If the time
the available:

\begin{itemize}
\tightlist
\item
  Track a position for 10 min and take StarTracker images.
\end{itemize}

}
\hdashrule[0.5ex]{\textwidth}{1pt}{3mm}
  Expected Result \\
{\footnotesize
\begin{itemize}
\tightlist
\item
  If time is available: The TMA is tracking a given position for 10 min
  and taking images.
\item
  At least one image is successfully taken with the StarTracker.
\end{itemize}

}

\begin{tabular}{p{2cm}}
\toprule
Step 206  \\ \hline
\end{tabular}
 Description \\
{\footnotesize
\textbf{On-the-fly Image Quality Check}\\
While tracking and taking images, check the images on RubinTV for an
astrometric solution.

}
\hdashrule[0.5ex]{\textwidth}{1pt}{3mm}
  Expected Result \\
{\footnotesize
RubinTV is showing an astrometric solution.

}

\begin{tabular}{p{2cm}}
\toprule
Step 207  \\ \hline
\end{tabular}
 Description \\
{\footnotesize
\textbf{Offline analysis results}\\
Offline analysis in Test case
\href{https://jira.lsstcorp.org/secure/Tests.jspa\#/testCase/LVV-T2739}{LVV-T2739}
Says that we do not have sufficient image quality.

}
\hdashrule[0.5ex]{\textwidth}{1pt}{3mm}
  Expected Result \\
{\footnotesize
Image quality is sufficient.

}

\begin{tabular}{p{2cm}}
\toprule
Step 208  \\ \hline
\end{tabular}
 Description \\
{\footnotesize
\textbf{Offline analysis results}\\
Offline analysis in Test case
\href{https://jira.lsstcorp.org/secure/Tests.jspa\#/testCase/LVV-T2739}{LVV-T2739}
Says that we do not have sufficient image quality.

}
\hdashrule[0.5ex]{\textwidth}{1pt}{3mm}
  Expected Result \\
{\footnotesize
Image quality is sufficient.

}

\begin{tabular}{p{2cm}}
\toprule
Step 209  \\ \hline
\end{tabular}
 Description \\
{\footnotesize
\textbf{Point} \textbf{the Dome:}\\
Command the Dome to {Pointing 10}⁠ to {90}⁠

}
\hdashrule[0.5ex]{\textwidth}{1pt}{3mm}
  Expected Result \\
{\footnotesize
The Dome starts moving.

}

\begin{tabular}{p{2cm}}
\toprule
Step 210  \\ \hline
\end{tabular}
 Description \\
{\footnotesize
Wait for the Dome to reach the commanded position.

}
\hdashrule[0.5ex]{\textwidth}{1pt}{3mm}
  Expected Result \\
{\footnotesize
The \emph{MTDome\_logevent\_azMotion} and
\emph{MTDome\_logevent\_elMotion} inPosition parameter = true.

}

\begin{tabular}{p{2cm}}
\toprule
Step 211  \\ \hline
\end{tabular}
 Description \\
{\footnotesize
\textbf{Point the TMA}\\
Command the TMA to {Pointing 10}⁠ at {90}⁠ , {45}⁠ .

}
\hdashrule[0.5ex]{\textwidth}{1pt}{3mm}
  Expected Result \\
{\footnotesize
The TMA starts moving

}

\begin{tabular}{p{2cm}}
\toprule
Step 212  \\ \hline
\end{tabular}
 Description \\
{\footnotesize
Wait for the TMA to reach the commanded position.

}
\hdashrule[0.5ex]{\textwidth}{1pt}{3mm}
  Expected Result \\
{\footnotesize
The \emph{MTMount\_logevent\_azimuthInPosition} and
\emph{MTMount\_logevent\_elevationInPosition} inPosition parameter =
true.

}

\begin{tabular}{p{2cm}}
\toprule
Step 213  \\ \hline
\end{tabular}
 Description \\
{\footnotesize
\textbf{Image preparation}\\
If the preparation to take images takes longer than 10sec, do
repositioning to target {{{90}⁠}}, {{{45}⁠~}}.

}
\hdashrule[0.5ex]{\textwidth}{1pt}{3mm}
  Expected Result \\
{\footnotesize
TMA reaches the commanded position.

}

\begin{tabular}{p{2cm}}
\toprule
Step 214  \\ \hline
\end{tabular}
 Description \\
{\footnotesize
\textbf{Track position and take images}\\[2\baselineskip]Take a
StarTracker image with 10s exposure time.\\[2\baselineskip]If the time
the available:

\begin{itemize}
\tightlist
\item
  Track a position for 10 min and take StarTracker images.
\end{itemize}

}
\hdashrule[0.5ex]{\textwidth}{1pt}{3mm}
  Expected Result \\
{\footnotesize
\begin{itemize}
\tightlist
\item
  If time is available: The TMA is tracking a given position for 10 min
  and taking images.
\item
  At least one image is successfully taken with the StarTracker.
\end{itemize}

}

\begin{tabular}{p{2cm}}
\toprule
Step 215  \\ \hline
\end{tabular}
 Description \\
{\footnotesize
\textbf{On-the-fly Image Quality Check}\\
While tracking and taking images, check the images on RubinTV for an
astrometric solution.

}
\hdashrule[0.5ex]{\textwidth}{1pt}{3mm}
  Expected Result \\
{\footnotesize
RubinTV is showing an astrometric solution.

}

\begin{tabular}{p{2cm}}
\toprule
Step 216  \\ \hline
\end{tabular}
 Description \\
{\footnotesize
\textbf{Offline analysis results}\\
Offline analysis in Test case
\href{https://jira.lsstcorp.org/secure/Tests.jspa\#/testCase/LVV-T2739}{LVV-T2739}
Says that we do not have sufficient image quality.

}
\hdashrule[0.5ex]{\textwidth}{1pt}{3mm}
  Expected Result \\
{\footnotesize
Image quality is sufficient.

}

\begin{tabular}{p{2cm}}
\toprule
Step 217  \\ \hline
\end{tabular}
 Description \\
{\footnotesize
\textbf{Point} \textbf{the Dome:}\\
Command the Dome to {Pointing 12}⁠ to {90}⁠

}
\hdashrule[0.5ex]{\textwidth}{1pt}{3mm}
  Expected Result \\
{\footnotesize
The Dome starts moving.

}

\begin{tabular}{p{2cm}}
\toprule
Step 218  \\ \hline
\end{tabular}
 Description \\
{\footnotesize
Wait for the Dome to reach the commanded position.

}
\hdashrule[0.5ex]{\textwidth}{1pt}{3mm}
  Expected Result \\
{\footnotesize
The \emph{MTDome\_logevent\_azMotion} and
\emph{MTDome\_logevent\_elMotion} inPosition parameter = true.

}

\begin{tabular}{p{2cm}}
\toprule
Step 219  \\ \hline
\end{tabular}
 Description \\
{\footnotesize
\textbf{Point the TMA}\\
Command the TMA to {Pointing 12}⁠ at {90}⁠ , {85}⁠ .

}
\hdashrule[0.5ex]{\textwidth}{1pt}{3mm}
  Expected Result \\
{\footnotesize
The TMA starts moving

}

\begin{tabular}{p{2cm}}
\toprule
Step 220  \\ \hline
\end{tabular}
 Description \\
{\footnotesize
Wait for the TMA to reach the commanded position.

}
\hdashrule[0.5ex]{\textwidth}{1pt}{3mm}
  Expected Result \\
{\footnotesize
The \emph{MTMount\_logevent\_azimuthInPosition} and
\emph{MTMount\_logevent\_elevationInPosition} inPosition parameter =
true.

}

\begin{tabular}{p{2cm}}
\toprule
Step 221  \\ \hline
\end{tabular}
 Description \\
{\footnotesize
\textbf{Image preparation}\\
If the preparation to take images takes longer than 10sec, do
repositioning to target {{{90}⁠}}, {{{85}⁠~}}.

}
\hdashrule[0.5ex]{\textwidth}{1pt}{3mm}
  Expected Result \\
{\footnotesize
TMA reaches the commanded position.

}

\begin{tabular}{p{2cm}}
\toprule
Step 222  \\ \hline
\end{tabular}
 Description \\
{\footnotesize
\textbf{Track position and take images}\\[2\baselineskip]Take a
StarTracker image with 10s exposure time.\\[2\baselineskip]If the time
the available:

\begin{itemize}
\tightlist
\item
  Track a position for 10 min and take StarTracker images.
\end{itemize}

}
\hdashrule[0.5ex]{\textwidth}{1pt}{3mm}
  Expected Result \\
{\footnotesize
\begin{itemize}
\tightlist
\item
  If time is available: The TMA is tracking a given position for 10 min
  and taking images.
\item
  At least one image is successfully taken with the StarTracker.
\end{itemize}

}

\begin{tabular}{p{2cm}}
\toprule
Step 223  \\ \hline
\end{tabular}
 Description \\
{\footnotesize
\textbf{On-the-fly Image Quality Check}\\
While tracking and taking images, check the images on RubinTV for an
astrometric solution.

}
\hdashrule[0.5ex]{\textwidth}{1pt}{3mm}
  Expected Result \\
{\footnotesize
RubinTV is showing an astrometric solution.

}

\begin{tabular}{p{2cm}}
\toprule
Step 224  \\ \hline
\end{tabular}
 Description \\
{\footnotesize
\textbf{Offline analysis results}\\
Offline analysis in Test case
\href{https://jira.lsstcorp.org/secure/Tests.jspa\#/testCase/LVV-T2739}{LVV-T2739}
Says that we do not have sufficient image quality.

}
\hdashrule[0.5ex]{\textwidth}{1pt}{3mm}
  Expected Result \\
{\footnotesize
Image quality is sufficient.

}

\paragraph{ LVV-T2715 - Configure Observatory Environment for Daytime Operations }\mbox{}\\

Version \textbf{1}.
Open  \href{https://jira.lsstcorp.org/secure/Tests.jspa#/testCase/LVV-T2715}{\textit{ LVV-T2715 } }
test case in Jira.

After using the observatory during the nighttime, prepare the
observatory for daytime operations.

\textbf{ Preconditions}:\\
The observatory was used during nighttime.

Final comment:\\


Detailed steps :

\begin{tabular}{p{2cm}}
\toprule
Step 1  \\ \hline
\end{tabular}
 Description \\
{\footnotesize
\textbf{CSCs}\\

\begin{itemize}
\tightlist
\item
  Transition the CSCs into STANDBY state
\end{itemize}

}
\hdashrule[0.5ex]{\textwidth}{1pt}{3mm}
  Expected Result \\
{\footnotesize
All CSCs are in their standbyState.

}

\begin{tabular}{p{2cm}}
\toprule
Step 2  \\ \hline
\end{tabular}
 Description \\
{\footnotesize
\textbf{Telescope daytime preparations:}

\begin{itemize}
\tightlist
\item
  Switch off or bring into standby the StarTracker and DIMM instruments
\item
  Install the caps on top of the StarTracker telescopes and the DIMM
\end{itemize}

}
\hdashrule[0.5ex]{\textwidth}{1pt}{3mm}
  Expected Result \\
{\footnotesize
The caps are installed.

}

\begin{tabular}{p{2cm}}
\toprule
Step 3  \\ \hline
\end{tabular}
 Description \\
{\footnotesize
\textbf{Dome:}\\

\begin{itemize}
\tightlist
\item
  Bring the dome into the park position
\end{itemize}

Until the dome shutter is motorized:\textbf{\\
}

\begin{itemize}
\tightlist
\item
  Send a message to the site manager :

  \begin{itemize}
  \tightlist
  \item
    confirming that nightly operations have finished~
  \item
    asking for a dome closer before the sun starts to shine on the
    StarTracker and the DIMM.
  \end{itemize}
\end{itemize}

}
\hdashrule[0.5ex]{\textwidth}{1pt}{3mm}
  Expected Result \\
{\footnotesize
Dome closure is organized.

}

\begin{tabular}{p{2cm}}
\toprule
Step 4  \\ \hline
\end{tabular}
 Description \\
{\footnotesize
\textbf{Auxillary systems~daytime preparations:}\\
If needed for daytime operations:\textbf{\\
}

\begin{itemize}
\tightlist
\item
  Switch on the UMA in the morning.
\item
  When available and need to be modified for the day:

  \begin{itemize}
  \tightlist
  \item
    Oil supply system on standby?
  \item
    Dynalyne into standby?
  \end{itemize}
\end{itemize}

}
\hdashrule[0.5ex]{\textwidth}{1pt}{3mm}
  Expected Result \\
{\footnotesize
All auxiliary systems are in the states suitable for daytime operations.

\begin{itemize}
\tightlist
\item
  The UMA is switched on.
\end{itemize}

}

\begin{tabular}{p{2cm}}
\toprule
Step 5  \\ \hline
\end{tabular}
 Description \\
{\footnotesize
\textbf{TMA position in the morning}\\

\begin{itemize}
\tightlist
\item
  Park the TMA in the position needed for the next day.
\end{itemize}

}
\hdashrule[0.5ex]{\textwidth}{1pt}{3mm}
  Expected Result \\
{\footnotesize
TMA parked in the corresponding position.

}

\begin{tabular}{p{2cm}}
\toprule
Step 6  \\ \hline
\end{tabular}
 Description \\
{\footnotesize

}
\hdashrule[0.5ex]{\textwidth}{1pt}{3mm}
  Expected Result \\
{\footnotesize

}

\begin{tabular}{p{2cm}}
\toprule
Step 7  \\ \hline
\end{tabular}
 Description \\
{\footnotesize
\textbf{Night log}\\

\begin{itemize}
\tightlist
\item
  Close the night log by writing a summary of the nightly events
\item
  Send a link with the summary to the site manager.
\end{itemize}

}
\hdashrule[0.5ex]{\textwidth}{1pt}{3mm}
  Expected Result \\
{\footnotesize
The night log is closed.

}

\subsection{Test Cycle LVV-C230 }

Open test cycle {\it \href{https://jira.lsstcorp.org/secure/Tests.jspa#/testrun/LVV-C230}{TMA Pointing and Tracking - Part 4 - Offset 0.2" + Slew and Settle + TMA
Tracking Jitter -- Using the DIMM.}} in Jira.

Test Cycle name: TMA Pointing and Tracking - Part 4 - Offset 0.2" + Slew and Settle + TMA
Tracking Jitter -- Using the DIMM.\\
Status: In Progress

Data collection for the pointing and tracking requirements verification
using the Star Tracker and the DIMM.\\
Star Tracker and the DIMM will be mounted on dedicated connector plates
to the top end of the TMA.\\[2\baselineskip]Slew and Settle:

\begin{itemize}
\tightlist
\item
  Track on position one, slew to position two, and Track on position
  two.
\item
  In four seconds, the mount should settle.
\item
  Timestamps come from the EFD.
\item
  acquire a bright star with the DIMM
\item
  initiate exposure series at \textasciitilde{}75 Hz for 15s. Mean and
  STD of the centroid
\item
  Offset to 3.5 deg. To be reached in 4s. The trigger is the offset
  command. We wait for the MCS command + a reasonable amount of time. To
  be done at survey cadence. No images at the offset position (likely no
  bright star there).
\item
  Issue the second command to go to the first position. We stay for 35
  seconds.
\item
  We use the same star and 5 times 3.5 AZ, 5 times 3.5 deg EL, and 5 x
  3.5 deg random positions.
\end{itemize}

\subsubsection{Software Version/Baseline}
Star Tracker software version:\\
Dimm software version:\\
CSC software version:\\
Analysis software repository:

\subsubsection{Configuration}
Not provided.

\subsubsection{Test Cases in LVV-C230 Test Cycle}

\paragraph{ LVV-T2707 - Evening Summit Tailgate Meeting - TMA and Dome Testing Safety Assurance }\mbox{}\\

Version \textbf{2}.
Open  \href{https://jira.lsstcorp.org/secure/Tests.jspa#/testCase/LVV-T2707}{\textit{ LVV-T2707 } }
test case in Jira.

Ensure the safety of observation with the main telescope during
nighttime operations.\\
\textbf{Tailgate Meeting:} Hold a tailgate for the upcoming task with
personnel on the summit working during the night. Go over any relevant
procedures, roles, and
responsibilities.\\[2\baselineskip]\textbf{Note:~}Version two is for
tests that do not involve moving or opening the dome.

\textbf{ Preconditions}:\\
All nonessential personnel has vacated the area.

Final comment:\\


Detailed steps :

\begin{tabular}{p{2cm}}
\toprule
Step 1  \\ \hline
\end{tabular}
 Description \\
{\footnotesize

}
\hdashrule[0.5ex]{\textwidth}{1pt}{3mm}
  Expected Result \\
{\footnotesize

}

\paragraph{ LVV-T2714 - Configure Observatory Environment for Nighttime Operations }\mbox{}\\

Version \textbf{1}.
Open  \href{https://jira.lsstcorp.org/secure/Tests.jspa#/testCase/LVV-T2714}{\textit{ LVV-T2714 } }
test case in Jira.

At the beginning of the night, prepare the observatory for nightly
operations.

\textbf{ Preconditions}:\\
Dome and the TMA, or at least the TMA must available for observations.

Final comment:\\


Detailed steps :

\begin{tabular}{p{2cm}}
\toprule
Step 1  \\ \hline
\end{tabular}
 Description \\
{\footnotesize
\textbf{Telescope preparation:}

\begin{itemize}
\tightlist
\item
  Remove the caps on top of the StarTracker telescopes and the DIMM.
\item
  Check the instrument's health status by taking a test image.
\end{itemize}

}
\hdashrule[0.5ex]{\textwidth}{1pt}{3mm}
  Expected Result \\
{\footnotesize
The caps are removed.\\
The test image is taken and stored correspondingly.

}

\begin{tabular}{p{2cm}}
\toprule
Step 2  \\ \hline
\end{tabular}
 Description \\
{\footnotesize
\textbf{Calibration images:}\\
Not needed at the moment, to be included if data analysis reveals the
need.

\begin{itemize}
\tightlist
\item
  Take 10 ``darks'' with the StarTracker and the DIMM instruments. Use
  the same exposure time as for the images.
\item
  Take 10 ``sky flats'' with the StarTracker and the DIMM instruments.
\end{itemize}

}
\hdashrule[0.5ex]{\textwidth}{1pt}{3mm}
  Expected Result \\
{\footnotesize
The Darks and flats are stored in the expected location.

}

\begin{tabular}{p{2cm}}
\toprule
Step 3  \\ \hline
\end{tabular}
 Description \\
{\footnotesize
\textbf{Auxillary systems~nighttime~preparations:}

\begin{itemize}
\tightlist
\item
  Switch off the UMA in the afternoon.
\end{itemize}

}
\hdashrule[0.5ex]{\textwidth}{1pt}{3mm}
  Expected Result \\
{\footnotesize
The UMA is switched off.

}

\begin{tabular}{p{2cm}}
\toprule
Step 4  \\ \hline
\end{tabular}
 Description \\
{\footnotesize
\textbf{Night logging page:}\\
Start the night log similar to the AuxTel night
log:\\[2\baselineskip]https://confluence.lsstcorp.org/display/LSSTCOM/Night+Logs\\[2\baselineskip]

}
\hdashrule[0.5ex]{\textwidth}{1pt}{3mm}
  Expected Result \\
{\footnotesize
Page created with template information.

}

\begin{tabular}{p{2cm}}
\toprule
Step 5  \\ \hline
\end{tabular}
 Description \\
{\footnotesize
\textbf{TMA preparation}

\begin{itemize}
\tightlist
\item
  Check the Oil Supply System (OSS) on the EUI
\item
  Follow the attached manual to startup the TMA.
\end{itemize}

}
\hdashrule[0.5ex]{\textwidth}{1pt}{3mm}
  Expected Result \\
{\footnotesize
The OSS is operational:

}

\begin{tabular}{p{2cm}}
\toprule
Step 6  \\ \hline
\end{tabular}
 Description \\
{\footnotesize
\textbf{CSC activation:}\\[2\baselineskip]Use L.O.V.E to bring the CSC
to the enabled state.\\[2\baselineskip]

}
\hdashrule[0.5ex]{\textwidth}{1pt}{3mm}
  Expected Result \\
{\footnotesize
All needed CSCs are in the enabled state.

}

\paragraph{ LVV-T2732 - StarTracker Pointing and Tracking Test - Pointing Offset 0.2" - Slew and
Settle - TMA Tracking Jitter Validation -- DIMM }\mbox{}\\

Version \textbf{1}.
Open  \href{https://jira.lsstcorp.org/secure/Tests.jspa#/testCase/LVV-T2732}{\textit{ LVV-T2732 } }
test case in Jira.

The objective of this test is

\begin{itemize}
\tightlist
\item
  the TMA achieves pointing repeatability within the value of 1arcsec
  RMS for any motion within the pointing range. (LTS-103-REQ-0011-V-01:
  2.1.5\_1)
\item
  to characterize the settling time and settling behavior after a move.~
\end{itemize}

The observations for the TMA settling characterization are done by

\begin{enumerate}
\tightlist
\item
  Centering star on the DIMM~
\item
  Slew off and back to center a star with the active damping active.
\item
  Slew off and back to center a star with the active damping
  deactivated.
\item
  Analyze the data using
  \href{https://jira.lsstcorp.org/secure/Tests.jspa\#/testCase/LVV-T2749}{LVV-T2749}\\[2\baselineskip]
\end{enumerate}

TMA Tracking Jitter Validation:

\begin{itemize}
\tightlist
\item
  Using the DIMM. Tracking siderially is obligatory.
\item
  Encoder stream at 50 Hz, Nyquist sampling at 100 -150 Hz.
\item
  The standard frequency is 75Hz smaller frames allow for higher.
\item
  StarTracker and DIMM are mounted and working at the same time.
\end{itemize}

\textbf{NOTE:}

\begin{itemize}
\tightlist
\item
  The Dome needs an offset of a min of 0.6 deg for the TMA motion during
  tracking, or else the Dome would need to be moved.
\item
  If not stated otherwise, the TMA damping is turned on.
\end{itemize}

\textbf{ Preconditions}:\\
These preconditions are taken from the FAT:\\

\begin{itemize}
\tightlist
\item
  The TMA interlock system is fully operative and tested.
\item
  MCS is ready for operation.
\item
  The encoder system is active and calibrated.
\item
  All the TMA subsystems hardware (mechanical and electrical) is
  available for operation and fully connected.
\item
  No alarms are active in TMA IS.
\end{itemize}

Preconditions for the summit tests:\\[2\baselineskip]This test case
needs the DIMM to reach the needed precision. The DIMM must be installed
and working.

Final comment:\\


Detailed steps :

\begin{tabular}{p{2cm}}
\toprule
Step 1  \\ \hline
\end{tabular}
 Description \\
{\footnotesize
\textbf{Position the dome}\\
Command the Dome to {Pointing 1}⁠ to {temp}⁠

}
\hdashrule[0.5ex]{\textwidth}{1pt}{3mm}
  Expected Result \\
{\footnotesize
The Dome starts moving.

}

\begin{tabular}{p{2cm}}
\toprule
Step 2  \\ \hline
\end{tabular}
 Description \\
{\footnotesize
\textbf{Point the TMA to (Az, El)-pattern position + DIMM pattern offset
\textbf{\textbf{and take DIMM images}}\\
}

\begin{itemize}
\tightlist
\item
  Point the TMA back to {Pointing 1}⁠ at {-270}⁠ + DIMM offset, {15}⁠ +
  DIMM offset.
\item
  While tracking, take DIMM images with XXXs exposure time and inspect
  the quality.
\end{itemize}

}
\hdashrule[0.5ex]{\textwidth}{1pt}{3mm}
  Expected Result \\
{\footnotesize
\begin{itemize}
\tightlist
\item
  TMA reaches the position.
\item
  DIMM image quality is sufficient
\end{itemize}

}

\begin{tabular}{p{2cm}}
\toprule
Step 3  \\ \hline
\end{tabular}
 Description \\
{\footnotesize
\textbf{Position the dome}\\
Command the Dome to {Pointing 10}⁠ to {\{Dome AZ\}}⁠

}
\hdashrule[0.5ex]{\textwidth}{1pt}{3mm}
  Expected Result \\
{\footnotesize
The Dome starts moving.

}

\begin{tabular}{p{2cm}}
\toprule
Step 4  \\ \hline
\end{tabular}
 Description \\
{\footnotesize
\textbf{Point the TMA to (Az, El)-pattern position + DIMM pattern offset
\textbf{\textbf{and take DIMM images}}\\
}

\begin{itemize}
\tightlist
\item
  Point the TMA back to {Pointing 10}⁠ at {-90}⁠ + DIMM offset, {45}⁠ +
  DIMM offset.
\item
  While tracking, take DIMM images with XXXs exposure time and inspect
  the quality.
\end{itemize}

}
\hdashrule[0.5ex]{\textwidth}{1pt}{3mm}
  Expected Result \\
{\footnotesize
\begin{itemize}
\tightlist
\item
  TMA reaches the position.
\item
  DIMM image quality is sufficient
\end{itemize}

}

\begin{tabular}{p{2cm}}
\toprule
Step 5  \\ \hline
\end{tabular}
 Description \\
{\footnotesize
Wait for the Dome to reach the commanded position.

}
\hdashrule[0.5ex]{\textwidth}{1pt}{3mm}
  Expected Result \\
{\footnotesize
The \emph{MTDome\_logevent\_azMotion} and
\emph{MTDome\_logevent\_elMotion} inPosition parameter = true.

}

\begin{tabular}{p{2cm}}
\toprule
Step 6  \\ \hline
\end{tabular}
 Description \\
{\footnotesize
\textbf{Point the TMA to (Az, El)-pattern position}\\
Point the TMA to {Pointing 10}⁠ at {-90}⁠ , {45}⁠ .

}
\hdashrule[0.5ex]{\textwidth}{1pt}{3mm}
  Expected Result \\
{\footnotesize
The TMA starts moving.

}

\begin{tabular}{p{2cm}}
\toprule
Step 7  \\ \hline
\end{tabular}
 Description \\
{\footnotesize
Wait for the TMA to reach the commanded position.

}
\hdashrule[0.5ex]{\textwidth}{1pt}{3mm}
  Expected Result \\
{\footnotesize
The \emph{MTMount\_logevent\_azimuthInPosition} and
\emph{MTMount\_logevent\_elevationInPosition} inPosition parameter =
true.

}

\begin{tabular}{p{2cm}}
\toprule
Step 8  \\ \hline
\end{tabular}
 Description \\
{\footnotesize
\textbf{Find DIMM Object and DIMM Pattern Offset}\\

\begin{itemize}
\tightlist
\item
  While tracking, take a 10-sec exposure with the StarTracker.
\item
  Load the image into an image viewer.
\item
  Overlay the GAIA catalog.
\item
  Select a star brighter than XXX mag (bright enough for the DIMM).
\item
  Calculate the pixel offset between the StarTracker and the DIMM.
\item
  Transform the offset into AZ and EL offsets.
\end{itemize}

}
\hdashrule[0.5ex]{\textwidth}{1pt}{3mm}
  Expected Result \\
{\footnotesize
\begin{itemize}
\tightlist
\item
  An image was successfully taken with the StarTracker and is of
  sufficient quality.
\item
  AZ and EL offsets are available.
\end{itemize}

}

\begin{tabular}{p{2cm}}
\toprule
Step 9  \\ \hline
\end{tabular}
 Description \\
{\footnotesize
\textbf{Move TMA to the 1. random distance of 3.5deg}\\
Point the TMA to a random 3.5 deg combined offset in AZ and EL from
{Pointing 10}⁠ at {-90}⁠, {45}⁠. Record the exact position of the offset
in AZ and El.

}
\hdashrule[0.5ex]{\textwidth}{1pt}{3mm}
  Expected Result \\
{\footnotesize
\begin{itemize}
\tightlist
\item
  The TMA reaches the commanded offset position.
\item
  The \emph{MTMount\_logevent\_azimuthInPosition} and
  \emph{MTMount\_logevent\_elevationInPosition}inPosition parameter =
  true.
\end{itemize}

}

\begin{tabular}{p{2cm}}
\toprule
Step 10  \\ \hline
\end{tabular}
 Description \\
{\footnotesize
\textbf{Position the dome}\\
Command the Dome to {Pointing 11}⁠ to {\{Dome AZ\}}⁠

}
\hdashrule[0.5ex]{\textwidth}{1pt}{3mm}
  Expected Result \\
{\footnotesize
The Dome starts moving.

}

\begin{tabular}{p{2cm}}
\toprule
Step 11  \\ \hline
\end{tabular}
 Description \\
{\footnotesize
\textbf{Point the TMA to (Az, El)-pattern position + DIMM pattern offset
\textbf{\textbf{and take DIMM images}}\\
}

\begin{itemize}
\tightlist
\item
  Point the TMA back to {Pointing 11}⁠ at {-90}⁠ + DIMM offset, {75}⁠ +
  DIMM offset.
\item
  While tracking, take DIMM images with XXXs exposure time and inspect
  the quality.
\end{itemize}

}
\hdashrule[0.5ex]{\textwidth}{1pt}{3mm}
  Expected Result \\
{\footnotesize
\begin{itemize}
\tightlist
\item
  TMA reaches the position.
\item
  DIMM image quality is sufficient
\end{itemize}

}

\begin{tabular}{p{2cm}}
\toprule
Step 12  \\ \hline
\end{tabular}
 Description \\
{\footnotesize
Wait for the Dome to reach the commanded position.

}
\hdashrule[0.5ex]{\textwidth}{1pt}{3mm}
  Expected Result \\
{\footnotesize
The \emph{MTDome\_logevent\_azMotion} and
\emph{MTDome\_logevent\_elMotion} inPosition parameter = true.

}

\begin{tabular}{p{2cm}}
\toprule
Step 13  \\ \hline
\end{tabular}
 Description \\
{\footnotesize
\textbf{Point the TMA to (Az, El)-pattern position}\\
Point the TMA to {Pointing 11}⁠ at {-90}⁠ , {75}⁠ .

}
\hdashrule[0.5ex]{\textwidth}{1pt}{3mm}
  Expected Result \\
{\footnotesize
The TMA starts moving.

}

\begin{tabular}{p{2cm}}
\toprule
Step 14  \\ \hline
\end{tabular}
 Description \\
{\footnotesize
Wait for the TMA to reach the commanded position.

}
\hdashrule[0.5ex]{\textwidth}{1pt}{3mm}
  Expected Result \\
{\footnotesize
The \emph{MTMount\_logevent\_azimuthInPosition} and
\emph{MTMount\_logevent\_elevationInPosition} inPosition parameter =
true.

}

\begin{tabular}{p{2cm}}
\toprule
Step 15  \\ \hline
\end{tabular}
 Description \\
{\footnotesize
\textbf{Find DIMM Object and DIMM Pattern Offset}\\

\begin{itemize}
\tightlist
\item
  While tracking, take a 10-sec exposure with the StarTracker.
\item
  Load the image into an image viewer.
\item
  Overlay the GAIA catalog.
\item
  Select a star brighter than XXX mag (bright enough for the DIMM).
\item
  Calculate the pixel offset between the StarTracker and the DIMM.
\item
  Transform the offset into AZ and EL offsets.
\end{itemize}

}
\hdashrule[0.5ex]{\textwidth}{1pt}{3mm}
  Expected Result \\
{\footnotesize
\begin{itemize}
\tightlist
\item
  An image was successfully taken with the StarTracker and is of
  sufficient quality.
\item
  AZ and EL offsets are available.
\end{itemize}

}

\begin{tabular}{p{2cm}}
\toprule
Step 16  \\ \hline
\end{tabular}
 Description \\
{\footnotesize
\textbf{Move TMA to the 1. random distance of 3.5deg}\\
Point the TMA to a random 3.5 deg combined offset in AZ and EL from
{Pointing 11}⁠ at {-90}⁠, {75}⁠. Record the exact position of the offset
in AZ and El.

}
\hdashrule[0.5ex]{\textwidth}{1pt}{3mm}
  Expected Result \\
{\footnotesize
\begin{itemize}
\tightlist
\item
  The TMA reaches the commanded offset position.
\item
  The \emph{MTMount\_logevent\_azimuthInPosition} and
  \emph{MTMount\_logevent\_elevationInPosition}inPosition parameter =
  true.
\end{itemize}

}

\begin{tabular}{p{2cm}}
\toprule
Step 17  \\ \hline
\end{tabular}
 Description \\
{\footnotesize
\textbf{Position the dome}\\
Command the Dome to {Pointing 12}⁠ to {\{Dome AZ\}}⁠

}
\hdashrule[0.5ex]{\textwidth}{1pt}{3mm}
  Expected Result \\
{\footnotesize
The Dome starts moving.

}

\begin{tabular}{p{2cm}}
\toprule
Step 18  \\ \hline
\end{tabular}
 Description \\
{\footnotesize
\textbf{Point the TMA to (Az, El)-pattern position + DIMM pattern offset
\textbf{\textbf{and take DIMM images}}\\
}

\begin{itemize}
\tightlist
\item
  Point the TMA back to {Pointing 12}⁠ at {-90}⁠ + DIMM offset, {86.5}⁠
  + DIMM offset.
\item
  While tracking, take DIMM images with XXXs exposure time and inspect
  the quality.
\end{itemize}

}
\hdashrule[0.5ex]{\textwidth}{1pt}{3mm}
  Expected Result \\
{\footnotesize
\begin{itemize}
\tightlist
\item
  TMA reaches the position.
\item
  DIMM image quality is sufficient
\end{itemize}

}

\begin{tabular}{p{2cm}}
\toprule
Step 19  \\ \hline
\end{tabular}
 Description \\
{\footnotesize
Wait for the Dome to reach the commanded position.

}
\hdashrule[0.5ex]{\textwidth}{1pt}{3mm}
  Expected Result \\
{\footnotesize
The \emph{MTDome\_logevent\_azMotion} and
\emph{MTDome\_logevent\_elMotion} inPosition parameter = true.

}

\begin{tabular}{p{2cm}}
\toprule
Step 20  \\ \hline
\end{tabular}
 Description \\
{\footnotesize
\textbf{Point the TMA to (Az, El)-pattern position}\\
Point the TMA to {Pointing 12}⁠ at {-90}⁠ , {86.5}⁠ .

}
\hdashrule[0.5ex]{\textwidth}{1pt}{3mm}
  Expected Result \\
{\footnotesize
The TMA starts moving.

}

\begin{tabular}{p{2cm}}
\toprule
Step 21  \\ \hline
\end{tabular}
 Description \\
{\footnotesize
Wait for the TMA to reach the commanded position.

}
\hdashrule[0.5ex]{\textwidth}{1pt}{3mm}
  Expected Result \\
{\footnotesize
The \emph{MTMount\_logevent\_azimuthInPosition} and
\emph{MTMount\_logevent\_elevationInPosition} inPosition parameter =
true.

}

\begin{tabular}{p{2cm}}
\toprule
Step 22  \\ \hline
\end{tabular}
 Description \\
{\footnotesize
\textbf{Find DIMM Object and DIMM Pattern Offset}\\

\begin{itemize}
\tightlist
\item
  While tracking, take a 10-sec exposure with the StarTracker.
\item
  Load the image into an image viewer.
\item
  Overlay the GAIA catalog.
\item
  Select a star brighter than XXX mag (bright enough for the DIMM).
\item
  Calculate the pixel offset between the StarTracker and the DIMM.
\item
  Transform the offset into AZ and EL offsets.
\end{itemize}

}
\hdashrule[0.5ex]{\textwidth}{1pt}{3mm}
  Expected Result \\
{\footnotesize
\begin{itemize}
\tightlist
\item
  An image was successfully taken with the StarTracker and is of
  sufficient quality.
\item
  AZ and EL offsets are available.
\end{itemize}

}

\begin{tabular}{p{2cm}}
\toprule
Step 23  \\ \hline
\end{tabular}
 Description \\
{\footnotesize
\textbf{Move TMA to the 1. random distance of 3.5deg}\\
Point the TMA to a random 3.5 deg combined offset in AZ and EL from
{Pointing 12}⁠ at {-90}⁠, {86.5}⁠. Record the exact position of the
offset in AZ and El.

}
\hdashrule[0.5ex]{\textwidth}{1pt}{3mm}
  Expected Result \\
{\footnotesize
\begin{itemize}
\tightlist
\item
  The TMA reaches the commanded offset position.
\item
  The \emph{MTMount\_logevent\_azimuthInPosition} and
  \emph{MTMount\_logevent\_elevationInPosition}inPosition parameter =
  true.
\end{itemize}

}

\begin{tabular}{p{2cm}}
\toprule
Step 24  \\ \hline
\end{tabular}
 Description \\
{\footnotesize
\textbf{Position the dome}\\
Command the Dome to {Pointing 13}⁠ to {\{Dome AZ\}}⁠

}
\hdashrule[0.5ex]{\textwidth}{1pt}{3mm}
  Expected Result \\
{\footnotesize
The Dome starts moving.

}

\begin{tabular}{p{2cm}}
\toprule
Step 25  \\ \hline
\end{tabular}
 Description \\
{\footnotesize
\textbf{Point the TMA to (Az, El)-pattern position + DIMM pattern offset
\textbf{\textbf{and take DIMM images}}\\
}

\begin{itemize}
\tightlist
\item
  Point the TMA back to {Pointing 13}⁠ at {0}⁠ + DIMM offset, {86.5}⁠ +
  DIMM offset.
\item
  While tracking, take DIMM images with XXXs exposure time and inspect
  the quality.
\end{itemize}

}
\hdashrule[0.5ex]{\textwidth}{1pt}{3mm}
  Expected Result \\
{\footnotesize
\begin{itemize}
\tightlist
\item
  TMA reaches the position.
\item
  DIMM image quality is sufficient
\end{itemize}

}

\begin{tabular}{p{2cm}}
\toprule
Step 26  \\ \hline
\end{tabular}
 Description \\
{\footnotesize
Wait for the Dome to reach the commanded position.

}
\hdashrule[0.5ex]{\textwidth}{1pt}{3mm}
  Expected Result \\
{\footnotesize
The \emph{MTDome\_logevent\_azMotion} and
\emph{MTDome\_logevent\_elMotion} inPosition parameter = true.

}

\begin{tabular}{p{2cm}}
\toprule
Step 27  \\ \hline
\end{tabular}
 Description \\
{\footnotesize
\textbf{Point the TMA to (Az, El)-pattern position}\\
Point the TMA to {Pointing 13}⁠ at {0}⁠ , {86.5}⁠ .

}
\hdashrule[0.5ex]{\textwidth}{1pt}{3mm}
  Expected Result \\
{\footnotesize
The TMA starts moving.

}

\begin{tabular}{p{2cm}}
\toprule
Step 28  \\ \hline
\end{tabular}
 Description \\
{\footnotesize
Wait for the TMA to reach the commanded position.

}
\hdashrule[0.5ex]{\textwidth}{1pt}{3mm}
  Expected Result \\
{\footnotesize
The \emph{MTMount\_logevent\_azimuthInPosition} and
\emph{MTMount\_logevent\_elevationInPosition} inPosition parameter =
true.

}

\begin{tabular}{p{2cm}}
\toprule
Step 29  \\ \hline
\end{tabular}
 Description \\
{\footnotesize
\textbf{Find DIMM Object and DIMM Pattern Offset}\\

\begin{itemize}
\tightlist
\item
  While tracking, take a 10-sec exposure with the StarTracker.
\item
  Load the image into an image viewer.
\item
  Overlay the GAIA catalog.
\item
  Select a star brighter than XXX mag (bright enough for the DIMM).
\item
  Calculate the pixel offset between the StarTracker and the DIMM.
\item
  Transform the offset into AZ and EL offsets.
\end{itemize}

}
\hdashrule[0.5ex]{\textwidth}{1pt}{3mm}
  Expected Result \\
{\footnotesize
\begin{itemize}
\tightlist
\item
  An image was successfully taken with the StarTracker and is of
  sufficient quality.
\item
  AZ and EL offsets are available.
\end{itemize}

}

\begin{tabular}{p{2cm}}
\toprule
Step 30  \\ \hline
\end{tabular}
 Description \\
{\footnotesize
\textbf{Move TMA to the 1. random distance of 3.5deg}\\
Point the TMA to a random 3.5 deg combined offset in AZ and EL from
{Pointing 13}⁠ at {0}⁠, {86.5}⁠. Record the exact position of the offset
in AZ and El.

}
\hdashrule[0.5ex]{\textwidth}{1pt}{3mm}
  Expected Result \\
{\footnotesize
\begin{itemize}
\tightlist
\item
  The TMA reaches the commanded offset position.
\item
  The \emph{MTMount\_logevent\_azimuthInPosition} and
  \emph{MTMount\_logevent\_elevationInPosition}inPosition parameter =
  true.
\end{itemize}

}

\begin{tabular}{p{2cm}}
\toprule
Step 31  \\ \hline
\end{tabular}
 Description \\
{\footnotesize
\textbf{Position the dome}\\
Command the Dome to {Pointing 14}⁠ to {\{Dome AZ\}}⁠

}
\hdashrule[0.5ex]{\textwidth}{1pt}{3mm}
  Expected Result \\
{\footnotesize
The Dome starts moving.

}

\begin{tabular}{p{2cm}}
\toprule
Step 32  \\ \hline
\end{tabular}
 Description \\
{\footnotesize
\textbf{Point the TMA to (Az, El)-pattern position + DIMM pattern offset
\textbf{\textbf{and take DIMM images}}\\
}

\begin{itemize}
\tightlist
\item
  Point the TMA back to {Pointing 14}⁠ at {0}⁠ + DIMM offset, {75}⁠ +
  DIMM offset.
\item
  While tracking, take DIMM images with XXXs exposure time and inspect
  the quality.
\end{itemize}

}
\hdashrule[0.5ex]{\textwidth}{1pt}{3mm}
  Expected Result \\
{\footnotesize
\begin{itemize}
\tightlist
\item
  TMA reaches the position.
\item
  DIMM image quality is sufficient
\end{itemize}

}

\begin{tabular}{p{2cm}}
\toprule
Step 33  \\ \hline
\end{tabular}
 Description \\
{\footnotesize
Wait for the Dome to reach the commanded position.

}
\hdashrule[0.5ex]{\textwidth}{1pt}{3mm}
  Expected Result \\
{\footnotesize
The \emph{MTDome\_logevent\_azMotion} and
\emph{MTDome\_logevent\_elMotion} inPosition parameter = true.

}

\begin{tabular}{p{2cm}}
\toprule
Step 34  \\ \hline
\end{tabular}
 Description \\
{\footnotesize
\textbf{Point the TMA to (Az, El)-pattern position}\\
Point the TMA to {Pointing 14}⁠ at {0}⁠ , {75}⁠ .

}
\hdashrule[0.5ex]{\textwidth}{1pt}{3mm}
  Expected Result \\
{\footnotesize
The TMA starts moving.

}

\begin{tabular}{p{2cm}}
\toprule
Step 35  \\ \hline
\end{tabular}
 Description \\
{\footnotesize
Wait for the TMA to reach the commanded position.

}
\hdashrule[0.5ex]{\textwidth}{1pt}{3mm}
  Expected Result \\
{\footnotesize
The \emph{MTMount\_logevent\_azimuthInPosition} and
\emph{MTMount\_logevent\_elevationInPosition} inPosition parameter =
true.

}

\begin{tabular}{p{2cm}}
\toprule
Step 36  \\ \hline
\end{tabular}
 Description \\
{\footnotesize
\textbf{Find DIMM Object and DIMM Pattern Offset}\\

\begin{itemize}
\tightlist
\item
  While tracking, take a 10-sec exposure with the StarTracker.
\item
  Load the image into an image viewer.
\item
  Overlay the GAIA catalog.
\item
  Select a star brighter than XXX mag (bright enough for the DIMM).
\item
  Calculate the pixel offset between the StarTracker and the DIMM.
\item
  Transform the offset into AZ and EL offsets.
\end{itemize}

}
\hdashrule[0.5ex]{\textwidth}{1pt}{3mm}
  Expected Result \\
{\footnotesize
\begin{itemize}
\tightlist
\item
  An image was successfully taken with the StarTracker and is of
  sufficient quality.
\item
  AZ and EL offsets are available.
\end{itemize}

}

\begin{tabular}{p{2cm}}
\toprule
Step 37  \\ \hline
\end{tabular}
 Description \\
{\footnotesize
\textbf{Move TMA to the 1. random distance of 3.5deg}\\
Point the TMA to a random 3.5 deg combined offset in AZ and EL from
{Pointing 14}⁠ at {0}⁠, {75}⁠. Record the exact position of the offset
in AZ and El.

}
\hdashrule[0.5ex]{\textwidth}{1pt}{3mm}
  Expected Result \\
{\footnotesize
\begin{itemize}
\tightlist
\item
  The TMA reaches the commanded offset position.
\item
  The \emph{MTMount\_logevent\_azimuthInPosition} and
  \emph{MTMount\_logevent\_elevationInPosition}inPosition parameter =
  true.
\end{itemize}

}

\begin{tabular}{p{2cm}}
\toprule
Step 38  \\ \hline
\end{tabular}
 Description \\
{\footnotesize
\textbf{Position the dome}\\
Command the Dome to {Pointing 15}⁠ to {\{Dome AZ\}}⁠

}
\hdashrule[0.5ex]{\textwidth}{1pt}{3mm}
  Expected Result \\
{\footnotesize
The Dome starts moving.

}

\begin{tabular}{p{2cm}}
\toprule
Step 39  \\ \hline
\end{tabular}
 Description \\
{\footnotesize
\textbf{Point the TMA to (Az, El)-pattern position + DIMM pattern offset
\textbf{\textbf{and take DIMM images}}\\
}

\begin{itemize}
\tightlist
\item
  Point the TMA back to {Pointing 15}⁠ at {0}⁠ + DIMM offset, {45}⁠ +
  DIMM offset.
\item
  While tracking, take DIMM images with XXXs exposure time and inspect
  the quality.
\end{itemize}

}
\hdashrule[0.5ex]{\textwidth}{1pt}{3mm}
  Expected Result \\
{\footnotesize
\begin{itemize}
\tightlist
\item
  TMA reaches the position.
\item
  DIMM image quality is sufficient
\end{itemize}

}

\begin{tabular}{p{2cm}}
\toprule
Step 40  \\ \hline
\end{tabular}
 Description \\
{\footnotesize
Wait for the Dome to reach the commanded position.

}
\hdashrule[0.5ex]{\textwidth}{1pt}{3mm}
  Expected Result \\
{\footnotesize
The \emph{MTDome\_logevent\_azMotion} and
\emph{MTDome\_logevent\_elMotion} inPosition parameter = true.

}

\begin{tabular}{p{2cm}}
\toprule
Step 41  \\ \hline
\end{tabular}
 Description \\
{\footnotesize
\textbf{Point the TMA to (Az, El)-pattern position}\\
Point the TMA to {Pointing 15}⁠ at {0}⁠ , {45}⁠ .

}
\hdashrule[0.5ex]{\textwidth}{1pt}{3mm}
  Expected Result \\
{\footnotesize
The TMA starts moving.

}

\begin{tabular}{p{2cm}}
\toprule
Step 42  \\ \hline
\end{tabular}
 Description \\
{\footnotesize
Wait for the TMA to reach the commanded position.

}
\hdashrule[0.5ex]{\textwidth}{1pt}{3mm}
  Expected Result \\
{\footnotesize
The \emph{MTMount\_logevent\_azimuthInPosition} and
\emph{MTMount\_logevent\_elevationInPosition} inPosition parameter =
true.

}

\begin{tabular}{p{2cm}}
\toprule
Step 43  \\ \hline
\end{tabular}
 Description \\
{\footnotesize
\textbf{Find DIMM Object and DIMM Pattern Offset}\\

\begin{itemize}
\tightlist
\item
  While tracking, take a 10-sec exposure with the StarTracker.
\item
  Load the image into an image viewer.
\item
  Overlay the GAIA catalog.
\item
  Select a star brighter than XXX mag (bright enough for the DIMM).
\item
  Calculate the pixel offset between the StarTracker and the DIMM.
\item
  Transform the offset into AZ and EL offsets.
\end{itemize}

}
\hdashrule[0.5ex]{\textwidth}{1pt}{3mm}
  Expected Result \\
{\footnotesize
\begin{itemize}
\tightlist
\item
  An image was successfully taken with the StarTracker and is of
  sufficient quality.
\item
  AZ and EL offsets are available.
\end{itemize}

}

\begin{tabular}{p{2cm}}
\toprule
Step 44  \\ \hline
\end{tabular}
 Description \\
{\footnotesize
\textbf{Move TMA to the 1. random distance of 3.5deg}\\
Point the TMA to a random 3.5 deg combined offset in AZ and EL from
{Pointing 15}⁠ at {0}⁠, {45}⁠. Record the exact position of the offset
in AZ and El.

}
\hdashrule[0.5ex]{\textwidth}{1pt}{3mm}
  Expected Result \\
{\footnotesize
\begin{itemize}
\tightlist
\item
  The TMA reaches the commanded offset position.
\item
  The \emph{MTMount\_logevent\_azimuthInPosition} and
  \emph{MTMount\_logevent\_elevationInPosition}inPosition parameter =
  true.
\end{itemize}

}

\begin{tabular}{p{2cm}}
\toprule
Step 45  \\ \hline
\end{tabular}
 Description \\
{\footnotesize
\textbf{Position the dome}\\
Command the Dome to {Pointing 16}⁠ to {\{Dome AZ\}}⁠

}
\hdashrule[0.5ex]{\textwidth}{1pt}{3mm}
  Expected Result \\
{\footnotesize
The Dome starts moving.

}

\begin{tabular}{p{2cm}}
\toprule
Step 46  \\ \hline
\end{tabular}
 Description \\
{\footnotesize
\textbf{Point the TMA to (Az, El)-pattern position + DIMM pattern offset
\textbf{\textbf{and take DIMM images}}\\
}

\begin{itemize}
\tightlist
\item
  Point the TMA back to {Pointing 16}⁠ at {0}⁠ + DIMM offset, {15}⁠ +
  DIMM offset.
\item
  While tracking, take DIMM images with XXXs exposure time and inspect
  the quality.
\end{itemize}

}
\hdashrule[0.5ex]{\textwidth}{1pt}{3mm}
  Expected Result \\
{\footnotesize
\begin{itemize}
\tightlist
\item
  TMA reaches the position.
\item
  DIMM image quality is sufficient
\end{itemize}

}

\begin{tabular}{p{2cm}}
\toprule
Step 47  \\ \hline
\end{tabular}
 Description \\
{\footnotesize
Wait for the Dome to reach the commanded position.

}
\hdashrule[0.5ex]{\textwidth}{1pt}{3mm}
  Expected Result \\
{\footnotesize
The \emph{MTDome\_logevent\_azMotion} and
\emph{MTDome\_logevent\_elMotion} inPosition parameter = true.

}

\begin{tabular}{p{2cm}}
\toprule
Step 48  \\ \hline
\end{tabular}
 Description \\
{\footnotesize
\textbf{Point the TMA to (Az, El)-pattern position}\\
Point the TMA to {Pointing 16}⁠ at {0}⁠ , {15}⁠ .

}
\hdashrule[0.5ex]{\textwidth}{1pt}{3mm}
  Expected Result \\
{\footnotesize
The TMA starts moving.

}

\begin{tabular}{p{2cm}}
\toprule
Step 49  \\ \hline
\end{tabular}
 Description \\
{\footnotesize
Wait for the TMA to reach the commanded position.

}
\hdashrule[0.5ex]{\textwidth}{1pt}{3mm}
  Expected Result \\
{\footnotesize
The \emph{MTMount\_logevent\_azimuthInPosition} and
\emph{MTMount\_logevent\_elevationInPosition} inPosition parameter =
true.

}

\begin{tabular}{p{2cm}}
\toprule
Step 50  \\ \hline
\end{tabular}
 Description \\
{\footnotesize
\textbf{Find DIMM Object and DIMM Pattern Offset}\\

\begin{itemize}
\tightlist
\item
  While tracking, take a 10-sec exposure with the StarTracker.
\item
  Load the image into an image viewer.
\item
  Overlay the GAIA catalog.
\item
  Select a star brighter than XXX mag (bright enough for the DIMM).
\item
  Calculate the pixel offset between the StarTracker and the DIMM.
\item
  Transform the offset into AZ and EL offsets.
\end{itemize}

}
\hdashrule[0.5ex]{\textwidth}{1pt}{3mm}
  Expected Result \\
{\footnotesize
\begin{itemize}
\tightlist
\item
  An image was successfully taken with the StarTracker and is of
  sufficient quality.
\item
  AZ and EL offsets are available.
\end{itemize}

}

\begin{tabular}{p{2cm}}
\toprule
Step 51  \\ \hline
\end{tabular}
 Description \\
{\footnotesize
\textbf{Move TMA to the 1. random distance of 3.5deg}\\
Point the TMA to a random 3.5 deg combined offset in AZ and EL from
{Pointing 16}⁠ at {0}⁠, {15}⁠. Record the exact position of the offset
in AZ and El.

}
\hdashrule[0.5ex]{\textwidth}{1pt}{3mm}
  Expected Result \\
{\footnotesize
\begin{itemize}
\tightlist
\item
  The TMA reaches the commanded offset position.
\item
  The \emph{MTMount\_logevent\_azimuthInPosition} and
  \emph{MTMount\_logevent\_elevationInPosition}inPosition parameter =
  true.
\end{itemize}

}

\begin{tabular}{p{2cm}}
\toprule
Step 52  \\ \hline
\end{tabular}
 Description \\
{\footnotesize
\textbf{Position the dome}\\
Command the Dome to {Pointing 17}⁠ to {\{Dome AZ\}}⁠

}
\hdashrule[0.5ex]{\textwidth}{1pt}{3mm}
  Expected Result \\
{\footnotesize
The Dome starts moving.

}

\begin{tabular}{p{2cm}}
\toprule
Step 53  \\ \hline
\end{tabular}
 Description \\
{\footnotesize
\textbf{Point the TMA to (Az, El)-pattern position + DIMM pattern offset
\textbf{\textbf{and take DIMM images}}\\
}

\begin{itemize}
\tightlist
\item
  Point the TMA back to {Pointing 17}⁠ at {90}⁠ + DIMM offset, {15}⁠ +
  DIMM offset.
\item
  While tracking, take DIMM images with XXXs exposure time and inspect
  the quality.
\end{itemize}

}
\hdashrule[0.5ex]{\textwidth}{1pt}{3mm}
  Expected Result \\
{\footnotesize
\begin{itemize}
\tightlist
\item
  TMA reaches the position.
\item
  DIMM image quality is sufficient
\end{itemize}

}

\begin{tabular}{p{2cm}}
\toprule
Step 54  \\ \hline
\end{tabular}
 Description \\
{\footnotesize
Wait for the Dome to reach the commanded position.

}
\hdashrule[0.5ex]{\textwidth}{1pt}{3mm}
  Expected Result \\
{\footnotesize
The \emph{MTDome\_logevent\_azMotion} and
\emph{MTDome\_logevent\_elMotion} inPosition parameter = true.

}

\begin{tabular}{p{2cm}}
\toprule
Step 55  \\ \hline
\end{tabular}
 Description \\
{\footnotesize
\textbf{Point the TMA to (Az, El)-pattern position}\\
Point the TMA to {Pointing 17}⁠ at {90}⁠ , {15}⁠ .

}
\hdashrule[0.5ex]{\textwidth}{1pt}{3mm}
  Expected Result \\
{\footnotesize
The TMA starts moving.

}

\begin{tabular}{p{2cm}}
\toprule
Step 56  \\ \hline
\end{tabular}
 Description \\
{\footnotesize
Wait for the TMA to reach the commanded position.

}
\hdashrule[0.5ex]{\textwidth}{1pt}{3mm}
  Expected Result \\
{\footnotesize
The \emph{MTMount\_logevent\_azimuthInPosition} and
\emph{MTMount\_logevent\_elevationInPosition} inPosition parameter =
true.

}

\begin{tabular}{p{2cm}}
\toprule
Step 57  \\ \hline
\end{tabular}
 Description \\
{\footnotesize
\textbf{Find DIMM Object and DIMM Pattern Offset}\\

\begin{itemize}
\tightlist
\item
  While tracking, take a 10-sec exposure with the StarTracker.
\item
  Load the image into an image viewer.
\item
  Overlay the GAIA catalog.
\item
  Select a star brighter than XXX mag (bright enough for the DIMM).
\item
  Calculate the pixel offset between the StarTracker and the DIMM.
\item
  Transform the offset into AZ and EL offsets.
\end{itemize}

}
\hdashrule[0.5ex]{\textwidth}{1pt}{3mm}
  Expected Result \\
{\footnotesize
\begin{itemize}
\tightlist
\item
  An image was successfully taken with the StarTracker and is of
  sufficient quality.
\item
  AZ and EL offsets are available.
\end{itemize}

}

\begin{tabular}{p{2cm}}
\toprule
Step 58  \\ \hline
\end{tabular}
 Description \\
{\footnotesize
\textbf{Move TMA to the 1. random distance of 3.5deg}\\
Point the TMA to a random 3.5 deg combined offset in AZ and EL from
{Pointing 17}⁠ at {90}⁠, {15}⁠. Record the exact position of the offset
in AZ and El.

}
\hdashrule[0.5ex]{\textwidth}{1pt}{3mm}
  Expected Result \\
{\footnotesize
\begin{itemize}
\tightlist
\item
  The TMA reaches the commanded offset position.
\item
  The \emph{MTMount\_logevent\_azimuthInPosition} and
  \emph{MTMount\_logevent\_elevationInPosition}inPosition parameter =
  true.
\end{itemize}

}

\begin{tabular}{p{2cm}}
\toprule
Step 59  \\ \hline
\end{tabular}
 Description \\
{\footnotesize
\textbf{Position the dome}\\
Command the Dome to {Pointing 18}⁠ to {\{Dome AZ\}}⁠

}
\hdashrule[0.5ex]{\textwidth}{1pt}{3mm}
  Expected Result \\
{\footnotesize
The Dome starts moving.

}

\begin{tabular}{p{2cm}}
\toprule
Step 60  \\ \hline
\end{tabular}
 Description \\
{\footnotesize
\textbf{Point the TMA to (Az, El)-pattern position + DIMM pattern offset
\textbf{\textbf{and take DIMM images}}\\
}

\begin{itemize}
\tightlist
\item
  Point the TMA back to {Pointing 18}⁠ at {90}⁠ + DIMM offset, {45}⁠ +
  DIMM offset.
\item
  While tracking, take DIMM images with XXXs exposure time and inspect
  the quality.
\end{itemize}

}
\hdashrule[0.5ex]{\textwidth}{1pt}{3mm}
  Expected Result \\
{\footnotesize
\begin{itemize}
\tightlist
\item
  TMA reaches the position.
\item
  DIMM image quality is sufficient
\end{itemize}

}

\begin{tabular}{p{2cm}}
\toprule
Step 61  \\ \hline
\end{tabular}
 Description \\
{\footnotesize
Wait for the Dome to reach the commanded position.

}
\hdashrule[0.5ex]{\textwidth}{1pt}{3mm}
  Expected Result \\
{\footnotesize
The \emph{MTDome\_logevent\_azMotion} and
\emph{MTDome\_logevent\_elMotion} inPosition parameter = true.

}

\begin{tabular}{p{2cm}}
\toprule
Step 62  \\ \hline
\end{tabular}
 Description \\
{\footnotesize
\textbf{Point the TMA to (Az, El)-pattern position}\\
Point the TMA to {Pointing 18}⁠ at {90}⁠ , {45}⁠ .

}
\hdashrule[0.5ex]{\textwidth}{1pt}{3mm}
  Expected Result \\
{\footnotesize
The TMA starts moving.

}

\begin{tabular}{p{2cm}}
\toprule
Step 63  \\ \hline
\end{tabular}
 Description \\
{\footnotesize
Wait for the TMA to reach the commanded position.

}
\hdashrule[0.5ex]{\textwidth}{1pt}{3mm}
  Expected Result \\
{\footnotesize
The \emph{MTMount\_logevent\_azimuthInPosition} and
\emph{MTMount\_logevent\_elevationInPosition} inPosition parameter =
true.

}

\begin{tabular}{p{2cm}}
\toprule
Step 64  \\ \hline
\end{tabular}
 Description \\
{\footnotesize
\textbf{Find DIMM Object and DIMM Pattern Offset}\\

\begin{itemize}
\tightlist
\item
  While tracking, take a 10-sec exposure with the StarTracker.
\item
  Load the image into an image viewer.
\item
  Overlay the GAIA catalog.
\item
  Select a star brighter than XXX mag (bright enough for the DIMM).
\item
  Calculate the pixel offset between the StarTracker and the DIMM.
\item
  Transform the offset into AZ and EL offsets.
\end{itemize}

}
\hdashrule[0.5ex]{\textwidth}{1pt}{3mm}
  Expected Result \\
{\footnotesize
\begin{itemize}
\tightlist
\item
  An image was successfully taken with the StarTracker and is of
  sufficient quality.
\item
  AZ and EL offsets are available.
\end{itemize}

}

\begin{tabular}{p{2cm}}
\toprule
Step 65  \\ \hline
\end{tabular}
 Description \\
{\footnotesize
\textbf{Move TMA to the 1. random distance of 3.5deg}\\
Point the TMA to a random 3.5 deg combined offset in AZ and EL from
{Pointing 18}⁠ at {90}⁠, {45}⁠. Record the exact position of the offset
in AZ and El.

}
\hdashrule[0.5ex]{\textwidth}{1pt}{3mm}
  Expected Result \\
{\footnotesize
\begin{itemize}
\tightlist
\item
  The TMA reaches the commanded offset position.
\item
  The \emph{MTMount\_logevent\_azimuthInPosition} and
  \emph{MTMount\_logevent\_elevationInPosition}inPosition parameter =
  true.
\end{itemize}

}

\begin{tabular}{p{2cm}}
\toprule
Step 66  \\ \hline
\end{tabular}
 Description \\
{\footnotesize
\textbf{Position the dome}\\
Command the Dome to {Pointing 19}⁠ to {\{Dome AZ\}}⁠

}
\hdashrule[0.5ex]{\textwidth}{1pt}{3mm}
  Expected Result \\
{\footnotesize
The Dome starts moving.

}

\begin{tabular}{p{2cm}}
\toprule
Step 67  \\ \hline
\end{tabular}
 Description \\
{\footnotesize
\textbf{Point the TMA to (Az, El)-pattern position + DIMM pattern offset
\textbf{\textbf{and take DIMM images}}\\
}

\begin{itemize}
\tightlist
\item
  Point the TMA back to {Pointing 19}⁠ at {90}⁠ + DIMM offset, {75}⁠ +
  DIMM offset.
\item
  While tracking, take DIMM images with XXXs exposure time and inspect
  the quality.
\end{itemize}

}
\hdashrule[0.5ex]{\textwidth}{1pt}{3mm}
  Expected Result \\
{\footnotesize
\begin{itemize}
\tightlist
\item
  TMA reaches the position.
\item
  DIMM image quality is sufficient
\end{itemize}

}

\begin{tabular}{p{2cm}}
\toprule
Step 68  \\ \hline
\end{tabular}
 Description \\
{\footnotesize
Wait for the Dome to reach the commanded position.

}
\hdashrule[0.5ex]{\textwidth}{1pt}{3mm}
  Expected Result \\
{\footnotesize
The \emph{MTDome\_logevent\_azMotion} and
\emph{MTDome\_logevent\_elMotion} inPosition parameter = true.

}

\begin{tabular}{p{2cm}}
\toprule
Step 69  \\ \hline
\end{tabular}
 Description \\
{\footnotesize
\textbf{Point the TMA to (Az, El)-pattern position}\\
Point the TMA to {Pointing 19}⁠ at {90}⁠ , {75}⁠ .

}
\hdashrule[0.5ex]{\textwidth}{1pt}{3mm}
  Expected Result \\
{\footnotesize
The TMA starts moving.

}

\begin{tabular}{p{2cm}}
\toprule
Step 70  \\ \hline
\end{tabular}
 Description \\
{\footnotesize
Wait for the TMA to reach the commanded position.

}
\hdashrule[0.5ex]{\textwidth}{1pt}{3mm}
  Expected Result \\
{\footnotesize
The \emph{MTMount\_logevent\_azimuthInPosition} and
\emph{MTMount\_logevent\_elevationInPosition} inPosition parameter =
true.

}

\begin{tabular}{p{2cm}}
\toprule
Step 71  \\ \hline
\end{tabular}
 Description \\
{\footnotesize
\textbf{Find DIMM Object and DIMM Pattern Offset}\\

\begin{itemize}
\tightlist
\item
  While tracking, take a 10-sec exposure with the StarTracker.
\item
  Load the image into an image viewer.
\item
  Overlay the GAIA catalog.
\item
  Select a star brighter than XXX mag (bright enough for the DIMM).
\item
  Calculate the pixel offset between the StarTracker and the DIMM.
\item
  Transform the offset into AZ and EL offsets.
\end{itemize}

}
\hdashrule[0.5ex]{\textwidth}{1pt}{3mm}
  Expected Result \\
{\footnotesize
\begin{itemize}
\tightlist
\item
  An image was successfully taken with the StarTracker and is of
  sufficient quality.
\item
  AZ and EL offsets are available.
\end{itemize}

}

\begin{tabular}{p{2cm}}
\toprule
Step 72  \\ \hline
\end{tabular}
 Description \\
{\footnotesize
\textbf{Move TMA to the 1. random distance of 3.5deg}\\
Point the TMA to a random 3.5 deg combined offset in AZ and EL from
{Pointing 19}⁠ at {90}⁠, {75}⁠. Record the exact position of the offset
in AZ and El.

}
\hdashrule[0.5ex]{\textwidth}{1pt}{3mm}
  Expected Result \\
{\footnotesize
\begin{itemize}
\tightlist
\item
  The TMA reaches the commanded offset position.
\item
  The \emph{MTMount\_logevent\_azimuthInPosition} and
  \emph{MTMount\_logevent\_elevationInPosition}inPosition parameter =
  true.
\end{itemize}

}

\begin{tabular}{p{2cm}}
\toprule
Step 73  \\ \hline
\end{tabular}
 Description \\
{\footnotesize
Wait for the Dome to reach the commanded position.

}
\hdashrule[0.5ex]{\textwidth}{1pt}{3mm}
  Expected Result \\
{\footnotesize
The \emph{MTDome\_logevent\_azMotion} and
\emph{MTDome\_logevent\_elMotion} inPosition parameter = true.

}

\begin{tabular}{p{2cm}}
\toprule
Step 74  \\ \hline
\end{tabular}
 Description \\
{\footnotesize
\textbf{Position the dome}\\
Command the Dome to {Pointing 2}⁠ to {\{Dome AZ\}}⁠

}
\hdashrule[0.5ex]{\textwidth}{1pt}{3mm}
  Expected Result \\
{\footnotesize
The Dome starts moving.

}

\begin{tabular}{p{2cm}}
\toprule
Step 75  \\ \hline
\end{tabular}
 Description \\
{\footnotesize
\textbf{Point the TMA to (Az, El)-pattern position + DIMM pattern offset
\textbf{\textbf{and take DIMM images}}\\
}

\begin{itemize}
\tightlist
\item
  Point the TMA back to {Pointing 2}⁠ at {-270}⁠ + DIMM offset, {45}⁠ +
  DIMM offset.
\item
  While tracking, take DIMM images with XXXs exposure time and inspect
  the quality.
\end{itemize}

}
\hdashrule[0.5ex]{\textwidth}{1pt}{3mm}
  Expected Result \\
{\footnotesize
\begin{itemize}
\tightlist
\item
  TMA reaches the position.
\item
  DIMM image quality is sufficient
\end{itemize}

}

\begin{tabular}{p{2cm}}
\toprule
Step 76  \\ \hline
\end{tabular}
 Description \\
{\footnotesize
Wait for the Dome to reach the commanded position.

}
\hdashrule[0.5ex]{\textwidth}{1pt}{3mm}
  Expected Result \\
{\footnotesize
The \emph{MTDome\_logevent\_azMotion} and
\emph{MTDome\_logevent\_elMotion} inPosition parameter = true.

}

\begin{tabular}{p{2cm}}
\toprule
Step 77  \\ \hline
\end{tabular}
 Description \\
{\footnotesize
\textbf{Point the TMA to (Az, El)-pattern position}\\
Point the TMA to {Pointing 2}⁠ at {-270}⁠ , {45}⁠ .

}
\hdashrule[0.5ex]{\textwidth}{1pt}{3mm}
  Expected Result \\
{\footnotesize
The TMA starts moving.

}

\begin{tabular}{p{2cm}}
\toprule
Step 78  \\ \hline
\end{tabular}
 Description \\
{\footnotesize
Wait for the TMA to reach the commanded position.

}
\hdashrule[0.5ex]{\textwidth}{1pt}{3mm}
  Expected Result \\
{\footnotesize
The \emph{MTMount\_logevent\_azimuthInPosition} and
\emph{MTMount\_logevent\_elevationInPosition} inPosition parameter =
true.

}

\begin{tabular}{p{2cm}}
\toprule
Step 79  \\ \hline
\end{tabular}
 Description \\
{\footnotesize
\textbf{Find DIMM Object and DIMM Pattern Offset}\\

\begin{itemize}
\tightlist
\item
  While tracking, take a 10-sec exposure with the StarTracker.
\item
  Load the image into an image viewer.
\item
  Overlay the GAIA catalog.
\item
  Select a star brighter than XXX mag (bright enough for the DIMM).
\item
  Calculate the pixel offset between the StarTracker and the DIMM.
\item
  Transform the offset into AZ and EL offsets.
\end{itemize}

}
\hdashrule[0.5ex]{\textwidth}{1pt}{3mm}
  Expected Result \\
{\footnotesize
\begin{itemize}
\tightlist
\item
  An image was successfully taken with the StarTracker and is of
  sufficient quality.
\item
  AZ and EL offsets are available.
\end{itemize}

}

\begin{tabular}{p{2cm}}
\toprule
Step 80  \\ \hline
\end{tabular}
 Description \\
{\footnotesize
\textbf{Move TMA to the 1. random distance of 3.5deg}\\
Point the TMA to a random 3.5 deg combined offset in AZ and EL from
{Pointing 2}⁠ at {-270}⁠, {45}⁠. Record the exact position of the offset
in AZ and El.

}
\hdashrule[0.5ex]{\textwidth}{1pt}{3mm}
  Expected Result \\
{\footnotesize
\begin{itemize}
\tightlist
\item
  The TMA reaches the commanded offset position.
\item
  The \emph{MTMount\_logevent\_azimuthInPosition} and
  \emph{MTMount\_logevent\_elevationInPosition}inPosition parameter =
  true.
\end{itemize}

}

\begin{tabular}{p{2cm}}
\toprule
Step 81  \\ \hline
\end{tabular}
 Description \\
{\footnotesize
\textbf{Position the dome}\\
Command the Dome to {Pointing 20}⁠ to {\{Dome AZ\}}⁠

}
\hdashrule[0.5ex]{\textwidth}{1pt}{3mm}
  Expected Result \\
{\footnotesize
The Dome starts moving.

}

\begin{tabular}{p{2cm}}
\toprule
Step 82  \\ \hline
\end{tabular}
 Description \\
{\footnotesize
\textbf{Point the TMA to (Az, El)-pattern position + DIMM pattern offset
\textbf{\textbf{and take DIMM images}}\\
}

\begin{itemize}
\tightlist
\item
  Point the TMA back to {Pointing 20}⁠ at {90}⁠ + DIMM offset, {86.5}⁠ +
  DIMM offset.
\item
  While tracking, take DIMM images with XXXs exposure time and inspect
  the quality.
\end{itemize}

}
\hdashrule[0.5ex]{\textwidth}{1pt}{3mm}
  Expected Result \\
{\footnotesize
\begin{itemize}
\tightlist
\item
  TMA reaches the position.
\item
  DIMM image quality is sufficient
\end{itemize}

}

\begin{tabular}{p{2cm}}
\toprule
Step 83  \\ \hline
\end{tabular}
 Description \\
{\footnotesize
Wait for the Dome to reach the commanded position.

}
\hdashrule[0.5ex]{\textwidth}{1pt}{3mm}
  Expected Result \\
{\footnotesize
The \emph{MTDome\_logevent\_azMotion} and
\emph{MTDome\_logevent\_elMotion} inPosition parameter = true.

}

\begin{tabular}{p{2cm}}
\toprule
Step 84  \\ \hline
\end{tabular}
 Description \\
{\footnotesize
\textbf{Point the TMA to (Az, El)-pattern position}\\
Point the TMA to {Pointing 20}⁠ at {90}⁠ , {86.5}⁠ .

}
\hdashrule[0.5ex]{\textwidth}{1pt}{3mm}
  Expected Result \\
{\footnotesize
The TMA starts moving.

}

\begin{tabular}{p{2cm}}
\toprule
Step 85  \\ \hline
\end{tabular}
 Description \\
{\footnotesize
Wait for the TMA to reach the commanded position.

}
\hdashrule[0.5ex]{\textwidth}{1pt}{3mm}
  Expected Result \\
{\footnotesize
The \emph{MTMount\_logevent\_azimuthInPosition} and
\emph{MTMount\_logevent\_elevationInPosition} inPosition parameter =
true.

}

\begin{tabular}{p{2cm}}
\toprule
Step 86  \\ \hline
\end{tabular}
 Description \\
{\footnotesize
\textbf{Find DIMM Object and DIMM Pattern Offset}\\

\begin{itemize}
\tightlist
\item
  While tracking, take a 10-sec exposure with the StarTracker.
\item
  Load the image into an image viewer.
\item
  Overlay the GAIA catalog.
\item
  Select a star brighter than XXX mag (bright enough for the DIMM).
\item
  Calculate the pixel offset between the StarTracker and the DIMM.
\item
  Transform the offset into AZ and EL offsets.
\end{itemize}

}
\hdashrule[0.5ex]{\textwidth}{1pt}{3mm}
  Expected Result \\
{\footnotesize
\begin{itemize}
\tightlist
\item
  An image was successfully taken with the StarTracker and is of
  sufficient quality.
\item
  AZ and EL offsets are available.
\end{itemize}

}

\begin{tabular}{p{2cm}}
\toprule
Step 87  \\ \hline
\end{tabular}
 Description \\
{\footnotesize
\textbf{Move TMA to the 1. random distance of 3.5deg}\\
Point the TMA to a random 3.5 deg combined offset in AZ and EL from
{Pointing 20}⁠ at {90}⁠, {86.5}⁠. Record the exact position of the
offset in AZ and El.

}
\hdashrule[0.5ex]{\textwidth}{1pt}{3mm}
  Expected Result \\
{\footnotesize
\begin{itemize}
\tightlist
\item
  The TMA reaches the commanded offset position.
\item
  The \emph{MTMount\_logevent\_azimuthInPosition} and
  \emph{MTMount\_logevent\_elevationInPosition}inPosition parameter =
  true.
\end{itemize}

}

\begin{tabular}{p{2cm}}
\toprule
Step 88  \\ \hline
\end{tabular}
 Description \\
{\footnotesize
\textbf{Position the dome}\\
Command the Dome to {Pointing 21}⁠ to {\{Dome AZ\}}⁠

}
\hdashrule[0.5ex]{\textwidth}{1pt}{3mm}
  Expected Result \\
{\footnotesize
The Dome starts moving.

}

\begin{tabular}{p{2cm}}
\toprule
Step 89  \\ \hline
\end{tabular}
 Description \\
{\footnotesize
\textbf{Point the TMA to (Az, El)-pattern position + DIMM pattern offset
\textbf{\textbf{and take DIMM images}}\\
}

\begin{itemize}
\tightlist
\item
  Point the TMA back to {Pointing 21}⁠ at {180}⁠ + DIMM offset, {86.5}⁠
  + DIMM offset.
\item
  While tracking, take DIMM images with XXXs exposure time and inspect
  the quality.
\end{itemize}

}
\hdashrule[0.5ex]{\textwidth}{1pt}{3mm}
  Expected Result \\
{\footnotesize
\begin{itemize}
\tightlist
\item
  TMA reaches the position.
\item
  DIMM image quality is sufficient
\end{itemize}

}

\begin{tabular}{p{2cm}}
\toprule
Step 90  \\ \hline
\end{tabular}
 Description \\
{\footnotesize
Wait for the Dome to reach the commanded position.

}
\hdashrule[0.5ex]{\textwidth}{1pt}{3mm}
  Expected Result \\
{\footnotesize
The \emph{MTDome\_logevent\_azMotion} and
\emph{MTDome\_logevent\_elMotion} inPosition parameter = true.

}

\begin{tabular}{p{2cm}}
\toprule
Step 91  \\ \hline
\end{tabular}
 Description \\
{\footnotesize
\textbf{Point the TMA to (Az, El)-pattern position}\\
Point the TMA to {Pointing 21}⁠ at {180}⁠ , {86.5}⁠ .

}
\hdashrule[0.5ex]{\textwidth}{1pt}{3mm}
  Expected Result \\
{\footnotesize
The TMA starts moving.

}

\begin{tabular}{p{2cm}}
\toprule
Step 92  \\ \hline
\end{tabular}
 Description \\
{\footnotesize
Wait for the TMA to reach the commanded position.

}
\hdashrule[0.5ex]{\textwidth}{1pt}{3mm}
  Expected Result \\
{\footnotesize
The \emph{MTMount\_logevent\_azimuthInPosition} and
\emph{MTMount\_logevent\_elevationInPosition} inPosition parameter =
true.

}

\begin{tabular}{p{2cm}}
\toprule
Step 93  \\ \hline
\end{tabular}
 Description \\
{\footnotesize
\textbf{Find DIMM Object and DIMM Pattern Offset}\\

\begin{itemize}
\tightlist
\item
  While tracking, take a 10-sec exposure with the StarTracker.
\item
  Load the image into an image viewer.
\item
  Overlay the GAIA catalog.
\item
  Select a star brighter than XXX mag (bright enough for the DIMM).
\item
  Calculate the pixel offset between the StarTracker and the DIMM.
\item
  Transform the offset into AZ and EL offsets.
\end{itemize}

}
\hdashrule[0.5ex]{\textwidth}{1pt}{3mm}
  Expected Result \\
{\footnotesize
\begin{itemize}
\tightlist
\item
  An image was successfully taken with the StarTracker and is of
  sufficient quality.
\item
  AZ and EL offsets are available.
\end{itemize}

}

\begin{tabular}{p{2cm}}
\toprule
Step 94  \\ \hline
\end{tabular}
 Description \\
{\footnotesize
\textbf{Move TMA to the 1. random distance of 3.5deg}\\
Point the TMA to a random 3.5 deg combined offset in AZ and EL from
{Pointing 21}⁠ at {180}⁠, {86.5}⁠. Record the exact position of the
offset in AZ and El.

}
\hdashrule[0.5ex]{\textwidth}{1pt}{3mm}
  Expected Result \\
{\footnotesize
\begin{itemize}
\tightlist
\item
  The TMA reaches the commanded offset position.
\item
  The \emph{MTMount\_logevent\_azimuthInPosition} and
  \emph{MTMount\_logevent\_elevationInPosition}inPosition parameter =
  true.
\end{itemize}

}

\begin{tabular}{p{2cm}}
\toprule
Step 95  \\ \hline
\end{tabular}
 Description \\
{\footnotesize
\textbf{Position the dome}\\
Command the Dome to {Pointing 22}⁠ to {\{Dome AZ\}}⁠

}
\hdashrule[0.5ex]{\textwidth}{1pt}{3mm}
  Expected Result \\
{\footnotesize
The Dome starts moving.

}

\begin{tabular}{p{2cm}}
\toprule
Step 96  \\ \hline
\end{tabular}
 Description \\
{\footnotesize
\textbf{Point the TMA to (Az, El)-pattern position + DIMM pattern offset
\textbf{\textbf{and take DIMM images}}\\
}

\begin{itemize}
\tightlist
\item
  Point the TMA back to {Pointing 22}⁠ at {180}⁠ + DIMM offset, {75}⁠ +
  DIMM offset.
\item
  While tracking, take DIMM images with XXXs exposure time and inspect
  the quality.
\end{itemize}

}
\hdashrule[0.5ex]{\textwidth}{1pt}{3mm}
  Expected Result \\
{\footnotesize
\begin{itemize}
\tightlist
\item
  TMA reaches the position.
\item
  DIMM image quality is sufficient
\end{itemize}

}

\begin{tabular}{p{2cm}}
\toprule
Step 97  \\ \hline
\end{tabular}
 Description \\
{\footnotesize
Wait for the Dome to reach the commanded position.

}
\hdashrule[0.5ex]{\textwidth}{1pt}{3mm}
  Expected Result \\
{\footnotesize
The \emph{MTDome\_logevent\_azMotion} and
\emph{MTDome\_logevent\_elMotion} inPosition parameter = true.

}

\begin{tabular}{p{2cm}}
\toprule
Step 98  \\ \hline
\end{tabular}
 Description \\
{\footnotesize
\textbf{Point the TMA to (Az, El)-pattern position}\\
Point the TMA to {Pointing 22}⁠ at {180}⁠ , {75}⁠ .

}
\hdashrule[0.5ex]{\textwidth}{1pt}{3mm}
  Expected Result \\
{\footnotesize
The TMA starts moving.

}

\begin{tabular}{p{2cm}}
\toprule
Step 99  \\ \hline
\end{tabular}
 Description \\
{\footnotesize
Wait for the TMA to reach the commanded position.

}
\hdashrule[0.5ex]{\textwidth}{1pt}{3mm}
  Expected Result \\
{\footnotesize
The \emph{MTMount\_logevent\_azimuthInPosition} and
\emph{MTMount\_logevent\_elevationInPosition} inPosition parameter =
true.

}

\begin{tabular}{p{2cm}}
\toprule
Step 100  \\ \hline
\end{tabular}
 Description \\
{\footnotesize
\textbf{Find DIMM Object and DIMM Pattern Offset}\\

\begin{itemize}
\tightlist
\item
  While tracking, take a 10-sec exposure with the StarTracker.
\item
  Load the image into an image viewer.
\item
  Overlay the GAIA catalog.
\item
  Select a star brighter than XXX mag (bright enough for the DIMM).
\item
  Calculate the pixel offset between the StarTracker and the DIMM.
\item
  Transform the offset into AZ and EL offsets.
\end{itemize}

}
\hdashrule[0.5ex]{\textwidth}{1pt}{3mm}
  Expected Result \\
{\footnotesize
\begin{itemize}
\tightlist
\item
  An image was successfully taken with the StarTracker and is of
  sufficient quality.
\item
  AZ and EL offsets are available.
\end{itemize}

}

\begin{tabular}{p{2cm}}
\toprule
Step 101  \\ \hline
\end{tabular}
 Description \\
{\footnotesize
\textbf{Move TMA to the 1. random distance of 3.5deg}\\
Point the TMA to a random 3.5 deg combined offset in AZ and EL from
{Pointing 22}⁠ at {180}⁠, {75}⁠. Record the exact position of the offset
in AZ and El.

}
\hdashrule[0.5ex]{\textwidth}{1pt}{3mm}
  Expected Result \\
{\footnotesize
\begin{itemize}
\tightlist
\item
  The TMA reaches the commanded offset position.
\item
  The \emph{MTMount\_logevent\_azimuthInPosition} and
  \emph{MTMount\_logevent\_elevationInPosition}inPosition parameter =
  true.
\end{itemize}

}

\begin{tabular}{p{2cm}}
\toprule
Step 102  \\ \hline
\end{tabular}
 Description \\
{\footnotesize
\textbf{Position the dome}\\
Command the Dome to {Pointing 23}⁠ to {\{Dome AZ\}}⁠

}
\hdashrule[0.5ex]{\textwidth}{1pt}{3mm}
  Expected Result \\
{\footnotesize
The Dome starts moving.

}

\begin{tabular}{p{2cm}}
\toprule
Step 103  \\ \hline
\end{tabular}
 Description \\
{\footnotesize
\textbf{Point the TMA to (Az, El)-pattern position + DIMM pattern offset
\textbf{\textbf{and take DIMM images}}\\
}

\begin{itemize}
\tightlist
\item
  Point the TMA back to {Pointing 23}⁠ at {180}⁠ + DIMM offset, {45}⁠ +
  DIMM offset.
\item
  While tracking, take DIMM images with XXXs exposure time and inspect
  the quality.
\end{itemize}

}
\hdashrule[0.5ex]{\textwidth}{1pt}{3mm}
  Expected Result \\
{\footnotesize
\begin{itemize}
\tightlist
\item
  TMA reaches the position.
\item
  DIMM image quality is sufficient
\end{itemize}

}

\begin{tabular}{p{2cm}}
\toprule
Step 104  \\ \hline
\end{tabular}
 Description \\
{\footnotesize
Wait for the Dome to reach the commanded position.

}
\hdashrule[0.5ex]{\textwidth}{1pt}{3mm}
  Expected Result \\
{\footnotesize
The \emph{MTDome\_logevent\_azMotion} and
\emph{MTDome\_logevent\_elMotion} inPosition parameter = true.

}

\begin{tabular}{p{2cm}}
\toprule
Step 105  \\ \hline
\end{tabular}
 Description \\
{\footnotesize
\textbf{Point the TMA to (Az, El)-pattern position}\\
Point the TMA to {Pointing 23}⁠ at {180}⁠ , {45}⁠ .

}
\hdashrule[0.5ex]{\textwidth}{1pt}{3mm}
  Expected Result \\
{\footnotesize
The TMA starts moving.

}

\begin{tabular}{p{2cm}}
\toprule
Step 106  \\ \hline
\end{tabular}
 Description \\
{\footnotesize
Wait for the TMA to reach the commanded position.

}
\hdashrule[0.5ex]{\textwidth}{1pt}{3mm}
  Expected Result \\
{\footnotesize
The \emph{MTMount\_logevent\_azimuthInPosition} and
\emph{MTMount\_logevent\_elevationInPosition} inPosition parameter =
true.

}

\begin{tabular}{p{2cm}}
\toprule
Step 107  \\ \hline
\end{tabular}
 Description \\
{\footnotesize
\textbf{Find DIMM Object and DIMM Pattern Offset}\\

\begin{itemize}
\tightlist
\item
  While tracking, take a 10-sec exposure with the StarTracker.
\item
  Load the image into an image viewer.
\item
  Overlay the GAIA catalog.
\item
  Select a star brighter than XXX mag (bright enough for the DIMM).
\item
  Calculate the pixel offset between the StarTracker and the DIMM.
\item
  Transform the offset into AZ and EL offsets.
\end{itemize}

}
\hdashrule[0.5ex]{\textwidth}{1pt}{3mm}
  Expected Result \\
{\footnotesize
\begin{itemize}
\tightlist
\item
  An image was successfully taken with the StarTracker and is of
  sufficient quality.
\item
  AZ and EL offsets are available.
\end{itemize}

}

\begin{tabular}{p{2cm}}
\toprule
Step 108  \\ \hline
\end{tabular}
 Description \\
{\footnotesize
\textbf{Move TMA to the 1. random distance of 3.5deg}\\
Point the TMA to a random 3.5 deg combined offset in AZ and EL from
{Pointing 23}⁠ at {180}⁠, {45}⁠. Record the exact position of the offset
in AZ and El.

}
\hdashrule[0.5ex]{\textwidth}{1pt}{3mm}
  Expected Result \\
{\footnotesize
\begin{itemize}
\tightlist
\item
  The TMA reaches the commanded offset position.
\item
  The \emph{MTMount\_logevent\_azimuthInPosition} and
  \emph{MTMount\_logevent\_elevationInPosition}inPosition parameter =
  true.
\end{itemize}

}

\begin{tabular}{p{2cm}}
\toprule
Step 109  \\ \hline
\end{tabular}
 Description \\
{\footnotesize
\textbf{Position the dome}\\
Command the Dome to {Pointing 24}⁠ to {\{Dome AZ\}}⁠

}
\hdashrule[0.5ex]{\textwidth}{1pt}{3mm}
  Expected Result \\
{\footnotesize
The Dome starts moving.

}

\begin{tabular}{p{2cm}}
\toprule
Step 110  \\ \hline
\end{tabular}
 Description \\
{\footnotesize
\textbf{Point the TMA to (Az, El)-pattern position + DIMM pattern offset
\textbf{\textbf{and take DIMM images}}\\
}

\begin{itemize}
\tightlist
\item
  Point the TMA back to {Pointing 24}⁠ at {180}⁠ + DIMM offset, {15}⁠ +
  DIMM offset.
\item
  While tracking, take DIMM images with XXXs exposure time and inspect
  the quality.
\end{itemize}

}
\hdashrule[0.5ex]{\textwidth}{1pt}{3mm}
  Expected Result \\
{\footnotesize
\begin{itemize}
\tightlist
\item
  TMA reaches the position.
\item
  DIMM image quality is sufficient
\end{itemize}

}

\begin{tabular}{p{2cm}}
\toprule
Step 111  \\ \hline
\end{tabular}
 Description \\
{\footnotesize
Wait for the Dome to reach the commanded position.

}
\hdashrule[0.5ex]{\textwidth}{1pt}{3mm}
  Expected Result \\
{\footnotesize
The \emph{MTDome\_logevent\_azMotion} and
\emph{MTDome\_logevent\_elMotion} inPosition parameter = true.

}

\begin{tabular}{p{2cm}}
\toprule
Step 112  \\ \hline
\end{tabular}
 Description \\
{\footnotesize
\textbf{Point the TMA to (Az, El)-pattern position}\\
Point the TMA to {Pointing 24}⁠ at {180}⁠ , {15}⁠ .

}
\hdashrule[0.5ex]{\textwidth}{1pt}{3mm}
  Expected Result \\
{\footnotesize
The TMA starts moving.

}

\begin{tabular}{p{2cm}}
\toprule
Step 113  \\ \hline
\end{tabular}
 Description \\
{\footnotesize
Wait for the TMA to reach the commanded position.

}
\hdashrule[0.5ex]{\textwidth}{1pt}{3mm}
  Expected Result \\
{\footnotesize
The \emph{MTMount\_logevent\_azimuthInPosition} and
\emph{MTMount\_logevent\_elevationInPosition} inPosition parameter =
true.

}

\begin{tabular}{p{2cm}}
\toprule
Step 114  \\ \hline
\end{tabular}
 Description \\
{\footnotesize
\textbf{Find DIMM Object and DIMM Pattern Offset}\\

\begin{itemize}
\tightlist
\item
  While tracking, take a 10-sec exposure with the StarTracker.
\item
  Load the image into an image viewer.
\item
  Overlay the GAIA catalog.
\item
  Select a star brighter than XXX mag (bright enough for the DIMM).
\item
  Calculate the pixel offset between the StarTracker and the DIMM.
\item
  Transform the offset into AZ and EL offsets.
\end{itemize}

}
\hdashrule[0.5ex]{\textwidth}{1pt}{3mm}
  Expected Result \\
{\footnotesize
\begin{itemize}
\tightlist
\item
  An image was successfully taken with the StarTracker and is of
  sufficient quality.
\item
  AZ and EL offsets are available.
\end{itemize}

}

\begin{tabular}{p{2cm}}
\toprule
Step 115  \\ \hline
\end{tabular}
 Description \\
{\footnotesize
\textbf{Move TMA to the 1. random distance of 3.5deg}\\
Point the TMA to a random 3.5 deg combined offset in AZ and EL from
{Pointing 24}⁠ at {180}⁠, {15}⁠. Record the exact position of the offset
in AZ and El.

}
\hdashrule[0.5ex]{\textwidth}{1pt}{3mm}
  Expected Result \\
{\footnotesize
\begin{itemize}
\tightlist
\item
  The TMA reaches the commanded offset position.
\item
  The \emph{MTMount\_logevent\_azimuthInPosition} and
  \emph{MTMount\_logevent\_elevationInPosition}inPosition parameter =
  true.
\end{itemize}

}

\begin{tabular}{p{2cm}}
\toprule
Step 116  \\ \hline
\end{tabular}
 Description \\
{\footnotesize
\textbf{Position the dome}\\
Command the Dome to {Pointing 25}⁠ to {\{Dome AZ\}}⁠

}
\hdashrule[0.5ex]{\textwidth}{1pt}{3mm}
  Expected Result \\
{\footnotesize
The Dome starts moving.

}

\begin{tabular}{p{2cm}}
\toprule
Step 117  \\ \hline
\end{tabular}
 Description \\
{\footnotesize
\textbf{Point the TMA to (Az, El)-pattern position + DIMM pattern offset
\textbf{\textbf{and take DIMM images}}\\
}

\begin{itemize}
\tightlist
\item
  Point the TMA back to {Pointing 25}⁠ at {270}⁠ + DIMM offset, {15}⁠ +
  DIMM offset.
\item
  While tracking, take DIMM images with XXXs exposure time and inspect
  the quality.
\end{itemize}

}
\hdashrule[0.5ex]{\textwidth}{1pt}{3mm}
  Expected Result \\
{\footnotesize
\begin{itemize}
\tightlist
\item
  TMA reaches the position.
\item
  DIMM image quality is sufficient
\end{itemize}

}

\begin{tabular}{p{2cm}}
\toprule
Step 118  \\ \hline
\end{tabular}
 Description \\
{\footnotesize
Wait for the Dome to reach the commanded position.

}
\hdashrule[0.5ex]{\textwidth}{1pt}{3mm}
  Expected Result \\
{\footnotesize
The \emph{MTDome\_logevent\_azMotion} and
\emph{MTDome\_logevent\_elMotion} inPosition parameter = true.

}

\begin{tabular}{p{2cm}}
\toprule
Step 119  \\ \hline
\end{tabular}
 Description \\
{\footnotesize
\textbf{Point the TMA to (Az, El)-pattern position}\\
Point the TMA to {Pointing 25}⁠ at {270}⁠ , {15}⁠ .

}
\hdashrule[0.5ex]{\textwidth}{1pt}{3mm}
  Expected Result \\
{\footnotesize
The TMA starts moving.

}

\begin{tabular}{p{2cm}}
\toprule
Step 120  \\ \hline
\end{tabular}
 Description \\
{\footnotesize
Wait for the TMA to reach the commanded position.

}
\hdashrule[0.5ex]{\textwidth}{1pt}{3mm}
  Expected Result \\
{\footnotesize
The \emph{MTMount\_logevent\_azimuthInPosition} and
\emph{MTMount\_logevent\_elevationInPosition} inPosition parameter =
true.

}

\begin{tabular}{p{2cm}}
\toprule
Step 121  \\ \hline
\end{tabular}
 Description \\
{\footnotesize
\textbf{Find DIMM Object and DIMM Pattern Offset}\\

\begin{itemize}
\tightlist
\item
  While tracking, take a 10-sec exposure with the StarTracker.
\item
  Load the image into an image viewer.
\item
  Overlay the GAIA catalog.
\item
  Select a star brighter than XXX mag (bright enough for the DIMM).
\item
  Calculate the pixel offset between the StarTracker and the DIMM.
\item
  Transform the offset into AZ and EL offsets.
\end{itemize}

}
\hdashrule[0.5ex]{\textwidth}{1pt}{3mm}
  Expected Result \\
{\footnotesize
\begin{itemize}
\tightlist
\item
  An image was successfully taken with the StarTracker and is of
  sufficient quality.
\item
  AZ and EL offsets are available.
\end{itemize}

}

\begin{tabular}{p{2cm}}
\toprule
Step 122  \\ \hline
\end{tabular}
 Description \\
{\footnotesize
\textbf{Move TMA to the 1. random distance of 3.5deg}\\
Point the TMA to a random 3.5 deg combined offset in AZ and EL from
{Pointing 25}⁠ at {270}⁠, {15}⁠. Record the exact position of the offset
in AZ and El.

}
\hdashrule[0.5ex]{\textwidth}{1pt}{3mm}
  Expected Result \\
{\footnotesize
\begin{itemize}
\tightlist
\item
  The TMA reaches the commanded offset position.
\item
  The \emph{MTMount\_logevent\_azimuthInPosition} and
  \emph{MTMount\_logevent\_elevationInPosition}inPosition parameter =
  true.
\end{itemize}

}

\begin{tabular}{p{2cm}}
\toprule
Step 123  \\ \hline
\end{tabular}
 Description \\
{\footnotesize
\textbf{Position the dome}\\
Command the Dome to {Pointing 26}⁠ to {\{Dome AZ\}}⁠

}
\hdashrule[0.5ex]{\textwidth}{1pt}{3mm}
  Expected Result \\
{\footnotesize
The Dome starts moving.

}

\begin{tabular}{p{2cm}}
\toprule
Step 124  \\ \hline
\end{tabular}
 Description \\
{\footnotesize
\textbf{Point the TMA to (Az, El)-pattern position + DIMM pattern offset
\textbf{\textbf{and take DIMM images}}\\
}

\begin{itemize}
\tightlist
\item
  Point the TMA back to {Pointing 26}⁠ at {270}⁠ + DIMM offset, {45}⁠ +
  DIMM offset.
\item
  While tracking, take DIMM images with XXXs exposure time and inspect
  the quality.
\end{itemize}

}
\hdashrule[0.5ex]{\textwidth}{1pt}{3mm}
  Expected Result \\
{\footnotesize
\begin{itemize}
\tightlist
\item
  TMA reaches the position.
\item
  DIMM image quality is sufficient
\end{itemize}

}

\begin{tabular}{p{2cm}}
\toprule
Step 125  \\ \hline
\end{tabular}
 Description \\
{\footnotesize
Wait for the Dome to reach the commanded position.

}
\hdashrule[0.5ex]{\textwidth}{1pt}{3mm}
  Expected Result \\
{\footnotesize
The \emph{MTDome\_logevent\_azMotion} and
\emph{MTDome\_logevent\_elMotion} inPosition parameter = true.

}

\begin{tabular}{p{2cm}}
\toprule
Step 126  \\ \hline
\end{tabular}
 Description \\
{\footnotesize
\textbf{Point the TMA to (Az, El)-pattern position}\\
Point the TMA to {Pointing 26}⁠ at {270}⁠ , {45}⁠ .

}
\hdashrule[0.5ex]{\textwidth}{1pt}{3mm}
  Expected Result \\
{\footnotesize
The TMA starts moving.

}

\begin{tabular}{p{2cm}}
\toprule
Step 127  \\ \hline
\end{tabular}
 Description \\
{\footnotesize
Wait for the TMA to reach the commanded position.

}
\hdashrule[0.5ex]{\textwidth}{1pt}{3mm}
  Expected Result \\
{\footnotesize
The \emph{MTMount\_logevent\_azimuthInPosition} and
\emph{MTMount\_logevent\_elevationInPosition} inPosition parameter =
true.

}

\begin{tabular}{p{2cm}}
\toprule
Step 128  \\ \hline
\end{tabular}
 Description \\
{\footnotesize
\textbf{Find DIMM Object and DIMM Pattern Offset}\\

\begin{itemize}
\tightlist
\item
  While tracking, take a 10-sec exposure with the StarTracker.
\item
  Load the image into an image viewer.
\item
  Overlay the GAIA catalog.
\item
  Select a star brighter than XXX mag (bright enough for the DIMM).
\item
  Calculate the pixel offset between the StarTracker and the DIMM.
\item
  Transform the offset into AZ and EL offsets.
\end{itemize}

}
\hdashrule[0.5ex]{\textwidth}{1pt}{3mm}
  Expected Result \\
{\footnotesize
\begin{itemize}
\tightlist
\item
  An image was successfully taken with the StarTracker and is of
  sufficient quality.
\item
  AZ and EL offsets are available.
\end{itemize}

}

\begin{tabular}{p{2cm}}
\toprule
Step 129  \\ \hline
\end{tabular}
 Description \\
{\footnotesize
\textbf{Move TMA to the 1. random distance of 3.5deg}\\
Point the TMA to a random 3.5 deg combined offset in AZ and EL from
{Pointing 26}⁠ at {270}⁠, {45}⁠. Record the exact position of the offset
in AZ and El.

}
\hdashrule[0.5ex]{\textwidth}{1pt}{3mm}
  Expected Result \\
{\footnotesize
\begin{itemize}
\tightlist
\item
  The TMA reaches the commanded offset position.
\item
  The \emph{MTMount\_logevent\_azimuthInPosition} and
  \emph{MTMount\_logevent\_elevationInPosition}inPosition parameter =
  true.
\end{itemize}

}

\begin{tabular}{p{2cm}}
\toprule
Step 130  \\ \hline
\end{tabular}
 Description \\
{\footnotesize
\textbf{Position the dome}\\
Command the Dome to {Pointing 27}⁠ to {\{Dome AZ\}}⁠

}
\hdashrule[0.5ex]{\textwidth}{1pt}{3mm}
  Expected Result \\
{\footnotesize
The Dome starts moving.

}

\begin{tabular}{p{2cm}}
\toprule
Step 131  \\ \hline
\end{tabular}
 Description \\
{\footnotesize
\textbf{Point the TMA to (Az, El)-pattern position + DIMM pattern offset
\textbf{\textbf{and take DIMM images}}\\
}

\begin{itemize}
\tightlist
\item
  Point the TMA back to {Pointing 27}⁠ at {270}⁠ + DIMM offset, {75}⁠ +
  DIMM offset.
\item
  While tracking, take DIMM images with XXXs exposure time and inspect
  the quality.
\end{itemize}

}
\hdashrule[0.5ex]{\textwidth}{1pt}{3mm}
  Expected Result \\
{\footnotesize
\begin{itemize}
\tightlist
\item
  TMA reaches the position.
\item
  DIMM image quality is sufficient
\end{itemize}

}

\begin{tabular}{p{2cm}}
\toprule
Step 132  \\ \hline
\end{tabular}
 Description \\
{\footnotesize
Wait for the Dome to reach the commanded position.

}
\hdashrule[0.5ex]{\textwidth}{1pt}{3mm}
  Expected Result \\
{\footnotesize
The \emph{MTDome\_logevent\_azMotion} and
\emph{MTDome\_logevent\_elMotion} inPosition parameter = true.

}

\begin{tabular}{p{2cm}}
\toprule
Step 133  \\ \hline
\end{tabular}
 Description \\
{\footnotesize
\textbf{Point the TMA to (Az, El)-pattern position}\\
Point the TMA to {Pointing 27}⁠ at {270}⁠ , {75}⁠ .

}
\hdashrule[0.5ex]{\textwidth}{1pt}{3mm}
  Expected Result \\
{\footnotesize
The TMA starts moving.

}

\begin{tabular}{p{2cm}}
\toprule
Step 134  \\ \hline
\end{tabular}
 Description \\
{\footnotesize
Wait for the TMA to reach the commanded position.

}
\hdashrule[0.5ex]{\textwidth}{1pt}{3mm}
  Expected Result \\
{\footnotesize
The \emph{MTMount\_logevent\_azimuthInPosition} and
\emph{MTMount\_logevent\_elevationInPosition} inPosition parameter =
true.

}

\begin{tabular}{p{2cm}}
\toprule
Step 135  \\ \hline
\end{tabular}
 Description \\
{\footnotesize
\textbf{Find DIMM Object and DIMM Pattern Offset}\\

\begin{itemize}
\tightlist
\item
  While tracking, take a 10-sec exposure with the StarTracker.
\item
  Load the image into an image viewer.
\item
  Overlay the GAIA catalog.
\item
  Select a star brighter than XXX mag (bright enough for the DIMM).
\item
  Calculate the pixel offset between the StarTracker and the DIMM.
\item
  Transform the offset into AZ and EL offsets.
\end{itemize}

}
\hdashrule[0.5ex]{\textwidth}{1pt}{3mm}
  Expected Result \\
{\footnotesize
\begin{itemize}
\tightlist
\item
  An image was successfully taken with the StarTracker and is of
  sufficient quality.
\item
  AZ and EL offsets are available.
\end{itemize}

}

\begin{tabular}{p{2cm}}
\toprule
Step 136  \\ \hline
\end{tabular}
 Description \\
{\footnotesize
\textbf{Move TMA to the 1. random distance of 3.5deg}\\
Point the TMA to a random 3.5 deg combined offset in AZ and EL from
{Pointing 27}⁠ at {270}⁠, {75}⁠. Record the exact position of the offset
in AZ and El.

}
\hdashrule[0.5ex]{\textwidth}{1pt}{3mm}
  Expected Result \\
{\footnotesize
\begin{itemize}
\tightlist
\item
  The TMA reaches the commanded offset position.
\item
  The \emph{MTMount\_logevent\_azimuthInPosition} and
  \emph{MTMount\_logevent\_elevationInPosition}inPosition parameter =
  true.
\end{itemize}

}

\begin{tabular}{p{2cm}}
\toprule
Step 137  \\ \hline
\end{tabular}
 Description \\
{\footnotesize
\textbf{Position the dome}\\
Command the Dome to {Pointing 28}⁠ to {\{Dome AZ\}}⁠

}
\hdashrule[0.5ex]{\textwidth}{1pt}{3mm}
  Expected Result \\
{\footnotesize
The Dome starts moving.

}

\begin{tabular}{p{2cm}}
\toprule
Step 138  \\ \hline
\end{tabular}
 Description \\
{\footnotesize
\textbf{Point the TMA to (Az, El)-pattern position + DIMM pattern offset
\textbf{\textbf{and take DIMM images}}\\
}

\begin{itemize}
\tightlist
\item
  Point the TMA back to {Pointing 28}⁠ at {270}⁠ + DIMM offset, {86.5}⁠
  + DIMM offset.
\item
  While tracking, take DIMM images with XXXs exposure time and inspect
  the quality.
\end{itemize}

}
\hdashrule[0.5ex]{\textwidth}{1pt}{3mm}
  Expected Result \\
{\footnotesize
\begin{itemize}
\tightlist
\item
  TMA reaches the position.
\item
  DIMM image quality is sufficient
\end{itemize}

}

\begin{tabular}{p{2cm}}
\toprule
Step 139  \\ \hline
\end{tabular}
 Description \\
{\footnotesize
\textbf{Move TMA to the 2. random distance of 3.5deg}

\begin{itemize}
\tightlist
\item
  Point the TMA to a random 3.5 deg combined offset in AZ and EL from
  {Pointing 1}⁠ at {-270}⁠, {15}⁠. Record the exact position of the
  offset in AZ and El.
\end{itemize}

}
\hdashrule[0.5ex]{\textwidth}{1pt}{3mm}
  Expected Result \\
{\footnotesize
The TMA reaches the commanded offset position.\\[2\baselineskip]

}

\begin{tabular}{p{2cm}}
\toprule
Step 140  \\ \hline
\end{tabular}
 Description \\
{\footnotesize
\textbf{Move TMA to the 2. random distance of 3.5deg}

\begin{itemize}
\tightlist
\item
  Point the TMA to a random 3.5 deg combined offset in AZ and EL from
  {Pointing 2}⁠ at {-270}⁠, {45}⁠. Record the exact position of the
  offset in AZ and El.
\end{itemize}

}
\hdashrule[0.5ex]{\textwidth}{1pt}{3mm}
  Expected Result \\
{\footnotesize
The TMA reaches the commanded offset position.\\[2\baselineskip]

}

\begin{tabular}{p{2cm}}
\toprule
Step 141  \\ \hline
\end{tabular}
 Description \\
{\footnotesize
\textbf{Move TMA to the 2. random distance of 3.5deg}

\begin{itemize}
\tightlist
\item
  Point the TMA to a random 3.5 deg combined offset in AZ and EL from
  {Pointing 3}⁠ at {-270}⁠, {75}⁠. Record the exact position of the
  offset in AZ and El.
\end{itemize}

}
\hdashrule[0.5ex]{\textwidth}{1pt}{3mm}
  Expected Result \\
{\footnotesize
The TMA reaches the commanded offset position.\\[2\baselineskip]

}

\begin{tabular}{p{2cm}}
\toprule
Step 142  \\ \hline
\end{tabular}
 Description \\
{\footnotesize
\textbf{Move TMA to the 2. random distance of 3.5deg}

\begin{itemize}
\tightlist
\item
  Point the TMA to a random 3.5 deg combined offset in AZ and EL from
  {Pointing 4}⁠ at {-270}⁠, {86.5}⁠. Record the exact position of the
  offset in AZ and El.
\end{itemize}

}
\hdashrule[0.5ex]{\textwidth}{1pt}{3mm}
  Expected Result \\
{\footnotesize
The TMA reaches the commanded offset position.\\[2\baselineskip]

}

\begin{tabular}{p{2cm}}
\toprule
Step 143  \\ \hline
\end{tabular}
 Description \\
{\footnotesize
\textbf{Move TMA to the 2. random distance of 3.5deg}

\begin{itemize}
\tightlist
\item
  Point the TMA to a random 3.5 deg combined offset in AZ and EL from
  {Pointing 5}⁠ at {-180}⁠, {86.5}⁠. Record the exact position of the
  offset in AZ and El.
\end{itemize}

}
\hdashrule[0.5ex]{\textwidth}{1pt}{3mm}
  Expected Result \\
{\footnotesize
The TMA reaches the commanded offset position.\\[2\baselineskip]

}

\begin{tabular}{p{2cm}}
\toprule
Step 144  \\ \hline
\end{tabular}
 Description \\
{\footnotesize
\textbf{Move TMA to the 2. random distance of 3.5deg}

\begin{itemize}
\tightlist
\item
  Point the TMA to a random 3.5 deg combined offset in AZ and EL from
  {Pointing 6}⁠ at {-180}⁠, {75}⁠. Record the exact position of the
  offset in AZ and El.
\end{itemize}

}
\hdashrule[0.5ex]{\textwidth}{1pt}{3mm}
  Expected Result \\
{\footnotesize
The TMA reaches the commanded offset position.\\[2\baselineskip]

}

\begin{tabular}{p{2cm}}
\toprule
Step 145  \\ \hline
\end{tabular}
 Description \\
{\footnotesize
\textbf{Move TMA to the 2. random distance of 3.5deg}

\begin{itemize}
\tightlist
\item
  Point the TMA to a random 3.5 deg combined offset in AZ and EL from
  {Pointing 7}⁠ at {-180}⁠, {45}⁠. Record the exact position of the
  offset in AZ and El.
\end{itemize}

}
\hdashrule[0.5ex]{\textwidth}{1pt}{3mm}
  Expected Result \\
{\footnotesize
The TMA reaches the commanded offset position.\\[2\baselineskip]

}

\begin{tabular}{p{2cm}}
\toprule
Step 146  \\ \hline
\end{tabular}
 Description \\
{\footnotesize
\textbf{Move TMA to the 2. random distance of 3.5deg}

\begin{itemize}
\tightlist
\item
  Point the TMA to a random 3.5 deg combined offset in AZ and EL from
  {Pointing 8}⁠ at {-180}⁠, {15}⁠. Record the exact position of the
  offset in AZ and El.
\end{itemize}

}
\hdashrule[0.5ex]{\textwidth}{1pt}{3mm}
  Expected Result \\
{\footnotesize
The TMA reaches the commanded offset position.\\[2\baselineskip]

}

\begin{tabular}{p{2cm}}
\toprule
Step 147  \\ \hline
\end{tabular}
 Description \\
{\footnotesize
\textbf{Move TMA to the 2. random distance of 3.5deg}

\begin{itemize}
\tightlist
\item
  Point the TMA to a random 3.5 deg combined offset in AZ and EL from
  {Pointing 9}⁠ at {-90}⁠, {15}⁠. Record the exact position of the
  offset in AZ and El.
\end{itemize}

}
\hdashrule[0.5ex]{\textwidth}{1pt}{3mm}
  Expected Result \\
{\footnotesize
The TMA reaches the commanded offset position.\\[2\baselineskip]

}

\begin{tabular}{p{2cm}}
\toprule
Step 148  \\ \hline
\end{tabular}
 Description \\
{\footnotesize
\textbf{Move TMA to the 2. random distance of 3.5deg}

\begin{itemize}
\tightlist
\item
  Point the TMA to a random 3.5 deg combined offset in AZ and EL from
  {Pointing 10}⁠ at {-90}⁠, {45}⁠. Record the exact position of the
  offset in AZ and El.
\end{itemize}

}
\hdashrule[0.5ex]{\textwidth}{1pt}{3mm}
  Expected Result \\
{\footnotesize
The TMA reaches the commanded offset position.\\[2\baselineskip]

}

\begin{tabular}{p{2cm}}
\toprule
Step 149  \\ \hline
\end{tabular}
 Description \\
{\footnotesize
\textbf{Move TMA to the 2. random distance of 3.5deg}

\begin{itemize}
\tightlist
\item
  Point the TMA to a random 3.5 deg combined offset in AZ and EL from
  {Pointing 11}⁠ at {-90}⁠, {75}⁠. Record the exact position of the
  offset in AZ and El.
\end{itemize}

}
\hdashrule[0.5ex]{\textwidth}{1pt}{3mm}
  Expected Result \\
{\footnotesize
The TMA reaches the commanded offset position.\\[2\baselineskip]

}

\begin{tabular}{p{2cm}}
\toprule
Step 150  \\ \hline
\end{tabular}
 Description \\
{\footnotesize
\textbf{Move TMA to the 2. random distance of 3.5deg}

\begin{itemize}
\tightlist
\item
  Point the TMA to a random 3.5 deg combined offset in AZ and EL from
  {Pointing 12}⁠ at {-90}⁠, {86.5}⁠. Record the exact position of the
  offset in AZ and El.
\end{itemize}

}
\hdashrule[0.5ex]{\textwidth}{1pt}{3mm}
  Expected Result \\
{\footnotesize
The TMA reaches the commanded offset position.\\[2\baselineskip]

}

\begin{tabular}{p{2cm}}
\toprule
Step 151  \\ \hline
\end{tabular}
 Description \\
{\footnotesize
\textbf{Move TMA to the 2. random distance of 3.5deg}

\begin{itemize}
\tightlist
\item
  Point the TMA to a random 3.5 deg combined offset in AZ and EL from
  {Pointing 13}⁠ at {0}⁠, {86.5}⁠. Record the exact position of the
  offset in AZ and El.
\end{itemize}

}
\hdashrule[0.5ex]{\textwidth}{1pt}{3mm}
  Expected Result \\
{\footnotesize
The TMA reaches the commanded offset position.\\[2\baselineskip]

}

\begin{tabular}{p{2cm}}
\toprule
Step 152  \\ \hline
\end{tabular}
 Description \\
{\footnotesize
\textbf{Move TMA to the 2. random distance of 3.5deg}

\begin{itemize}
\tightlist
\item
  Point the TMA to a random 3.5 deg combined offset in AZ and EL from
  {Pointing 14}⁠ at {0}⁠, {75}⁠. Record the exact position of the offset
  in AZ and El.
\end{itemize}

}
\hdashrule[0.5ex]{\textwidth}{1pt}{3mm}
  Expected Result \\
{\footnotesize
The TMA reaches the commanded offset position.\\[2\baselineskip]

}

\begin{tabular}{p{2cm}}
\toprule
Step 153  \\ \hline
\end{tabular}
 Description \\
{\footnotesize
\textbf{Move TMA to the 2. random distance of 3.5deg}

\begin{itemize}
\tightlist
\item
  Point the TMA to a random 3.5 deg combined offset in AZ and EL from
  {Pointing 15}⁠ at {0}⁠, {45}⁠. Record the exact position of the offset
  in AZ and El.
\end{itemize}

}
\hdashrule[0.5ex]{\textwidth}{1pt}{3mm}
  Expected Result \\
{\footnotesize
The TMA reaches the commanded offset position.\\[2\baselineskip]

}

\begin{tabular}{p{2cm}}
\toprule
Step 154  \\ \hline
\end{tabular}
 Description \\
{\footnotesize
\textbf{Move TMA to the 2. random distance of 3.5deg}

\begin{itemize}
\tightlist
\item
  Point the TMA to a random 3.5 deg combined offset in AZ and EL from
  {Pointing 16}⁠ at {0}⁠, {15}⁠. Record the exact position of the offset
  in AZ and El.
\end{itemize}

}
\hdashrule[0.5ex]{\textwidth}{1pt}{3mm}
  Expected Result \\
{\footnotesize
The TMA reaches the commanded offset position.\\[2\baselineskip]

}

\begin{tabular}{p{2cm}}
\toprule
Step 155  \\ \hline
\end{tabular}
 Description \\
{\footnotesize
\textbf{Move TMA to the 2. random distance of 3.5deg}

\begin{itemize}
\tightlist
\item
  Point the TMA to a random 3.5 deg combined offset in AZ and EL from
  {Pointing 17}⁠ at {90}⁠, {15}⁠. Record the exact position of the
  offset in AZ and El.
\end{itemize}

}
\hdashrule[0.5ex]{\textwidth}{1pt}{3mm}
  Expected Result \\
{\footnotesize
The TMA reaches the commanded offset position.\\[2\baselineskip]

}

\begin{tabular}{p{2cm}}
\toprule
Step 156  \\ \hline
\end{tabular}
 Description \\
{\footnotesize
\textbf{Move TMA to the 2. random distance of 3.5deg}

\begin{itemize}
\tightlist
\item
  Point the TMA to a random 3.5 deg combined offset in AZ and EL from
  {Pointing 18}⁠ at {90}⁠, {45}⁠. Record the exact position of the
  offset in AZ and El.
\end{itemize}

}
\hdashrule[0.5ex]{\textwidth}{1pt}{3mm}
  Expected Result \\
{\footnotesize
The TMA reaches the commanded offset position.\\[2\baselineskip]

}

\begin{tabular}{p{2cm}}
\toprule
Step 157  \\ \hline
\end{tabular}
 Description \\
{\footnotesize
\textbf{Move TMA to the 2. random distance of 3.5deg}

\begin{itemize}
\tightlist
\item
  Point the TMA to a random 3.5 deg combined offset in AZ and EL from
  {Pointing 19}⁠ at {90}⁠, {75}⁠. Record the exact position of the
  offset in AZ and El.
\end{itemize}

}
\hdashrule[0.5ex]{\textwidth}{1pt}{3mm}
  Expected Result \\
{\footnotesize
The TMA reaches the commanded offset position.\\[2\baselineskip]

}

\begin{tabular}{p{2cm}}
\toprule
Step 158  \\ \hline
\end{tabular}
 Description \\
{\footnotesize
\textbf{Move TMA to the 2. random distance of 3.5deg}

\begin{itemize}
\tightlist
\item
  Point the TMA to a random 3.5 deg combined offset in AZ and EL from
  {Pointing 20}⁠ at {90}⁠, {86.5}⁠. Record the exact position of the
  offset in AZ and El.
\end{itemize}

}
\hdashrule[0.5ex]{\textwidth}{1pt}{3mm}
  Expected Result \\
{\footnotesize
The TMA reaches the commanded offset position.\\[2\baselineskip]

}

\begin{tabular}{p{2cm}}
\toprule
Step 159  \\ \hline
\end{tabular}
 Description \\
{\footnotesize
\textbf{Move TMA to the 2. random distance of 3.5deg}

\begin{itemize}
\tightlist
\item
  Point the TMA to a random 3.5 deg combined offset in AZ and EL from
  {Pointing 21}⁠ at {180}⁠, {86.5}⁠. Record the exact position of the
  offset in AZ and El.
\end{itemize}

}
\hdashrule[0.5ex]{\textwidth}{1pt}{3mm}
  Expected Result \\
{\footnotesize
The TMA reaches the commanded offset position.\\[2\baselineskip]

}

\begin{tabular}{p{2cm}}
\toprule
Step 160  \\ \hline
\end{tabular}
 Description \\
{\footnotesize
\textbf{Move TMA to the 2. random distance of 3.5deg}

\begin{itemize}
\tightlist
\item
  Point the TMA to a random 3.5 deg combined offset in AZ and EL from
  {Pointing 22}⁠ at {180}⁠, {75}⁠. Record the exact position of the
  offset in AZ and El.
\end{itemize}

}
\hdashrule[0.5ex]{\textwidth}{1pt}{3mm}
  Expected Result \\
{\footnotesize
The TMA reaches the commanded offset position.\\[2\baselineskip]

}

\begin{tabular}{p{2cm}}
\toprule
Step 161  \\ \hline
\end{tabular}
 Description \\
{\footnotesize
\textbf{Move TMA to the 2. random distance of 3.5deg}

\begin{itemize}
\tightlist
\item
  Point the TMA to a random 3.5 deg combined offset in AZ and EL from
  {Pointing 23}⁠ at {180}⁠, {45}⁠. Record the exact position of the
  offset in AZ and El.
\end{itemize}

}
\hdashrule[0.5ex]{\textwidth}{1pt}{3mm}
  Expected Result \\
{\footnotesize
The TMA reaches the commanded offset position.\\[2\baselineskip]

}

\begin{tabular}{p{2cm}}
\toprule
Step 162  \\ \hline
\end{tabular}
 Description \\
{\footnotesize
\textbf{Move TMA to the 2. random distance of 3.5deg}

\begin{itemize}
\tightlist
\item
  Point the TMA to a random 3.5 deg combined offset in AZ and EL from
  {Pointing 24}⁠ at {180}⁠, {15}⁠. Record the exact position of the
  offset in AZ and El.
\end{itemize}

}
\hdashrule[0.5ex]{\textwidth}{1pt}{3mm}
  Expected Result \\
{\footnotesize
The TMA reaches the commanded offset position.\\[2\baselineskip]

}

\begin{tabular}{p{2cm}}
\toprule
Step 163  \\ \hline
\end{tabular}
 Description \\
{\footnotesize
\textbf{Move TMA to the 2. random distance of 3.5deg}

\begin{itemize}
\tightlist
\item
  Point the TMA to a random 3.5 deg combined offset in AZ and EL from
  {Pointing 25}⁠ at {270}⁠, {15}⁠. Record the exact position of the
  offset in AZ and El.
\end{itemize}

}
\hdashrule[0.5ex]{\textwidth}{1pt}{3mm}
  Expected Result \\
{\footnotesize
The TMA reaches the commanded offset position.\\[2\baselineskip]

}

\begin{tabular}{p{2cm}}
\toprule
Step 164  \\ \hline
\end{tabular}
 Description \\
{\footnotesize
\textbf{Move TMA to the 2. random distance of 3.5deg}

\begin{itemize}
\tightlist
\item
  Point the TMA to a random 3.5 deg combined offset in AZ and EL from
  {Pointing 26}⁠ at {270}⁠, {45}⁠. Record the exact position of the
  offset in AZ and El.
\end{itemize}

}
\hdashrule[0.5ex]{\textwidth}{1pt}{3mm}
  Expected Result \\
{\footnotesize
The TMA reaches the commanded offset position.\\[2\baselineskip]

}

\begin{tabular}{p{2cm}}
\toprule
Step 165  \\ \hline
\end{tabular}
 Description \\
{\footnotesize
\textbf{Move TMA to the 2. random distance of 3.5deg}

\begin{itemize}
\tightlist
\item
  Point the TMA to a random 3.5 deg combined offset in AZ and EL from
  {Pointing 27}⁠ at {270}⁠, {75}⁠. Record the exact position of the
  offset in AZ and El.
\end{itemize}

}
\hdashrule[0.5ex]{\textwidth}{1pt}{3mm}
  Expected Result \\
{\footnotesize
The TMA reaches the commanded offset position.\\[2\baselineskip]

}

\begin{tabular}{p{2cm}}
\toprule
Step 166  \\ \hline
\end{tabular}
 Description \\
{\footnotesize
\textbf{Move TMA to the 2. random distance of 3.5deg}

\begin{itemize}
\tightlist
\item
  Point the TMA to a random 3.5 deg combined offset in AZ and EL from
  {Pointing 13}⁠ at {0}⁠, {86.5}⁠. Record the exact position of the
  offset in AZ and El.
\end{itemize}

}
\hdashrule[0.5ex]{\textwidth}{1pt}{3mm}
  Expected Result \\
{\footnotesize
The TMA reaches the commanded offset position.\\[2\baselineskip]

}

\begin{tabular}{p{2cm}}
\toprule
Step 167  \\ \hline
\end{tabular}
 Description \\
{\footnotesize
\textbf{Move TMA to the 2. random distance of 3.5deg}

\begin{itemize}
\tightlist
\item
  Point the TMA to a random 3.5 deg combined offset in AZ and EL from
  {Pointing 28}⁠ at {270}⁠, {86.5}⁠. Record the exact position of the
  offset in AZ and El.
\end{itemize}

}
\hdashrule[0.5ex]{\textwidth}{1pt}{3mm}
  Expected Result \\
{\footnotesize
The TMA reaches the commanded offset position.\\[2\baselineskip]

}

\begin{tabular}{p{2cm}}
\toprule
Step 168  \\ \hline
\end{tabular}
 Description \\
{\footnotesize
\textbf{Move TMA to the 2. random distance of 3.5deg}

\begin{itemize}
\tightlist
\item
  Point the TMA to a random 3.5 deg combined offset in AZ and EL from
  {Pointing 14}⁠ at {0}⁠, {75}⁠. Record the exact position of the offset
  in AZ and El.
\end{itemize}

}
\hdashrule[0.5ex]{\textwidth}{1pt}{3mm}
  Expected Result \\
{\footnotesize
The TMA reaches the commanded offset position.\\[2\baselineskip]

}

\begin{tabular}{p{2cm}}
\toprule
Step 169  \\ \hline
\end{tabular}
 Description \\
{\footnotesize
\textbf{Move TMA to the 2. random distance of 3.5deg}

\begin{itemize}
\tightlist
\item
  Point the TMA to a random 3.5 deg combined offset in AZ and EL from
  {Pointing 15}⁠ at {0}⁠, {45}⁠. Record the exact position of the offset
  in AZ and El.
\end{itemize}

}
\hdashrule[0.5ex]{\textwidth}{1pt}{3mm}
  Expected Result \\
{\footnotesize
The TMA reaches the commanded offset position.\\[2\baselineskip]

}

\begin{tabular}{p{2cm}}
\toprule
Step 170  \\ \hline
\end{tabular}
 Description \\
{\footnotesize
\textbf{Move TMA to the 2. random distance of 3.5deg}

\begin{itemize}
\tightlist
\item
  Point the TMA to a random 3.5 deg combined offset in AZ and EL from
  {Pointing 16}⁠ at {0}⁠, {15}⁠. Record the exact position of the offset
  in AZ and El.
\end{itemize}

}
\hdashrule[0.5ex]{\textwidth}{1pt}{3mm}
  Expected Result \\
{\footnotesize
The TMA reaches the commanded offset position.\\[2\baselineskip]

}

\begin{tabular}{p{2cm}}
\toprule
Step 171  \\ \hline
\end{tabular}
 Description \\
{\footnotesize
\textbf{Move TMA to the 2. random distance of 3.5deg}

\begin{itemize}
\tightlist
\item
  Point the TMA to a random 3.5 deg combined offset in AZ and EL from
  {Pointing 17}⁠ at {90}⁠, {15}⁠. Record the exact position of the
  offset in AZ and El.
\end{itemize}

}
\hdashrule[0.5ex]{\textwidth}{1pt}{3mm}
  Expected Result \\
{\footnotesize
The TMA reaches the commanded offset position.\\[2\baselineskip]

}

\begin{tabular}{p{2cm}}
\toprule
Step 172  \\ \hline
\end{tabular}
 Description \\
{\footnotesize
\textbf{Move TMA to the 2. random distance of 3.5deg}

\begin{itemize}
\tightlist
\item
  Point the TMA to a random 3.5 deg combined offset in AZ and EL from
  {Pointing 18}⁠ at {90}⁠, {45}⁠. Record the exact position of the
  offset in AZ and El.
\end{itemize}

}
\hdashrule[0.5ex]{\textwidth}{1pt}{3mm}
  Expected Result \\
{\footnotesize
The TMA reaches the commanded offset position.\\[2\baselineskip]

}

\begin{tabular}{p{2cm}}
\toprule
Step 173  \\ \hline
\end{tabular}
 Description \\
{\footnotesize
\textbf{Move TMA to the 2. random distance of 3.5deg}

\begin{itemize}
\tightlist
\item
  Point the TMA to a random 3.5 deg combined offset in AZ and EL from
  {Pointing 19}⁠ at {90}⁠, {75}⁠. Record the exact position of the
  offset in AZ and El.
\end{itemize}

}
\hdashrule[0.5ex]{\textwidth}{1pt}{3mm}
  Expected Result \\
{\footnotesize
The TMA reaches the commanded offset position.\\[2\baselineskip]

}

\begin{tabular}{p{2cm}}
\toprule
Step 174  \\ \hline
\end{tabular}
 Description \\
{\footnotesize
\textbf{Move TMA to the 2. random distance of 3.5deg}

\begin{itemize}
\tightlist
\item
  Point the TMA to a random 3.5 deg combined offset in AZ and EL from
  {Pointing 20}⁠ at {90}⁠, {86.5}⁠. Record the exact position of the
  offset in AZ and El.
\end{itemize}

}
\hdashrule[0.5ex]{\textwidth}{1pt}{3mm}
  Expected Result \\
{\footnotesize
The TMA reaches the commanded offset position.\\[2\baselineskip]

}

\begin{tabular}{p{2cm}}
\toprule
Step 175  \\ \hline
\end{tabular}
 Description \\
{\footnotesize
\textbf{Move TMA to the 2. random distance of 3.5deg}

\begin{itemize}
\tightlist
\item
  Point the TMA to a random 3.5 deg combined offset in AZ and EL from
  {Pointing 21}⁠ at {180}⁠, {86.5}⁠. Record the exact position of the
  offset in AZ and El.
\end{itemize}

}
\hdashrule[0.5ex]{\textwidth}{1pt}{3mm}
  Expected Result \\
{\footnotesize
The TMA reaches the commanded offset position.\\[2\baselineskip]

}

\begin{tabular}{p{2cm}}
\toprule
Step 176  \\ \hline
\end{tabular}
 Description \\
{\footnotesize
\textbf{Move TMA to the 2. random distance of 3.5deg}

\begin{itemize}
\tightlist
\item
  Point the TMA to a random 3.5 deg combined offset in AZ and EL from
  {Pointing 22}⁠ at {180}⁠, {75}⁠. Record the exact position of the
  offset in AZ and El.
\end{itemize}

}
\hdashrule[0.5ex]{\textwidth}{1pt}{3mm}
  Expected Result \\
{\footnotesize
The TMA reaches the commanded offset position.\\[2\baselineskip]

}

\begin{tabular}{p{2cm}}
\toprule
Step 177  \\ \hline
\end{tabular}
 Description \\
{\footnotesize
\textbf{Move TMA to the 2. random distance of 3.5deg}

\begin{itemize}
\tightlist
\item
  Point the TMA to a random 3.5 deg combined offset in AZ and EL from
  {Pointing 23}⁠ at {180}⁠, {45}⁠. Record the exact position of the
  offset in AZ and El.
\end{itemize}

}
\hdashrule[0.5ex]{\textwidth}{1pt}{3mm}
  Expected Result \\
{\footnotesize
The TMA reaches the commanded offset position.\\[2\baselineskip]

}

\begin{tabular}{p{2cm}}
\toprule
Step 178  \\ \hline
\end{tabular}
 Description \\
{\footnotesize
\textbf{Move TMA to the 2. random distance of 3.5deg}

\begin{itemize}
\tightlist
\item
  Point the TMA to a random 3.5 deg combined offset in AZ and EL from
  {Pointing 24}⁠ at {180}⁠, {15}⁠. Record the exact position of the
  offset in AZ and El.
\end{itemize}

}
\hdashrule[0.5ex]{\textwidth}{1pt}{3mm}
  Expected Result \\
{\footnotesize
The TMA reaches the commanded offset position.\\[2\baselineskip]

}

\begin{tabular}{p{2cm}}
\toprule
Step 179  \\ \hline
\end{tabular}
 Description \\
{\footnotesize
\textbf{Move TMA to the 2. random distance of 3.5deg}

\begin{itemize}
\tightlist
\item
  Point the TMA to a random 3.5 deg combined offset in AZ and EL from
  {Pointing 25}⁠ at {270}⁠, {15}⁠. Record the exact position of the
  offset in AZ and El.
\end{itemize}

}
\hdashrule[0.5ex]{\textwidth}{1pt}{3mm}
  Expected Result \\
{\footnotesize
The TMA reaches the commanded offset position.\\[2\baselineskip]

}

\begin{tabular}{p{2cm}}
\toprule
Step 180  \\ \hline
\end{tabular}
 Description \\
{\footnotesize
\textbf{Move TMA to the 2. random distance of 3.5deg}

\begin{itemize}
\tightlist
\item
  Point the TMA to a random 3.5 deg combined offset in AZ and EL from
  {Pointing 26}⁠ at {270}⁠, {45}⁠. Record the exact position of the
  offset in AZ and El.
\end{itemize}

}
\hdashrule[0.5ex]{\textwidth}{1pt}{3mm}
  Expected Result \\
{\footnotesize
The TMA reaches the commanded offset position.\\[2\baselineskip]

}

\begin{tabular}{p{2cm}}
\toprule
Step 181  \\ \hline
\end{tabular}
 Description \\
{\footnotesize
\textbf{Move TMA to the 2. random distance of 3.5deg}

\begin{itemize}
\tightlist
\item
  Point the TMA to a random 3.5 deg combined offset in AZ and EL from
  {Pointing 27}⁠ at {270}⁠, {75}⁠. Record the exact position of the
  offset in AZ and El.
\end{itemize}

}
\hdashrule[0.5ex]{\textwidth}{1pt}{3mm}
  Expected Result \\
{\footnotesize
The TMA reaches the commanded offset position.\\[2\baselineskip]

}

\begin{tabular}{p{2cm}}
\toprule
Step 182  \\ \hline
\end{tabular}
 Description \\
{\footnotesize
\textbf{Move TMA to the 2. random distance of 3.5deg}

\begin{itemize}
\tightlist
\item
  Point the TMA to a random 3.5 deg combined offset in AZ and EL from
  {Pointing 28}⁠ at {270}⁠, {86.5}⁠. Record the exact position of the
  offset in AZ and El.
\end{itemize}

}
\hdashrule[0.5ex]{\textwidth}{1pt}{3mm}
  Expected Result \\
{\footnotesize
The TMA reaches the commanded offset position.\\[2\baselineskip]

}

\begin{tabular}{p{2cm}}
\toprule
Step 183  \\ \hline
\end{tabular}
 Description \\
{\footnotesize
\textbf{Find DIMM Object and DIMM Offset}\\

\begin{itemize}
\tightlist
\item
  While tracking, take a 10-sec exposure with the StarTracker.
\item
  Load the image into an image viewer.
\item
  Overlay the GAIA catalog.
\item
  Select a star brighter than XXX mag (bright enough for the DIMM).
\item
  Calculate the pixel offset between the StarTracker and the DIMM.
\item
  Transform the offset into AZ and EL offsets.
\end{itemize}

}
\hdashrule[0.5ex]{\textwidth}{1pt}{3mm}
  Expected Result \\
{\footnotesize
\begin{itemize}
\tightlist
\item
  An image was successfully taken with the StarTracker and is of
  sufficient quality.
\item
  AZ and EL offsets are available.
\end{itemize}

}

\begin{tabular}{p{2cm}}
\toprule
Step 184  \\ \hline
\end{tabular}
 Description \\
{\footnotesize
\textbf{Find DIMM Object and DIMM Offset}\\

\begin{itemize}
\tightlist
\item
  While tracking, take a 10-sec exposure with the StarTracker.
\item
  Load the image into an image viewer.
\item
  Overlay the GAIA catalog.
\item
  Select a star brighter than XXX mag (bright enough for the DIMM).
\item
  Calculate the pixel offset between the StarTracker and the DIMM.
\item
  Transform the offset into AZ and EL offsets.
\end{itemize}

}
\hdashrule[0.5ex]{\textwidth}{1pt}{3mm}
  Expected Result \\
{\footnotesize
\begin{itemize}
\tightlist
\item
  An image was successfully taken with the StarTracker and is of
  sufficient quality.
\item
  AZ and EL offsets are available.
\end{itemize}

}

\begin{tabular}{p{2cm}}
\toprule
Step 185  \\ \hline
\end{tabular}
 Description \\
{\footnotesize
\textbf{Find DIMM Object and DIMM Offset}\\

\begin{itemize}
\tightlist
\item
  While tracking, take a 10-sec exposure with the StarTracker.
\item
  Load the image into an image viewer.
\item
  Overlay the GAIA catalog.
\item
  Select a star brighter than XXX mag (bright enough for the DIMM).
\item
  Calculate the pixel offset between the StarTracker and the DIMM.
\item
  Transform the offset into AZ and EL offsets.
\end{itemize}

}
\hdashrule[0.5ex]{\textwidth}{1pt}{3mm}
  Expected Result \\
{\footnotesize
\begin{itemize}
\tightlist
\item
  An image was successfully taken with the StarTracker and is of
  sufficient quality.
\item
  AZ and EL offsets are available.
\end{itemize}

}

\begin{tabular}{p{2cm}}
\toprule
Step 186  \\ \hline
\end{tabular}
 Description \\
{\footnotesize
\textbf{Find DIMM Object and DIMM Offset}\\

\begin{itemize}
\tightlist
\item
  While tracking, take a 10-sec exposure with the StarTracker.
\item
  Load the image into an image viewer.
\item
  Overlay the GAIA catalog.
\item
  Select a star brighter than XXX mag (bright enough for the DIMM).
\item
  Calculate the pixel offset between the StarTracker and the DIMM.
\item
  Transform the offset into AZ and EL offsets.
\end{itemize}

}
\hdashrule[0.5ex]{\textwidth}{1pt}{3mm}
  Expected Result \\
{\footnotesize
\begin{itemize}
\tightlist
\item
  An image was successfully taken with the StarTracker and is of
  sufficient quality.
\item
  AZ and EL offsets are available.
\end{itemize}

}

\begin{tabular}{p{2cm}}
\toprule
Step 187  \\ \hline
\end{tabular}
 Description \\
{\footnotesize
\textbf{Find DIMM Object and DIMM Offset}\\

\begin{itemize}
\tightlist
\item
  While tracking, take a 10-sec exposure with the StarTracker.
\item
  Load the image into an image viewer.
\item
  Overlay the GAIA catalog.
\item
  Select a star brighter than XXX mag (bright enough for the DIMM).
\item
  Calculate the pixel offset between the StarTracker and the DIMM.
\item
  Transform the offset into AZ and EL offsets.
\end{itemize}

}
\hdashrule[0.5ex]{\textwidth}{1pt}{3mm}
  Expected Result \\
{\footnotesize
\begin{itemize}
\tightlist
\item
  An image was successfully taken with the StarTracker and is of
  sufficient quality.
\item
  AZ and EL offsets are available.
\end{itemize}

}

\begin{tabular}{p{2cm}}
\toprule
Step 188  \\ \hline
\end{tabular}
 Description \\
{\footnotesize
\textbf{Find DIMM Object and DIMM Offset}\\

\begin{itemize}
\tightlist
\item
  While tracking, take a 10-sec exposure with the StarTracker.
\item
  Load the image into an image viewer.
\item
  Overlay the GAIA catalog.
\item
  Select a star brighter than XXX mag (bright enough for the DIMM).
\item
  Calculate the pixel offset between the StarTracker and the DIMM.
\item
  Transform the offset into AZ and EL offsets.
\end{itemize}

}
\hdashrule[0.5ex]{\textwidth}{1pt}{3mm}
  Expected Result \\
{\footnotesize
\begin{itemize}
\tightlist
\item
  An image was successfully taken with the StarTracker and is of
  sufficient quality.
\item
  AZ and EL offsets are available.
\end{itemize}

}

\begin{tabular}{p{2cm}}
\toprule
Step 189  \\ \hline
\end{tabular}
 Description \\
{\footnotesize
\textbf{Find DIMM Object and DIMM Offset}\\

\begin{itemize}
\tightlist
\item
  While tracking, take a 10-sec exposure with the StarTracker.
\item
  Load the image into an image viewer.
\item
  Overlay the GAIA catalog.
\item
  Select a star brighter than XXX mag (bright enough for the DIMM).
\item
  Calculate the pixel offset between the StarTracker and the DIMM.
\item
  Transform the offset into AZ and EL offsets.
\end{itemize}

}
\hdashrule[0.5ex]{\textwidth}{1pt}{3mm}
  Expected Result \\
{\footnotesize
\begin{itemize}
\tightlist
\item
  An image was successfully taken with the StarTracker and is of
  sufficient quality.
\item
  AZ and EL offsets are available.
\end{itemize}

}

\begin{tabular}{p{2cm}}
\toprule
Step 190  \\ \hline
\end{tabular}
 Description \\
{\footnotesize
\textbf{Find DIMM Object and DIMM Offset}\\

\begin{itemize}
\tightlist
\item
  While tracking, take a 10-sec exposure with the StarTracker.
\item
  Load the image into an image viewer.
\item
  Overlay the GAIA catalog.
\item
  Select a star brighter than XXX mag (bright enough for the DIMM).
\item
  Calculate the pixel offset between the StarTracker and the DIMM.
\item
  Transform the offset into AZ and EL offsets.
\end{itemize}

}
\hdashrule[0.5ex]{\textwidth}{1pt}{3mm}
  Expected Result \\
{\footnotesize
\begin{itemize}
\tightlist
\item
  An image was successfully taken with the StarTracker and is of
  sufficient quality.
\item
  AZ and EL offsets are available.
\end{itemize}

}

\begin{tabular}{p{2cm}}
\toprule
Step 191  \\ \hline
\end{tabular}
 Description \\
{\footnotesize
\textbf{Find DIMM Object and DIMM Offset}\\

\begin{itemize}
\tightlist
\item
  While tracking, take a 10-sec exposure with the StarTracker.
\item
  Load the image into an image viewer.
\item
  Overlay the GAIA catalog.
\item
  Select a star brighter than XXX mag (bright enough for the DIMM).
\item
  Calculate the pixel offset between the StarTracker and the DIMM.
\item
  Transform the offset into AZ and EL offsets.
\end{itemize}

}
\hdashrule[0.5ex]{\textwidth}{1pt}{3mm}
  Expected Result \\
{\footnotesize
\begin{itemize}
\tightlist
\item
  An image was successfully taken with the StarTracker and is of
  sufficient quality.
\item
  AZ and EL offsets are available.
\end{itemize}

}

\begin{tabular}{p{2cm}}
\toprule
Step 192  \\ \hline
\end{tabular}
 Description \\
{\footnotesize
\textbf{Find DIMM Object and DIMM Offset}\\

\begin{itemize}
\tightlist
\item
  While tracking, take a 10-sec exposure with the StarTracker.
\item
  Load the image into an image viewer.
\item
  Overlay the GAIA catalog.
\item
  Select a star brighter than XXX mag (bright enough for the DIMM).
\item
  Calculate the pixel offset between the StarTracker and the DIMM.
\item
  Transform the offset into AZ and EL offsets.
\end{itemize}

}
\hdashrule[0.5ex]{\textwidth}{1pt}{3mm}
  Expected Result \\
{\footnotesize
\begin{itemize}
\tightlist
\item
  An image was successfully taken with the StarTracker and is of
  sufficient quality.
\item
  AZ and EL offsets are available.
\end{itemize}

}

\begin{tabular}{p{2cm}}
\toprule
Step 193  \\ \hline
\end{tabular}
 Description \\
{\footnotesize
\textbf{Find DIMM Object and DIMM Offset}\\

\begin{itemize}
\tightlist
\item
  While tracking, take a 10-sec exposure with the StarTracker.
\item
  Load the image into an image viewer.
\item
  Overlay the GAIA catalog.
\item
  Select a star brighter than XXX mag (bright enough for the DIMM).
\item
  Calculate the pixel offset between the StarTracker and the DIMM.
\item
  Transform the offset into AZ and EL offsets.
\end{itemize}

}
\hdashrule[0.5ex]{\textwidth}{1pt}{3mm}
  Expected Result \\
{\footnotesize
\begin{itemize}
\tightlist
\item
  An image was successfully taken with the StarTracker and is of
  sufficient quality.
\item
  AZ and EL offsets are available.
\end{itemize}

}

\begin{tabular}{p{2cm}}
\toprule
Step 194  \\ \hline
\end{tabular}
 Description \\
{\footnotesize
\textbf{Find DIMM Object and DIMM Offset}\\

\begin{itemize}
\tightlist
\item
  While tracking, take a 10-sec exposure with the StarTracker.
\item
  Load the image into an image viewer.
\item
  Overlay the GAIA catalog.
\item
  Select a star brighter than XXX mag (bright enough for the DIMM).
\item
  Calculate the pixel offset between the StarTracker and the DIMM.
\item
  Transform the offset into AZ and EL offsets.
\end{itemize}

}
\hdashrule[0.5ex]{\textwidth}{1pt}{3mm}
  Expected Result \\
{\footnotesize
\begin{itemize}
\tightlist
\item
  An image was successfully taken with the StarTracker and is of
  sufficient quality.
\item
  AZ and EL offsets are available.
\end{itemize}

}

\begin{tabular}{p{2cm}}
\toprule
Step 195  \\ \hline
\end{tabular}
 Description \\
{\footnotesize
\textbf{Find DIMM Object and DIMM Offset}\\

\begin{itemize}
\tightlist
\item
  While tracking, take a 10-sec exposure with the StarTracker.
\item
  Load the image into an image viewer.
\item
  Overlay the GAIA catalog.
\item
  Select a star brighter than XXX mag (bright enough for the DIMM).
\item
  Calculate the pixel offset between the StarTracker and the DIMM.
\item
  Transform the offset into AZ and EL offsets.
\end{itemize}

}
\hdashrule[0.5ex]{\textwidth}{1pt}{3mm}
  Expected Result \\
{\footnotesize
\begin{itemize}
\tightlist
\item
  An image was successfully taken with the StarTracker and is of
  sufficient quality.
\item
  AZ and EL offsets are available.
\end{itemize}

}

\begin{tabular}{p{2cm}}
\toprule
Step 196  \\ \hline
\end{tabular}
 Description \\
{\footnotesize
\textbf{Find DIMM Object and DIMM Offset}\\

\begin{itemize}
\tightlist
\item
  While tracking, take a 10-sec exposure with the StarTracker.
\item
  Load the image into an image viewer.
\item
  Overlay the GAIA catalog.
\item
  Select a star brighter than XXX mag (bright enough for the DIMM).
\item
  Calculate the pixel offset between the StarTracker and the DIMM.
\item
  Transform the offset into AZ and EL offsets.
\end{itemize}

}
\hdashrule[0.5ex]{\textwidth}{1pt}{3mm}
  Expected Result \\
{\footnotesize
\begin{itemize}
\tightlist
\item
  An image was successfully taken with the StarTracker and is of
  sufficient quality.
\item
  AZ and EL offsets are available.
\end{itemize}

}

\begin{tabular}{p{2cm}}
\toprule
Step 197  \\ \hline
\end{tabular}
 Description \\
{\footnotesize
\textbf{Find DIMM Object and DIMM Offset}\\

\begin{itemize}
\tightlist
\item
  While tracking, take a 10-sec exposure with the StarTracker.
\item
  Load the image into an image viewer.
\item
  Overlay the GAIA catalog.
\item
  Select a star brighter than XXX mag (bright enough for the DIMM).
\item
  Calculate the pixel offset between the StarTracker and the DIMM.
\item
  Transform the offset into AZ and EL offsets.
\end{itemize}

}
\hdashrule[0.5ex]{\textwidth}{1pt}{3mm}
  Expected Result \\
{\footnotesize
\begin{itemize}
\tightlist
\item
  An image was successfully taken with the StarTracker and is of
  sufficient quality.
\item
  AZ and EL offsets are available.
\end{itemize}

}

\begin{tabular}{p{2cm}}
\toprule
Step 198  \\ \hline
\end{tabular}
 Description \\
{\footnotesize
\textbf{Find DIMM Object and DIMM Offset}\\

\begin{itemize}
\tightlist
\item
  While tracking, take a 10-sec exposure with the StarTracker.
\item
  Load the image into an image viewer.
\item
  Overlay the GAIA catalog.
\item
  Select a star brighter than XXX mag (bright enough for the DIMM).
\item
  Calculate the pixel offset between the StarTracker and the DIMM.
\item
  Transform the offset into AZ and EL offsets.
\end{itemize}

}
\hdashrule[0.5ex]{\textwidth}{1pt}{3mm}
  Expected Result \\
{\footnotesize
\begin{itemize}
\tightlist
\item
  An image was successfully taken with the StarTracker and is of
  sufficient quality.
\item
  AZ and EL offsets are available.
\end{itemize}

}

\begin{tabular}{p{2cm}}
\toprule
Step 199  \\ \hline
\end{tabular}
 Description \\
{\footnotesize
\textbf{Find DIMM Object and DIMM Offset}\\

\begin{itemize}
\tightlist
\item
  While tracking, take a 10-sec exposure with the StarTracker.
\item
  Load the image into an image viewer.
\item
  Overlay the GAIA catalog.
\item
  Select a star brighter than XXX mag (bright enough for the DIMM).
\item
  Calculate the pixel offset between the StarTracker and the DIMM.
\item
  Transform the offset into AZ and EL offsets.
\end{itemize}

}
\hdashrule[0.5ex]{\textwidth}{1pt}{3mm}
  Expected Result \\
{\footnotesize
\begin{itemize}
\tightlist
\item
  An image was successfully taken with the StarTracker and is of
  sufficient quality.
\item
  AZ and EL offsets are available.
\end{itemize}

}

\begin{tabular}{p{2cm}}
\toprule
Step 200  \\ \hline
\end{tabular}
 Description \\
{\footnotesize
\textbf{Find DIMM Object and DIMM Offset}\\

\begin{itemize}
\tightlist
\item
  While tracking, take a 10-sec exposure with the StarTracker.
\item
  Load the image into an image viewer.
\item
  Overlay the GAIA catalog.
\item
  Select a star brighter than XXX mag (bright enough for the DIMM).
\item
  Calculate the pixel offset between the StarTracker and the DIMM.
\item
  Transform the offset into AZ and EL offsets.
\end{itemize}

}
\hdashrule[0.5ex]{\textwidth}{1pt}{3mm}
  Expected Result \\
{\footnotesize
\begin{itemize}
\tightlist
\item
  An image was successfully taken with the StarTracker and is of
  sufficient quality.
\item
  AZ and EL offsets are available.
\end{itemize}

}

\begin{tabular}{p{2cm}}
\toprule
Step 201  \\ \hline
\end{tabular}
 Description \\
{\footnotesize
\textbf{Find DIMM Object and DIMM Offset}\\

\begin{itemize}
\tightlist
\item
  While tracking, take a 10-sec exposure with the StarTracker.
\item
  Load the image into an image viewer.
\item
  Overlay the GAIA catalog.
\item
  Select a star brighter than XXX mag (bright enough for the DIMM).
\item
  Calculate the pixel offset between the StarTracker and the DIMM.
\item
  Transform the offset into AZ and EL offsets.
\end{itemize}

}
\hdashrule[0.5ex]{\textwidth}{1pt}{3mm}
  Expected Result \\
{\footnotesize
\begin{itemize}
\tightlist
\item
  An image was successfully taken with the StarTracker and is of
  sufficient quality.
\item
  AZ and EL offsets are available.
\end{itemize}

}

\begin{tabular}{p{2cm}}
\toprule
Step 202  \\ \hline
\end{tabular}
 Description \\
{\footnotesize
\textbf{Find DIMM Object and DIMM Offset}\\

\begin{itemize}
\tightlist
\item
  While tracking, take a 10-sec exposure with the StarTracker.
\item
  Load the image into an image viewer.
\item
  Overlay the GAIA catalog.
\item
  Select a star brighter than XXX mag (bright enough for the DIMM).
\item
  Calculate the pixel offset between the StarTracker and the DIMM.
\item
  Transform the offset into AZ and EL offsets.
\end{itemize}

}
\hdashrule[0.5ex]{\textwidth}{1pt}{3mm}
  Expected Result \\
{\footnotesize
\begin{itemize}
\tightlist
\item
  An image was successfully taken with the StarTracker and is of
  sufficient quality.
\item
  AZ and EL offsets are available.
\end{itemize}

}

\begin{tabular}{p{2cm}}
\toprule
Step 203  \\ \hline
\end{tabular}
 Description \\
{\footnotesize
\textbf{Find DIMM Object and DIMM Offset}\\

\begin{itemize}
\tightlist
\item
  While tracking, take a 10-sec exposure with the StarTracker.
\item
  Load the image into an image viewer.
\item
  Overlay the GAIA catalog.
\item
  Select a star brighter than XXX mag (bright enough for the DIMM).
\item
  Calculate the pixel offset between the StarTracker and the DIMM.
\item
  Transform the offset into AZ and EL offsets.
\end{itemize}

}
\hdashrule[0.5ex]{\textwidth}{1pt}{3mm}
  Expected Result \\
{\footnotesize
\begin{itemize}
\tightlist
\item
  An image was successfully taken with the StarTracker and is of
  sufficient quality.
\item
  AZ and EL offsets are available.
\end{itemize}

}

\begin{tabular}{p{2cm}}
\toprule
Step 204  \\ \hline
\end{tabular}
 Description \\
{\footnotesize
\textbf{Find DIMM Object and DIMM Offset}\\

\begin{itemize}
\tightlist
\item
  While tracking, take a 10-sec exposure with the StarTracker.
\item
  Load the image into an image viewer.
\item
  Overlay the GAIA catalog.
\item
  Select a star brighter than XXX mag (bright enough for the DIMM).
\item
  Calculate the pixel offset between the StarTracker and the DIMM.
\item
  Transform the offset into AZ and EL offsets.
\end{itemize}

}
\hdashrule[0.5ex]{\textwidth}{1pt}{3mm}
  Expected Result \\
{\footnotesize
\begin{itemize}
\tightlist
\item
  An image was successfully taken with the StarTracker and is of
  sufficient quality.
\item
  AZ and EL offsets are available.
\end{itemize}

}

\begin{tabular}{p{2cm}}
\toprule
Step 205  \\ \hline
\end{tabular}
 Description \\
{\footnotesize
\textbf{Find DIMM Object and DIMM Offset}\\

\begin{itemize}
\tightlist
\item
  While tracking, take a 10-sec exposure with the StarTracker.
\item
  Load the image into an image viewer.
\item
  Overlay the GAIA catalog.
\item
  Select a star brighter than XXX mag (bright enough for the DIMM).
\item
  Calculate the pixel offset between the StarTracker and the DIMM.
\item
  Transform the offset into AZ and EL offsets.
\end{itemize}

}
\hdashrule[0.5ex]{\textwidth}{1pt}{3mm}
  Expected Result \\
{\footnotesize
\begin{itemize}
\tightlist
\item
  An image was successfully taken with the StarTracker and is of
  sufficient quality.
\item
  AZ and EL offsets are available.
\end{itemize}

}

\begin{tabular}{p{2cm}}
\toprule
Step 206  \\ \hline
\end{tabular}
 Description \\
{\footnotesize
\textbf{Find DIMM Object and DIMM Offset}\\

\begin{itemize}
\tightlist
\item
  While tracking, take a 10-sec exposure with the StarTracker.
\item
  Load the image into an image viewer.
\item
  Overlay the GAIA catalog.
\item
  Select a star brighter than XXX mag (bright enough for the DIMM).
\item
  Calculate the pixel offset between the StarTracker and the DIMM.
\item
  Transform the offset into AZ and EL offsets.
\end{itemize}

}
\hdashrule[0.5ex]{\textwidth}{1pt}{3mm}
  Expected Result \\
{\footnotesize
\begin{itemize}
\tightlist
\item
  An image was successfully taken with the StarTracker and is of
  sufficient quality.
\item
  AZ and EL offsets are available.
\end{itemize}

}

\begin{tabular}{p{2cm}}
\toprule
Step 207  \\ \hline
\end{tabular}
 Description \\
{\footnotesize
\textbf{Find DIMM Object and DIMM Offset}\\

\begin{itemize}
\tightlist
\item
  While tracking, take a 10-sec exposure with the StarTracker.
\item
  Load the image into an image viewer.
\item
  Overlay the GAIA catalog.
\item
  Select a star brighter than XXX mag (bright enough for the DIMM).
\item
  Calculate the pixel offset between the StarTracker and the DIMM.
\item
  Transform the offset into AZ and EL offsets.
\end{itemize}

}
\hdashrule[0.5ex]{\textwidth}{1pt}{3mm}
  Expected Result \\
{\footnotesize
\begin{itemize}
\tightlist
\item
  An image was successfully taken with the StarTracker and is of
  sufficient quality.
\item
  AZ and EL offsets are available.
\end{itemize}

}

\begin{tabular}{p{2cm}}
\toprule
Step 208  \\ \hline
\end{tabular}
 Description \\
{\footnotesize
\textbf{Find DIMM Object and DIMM Offset}\\

\begin{itemize}
\tightlist
\item
  While tracking, take a 10-sec exposure with the StarTracker.
\item
  Load the image into an image viewer.
\item
  Overlay the GAIA catalog.
\item
  Select a star brighter than XXX mag (bright enough for the DIMM).
\item
  Calculate the pixel offset between the StarTracker and the DIMM.
\item
  Transform the offset into AZ and EL offsets.
\end{itemize}

}
\hdashrule[0.5ex]{\textwidth}{1pt}{3mm}
  Expected Result \\
{\footnotesize
\begin{itemize}
\tightlist
\item
  An image was successfully taken with the StarTracker and is of
  sufficient quality.
\item
  AZ and EL offsets are available.
\end{itemize}

}

\begin{tabular}{p{2cm}}
\toprule
Step 209  \\ \hline
\end{tabular}
 Description \\
{\footnotesize
\textbf{Find DIMM Object and DIMM Offset}\\

\begin{itemize}
\tightlist
\item
  While tracking, take a 10-sec exposure with the StarTracker.
\item
  Load the image into an image viewer.
\item
  Overlay the GAIA catalog.
\item
  Select a star brighter than XXX mag (bright enough for the DIMM).
\item
  Calculate the pixel offset between the StarTracker and the DIMM.
\item
  Transform the offset into AZ and EL offsets.
\end{itemize}

}
\hdashrule[0.5ex]{\textwidth}{1pt}{3mm}
  Expected Result \\
{\footnotesize
\begin{itemize}
\tightlist
\item
  An image was successfully taken with the StarTracker and is of
  sufficient quality.
\item
  AZ and EL offsets are available.
\end{itemize}

}

\begin{tabular}{p{2cm}}
\toprule
Step 210  \\ \hline
\end{tabular}
 Description \\
{\footnotesize
\textbf{Find DIMM Object and DIMM Offset}\\

\begin{itemize}
\tightlist
\item
  While tracking, take a 10-sec exposure with the StarTracker.
\item
  Load the image into an image viewer.
\item
  Overlay the GAIA catalog.
\item
  Select a star brighter than XXX mag (bright enough for the DIMM).
\item
  Calculate the pixel offset between the StarTracker and the DIMM.
\item
  Transform the offset into AZ and EL offsets.
\end{itemize}

}
\hdashrule[0.5ex]{\textwidth}{1pt}{3mm}
  Expected Result \\
{\footnotesize
\begin{itemize}
\tightlist
\item
  An image was successfully taken with the StarTracker and is of
  sufficient quality.
\item
  AZ and EL offsets are available.
\end{itemize}

}

\begin{tabular}{p{2cm}}
\toprule
Step 211  \\ \hline
\end{tabular}
 Description \\
{\footnotesize
\textbf{Find DIMM Object and DIMM Offset}\\

\begin{itemize}
\tightlist
\item
  While tracking, take a 10-sec exposure with the StarTracker.
\item
  Load the image into an image viewer.
\item
  Overlay the GAIA catalog.
\item
  Select a star brighter than XXX mag (bright enough for the DIMM).
\item
  Calculate the pixel offset between the StarTracker and the DIMM.
\item
  Transform the offset into AZ and EL offsets.
\end{itemize}

}
\hdashrule[0.5ex]{\textwidth}{1pt}{3mm}
  Expected Result \\
{\footnotesize
\begin{itemize}
\tightlist
\item
  An image was successfully taken with the StarTracker and is of
  sufficient quality.
\item
  AZ and EL offsets are available.
\end{itemize}

}

\begin{tabular}{p{2cm}}
\toprule
Step 212  \\ \hline
\end{tabular}
 Description \\
{\footnotesize
\textbf{Find DIMM Object and DIMM Offset}\\

\begin{itemize}
\tightlist
\item
  While tracking, take a 10-sec exposure with the StarTracker.
\item
  Load the image into an image viewer.
\item
  Overlay the GAIA catalog.
\item
  Select a star brighter than XXX mag (bright enough for the DIMM).
\item
  Calculate the pixel offset between the StarTracker and the DIMM.
\item
  Transform the offset into AZ and EL offsets.
\end{itemize}

}
\hdashrule[0.5ex]{\textwidth}{1pt}{3mm}
  Expected Result \\
{\footnotesize
\begin{itemize}
\tightlist
\item
  An image was successfully taken with the StarTracker and is of
  sufficient quality.
\item
  AZ and EL offsets are available.
\end{itemize}

}

\begin{tabular}{p{2cm}}
\toprule
Step 213  \\ \hline
\end{tabular}
 Description \\
{\footnotesize
\textbf{Find DIMM Object and DIMM Offset}\\

\begin{itemize}
\tightlist
\item
  While tracking, take a 10-sec exposure with the StarTracker.
\item
  Load the image into an image viewer.
\item
  Overlay the GAIA catalog.
\item
  Select a star brighter than XXX mag (bright enough for the DIMM).
\item
  Calculate the pixel offset between the StarTracker and the DIMM.
\item
  Transform the offset into AZ and EL offsets.
\end{itemize}

}
\hdashrule[0.5ex]{\textwidth}{1pt}{3mm}
  Expected Result \\
{\footnotesize
\begin{itemize}
\tightlist
\item
  An image was successfully taken with the StarTracker and is of
  sufficient quality.
\item
  AZ and EL offsets are available.
\end{itemize}

}

\begin{tabular}{p{2cm}}
\toprule
Step 214  \\ \hline
\end{tabular}
 Description \\
{\footnotesize
\textbf{Find DIMM Object and DIMM Offset}\\

\begin{itemize}
\tightlist
\item
  While tracking, take a 10-sec exposure with the StarTracker.
\item
  Load the image into an image viewer.
\item
  Overlay the GAIA catalog.
\item
  Select a star brighter than XXX mag (bright enough for the DIMM).
\item
  Calculate the pixel offset between the StarTracker and the DIMM.
\item
  Transform the offset into AZ and EL offsets.
\end{itemize}

}
\hdashrule[0.5ex]{\textwidth}{1pt}{3mm}
  Expected Result \\
{\footnotesize
\begin{itemize}
\tightlist
\item
  An image was successfully taken with the StarTracker and is of
  sufficient quality.
\item
  AZ and EL offsets are available.
\end{itemize}

}

\begin{tabular}{p{2cm}}
\toprule
Step 215  \\ \hline
\end{tabular}
 Description \\
{\footnotesize
\textbf{Find DIMM Object and DIMM Offset}\\

\begin{itemize}
\tightlist
\item
  While tracking, take a 10-sec exposure with the StarTracker.
\item
  Load the image into an image viewer.
\item
  Overlay the GAIA catalog.
\item
  Select a star brighter than XXX mag (bright enough for the DIMM).
\item
  Calculate the pixel offset between the StarTracker and the DIMM.
\item
  Transform the offset into AZ and EL offsets.
\end{itemize}

}
\hdashrule[0.5ex]{\textwidth}{1pt}{3mm}
  Expected Result \\
{\footnotesize
\begin{itemize}
\tightlist
\item
  An image was successfully taken with the StarTracker and is of
  sufficient quality.
\item
  AZ and EL offsets are available.
\end{itemize}

}

\begin{tabular}{p{2cm}}
\toprule
Step 216  \\ \hline
\end{tabular}
 Description \\
{\footnotesize
\textbf{Find DIMM Object and DIMM Offset}\\

\begin{itemize}
\tightlist
\item
  While tracking, take a 10-sec exposure with the StarTracker.
\item
  Load the image into an image viewer.
\item
  Overlay the GAIA catalog.
\item
  Select a star brighter than XXX mag (bright enough for the DIMM).
\item
  Calculate the pixel offset between the StarTracker and the DIMM.
\item
  Transform the offset into AZ and EL offsets.
\end{itemize}

}
\hdashrule[0.5ex]{\textwidth}{1pt}{3mm}
  Expected Result \\
{\footnotesize
\begin{itemize}
\tightlist
\item
  An image was successfully taken with the StarTracker and is of
  sufficient quality.
\item
  AZ and EL offsets are available.
\end{itemize}

}

\begin{tabular}{p{2cm}}
\toprule
Step 217  \\ \hline
\end{tabular}
 Description \\
{\footnotesize
\textbf{Find DIMM Object and DIMM Offset}\\

\begin{itemize}
\tightlist
\item
  While tracking, take a 10-sec exposure with the StarTracker.
\item
  Load the image into an image viewer.
\item
  Overlay the GAIA catalog.
\item
  Select a star brighter than XXX mag (bright enough for the DIMM).
\item
  Calculate the pixel offset between the StarTracker and the DIMM.
\item
  Transform the offset into AZ and EL offsets.
\end{itemize}

}
\hdashrule[0.5ex]{\textwidth}{1pt}{3mm}
  Expected Result \\
{\footnotesize
\begin{itemize}
\tightlist
\item
  An image was successfully taken with the StarTracker and is of
  sufficient quality.
\item
  AZ and EL offsets are available.
\end{itemize}

}

\begin{tabular}{p{2cm}}
\toprule
Step 218  \\ \hline
\end{tabular}
 Description \\
{\footnotesize
\textbf{Find DIMM Object and DIMM Offset}\\

\begin{itemize}
\tightlist
\item
  While tracking, take a 10-sec exposure with the StarTracker.
\item
  Load the image into an image viewer.
\item
  Overlay the GAIA catalog.
\item
  Select a star brighter than XXX mag (bright enough for the DIMM).
\item
  Calculate the pixel offset between the StarTracker and the DIMM.
\item
  Transform the offset into AZ and EL offsets.
\end{itemize}

}
\hdashrule[0.5ex]{\textwidth}{1pt}{3mm}
  Expected Result \\
{\footnotesize
\begin{itemize}
\tightlist
\item
  An image was successfully taken with the StarTracker and is of
  sufficient quality.
\item
  AZ and EL offsets are available.
\end{itemize}

}

\begin{tabular}{p{2cm}}
\toprule
Step 219  \\ \hline
\end{tabular}
 Description \\
{\footnotesize
\textbf{Find DIMM Object and DIMM Offset}\\

\begin{itemize}
\tightlist
\item
  While tracking, take a 10-sec exposure with the StarTracker.
\item
  Load the image into an image viewer.
\item
  Overlay the GAIA catalog.
\item
  Select a star brighter than XXX mag (bright enough for the DIMM).
\item
  Calculate the pixel offset between the StarTracker and the DIMM.
\item
  Transform the offset into AZ and EL offsets.
\end{itemize}

}
\hdashrule[0.5ex]{\textwidth}{1pt}{3mm}
  Expected Result \\
{\footnotesize
\begin{itemize}
\tightlist
\item
  An image was successfully taken with the StarTracker and is of
  sufficient quality.
\item
  AZ and EL offsets are available.
\end{itemize}

}

\begin{tabular}{p{2cm}}
\toprule
Step 220  \\ \hline
\end{tabular}
 Description \\
{\footnotesize
\textbf{Find DIMM Object and DIMM Offset}\\

\begin{itemize}
\tightlist
\item
  While tracking, take a 10-sec exposure with the StarTracker.
\item
  Load the image into an image viewer.
\item
  Overlay the GAIA catalog.
\item
  Select a star brighter than XXX mag (bright enough for the DIMM).
\item
  Calculate the pixel offset between the StarTracker and the DIMM.
\item
  Transform the offset into AZ and EL offsets.
\end{itemize}

}
\hdashrule[0.5ex]{\textwidth}{1pt}{3mm}
  Expected Result \\
{\footnotesize
\begin{itemize}
\tightlist
\item
  An image was successfully taken with the StarTracker and is of
  sufficient quality.
\item
  AZ and EL offsets are available.
\end{itemize}

}

\begin{tabular}{p{2cm}}
\toprule
Step 221  \\ \hline
\end{tabular}
 Description \\
{\footnotesize
\textbf{Find DIMM Object and DIMM Offset}\\

\begin{itemize}
\tightlist
\item
  While tracking, take a 10-sec exposure with the StarTracker.
\item
  Load the image into an image viewer.
\item
  Overlay the GAIA catalog.
\item
  Select a star brighter than XXX mag (bright enough for the DIMM).
\item
  Calculate the pixel offset between the StarTracker and the DIMM.
\item
  Transform the offset into AZ and EL offsets.
\end{itemize}

}
\hdashrule[0.5ex]{\textwidth}{1pt}{3mm}
  Expected Result \\
{\footnotesize
\begin{itemize}
\tightlist
\item
  An image was successfully taken with the StarTracker and is of
  sufficient quality.
\item
  AZ and EL offsets are available.
\end{itemize}

}

\begin{tabular}{p{2cm}}
\toprule
Step 222  \\ \hline
\end{tabular}
 Description \\
{\footnotesize
\textbf{Find DIMM Object and DIMM Offset}\\

\begin{itemize}
\tightlist
\item
  While tracking, take a 10-sec exposure with the StarTracker.
\item
  Load the image into an image viewer.
\item
  Overlay the GAIA catalog.
\item
  Select a star brighter than XXX mag (bright enough for the DIMM).
\item
  Calculate the pixel offset between the StarTracker and the DIMM.
\item
  Transform the offset into AZ and EL offsets.
\end{itemize}

}
\hdashrule[0.5ex]{\textwidth}{1pt}{3mm}
  Expected Result \\
{\footnotesize
\begin{itemize}
\tightlist
\item
  An image was successfully taken with the StarTracker and is of
  sufficient quality.
\item
  AZ and EL offsets are available.
\end{itemize}

}

\begin{tabular}{p{2cm}}
\toprule
Step 223  \\ \hline
\end{tabular}
 Description \\
{\footnotesize
\textbf{Find DIMM Object and DIMM Offset}\\

\begin{itemize}
\tightlist
\item
  While tracking, take a 10-sec exposure with the StarTracker.
\item
  Load the image into an image viewer.
\item
  Overlay the GAIA catalog.
\item
  Select a star brighter than XXX mag (bright enough for the DIMM).
\item
  Calculate the pixel offset between the StarTracker and the DIMM.
\item
  Transform the offset into AZ and EL offsets.
\end{itemize}

}
\hdashrule[0.5ex]{\textwidth}{1pt}{3mm}
  Expected Result \\
{\footnotesize
\begin{itemize}
\tightlist
\item
  An image was successfully taken with the StarTracker and is of
  sufficient quality.
\item
  AZ and EL offsets are available.
\end{itemize}

}

\begin{tabular}{p{2cm}}
\toprule
Step 224  \\ \hline
\end{tabular}
 Description \\
{\footnotesize
\textbf{Find DIMM Object and DIMM Offset}\\

\begin{itemize}
\tightlist
\item
  While tracking, take a 10-sec exposure with the StarTracker.
\item
  Load the image into an image viewer.
\item
  Overlay the GAIA catalog.
\item
  Select a star brighter than XXX mag (bright enough for the DIMM).
\item
  Calculate the pixel offset between the StarTracker and the DIMM.
\item
  Transform the offset into AZ and EL offsets.
\end{itemize}

}
\hdashrule[0.5ex]{\textwidth}{1pt}{3mm}
  Expected Result \\
{\footnotesize
\begin{itemize}
\tightlist
\item
  An image was successfully taken with the StarTracker and is of
  sufficient quality.
\item
  AZ and EL offsets are available.
\end{itemize}

}

\begin{tabular}{p{2cm}}
\toprule
Step 225  \\ \hline
\end{tabular}
 Description \\
{\footnotesize
\textbf{Find DIMM Object and DIMM Offset}\\

\begin{itemize}
\tightlist
\item
  While tracking, take a 10-sec exposure with the StarTracker.
\item
  Load the image into an image viewer.
\item
  Overlay the GAIA catalog.
\item
  Select a star brighter than XXX mag (bright enough for the DIMM).
\item
  Calculate the pixel offset between the StarTracker and the DIMM.
\item
  Transform the offset into AZ and EL offsets.
\end{itemize}

}
\hdashrule[0.5ex]{\textwidth}{1pt}{3mm}
  Expected Result \\
{\footnotesize
\begin{itemize}
\tightlist
\item
  An image was successfully taken with the StarTracker and is of
  sufficient quality.
\item
  AZ and EL offsets are available.
\end{itemize}

}

\begin{tabular}{p{2cm}}
\toprule
Step 226  \\ \hline
\end{tabular}
 Description \\
{\footnotesize
\textbf{Find DIMM Object and DIMM Offset}\\

\begin{itemize}
\tightlist
\item
  While tracking, take a 10-sec exposure with the StarTracker.
\item
  Load the image into an image viewer.
\item
  Overlay the GAIA catalog.
\item
  Select a star brighter than XXX mag (bright enough for the DIMM).
\item
  Calculate the pixel offset between the StarTracker and the DIMM.
\item
  Transform the offset into AZ and EL offsets.
\end{itemize}

}
\hdashrule[0.5ex]{\textwidth}{1pt}{3mm}
  Expected Result \\
{\footnotesize
\begin{itemize}
\tightlist
\item
  An image was successfully taken with the StarTracker and is of
  sufficient quality.
\item
  AZ and EL offsets are available.
\end{itemize}

}

\begin{tabular}{p{2cm}}
\toprule
Step 227  \\ \hline
\end{tabular}
 Description \\
{\footnotesize
\textbf{Move TMA to the DIMM position and \textbf{Take DIMM images}}\\

\begin{itemize}
\tightlist
\item
  Command the TMA to the DIMM position by applying the offsets
\item
  While tracking, take DIMM images with XXXs exposure time and inspect
  the quality.
\end{itemize}

}
\hdashrule[0.5ex]{\textwidth}{1pt}{3mm}
  Expected Result \\
{\footnotesize
\begin{itemize}
\tightlist
\item
  TMA reaches the DIMM position.
\item
  DIMM imaging quality is sufficient.
\end{itemize}

}

\begin{tabular}{p{2cm}}
\toprule
Step 228  \\ \hline
\end{tabular}
 Description \\
{\footnotesize
\textbf{Move TMA to the DIMM position and \textbf{Take DIMM images}}\\

\begin{itemize}
\tightlist
\item
  Command the TMA to the DIMM position by applying the offsets
\item
  While tracking, take DIMM images with XXXs exposure time and inspect
  the quality.
\end{itemize}

}
\hdashrule[0.5ex]{\textwidth}{1pt}{3mm}
  Expected Result \\
{\footnotesize
\begin{itemize}
\tightlist
\item
  TMA reaches the DIMM position.
\item
  DIMM imaging quality is sufficient.
\end{itemize}

}

\begin{tabular}{p{2cm}}
\toprule
Step 229  \\ \hline
\end{tabular}
 Description \\
{\footnotesize
\textbf{Move TMA to the DIMM position and \textbf{Take DIMM images}}\\

\begin{itemize}
\tightlist
\item
  Command the TMA to the DIMM position by applying the offsets
\item
  While tracking, take DIMM images with XXXs exposure time and inspect
  the quality.
\end{itemize}

}
\hdashrule[0.5ex]{\textwidth}{1pt}{3mm}
  Expected Result \\
{\footnotesize
\begin{itemize}
\tightlist
\item
  TMA reaches the DIMM position.
\item
  DIMM imaging quality is sufficient.
\end{itemize}

}

\begin{tabular}{p{2cm}}
\toprule
Step 230  \\ \hline
\end{tabular}
 Description \\
{\footnotesize
\textbf{Move TMA to the DIMM position and \textbf{Take DIMM images}}\\

\begin{itemize}
\tightlist
\item
  Command the TMA to the DIMM position by applying the offsets
\item
  While tracking, take DIMM images with XXXs exposure time and inspect
  the quality.
\end{itemize}

}
\hdashrule[0.5ex]{\textwidth}{1pt}{3mm}
  Expected Result \\
{\footnotesize
\begin{itemize}
\tightlist
\item
  TMA reaches the DIMM position.
\item
  DIMM imaging quality is sufficient.
\end{itemize}

}

\begin{tabular}{p{2cm}}
\toprule
Step 231  \\ \hline
\end{tabular}
 Description \\
{\footnotesize
\textbf{Move TMA to the DIMM position and \textbf{Take DIMM images}}\\

\begin{itemize}
\tightlist
\item
  Command the TMA to the DIMM position by applying the offsets
\item
  While tracking, take DIMM images with XXXs exposure time and inspect
  the quality.
\end{itemize}

}
\hdashrule[0.5ex]{\textwidth}{1pt}{3mm}
  Expected Result \\
{\footnotesize
\begin{itemize}
\tightlist
\item
  TMA reaches the DIMM position.
\item
  DIMM imaging quality is sufficient.
\end{itemize}

}

\begin{tabular}{p{2cm}}
\toprule
Step 232  \\ \hline
\end{tabular}
 Description \\
{\footnotesize
\textbf{Move TMA to the DIMM position and \textbf{Take DIMM images}}\\

\begin{itemize}
\tightlist
\item
  Command the TMA to the DIMM position by applying the offsets
\item
  While tracking, take DIMM images with XXXs exposure time and inspect
  the quality.
\end{itemize}

}
\hdashrule[0.5ex]{\textwidth}{1pt}{3mm}
  Expected Result \\
{\footnotesize
\begin{itemize}
\tightlist
\item
  TMA reaches the DIMM position.
\item
  DIMM imaging quality is sufficient.
\end{itemize}

}

\begin{tabular}{p{2cm}}
\toprule
Step 233  \\ \hline
\end{tabular}
 Description \\
{\footnotesize
\textbf{Move TMA to the DIMM position and \textbf{Take DIMM images}}\\

\begin{itemize}
\tightlist
\item
  Command the TMA to the DIMM position by applying the offsets
\item
  While tracking, take DIMM images with XXXs exposure time and inspect
  the quality.
\end{itemize}

}
\hdashrule[0.5ex]{\textwidth}{1pt}{3mm}
  Expected Result \\
{\footnotesize
\begin{itemize}
\tightlist
\item
  TMA reaches the DIMM position.
\item
  DIMM imaging quality is sufficient.
\end{itemize}

}

\begin{tabular}{p{2cm}}
\toprule
Step 234  \\ \hline
\end{tabular}
 Description \\
{\footnotesize
\textbf{Move TMA to the DIMM position and \textbf{Take DIMM images}}\\

\begin{itemize}
\tightlist
\item
  Command the TMA to the DIMM position by applying the offsets
\item
  While tracking, take DIMM images with XXXs exposure time and inspect
  the quality.
\end{itemize}

}
\hdashrule[0.5ex]{\textwidth}{1pt}{3mm}
  Expected Result \\
{\footnotesize
\begin{itemize}
\tightlist
\item
  TMA reaches the DIMM position.
\item
  DIMM imaging quality is sufficient.
\end{itemize}

}

\begin{tabular}{p{2cm}}
\toprule
Step 235  \\ \hline
\end{tabular}
 Description \\
{\footnotesize
\textbf{Move TMA to the DIMM position and \textbf{Take DIMM images}}\\

\begin{itemize}
\tightlist
\item
  Command the TMA to the DIMM position by applying the offsets
\item
  While tracking, take DIMM images with XXXs exposure time and inspect
  the quality.
\end{itemize}

}
\hdashrule[0.5ex]{\textwidth}{1pt}{3mm}
  Expected Result \\
{\footnotesize
\begin{itemize}
\tightlist
\item
  TMA reaches the DIMM position.
\item
  DIMM imaging quality is sufficient.
\end{itemize}

}

\begin{tabular}{p{2cm}}
\toprule
Step 236  \\ \hline
\end{tabular}
 Description \\
{\footnotesize
\textbf{Move TMA to the DIMM position and \textbf{Take DIMM images}}\\

\begin{itemize}
\tightlist
\item
  Command the TMA to the DIMM position by applying the offsets
\item
  While tracking, take DIMM images with XXXs exposure time and inspect
  the quality.
\end{itemize}

}
\hdashrule[0.5ex]{\textwidth}{1pt}{3mm}
  Expected Result \\
{\footnotesize
\begin{itemize}
\tightlist
\item
  TMA reaches the DIMM position.
\item
  DIMM imaging quality is sufficient.
\end{itemize}

}

\begin{tabular}{p{2cm}}
\toprule
Step 237  \\ \hline
\end{tabular}
 Description \\
{\footnotesize
\textbf{Move TMA to the DIMM position and \textbf{Take DIMM images}}\\

\begin{itemize}
\tightlist
\item
  Command the TMA to the DIMM position by applying the offsets
\item
  While tracking, take DIMM images with XXXs exposure time and inspect
  the quality.
\end{itemize}

}
\hdashrule[0.5ex]{\textwidth}{1pt}{3mm}
  Expected Result \\
{\footnotesize
\begin{itemize}
\tightlist
\item
  TMA reaches the DIMM position.
\item
  DIMM imaging quality is sufficient.
\end{itemize}

}

\begin{tabular}{p{2cm}}
\toprule
Step 238  \\ \hline
\end{tabular}
 Description \\
{\footnotesize
\textbf{Move TMA to the DIMM position and \textbf{Take DIMM images}}\\

\begin{itemize}
\tightlist
\item
  Command the TMA to the DIMM position by applying the offsets
\item
  While tracking, take DIMM images with XXXs exposure time and inspect
  the quality.
\end{itemize}

}
\hdashrule[0.5ex]{\textwidth}{1pt}{3mm}
  Expected Result \\
{\footnotesize
\begin{itemize}
\tightlist
\item
  TMA reaches the DIMM position.
\item
  DIMM imaging quality is sufficient.
\end{itemize}

}

\begin{tabular}{p{2cm}}
\toprule
Step 239  \\ \hline
\end{tabular}
 Description \\
{\footnotesize
\textbf{Move TMA to the DIMM position and \textbf{Take DIMM images}}\\

\begin{itemize}
\tightlist
\item
  Command the TMA to the DIMM position by applying the offsets
\item
  While tracking, take DIMM images with XXXs exposure time and inspect
  the quality.
\end{itemize}

}
\hdashrule[0.5ex]{\textwidth}{1pt}{3mm}
  Expected Result \\
{\footnotesize
\begin{itemize}
\tightlist
\item
  TMA reaches the DIMM position.
\item
  DIMM imaging quality is sufficient.
\end{itemize}

}

\begin{tabular}{p{2cm}}
\toprule
Step 240  \\ \hline
\end{tabular}
 Description \\
{\footnotesize
\textbf{Move TMA to the DIMM position and \textbf{Take DIMM images}}\\

\begin{itemize}
\tightlist
\item
  Command the TMA to the DIMM position by applying the offsets
\item
  While tracking, take DIMM images with XXXs exposure time and inspect
  the quality.
\end{itemize}

}
\hdashrule[0.5ex]{\textwidth}{1pt}{3mm}
  Expected Result \\
{\footnotesize
\begin{itemize}
\tightlist
\item
  TMA reaches the DIMM position.
\item
  DIMM imaging quality is sufficient.
\end{itemize}

}

\begin{tabular}{p{2cm}}
\toprule
Step 241  \\ \hline
\end{tabular}
 Description \\
{\footnotesize
\textbf{Move TMA to the DIMM position and \textbf{Take DIMM images}}\\

\begin{itemize}
\tightlist
\item
  Command the TMA to the DIMM position by applying the offsets
\item
  While tracking, take DIMM images with XXXs exposure time and inspect
  the quality.
\end{itemize}

}
\hdashrule[0.5ex]{\textwidth}{1pt}{3mm}
  Expected Result \\
{\footnotesize
\begin{itemize}
\tightlist
\item
  TMA reaches the DIMM position.
\item
  DIMM imaging quality is sufficient.
\end{itemize}

}

\begin{tabular}{p{2cm}}
\toprule
Step 242  \\ \hline
\end{tabular}
 Description \\
{\footnotesize
\textbf{Move TMA to the DIMM position and \textbf{Take DIMM images}}\\

\begin{itemize}
\tightlist
\item
  Command the TMA to the DIMM position by applying the offsets
\item
  While tracking, take DIMM images with XXXs exposure time and inspect
  the quality.
\end{itemize}

}
\hdashrule[0.5ex]{\textwidth}{1pt}{3mm}
  Expected Result \\
{\footnotesize
\begin{itemize}
\tightlist
\item
  TMA reaches the DIMM position.
\item
  DIMM imaging quality is sufficient.
\end{itemize}

}

\begin{tabular}{p{2cm}}
\toprule
Step 243  \\ \hline
\end{tabular}
 Description \\
{\footnotesize
\textbf{Move TMA to the DIMM position and \textbf{Take DIMM images}}\\

\begin{itemize}
\tightlist
\item
  Command the TMA to the DIMM position by applying the offsets
\item
  While tracking, take DIMM images with XXXs exposure time and inspect
  the quality.
\end{itemize}

}
\hdashrule[0.5ex]{\textwidth}{1pt}{3mm}
  Expected Result \\
{\footnotesize
\begin{itemize}
\tightlist
\item
  TMA reaches the DIMM position.
\item
  DIMM imaging quality is sufficient.
\end{itemize}

}

\begin{tabular}{p{2cm}}
\toprule
Step 244  \\ \hline
\end{tabular}
 Description \\
{\footnotesize
\textbf{Move TMA to the DIMM position and \textbf{Take DIMM images}}\\

\begin{itemize}
\tightlist
\item
  Command the TMA to the DIMM position by applying the offsets
\item
  While tracking, take DIMM images with XXXs exposure time and inspect
  the quality.
\end{itemize}

}
\hdashrule[0.5ex]{\textwidth}{1pt}{3mm}
  Expected Result \\
{\footnotesize
\begin{itemize}
\tightlist
\item
  TMA reaches the DIMM position.
\item
  DIMM imaging quality is sufficient.
\end{itemize}

}

\begin{tabular}{p{2cm}}
\toprule
Step 245  \\ \hline
\end{tabular}
 Description \\
{\footnotesize
\textbf{Move TMA to the DIMM position and \textbf{Take DIMM images}}\\

\begin{itemize}
\tightlist
\item
  Command the TMA to the DIMM position by applying the offsets
\item
  While tracking, take DIMM images with XXXs exposure time and inspect
  the quality.
\end{itemize}

}
\hdashrule[0.5ex]{\textwidth}{1pt}{3mm}
  Expected Result \\
{\footnotesize
\begin{itemize}
\tightlist
\item
  TMA reaches the DIMM position.
\item
  DIMM imaging quality is sufficient.
\end{itemize}

}

\begin{tabular}{p{2cm}}
\toprule
Step 246  \\ \hline
\end{tabular}
 Description \\
{\footnotesize
\textbf{Move TMA to the DIMM position and \textbf{Take DIMM images}}\\

\begin{itemize}
\tightlist
\item
  Command the TMA to the DIMM position by applying the offsets
\item
  While tracking, take DIMM images with XXXs exposure time and inspect
  the quality.
\end{itemize}

}
\hdashrule[0.5ex]{\textwidth}{1pt}{3mm}
  Expected Result \\
{\footnotesize
\begin{itemize}
\tightlist
\item
  TMA reaches the DIMM position.
\item
  DIMM imaging quality is sufficient.
\end{itemize}

}

\begin{tabular}{p{2cm}}
\toprule
Step 247  \\ \hline
\end{tabular}
 Description \\
{\footnotesize
\textbf{Move TMA to the DIMM position and \textbf{Take DIMM images}}\\

\begin{itemize}
\tightlist
\item
  Command the TMA to the DIMM position by applying the offsets
\item
  While tracking, take DIMM images with XXXs exposure time and inspect
  the quality.
\end{itemize}

}
\hdashrule[0.5ex]{\textwidth}{1pt}{3mm}
  Expected Result \\
{\footnotesize
\begin{itemize}
\tightlist
\item
  TMA reaches the DIMM position.
\item
  DIMM imaging quality is sufficient.
\end{itemize}

}

\begin{tabular}{p{2cm}}
\toprule
Step 248  \\ \hline
\end{tabular}
 Description \\
{\footnotesize
\textbf{Move TMA to the DIMM position and \textbf{Take DIMM images}}\\

\begin{itemize}
\tightlist
\item
  Command the TMA to the DIMM position by applying the offsets
\item
  While tracking, take DIMM images with XXXs exposure time and inspect
  the quality.
\end{itemize}

}
\hdashrule[0.5ex]{\textwidth}{1pt}{3mm}
  Expected Result \\
{\footnotesize
\begin{itemize}
\tightlist
\item
  TMA reaches the DIMM position.
\item
  DIMM imaging quality is sufficient.
\end{itemize}

}

\begin{tabular}{p{2cm}}
\toprule
Step 249  \\ \hline
\end{tabular}
 Description \\
{\footnotesize
\textbf{Move TMA to the DIMM position and \textbf{Take DIMM images}}\\

\begin{itemize}
\tightlist
\item
  Command the TMA to the DIMM position by applying the offsets
\item
  While tracking, take DIMM images with XXXs exposure time and inspect
  the quality.
\end{itemize}

}
\hdashrule[0.5ex]{\textwidth}{1pt}{3mm}
  Expected Result \\
{\footnotesize
\begin{itemize}
\tightlist
\item
  TMA reaches the DIMM position.
\item
  DIMM imaging quality is sufficient.
\end{itemize}

}

\begin{tabular}{p{2cm}}
\toprule
Step 250  \\ \hline
\end{tabular}
 Description \\
{\footnotesize
\textbf{Move TMA to the DIMM position and \textbf{Take DIMM images}}\\

\begin{itemize}
\tightlist
\item
  Command the TMA to the DIMM position by applying the offsets
\item
  While tracking, take DIMM images with XXXs exposure time and inspect
  the quality.
\end{itemize}

}
\hdashrule[0.5ex]{\textwidth}{1pt}{3mm}
  Expected Result \\
{\footnotesize
\begin{itemize}
\tightlist
\item
  TMA reaches the DIMM position.
\item
  DIMM imaging quality is sufficient.
\end{itemize}

}

\begin{tabular}{p{2cm}}
\toprule
Step 251  \\ \hline
\end{tabular}
 Description \\
{\footnotesize
\textbf{Move TMA to the DIMM position and \textbf{Take DIMM images}}\\

\begin{itemize}
\tightlist
\item
  Command the TMA to the DIMM position by applying the offsets
\item
  While tracking, take DIMM images with XXXs exposure time and inspect
  the quality.
\end{itemize}

}
\hdashrule[0.5ex]{\textwidth}{1pt}{3mm}
  Expected Result \\
{\footnotesize
\begin{itemize}
\tightlist
\item
  TMA reaches the DIMM position.
\item
  DIMM imaging quality is sufficient.
\end{itemize}

}

\begin{tabular}{p{2cm}}
\toprule
Step 252  \\ \hline
\end{tabular}
 Description \\
{\footnotesize
\textbf{Move TMA to the DIMM position and \textbf{Take DIMM images}}\\

\begin{itemize}
\tightlist
\item
  Command the TMA to the DIMM position by applying the offsets
\item
  While tracking, take DIMM images with XXXs exposure time and inspect
  the quality.
\end{itemize}

}
\hdashrule[0.5ex]{\textwidth}{1pt}{3mm}
  Expected Result \\
{\footnotesize
\begin{itemize}
\tightlist
\item
  TMA reaches the DIMM position.
\item
  DIMM imaging quality is sufficient.
\end{itemize}

}

\begin{tabular}{p{2cm}}
\toprule
Step 253  \\ \hline
\end{tabular}
 Description \\
{\footnotesize
\textbf{Move TMA to the DIMM position and \textbf{Take DIMM images}}\\

\begin{itemize}
\tightlist
\item
  Command the TMA to the DIMM position by applying the offsets
\item
  While tracking, take DIMM images with XXXs exposure time and inspect
  the quality.
\end{itemize}

}
\hdashrule[0.5ex]{\textwidth}{1pt}{3mm}
  Expected Result \\
{\footnotesize
\begin{itemize}
\tightlist
\item
  TMA reaches the DIMM position.
\item
  DIMM imaging quality is sufficient.
\end{itemize}

}

\begin{tabular}{p{2cm}}
\toprule
Step 254  \\ \hline
\end{tabular}
 Description \\
{\footnotesize
\textbf{Move TMA to the DIMM position and \textbf{Take DIMM images}}\\

\begin{itemize}
\tightlist
\item
  Command the TMA to the DIMM position by applying the offsets
\item
  While tracking, take DIMM images with XXXs exposure time and inspect
  the quality.
\end{itemize}

}
\hdashrule[0.5ex]{\textwidth}{1pt}{3mm}
  Expected Result \\
{\footnotesize
\begin{itemize}
\tightlist
\item
  TMA reaches the DIMM position.
\item
  DIMM imaging quality is sufficient.
\end{itemize}

}

\begin{tabular}{p{2cm}}
\toprule
Step 255  \\ \hline
\end{tabular}
 Description \\
{\footnotesize
\textbf{Move TMA to the DIMM position and \textbf{Take DIMM images}}\\

\begin{itemize}
\tightlist
\item
  Command the TMA to the DIMM position by applying the offsets
\item
  While tracking, take DIMM images with XXXs exposure time and inspect
  the quality.
\end{itemize}

}
\hdashrule[0.5ex]{\textwidth}{1pt}{3mm}
  Expected Result \\
{\footnotesize
\begin{itemize}
\tightlist
\item
  TMA reaches the DIMM position.
\item
  DIMM imaging quality is sufficient.
\end{itemize}

}

\begin{tabular}{p{2cm}}
\toprule
Step 256  \\ \hline
\end{tabular}
 Description \\
{\footnotesize
\textbf{Move TMA to the DIMM position and \textbf{Take DIMM images}}\\

\begin{itemize}
\tightlist
\item
  Command the TMA to the DIMM position by applying the offsets
\item
  While tracking, take DIMM images with XXXs exposure time and inspect
  the quality.
\end{itemize}

}
\hdashrule[0.5ex]{\textwidth}{1pt}{3mm}
  Expected Result \\
{\footnotesize
\begin{itemize}
\tightlist
\item
  TMA reaches the DIMM position.
\item
  DIMM imaging quality is sufficient.
\end{itemize}

}

\begin{tabular}{p{2cm}}
\toprule
Step 257  \\ \hline
\end{tabular}
 Description \\
{\footnotesize
\textbf{Move TMA to the DIMM position and \textbf{Take DIMM images}}\\

\begin{itemize}
\tightlist
\item
  Command the TMA to the DIMM position by applying the offsets
\item
  While tracking, take DIMM images with XXXs exposure time and inspect
  the quality.
\end{itemize}

}
\hdashrule[0.5ex]{\textwidth}{1pt}{3mm}
  Expected Result \\
{\footnotesize
\begin{itemize}
\tightlist
\item
  TMA reaches the DIMM position.
\item
  DIMM imaging quality is sufficient.
\end{itemize}

}

\begin{tabular}{p{2cm}}
\toprule
Step 258  \\ \hline
\end{tabular}
 Description \\
{\footnotesize
\textbf{Move TMA to the DIMM position and \textbf{Take DIMM images}}\\

\begin{itemize}
\tightlist
\item
  Command the TMA to the DIMM position by applying the offsets
\item
  While tracking, take DIMM images with XXXs exposure time and inspect
  the quality.
\end{itemize}

}
\hdashrule[0.5ex]{\textwidth}{1pt}{3mm}
  Expected Result \\
{\footnotesize
\begin{itemize}
\tightlist
\item
  TMA reaches the DIMM position.
\item
  DIMM imaging quality is sufficient.
\end{itemize}

}

\begin{tabular}{p{2cm}}
\toprule
Step 259  \\ \hline
\end{tabular}
 Description \\
{\footnotesize
\textbf{Move TMA to the DIMM position and \textbf{Take DIMM images}}\\

\begin{itemize}
\tightlist
\item
  Command the TMA to the DIMM position by applying the offsets
\item
  While tracking, take DIMM images with XXXs exposure time and inspect
  the quality.
\end{itemize}

}
\hdashrule[0.5ex]{\textwidth}{1pt}{3mm}
  Expected Result \\
{\footnotesize
\begin{itemize}
\tightlist
\item
  TMA reaches the DIMM position.
\item
  DIMM imaging quality is sufficient.
\end{itemize}

}

\begin{tabular}{p{2cm}}
\toprule
Step 260  \\ \hline
\end{tabular}
 Description \\
{\footnotesize
\textbf{Move TMA to the DIMM position and \textbf{Take DIMM images}}\\

\begin{itemize}
\tightlist
\item
  Command the TMA to the DIMM position by applying the offsets
\item
  While tracking, take DIMM images with XXXs exposure time and inspect
  the quality.
\end{itemize}

}
\hdashrule[0.5ex]{\textwidth}{1pt}{3mm}
  Expected Result \\
{\footnotesize
\begin{itemize}
\tightlist
\item
  TMA reaches the DIMM position.
\item
  DIMM imaging quality is sufficient.
\end{itemize}

}

\begin{tabular}{p{2cm}}
\toprule
Step 261  \\ \hline
\end{tabular}
 Description \\
{\footnotesize
\textbf{Move TMA to the DIMM position and \textbf{Take DIMM images}}\\

\begin{itemize}
\tightlist
\item
  Command the TMA to the DIMM position by applying the offsets
\item
  While tracking, take DIMM images with XXXs exposure time and inspect
  the quality.
\end{itemize}

}
\hdashrule[0.5ex]{\textwidth}{1pt}{3mm}
  Expected Result \\
{\footnotesize
\begin{itemize}
\tightlist
\item
  TMA reaches the DIMM position.
\item
  DIMM imaging quality is sufficient.
\end{itemize}

}

\begin{tabular}{p{2cm}}
\toprule
Step 262  \\ \hline
\end{tabular}
 Description \\
{\footnotesize
\textbf{Move TMA to the DIMM position and \textbf{Take DIMM images}}\\

\begin{itemize}
\tightlist
\item
  Command the TMA to the DIMM position by applying the offsets
\item
  While tracking, take DIMM images with XXXs exposure time and inspect
  the quality.
\end{itemize}

}
\hdashrule[0.5ex]{\textwidth}{1pt}{3mm}
  Expected Result \\
{\footnotesize
\begin{itemize}
\tightlist
\item
  TMA reaches the DIMM position.
\item
  DIMM imaging quality is sufficient.
\end{itemize}

}

\begin{tabular}{p{2cm}}
\toprule
Step 263  \\ \hline
\end{tabular}
 Description \\
{\footnotesize
\textbf{Move TMA to the DIMM position and \textbf{Take DIMM images}}\\

\begin{itemize}
\tightlist
\item
  Command the TMA to the DIMM position by applying the offsets
\item
  While tracking, take DIMM images with XXXs exposure time and inspect
  the quality.
\end{itemize}

}
\hdashrule[0.5ex]{\textwidth}{1pt}{3mm}
  Expected Result \\
{\footnotesize
\begin{itemize}
\tightlist
\item
  TMA reaches the DIMM position.
\item
  DIMM imaging quality is sufficient.
\end{itemize}

}

\begin{tabular}{p{2cm}}
\toprule
Step 264  \\ \hline
\end{tabular}
 Description \\
{\footnotesize
\textbf{Move TMA to the DIMM position and \textbf{Take DIMM images}}\\

\begin{itemize}
\tightlist
\item
  Command the TMA to the DIMM position by applying the offsets
\item
  While tracking, take DIMM images with XXXs exposure time and inspect
  the quality.
\end{itemize}

}
\hdashrule[0.5ex]{\textwidth}{1pt}{3mm}
  Expected Result \\
{\footnotesize
\begin{itemize}
\tightlist
\item
  TMA reaches the DIMM position.
\item
  DIMM imaging quality is sufficient.
\end{itemize}

}

\begin{tabular}{p{2cm}}
\toprule
Step 265  \\ \hline
\end{tabular}
 Description \\
{\footnotesize
\textbf{Move TMA to the DIMM position and \textbf{Take DIMM images}}\\

\begin{itemize}
\tightlist
\item
  Command the TMA to the DIMM position by applying the offsets
\item
  While tracking, take DIMM images with XXXs exposure time and inspect
  the quality.
\end{itemize}

}
\hdashrule[0.5ex]{\textwidth}{1pt}{3mm}
  Expected Result \\
{\footnotesize
\begin{itemize}
\tightlist
\item
  TMA reaches the DIMM position.
\item
  DIMM imaging quality is sufficient.
\end{itemize}

}

\begin{tabular}{p{2cm}}
\toprule
Step 266  \\ \hline
\end{tabular}
 Description \\
{\footnotesize
\textbf{Move TMA to the DIMM position and \textbf{Take DIMM images}}\\

\begin{itemize}
\tightlist
\item
  Command the TMA to the DIMM position by applying the offsets
\item
  While tracking, take DIMM images with XXXs exposure time and inspect
  the quality.
\end{itemize}

}
\hdashrule[0.5ex]{\textwidth}{1pt}{3mm}
  Expected Result \\
{\footnotesize
\begin{itemize}
\tightlist
\item
  TMA reaches the DIMM position.
\item
  DIMM imaging quality is sufficient.
\end{itemize}

}

\begin{tabular}{p{2cm}}
\toprule
Step 267  \\ \hline
\end{tabular}
 Description \\
{\footnotesize
\textbf{Move TMA to the DIMM position and \textbf{Take DIMM images}}\\

\begin{itemize}
\tightlist
\item
  Command the TMA to the DIMM position by applying the offsets
\item
  While tracking, take DIMM images with XXXs exposure time and inspect
  the quality.
\end{itemize}

}
\hdashrule[0.5ex]{\textwidth}{1pt}{3mm}
  Expected Result \\
{\footnotesize
\begin{itemize}
\tightlist
\item
  TMA reaches the DIMM position.
\item
  DIMM imaging quality is sufficient.
\end{itemize}

}

\begin{tabular}{p{2cm}}
\toprule
Step 268  \\ \hline
\end{tabular}
 Description \\
{\footnotesize
\textbf{Move TMA to the DIMM position and \textbf{Take DIMM images}}\\

\begin{itemize}
\tightlist
\item
  Command the TMA to the DIMM position by applying the offsets
\item
  While tracking, take DIMM images with XXXs exposure time and inspect
  the quality.
\end{itemize}

}
\hdashrule[0.5ex]{\textwidth}{1pt}{3mm}
  Expected Result \\
{\footnotesize
\begin{itemize}
\tightlist
\item
  TMA reaches the DIMM position.
\item
  DIMM imaging quality is sufficient.
\end{itemize}

}

\begin{tabular}{p{2cm}}
\toprule
Step 269  \\ \hline
\end{tabular}
 Description \\
{\footnotesize
\textbf{Move TMA to the DIMM position and \textbf{Take DIMM images}}\\

\begin{itemize}
\tightlist
\item
  Command the TMA to the DIMM position by applying the offsets
\item
  While tracking, take DIMM images with XXXs exposure time and inspect
  the quality.
\end{itemize}

}
\hdashrule[0.5ex]{\textwidth}{1pt}{3mm}
  Expected Result \\
{\footnotesize
\begin{itemize}
\tightlist
\item
  TMA reaches the DIMM position.
\item
  DIMM imaging quality is sufficient.
\end{itemize}

}

\begin{tabular}{p{2cm}}
\toprule
Step 270  \\ \hline
\end{tabular}
 Description \\
{\footnotesize
\textbf{Move TMA to the DIMM position and \textbf{Take DIMM images}}\\

\begin{itemize}
\tightlist
\item
  Command the TMA to the DIMM position by applying the offsets
\item
  While tracking, take DIMM images with XXXs exposure time and inspect
  the quality.
\end{itemize}

}
\hdashrule[0.5ex]{\textwidth}{1pt}{3mm}
  Expected Result \\
{\footnotesize
\begin{itemize}
\tightlist
\item
  TMA reaches the DIMM position.
\item
  DIMM imaging quality is sufficient.
\end{itemize}

}

\begin{tabular}{p{2cm}}
\toprule
Step 271  \\ \hline
\end{tabular}
 Description \\
{\footnotesize
\textbf{Point the TMA to (Az, El)-pattern position + DIMM pattern offset
\textbf{and take DIMM images}\\
}

\begin{itemize}
\tightlist
\item
  Point the TMA back to {Pointing 1}⁠ at {-270}⁠ + DIMM offset, {15}⁠ +
  DIMM offset.
\item
  While tracking, take DIMM images with XXXs exposure time and inspect
  the quality.
\end{itemize}

}
\hdashrule[0.5ex]{\textwidth}{1pt}{3mm}
  Expected Result \\
{\footnotesize
\begin{itemize}
\tightlist
\item
  TMA reaches the position
\item
  DIMM image quality is sufficient
\end{itemize}

}

\begin{tabular}{p{2cm}}
\toprule
Step 272  \\ \hline
\end{tabular}
 Description \\
{\footnotesize
\textbf{Point the TMA to (Az, El)-pattern position + DIMM pattern offset
\textbf{and take DIMM images}\\
}

\begin{itemize}
\tightlist
\item
  Point the TMA back to {Pointing 2}⁠ at {-270}⁠ + DIMM offset, {45}⁠ +
  DIMM offset.
\item
  While tracking, take DIMM images with XXXs exposure time and inspect
  the quality.
\end{itemize}

}
\hdashrule[0.5ex]{\textwidth}{1pt}{3mm}
  Expected Result \\
{\footnotesize
\begin{itemize}
\tightlist
\item
  TMA reaches the position
\item
  DIMM image quality is sufficient
\end{itemize}

}

\begin{tabular}{p{2cm}}
\toprule
Step 273  \\ \hline
\end{tabular}
 Description \\
{\footnotesize
\textbf{Point the TMA to (Az, El)-pattern position + DIMM pattern offset
\textbf{and take DIMM images}\\
}

\begin{itemize}
\tightlist
\item
  Point the TMA back to {Pointing 3}⁠ at {-270}⁠ + DIMM offset, {75}⁠ +
  DIMM offset.
\item
  While tracking, take DIMM images with XXXs exposure time and inspect
  the quality.
\end{itemize}

}
\hdashrule[0.5ex]{\textwidth}{1pt}{3mm}
  Expected Result \\
{\footnotesize
\begin{itemize}
\tightlist
\item
  TMA reaches the position
\item
  DIMM image quality is sufficient
\end{itemize}

}

\begin{tabular}{p{2cm}}
\toprule
Step 274  \\ \hline
\end{tabular}
 Description \\
{\footnotesize
\textbf{Point the TMA to (Az, El)-pattern position + DIMM pattern offset
\textbf{and take DIMM images}\\
}

\begin{itemize}
\tightlist
\item
  Point the TMA back to {Pointing 4}⁠ at {-270}⁠ + DIMM offset, {86.5}⁠
  + DIMM offset.
\item
  While tracking, take DIMM images with XXXs exposure time and inspect
  the quality.
\end{itemize}

}
\hdashrule[0.5ex]{\textwidth}{1pt}{3mm}
  Expected Result \\
{\footnotesize
\begin{itemize}
\tightlist
\item
  TMA reaches the position
\item
  DIMM image quality is sufficient
\end{itemize}

}

\begin{tabular}{p{2cm}}
\toprule
Step 275  \\ \hline
\end{tabular}
 Description \\
{\footnotesize
\textbf{Point the TMA to (Az, El)-pattern position + DIMM pattern offset
\textbf{and take DIMM images}\\
}

\begin{itemize}
\tightlist
\item
  Point the TMA back to {Pointing 5}⁠ at {-180}⁠ + DIMM offset, {86.5}⁠
  + DIMM offset.
\item
  While tracking, take DIMM images with XXXs exposure time and inspect
  the quality.
\end{itemize}

}
\hdashrule[0.5ex]{\textwidth}{1pt}{3mm}
  Expected Result \\
{\footnotesize
\begin{itemize}
\tightlist
\item
  TMA reaches the position
\item
  DIMM image quality is sufficient
\end{itemize}

}

\begin{tabular}{p{2cm}}
\toprule
Step 276  \\ \hline
\end{tabular}
 Description \\
{\footnotesize
\textbf{Point the TMA to (Az, El)-pattern position + DIMM pattern offset
\textbf{and take DIMM images}\\
}

\begin{itemize}
\tightlist
\item
  Point the TMA back to {Pointing 6}⁠ at {-180}⁠ + DIMM offset, {75}⁠ +
  DIMM offset.
\item
  While tracking, take DIMM images with XXXs exposure time and inspect
  the quality.
\end{itemize}

}
\hdashrule[0.5ex]{\textwidth}{1pt}{3mm}
  Expected Result \\
{\footnotesize
\begin{itemize}
\tightlist
\item
  TMA reaches the position
\item
  DIMM image quality is sufficient
\end{itemize}

}

\begin{tabular}{p{2cm}}
\toprule
Step 277  \\ \hline
\end{tabular}
 Description \\
{\footnotesize
\textbf{Point the TMA to (Az, El)-pattern position + DIMM pattern offset
\textbf{and take DIMM images}\\
}

\begin{itemize}
\tightlist
\item
  Point the TMA back to {Pointing 7}⁠ at {-180}⁠ + DIMM offset, {45}⁠ +
  DIMM offset.
\item
  While tracking, take DIMM images with XXXs exposure time and inspect
  the quality.
\end{itemize}

}
\hdashrule[0.5ex]{\textwidth}{1pt}{3mm}
  Expected Result \\
{\footnotesize
\begin{itemize}
\tightlist
\item
  TMA reaches the position
\item
  DIMM image quality is sufficient
\end{itemize}

}

\begin{tabular}{p{2cm}}
\toprule
Step 278  \\ \hline
\end{tabular}
 Description \\
{\footnotesize
\textbf{Point the TMA to (Az, El)-pattern position + DIMM pattern offset
\textbf{and take DIMM images}\\
}

\begin{itemize}
\tightlist
\item
  Point the TMA back to {Pointing 8}⁠ at {-180}⁠ + DIMM offset, {15}⁠ +
  DIMM offset.
\item
  While tracking, take DIMM images with XXXs exposure time and inspect
  the quality.
\end{itemize}

}
\hdashrule[0.5ex]{\textwidth}{1pt}{3mm}
  Expected Result \\
{\footnotesize
\begin{itemize}
\tightlist
\item
  TMA reaches the position
\item
  DIMM image quality is sufficient
\end{itemize}

}

\begin{tabular}{p{2cm}}
\toprule
Step 279  \\ \hline
\end{tabular}
 Description \\
{\footnotesize
\textbf{Point the TMA to (Az, El)-pattern position + DIMM pattern offset
\textbf{and take DIMM images}\\
}

\begin{itemize}
\tightlist
\item
  Point the TMA back to {Pointing 9}⁠ at {-90}⁠ + DIMM offset, {15}⁠ +
  DIMM offset.
\item
  While tracking, take DIMM images with XXXs exposure time and inspect
  the quality.
\end{itemize}

}
\hdashrule[0.5ex]{\textwidth}{1pt}{3mm}
  Expected Result \\
{\footnotesize
\begin{itemize}
\tightlist
\item
  TMA reaches the position
\item
  DIMM image quality is sufficient
\end{itemize}

}

\begin{tabular}{p{2cm}}
\toprule
Step 280  \\ \hline
\end{tabular}
 Description \\
{\footnotesize
\textbf{Point the TMA to (Az, El)-pattern position + DIMM pattern offset
\textbf{and take DIMM images}\\
}

\begin{itemize}
\tightlist
\item
  Point the TMA back to {Pointing 10}⁠ at {-90}⁠ + DIMM offset, {45}⁠ +
  DIMM offset.
\item
  While tracking, take DIMM images with XXXs exposure time and inspect
  the quality.
\end{itemize}

}
\hdashrule[0.5ex]{\textwidth}{1pt}{3mm}
  Expected Result \\
{\footnotesize
\begin{itemize}
\tightlist
\item
  TMA reaches the position
\item
  DIMM image quality is sufficient
\end{itemize}

}

\begin{tabular}{p{2cm}}
\toprule
Step 281  \\ \hline
\end{tabular}
 Description \\
{\footnotesize
\textbf{Point the TMA to (Az, El)-pattern position + DIMM pattern offset
\textbf{and take DIMM images}\\
}

\begin{itemize}
\tightlist
\item
  Point the TMA back to {Pointing 11}⁠ at {-90}⁠ + DIMM offset, {75}⁠ +
  DIMM offset.
\item
  While tracking, take DIMM images with XXXs exposure time and inspect
  the quality.
\end{itemize}

}
\hdashrule[0.5ex]{\textwidth}{1pt}{3mm}
  Expected Result \\
{\footnotesize
\begin{itemize}
\tightlist
\item
  TMA reaches the position
\item
  DIMM image quality is sufficient
\end{itemize}

}

\begin{tabular}{p{2cm}}
\toprule
Step 282  \\ \hline
\end{tabular}
 Description \\
{\footnotesize
\textbf{Point the TMA to (Az, El)-pattern position + DIMM pattern offset
\textbf{and take DIMM images}\\
}

\begin{itemize}
\tightlist
\item
  Point the TMA back to {Pointing 12}⁠ at {-90}⁠ + DIMM offset, {86.5}⁠
  + DIMM offset.
\item
  While tracking, take DIMM images with XXXs exposure time and inspect
  the quality.
\end{itemize}

}
\hdashrule[0.5ex]{\textwidth}{1pt}{3mm}
  Expected Result \\
{\footnotesize
\begin{itemize}
\tightlist
\item
  TMA reaches the position
\item
  DIMM image quality is sufficient
\end{itemize}

}

\begin{tabular}{p{2cm}}
\toprule
Step 283  \\ \hline
\end{tabular}
 Description \\
{\footnotesize
\textbf{Point the TMA to (Az, El)-pattern position + DIMM pattern offset
\textbf{and take DIMM images}\\
}

\begin{itemize}
\tightlist
\item
  Point the TMA back to {Pointing 13}⁠ at {0}⁠ + DIMM offset, {86.5}⁠ +
  DIMM offset.
\item
  While tracking, take DIMM images with XXXs exposure time and inspect
  the quality.
\end{itemize}

}
\hdashrule[0.5ex]{\textwidth}{1pt}{3mm}
  Expected Result \\
{\footnotesize
\begin{itemize}
\tightlist
\item
  TMA reaches the position
\item
  DIMM image quality is sufficient
\end{itemize}

}

\begin{tabular}{p{2cm}}
\toprule
Step 284  \\ \hline
\end{tabular}
 Description \\
{\footnotesize
\textbf{Point the TMA to (Az, El)-pattern position + DIMM pattern offset
\textbf{and take DIMM images}\\
}

\begin{itemize}
\tightlist
\item
  Point the TMA back to {Pointing 14}⁠ at {0}⁠ + DIMM offset, {75}⁠ +
  DIMM offset.
\item
  While tracking, take DIMM images with XXXs exposure time and inspect
  the quality.
\end{itemize}

}
\hdashrule[0.5ex]{\textwidth}{1pt}{3mm}
  Expected Result \\
{\footnotesize
\begin{itemize}
\tightlist
\item
  TMA reaches the position
\item
  DIMM image quality is sufficient
\end{itemize}

}

\begin{tabular}{p{2cm}}
\toprule
Step 285  \\ \hline
\end{tabular}
 Description \\
{\footnotesize
\textbf{Point the TMA to (Az, El)-pattern position + DIMM pattern offset
\textbf{and take DIMM images}\\
}

\begin{itemize}
\tightlist
\item
  Point the TMA back to {Pointing 15}⁠ at {0}⁠ + DIMM offset, {45}⁠ +
  DIMM offset.
\item
  While tracking, take DIMM images with XXXs exposure time and inspect
  the quality.
\end{itemize}

}
\hdashrule[0.5ex]{\textwidth}{1pt}{3mm}
  Expected Result \\
{\footnotesize
\begin{itemize}
\tightlist
\item
  TMA reaches the position
\item
  DIMM image quality is sufficient
\end{itemize}

}

\begin{tabular}{p{2cm}}
\toprule
Step 286  \\ \hline
\end{tabular}
 Description \\
{\footnotesize
\textbf{Point the TMA to (Az, El)-pattern position + DIMM pattern offset
\textbf{and take DIMM images}\\
}

\begin{itemize}
\tightlist
\item
  Point the TMA back to {Pointing 16}⁠ at {0}⁠ + DIMM offset, {15}⁠ +
  DIMM offset.
\item
  While tracking, take DIMM images with XXXs exposure time and inspect
  the quality.
\end{itemize}

}
\hdashrule[0.5ex]{\textwidth}{1pt}{3mm}
  Expected Result \\
{\footnotesize
\begin{itemize}
\tightlist
\item
  TMA reaches the position
\item
  DIMM image quality is sufficient
\end{itemize}

}

\begin{tabular}{p{2cm}}
\toprule
Step 287  \\ \hline
\end{tabular}
 Description \\
{\footnotesize
\textbf{Point the TMA to (Az, El)-pattern position + DIMM pattern offset
\textbf{and take DIMM images}\\
}

\begin{itemize}
\tightlist
\item
  Point the TMA back to {Pointing 17}⁠ at {90}⁠ + DIMM offset, {15}⁠ +
  DIMM offset.
\item
  While tracking, take DIMM images with XXXs exposure time and inspect
  the quality.
\end{itemize}

}
\hdashrule[0.5ex]{\textwidth}{1pt}{3mm}
  Expected Result \\
{\footnotesize
\begin{itemize}
\tightlist
\item
  TMA reaches the position
\item
  DIMM image quality is sufficient
\end{itemize}

}

\begin{tabular}{p{2cm}}
\toprule
Step 288  \\ \hline
\end{tabular}
 Description \\
{\footnotesize
\textbf{Point the TMA to (Az, El)-pattern position + DIMM pattern offset
\textbf{and take DIMM images}\\
}

\begin{itemize}
\tightlist
\item
  Point the TMA back to {Pointing 18}⁠ at {90}⁠ + DIMM offset, {45}⁠ +
  DIMM offset.
\item
  While tracking, take DIMM images with XXXs exposure time and inspect
  the quality.
\end{itemize}

}
\hdashrule[0.5ex]{\textwidth}{1pt}{3mm}
  Expected Result \\
{\footnotesize
\begin{itemize}
\tightlist
\item
  TMA reaches the position
\item
  DIMM image quality is sufficient
\end{itemize}

}

\begin{tabular}{p{2cm}}
\toprule
Step 289  \\ \hline
\end{tabular}
 Description \\
{\footnotesize
\textbf{Point the TMA to (Az, El)-pattern position + DIMM pattern offset
\textbf{and take DIMM images}\\
}

\begin{itemize}
\tightlist
\item
  Point the TMA back to {Pointing 19}⁠ at {90}⁠ + DIMM offset, {75}⁠ +
  DIMM offset.
\item
  While tracking, take DIMM images with XXXs exposure time and inspect
  the quality.
\end{itemize}

}
\hdashrule[0.5ex]{\textwidth}{1pt}{3mm}
  Expected Result \\
{\footnotesize
\begin{itemize}
\tightlist
\item
  TMA reaches the position
\item
  DIMM image quality is sufficient
\end{itemize}

}

\begin{tabular}{p{2cm}}
\toprule
Step 290  \\ \hline
\end{tabular}
 Description \\
{\footnotesize
\textbf{Point the TMA to (Az, El)-pattern position + DIMM pattern offset
\textbf{and take DIMM images}\\
}

\begin{itemize}
\tightlist
\item
  Point the TMA back to {Pointing 20}⁠ at {90}⁠ + DIMM offset, {86.5}⁠ +
  DIMM offset.
\item
  While tracking, take DIMM images with XXXs exposure time and inspect
  the quality.
\end{itemize}

}
\hdashrule[0.5ex]{\textwidth}{1pt}{3mm}
  Expected Result \\
{\footnotesize
\begin{itemize}
\tightlist
\item
  TMA reaches the position
\item
  DIMM image quality is sufficient
\end{itemize}

}

\begin{tabular}{p{2cm}}
\toprule
Step 291  \\ \hline
\end{tabular}
 Description \\
{\footnotesize
\textbf{Point the TMA to (Az, El)-pattern position + DIMM pattern offset
\textbf{and take DIMM images}\\
}

\begin{itemize}
\tightlist
\item
  Point the TMA back to {Pointing 21}⁠ at {180}⁠ + DIMM offset, {86.5}⁠
  + DIMM offset.
\item
  While tracking, take DIMM images with XXXs exposure time and inspect
  the quality.
\end{itemize}

}
\hdashrule[0.5ex]{\textwidth}{1pt}{3mm}
  Expected Result \\
{\footnotesize
\begin{itemize}
\tightlist
\item
  TMA reaches the position
\item
  DIMM image quality is sufficient
\end{itemize}

}

\begin{tabular}{p{2cm}}
\toprule
Step 292  \\ \hline
\end{tabular}
 Description \\
{\footnotesize
\textbf{Point the TMA to (Az, El)-pattern position + DIMM pattern offset
\textbf{and take DIMM images}\\
}

\begin{itemize}
\tightlist
\item
  Point the TMA back to {Pointing 22}⁠ at {180}⁠ + DIMM offset, {75}⁠ +
  DIMM offset.
\item
  While tracking, take DIMM images with XXXs exposure time and inspect
  the quality.
\end{itemize}

}
\hdashrule[0.5ex]{\textwidth}{1pt}{3mm}
  Expected Result \\
{\footnotesize
\begin{itemize}
\tightlist
\item
  TMA reaches the position
\item
  DIMM image quality is sufficient
\end{itemize}

}

\begin{tabular}{p{2cm}}
\toprule
Step 293  \\ \hline
\end{tabular}
 Description \\
{\footnotesize
\textbf{Point the TMA to (Az, El)-pattern position + DIMM pattern offset
\textbf{and take DIMM images}\\
}

\begin{itemize}
\tightlist
\item
  Point the TMA back to {Pointing 23}⁠ at {180}⁠ + DIMM offset, {45}⁠ +
  DIMM offset.
\item
  While tracking, take DIMM images with XXXs exposure time and inspect
  the quality.
\end{itemize}

}
\hdashrule[0.5ex]{\textwidth}{1pt}{3mm}
  Expected Result \\
{\footnotesize
\begin{itemize}
\tightlist
\item
  TMA reaches the position
\item
  DIMM image quality is sufficient
\end{itemize}

}

\begin{tabular}{p{2cm}}
\toprule
Step 294  \\ \hline
\end{tabular}
 Description \\
{\footnotesize
\textbf{Point the TMA to (Az, El)-pattern position + DIMM pattern offset
\textbf{and take DIMM images}\\
}

\begin{itemize}
\tightlist
\item
  Point the TMA back to {Pointing 24}⁠ at {180}⁠ + DIMM offset, {15}⁠ +
  DIMM offset.
\item
  While tracking, take DIMM images with XXXs exposure time and inspect
  the quality.
\end{itemize}

}
\hdashrule[0.5ex]{\textwidth}{1pt}{3mm}
  Expected Result \\
{\footnotesize
\begin{itemize}
\tightlist
\item
  TMA reaches the position
\item
  DIMM image quality is sufficient
\end{itemize}

}

\begin{tabular}{p{2cm}}
\toprule
Step 295  \\ \hline
\end{tabular}
 Description \\
{\footnotesize
\textbf{Point the TMA to (Az, El)-pattern position + DIMM pattern offset
\textbf{and take DIMM images}\\
}

\begin{itemize}
\tightlist
\item
  Point the TMA back to {Pointing 25}⁠ at {270}⁠ + DIMM offset, {15}⁠ +
  DIMM offset.
\item
  While tracking, take DIMM images with XXXs exposure time and inspect
  the quality.
\end{itemize}

}
\hdashrule[0.5ex]{\textwidth}{1pt}{3mm}
  Expected Result \\
{\footnotesize
\begin{itemize}
\tightlist
\item
  TMA reaches the position
\item
  DIMM image quality is sufficient
\end{itemize}

}

\begin{tabular}{p{2cm}}
\toprule
Step 296  \\ \hline
\end{tabular}
 Description \\
{\footnotesize
\textbf{Point the TMA to (Az, El)-pattern position + DIMM pattern offset
\textbf{and take DIMM images}\\
}

\begin{itemize}
\tightlist
\item
  Point the TMA back to {Pointing 26}⁠ at {270}⁠ + DIMM offset, {45}⁠ +
  DIMM offset.
\item
  While tracking, take DIMM images with XXXs exposure time and inspect
  the quality.
\end{itemize}

}
\hdashrule[0.5ex]{\textwidth}{1pt}{3mm}
  Expected Result \\
{\footnotesize
\begin{itemize}
\tightlist
\item
  TMA reaches the position
\item
  DIMM image quality is sufficient
\end{itemize}

}

\begin{tabular}{p{2cm}}
\toprule
Step 297  \\ \hline
\end{tabular}
 Description \\
{\footnotesize
\textbf{Point the TMA to (Az, El)-pattern position + DIMM pattern offset
\textbf{and take DIMM images}\\
}

\begin{itemize}
\tightlist
\item
  Point the TMA back to {Pointing 27}⁠ at {270}⁠ + DIMM offset, {75}⁠ +
  DIMM offset.
\item
  While tracking, take DIMM images with XXXs exposure time and inspect
  the quality.
\end{itemize}

}
\hdashrule[0.5ex]{\textwidth}{1pt}{3mm}
  Expected Result \\
{\footnotesize
\begin{itemize}
\tightlist
\item
  TMA reaches the position
\item
  DIMM image quality is sufficient
\end{itemize}

}

\begin{tabular}{p{2cm}}
\toprule
Step 298  \\ \hline
\end{tabular}
 Description \\
{\footnotesize
\textbf{Point the TMA to (Az, El)-pattern position + DIMM pattern offset
\textbf{and take DIMM images}\\
}

\begin{itemize}
\tightlist
\item
  Point the TMA back to {Pointing 13}⁠ at {0}⁠ + DIMM offset, {86.5}⁠ +
  DIMM offset.
\item
  While tracking, take DIMM images with XXXs exposure time and inspect
  the quality.
\end{itemize}

}
\hdashrule[0.5ex]{\textwidth}{1pt}{3mm}
  Expected Result \\
{\footnotesize
\begin{itemize}
\tightlist
\item
  TMA reaches the position
\item
  DIMM image quality is sufficient
\end{itemize}

}

\begin{tabular}{p{2cm}}
\toprule
Step 299  \\ \hline
\end{tabular}
 Description \\
{\footnotesize
\textbf{Point the TMA to (Az, El)-pattern position + DIMM pattern offset
\textbf{and take DIMM images}\\
}

\begin{itemize}
\tightlist
\item
  Point the TMA back to {Pointing 28}⁠ at {270}⁠ + DIMM offset, {86.5}⁠
  + DIMM offset.
\item
  While tracking, take DIMM images with XXXs exposure time and inspect
  the quality.
\end{itemize}

}
\hdashrule[0.5ex]{\textwidth}{1pt}{3mm}
  Expected Result \\
{\footnotesize
\begin{itemize}
\tightlist
\item
  TMA reaches the position
\item
  DIMM image quality is sufficient
\end{itemize}

}

\begin{tabular}{p{2cm}}
\toprule
Step 300  \\ \hline
\end{tabular}
 Description \\
{\footnotesize
\textbf{Point the TMA to (Az, El)-pattern position + DIMM pattern offset
\textbf{and take DIMM images}\\
}

\begin{itemize}
\tightlist
\item
  Point the TMA back to {Pointing 14}⁠ at {0}⁠ + DIMM offset, {75}⁠ +
  DIMM offset.
\item
  While tracking, take DIMM images with XXXs exposure time and inspect
  the quality.
\end{itemize}

}
\hdashrule[0.5ex]{\textwidth}{1pt}{3mm}
  Expected Result \\
{\footnotesize
\begin{itemize}
\tightlist
\item
  TMA reaches the position
\item
  DIMM image quality is sufficient
\end{itemize}

}

\begin{tabular}{p{2cm}}
\toprule
Step 301  \\ \hline
\end{tabular}
 Description \\
{\footnotesize
\textbf{Point the TMA to (Az, El)-pattern position + DIMM pattern offset
\textbf{and take DIMM images}\\
}

\begin{itemize}
\tightlist
\item
  Point the TMA back to {Pointing 15}⁠ at {0}⁠ + DIMM offset, {45}⁠ +
  DIMM offset.
\item
  While tracking, take DIMM images with XXXs exposure time and inspect
  the quality.
\end{itemize}

}
\hdashrule[0.5ex]{\textwidth}{1pt}{3mm}
  Expected Result \\
{\footnotesize
\begin{itemize}
\tightlist
\item
  TMA reaches the position
\item
  DIMM image quality is sufficient
\end{itemize}

}

\begin{tabular}{p{2cm}}
\toprule
Step 302  \\ \hline
\end{tabular}
 Description \\
{\footnotesize
\textbf{Point the TMA to (Az, El)-pattern position + DIMM pattern offset
\textbf{and take DIMM images}\\
}

\begin{itemize}
\tightlist
\item
  Point the TMA back to {Pointing 16}⁠ at {0}⁠ + DIMM offset, {15}⁠ +
  DIMM offset.
\item
  While tracking, take DIMM images with XXXs exposure time and inspect
  the quality.
\end{itemize}

}
\hdashrule[0.5ex]{\textwidth}{1pt}{3mm}
  Expected Result \\
{\footnotesize
\begin{itemize}
\tightlist
\item
  TMA reaches the position
\item
  DIMM image quality is sufficient
\end{itemize}

}

\begin{tabular}{p{2cm}}
\toprule
Step 303  \\ \hline
\end{tabular}
 Description \\
{\footnotesize
\textbf{Point the TMA to (Az, El)-pattern position + DIMM pattern offset
\textbf{and take DIMM images}\\
}

\begin{itemize}
\tightlist
\item
  Point the TMA back to {Pointing 17}⁠ at {90}⁠ + DIMM offset, {15}⁠ +
  DIMM offset.
\item
  While tracking, take DIMM images with XXXs exposure time and inspect
  the quality.
\end{itemize}

}
\hdashrule[0.5ex]{\textwidth}{1pt}{3mm}
  Expected Result \\
{\footnotesize
\begin{itemize}
\tightlist
\item
  TMA reaches the position
\item
  DIMM image quality is sufficient
\end{itemize}

}

\begin{tabular}{p{2cm}}
\toprule
Step 304  \\ \hline
\end{tabular}
 Description \\
{\footnotesize
\textbf{Point the TMA to (Az, El)-pattern position + DIMM pattern offset
\textbf{and take DIMM images}\\
}

\begin{itemize}
\tightlist
\item
  Point the TMA back to {Pointing 18}⁠ at {90}⁠ + DIMM offset, {45}⁠ +
  DIMM offset.
\item
  While tracking, take DIMM images with XXXs exposure time and inspect
  the quality.
\end{itemize}

}
\hdashrule[0.5ex]{\textwidth}{1pt}{3mm}
  Expected Result \\
{\footnotesize
\begin{itemize}
\tightlist
\item
  TMA reaches the position
\item
  DIMM image quality is sufficient
\end{itemize}

}

\begin{tabular}{p{2cm}}
\toprule
Step 305  \\ \hline
\end{tabular}
 Description \\
{\footnotesize
\textbf{Point the TMA to (Az, El)-pattern position + DIMM pattern offset
\textbf{and take DIMM images}\\
}

\begin{itemize}
\tightlist
\item
  Point the TMA back to {Pointing 19}⁠ at {90}⁠ + DIMM offset, {75}⁠ +
  DIMM offset.
\item
  While tracking, take DIMM images with XXXs exposure time and inspect
  the quality.
\end{itemize}

}
\hdashrule[0.5ex]{\textwidth}{1pt}{3mm}
  Expected Result \\
{\footnotesize
\begin{itemize}
\tightlist
\item
  TMA reaches the position
\item
  DIMM image quality is sufficient
\end{itemize}

}

\begin{tabular}{p{2cm}}
\toprule
Step 306  \\ \hline
\end{tabular}
 Description \\
{\footnotesize
\textbf{Point the TMA to (Az, El)-pattern position + DIMM pattern offset
\textbf{and take DIMM images}\\
}

\begin{itemize}
\tightlist
\item
  Point the TMA back to {Pointing 20}⁠ at {90}⁠ + DIMM offset, {86.5}⁠ +
  DIMM offset.
\item
  While tracking, take DIMM images with XXXs exposure time and inspect
  the quality.
\end{itemize}

}
\hdashrule[0.5ex]{\textwidth}{1pt}{3mm}
  Expected Result \\
{\footnotesize
\begin{itemize}
\tightlist
\item
  TMA reaches the position
\item
  DIMM image quality is sufficient
\end{itemize}

}

\begin{tabular}{p{2cm}}
\toprule
Step 307  \\ \hline
\end{tabular}
 Description \\
{\footnotesize
\textbf{Point the TMA to (Az, El)-pattern position + DIMM pattern offset
\textbf{and take DIMM images}\\
}

\begin{itemize}
\tightlist
\item
  Point the TMA back to {Pointing 21}⁠ at {180}⁠ + DIMM offset, {86.5}⁠
  + DIMM offset.
\item
  While tracking, take DIMM images with XXXs exposure time and inspect
  the quality.
\end{itemize}

}
\hdashrule[0.5ex]{\textwidth}{1pt}{3mm}
  Expected Result \\
{\footnotesize
\begin{itemize}
\tightlist
\item
  TMA reaches the position
\item
  DIMM image quality is sufficient
\end{itemize}

}

\begin{tabular}{p{2cm}}
\toprule
Step 308  \\ \hline
\end{tabular}
 Description \\
{\footnotesize
\textbf{Point the TMA to (Az, El)-pattern position + DIMM pattern offset
\textbf{and take DIMM images}\\
}

\begin{itemize}
\tightlist
\item
  Point the TMA back to {Pointing 22}⁠ at {180}⁠ + DIMM offset, {75}⁠ +
  DIMM offset.
\item
  While tracking, take DIMM images with XXXs exposure time and inspect
  the quality.
\end{itemize}

}
\hdashrule[0.5ex]{\textwidth}{1pt}{3mm}
  Expected Result \\
{\footnotesize
\begin{itemize}
\tightlist
\item
  TMA reaches the position
\item
  DIMM image quality is sufficient
\end{itemize}

}

\begin{tabular}{p{2cm}}
\toprule
Step 309  \\ \hline
\end{tabular}
 Description \\
{\footnotesize
\textbf{Point the TMA to (Az, El)-pattern position + DIMM pattern offset
\textbf{and take DIMM images}\\
}

\begin{itemize}
\tightlist
\item
  Point the TMA back to {Pointing 23}⁠ at {180}⁠ + DIMM offset, {45}⁠ +
  DIMM offset.
\item
  While tracking, take DIMM images with XXXs exposure time and inspect
  the quality.
\end{itemize}

}
\hdashrule[0.5ex]{\textwidth}{1pt}{3mm}
  Expected Result \\
{\footnotesize
\begin{itemize}
\tightlist
\item
  TMA reaches the position
\item
  DIMM image quality is sufficient
\end{itemize}

}

\begin{tabular}{p{2cm}}
\toprule
Step 310  \\ \hline
\end{tabular}
 Description \\
{\footnotesize
\textbf{Point the TMA to (Az, El)-pattern position + DIMM pattern offset
\textbf{and take DIMM images}\\
}

\begin{itemize}
\tightlist
\item
  Point the TMA back to {Pointing 24}⁠ at {180}⁠ + DIMM offset, {15}⁠ +
  DIMM offset.
\item
  While tracking, take DIMM images with XXXs exposure time and inspect
  the quality.
\end{itemize}

}
\hdashrule[0.5ex]{\textwidth}{1pt}{3mm}
  Expected Result \\
{\footnotesize
\begin{itemize}
\tightlist
\item
  TMA reaches the position
\item
  DIMM image quality is sufficient
\end{itemize}

}

\begin{tabular}{p{2cm}}
\toprule
Step 311  \\ \hline
\end{tabular}
 Description \\
{\footnotesize
\textbf{Point the TMA to (Az, El)-pattern position + DIMM pattern offset
\textbf{and take DIMM images}\\
}

\begin{itemize}
\tightlist
\item
  Point the TMA back to {Pointing 25}⁠ at {270}⁠ + DIMM offset, {15}⁠ +
  DIMM offset.
\item
  While tracking, take DIMM images with XXXs exposure time and inspect
  the quality.
\end{itemize}

}
\hdashrule[0.5ex]{\textwidth}{1pt}{3mm}
  Expected Result \\
{\footnotesize
\begin{itemize}
\tightlist
\item
  TMA reaches the position
\item
  DIMM image quality is sufficient
\end{itemize}

}

\begin{tabular}{p{2cm}}
\toprule
Step 312  \\ \hline
\end{tabular}
 Description \\
{\footnotesize
\textbf{Point the TMA to (Az, El)-pattern position + DIMM pattern offset
\textbf{and take DIMM images}\\
}

\begin{itemize}
\tightlist
\item
  Point the TMA back to {Pointing 26}⁠ at {270}⁠ + DIMM offset, {45}⁠ +
  DIMM offset.
\item
  While tracking, take DIMM images with XXXs exposure time and inspect
  the quality.
\end{itemize}

}
\hdashrule[0.5ex]{\textwidth}{1pt}{3mm}
  Expected Result \\
{\footnotesize
\begin{itemize}
\tightlist
\item
  TMA reaches the position
\item
  DIMM image quality is sufficient
\end{itemize}

}

\begin{tabular}{p{2cm}}
\toprule
Step 313  \\ \hline
\end{tabular}
 Description \\
{\footnotesize
\textbf{Point the TMA to (Az, El)-pattern position + DIMM pattern offset
\textbf{and take DIMM images}\\
}

\begin{itemize}
\tightlist
\item
  Point the TMA back to {Pointing 27}⁠ at {270}⁠ + DIMM offset, {75}⁠ +
  DIMM offset.
\item
  While tracking, take DIMM images with XXXs exposure time and inspect
  the quality.
\end{itemize}

}
\hdashrule[0.5ex]{\textwidth}{1pt}{3mm}
  Expected Result \\
{\footnotesize
\begin{itemize}
\tightlist
\item
  TMA reaches the position
\item
  DIMM image quality is sufficient
\end{itemize}

}

\begin{tabular}{p{2cm}}
\toprule
Step 314  \\ \hline
\end{tabular}
 Description \\
{\footnotesize
\textbf{Point the TMA to (Az, El)-pattern position + DIMM pattern offset
\textbf{and take DIMM images}\\
}

\begin{itemize}
\tightlist
\item
  Point the TMA back to {Pointing 28}⁠ at {270}⁠ + DIMM offset, {86.5}⁠
  + DIMM offset.
\item
  While tracking, take DIMM images with XXXs exposure time and inspect
  the quality.
\end{itemize}

}
\hdashrule[0.5ex]{\textwidth}{1pt}{3mm}
  Expected Result \\
{\footnotesize
\begin{itemize}
\tightlist
\item
  TMA reaches the position
\item
  DIMM image quality is sufficient
\end{itemize}

}

\begin{tabular}{p{2cm}}
\toprule
Step 315  \\ \hline
\end{tabular}
 Description \\
{\footnotesize
\textbf{Move TMA to the 3. random distance of 3.5deg\\
}

\begin{itemize}
\tightlist
\item
  Point the TMA to a random 3.5 deg combined offset in AZ and EL from
  {Pointing 1}⁠ at {-270}⁠, {15}⁠. Record the exact position of the
  offset in AZ and El
\end{itemize}

}
\hdashrule[0.5ex]{\textwidth}{1pt}{3mm}
  Expected Result \\
{\footnotesize
\begin{itemize}
\tightlist
\item
  The TMA reaches the commanded offset position.
\end{itemize}

}

\begin{tabular}{p{2cm}}
\toprule
Step 316  \\ \hline
\end{tabular}
 Description \\
{\footnotesize
\textbf{Move TMA to the 3. random distance of 3.5deg\\
}

\begin{itemize}
\tightlist
\item
  Point the TMA to a random 3.5 deg combined offset in AZ and EL from
  {Pointing 2}⁠ at {-270}⁠, {45}⁠. Record the exact position of the
  offset in AZ and El
\end{itemize}

}
\hdashrule[0.5ex]{\textwidth}{1pt}{3mm}
  Expected Result \\
{\footnotesize
\begin{itemize}
\tightlist
\item
  The TMA reaches the commanded offset position.
\end{itemize}

}

\begin{tabular}{p{2cm}}
\toprule
Step 317  \\ \hline
\end{tabular}
 Description \\
{\footnotesize
\textbf{Move TMA to the 3. random distance of 3.5deg\\
}

\begin{itemize}
\tightlist
\item
  Point the TMA to a random 3.5 deg combined offset in AZ and EL from
  {Pointing 3}⁠ at {-270}⁠, {75}⁠. Record the exact position of the
  offset in AZ and El
\end{itemize}

}
\hdashrule[0.5ex]{\textwidth}{1pt}{3mm}
  Expected Result \\
{\footnotesize
\begin{itemize}
\tightlist
\item
  The TMA reaches the commanded offset position.
\end{itemize}

}

\begin{tabular}{p{2cm}}
\toprule
Step 318  \\ \hline
\end{tabular}
 Description \\
{\footnotesize
\textbf{Move TMA to the 3. random distance of 3.5deg\\
}

\begin{itemize}
\tightlist
\item
  Point the TMA to a random 3.5 deg combined offset in AZ and EL from
  {Pointing 4}⁠ at {-270}⁠, {86.5}⁠. Record the exact position of the
  offset in AZ and El
\end{itemize}

}
\hdashrule[0.5ex]{\textwidth}{1pt}{3mm}
  Expected Result \\
{\footnotesize
\begin{itemize}
\tightlist
\item
  The TMA reaches the commanded offset position.
\end{itemize}

}

\begin{tabular}{p{2cm}}
\toprule
Step 319  \\ \hline
\end{tabular}
 Description \\
{\footnotesize
\textbf{Move TMA to the 3. random distance of 3.5deg\\
}

\begin{itemize}
\tightlist
\item
  Point the TMA to a random 3.5 deg combined offset in AZ and EL from
  {Pointing 5}⁠ at {-180}⁠, {86.5}⁠. Record the exact position of the
  offset in AZ and El
\end{itemize}

}
\hdashrule[0.5ex]{\textwidth}{1pt}{3mm}
  Expected Result \\
{\footnotesize
\begin{itemize}
\tightlist
\item
  The TMA reaches the commanded offset position.
\end{itemize}

}

\begin{tabular}{p{2cm}}
\toprule
Step 320  \\ \hline
\end{tabular}
 Description \\
{\footnotesize
\textbf{Move TMA to the 3. random distance of 3.5deg\\
}

\begin{itemize}
\tightlist
\item
  Point the TMA to a random 3.5 deg combined offset in AZ and EL from
  {Pointing 6}⁠ at {-180}⁠, {75}⁠. Record the exact position of the
  offset in AZ and El
\end{itemize}

}
\hdashrule[0.5ex]{\textwidth}{1pt}{3mm}
  Expected Result \\
{\footnotesize
\begin{itemize}
\tightlist
\item
  The TMA reaches the commanded offset position.
\end{itemize}

}

\begin{tabular}{p{2cm}}
\toprule
Step 321  \\ \hline
\end{tabular}
 Description \\
{\footnotesize
\textbf{Move TMA to the 3. random distance of 3.5deg\\
}

\begin{itemize}
\tightlist
\item
  Point the TMA to a random 3.5 deg combined offset in AZ and EL from
  {Pointing 7}⁠ at {-180}⁠, {45}⁠. Record the exact position of the
  offset in AZ and El
\end{itemize}

}
\hdashrule[0.5ex]{\textwidth}{1pt}{3mm}
  Expected Result \\
{\footnotesize
\begin{itemize}
\tightlist
\item
  The TMA reaches the commanded offset position.
\end{itemize}

}

\begin{tabular}{p{2cm}}
\toprule
Step 322  \\ \hline
\end{tabular}
 Description \\
{\footnotesize
\textbf{Move TMA to the 3. random distance of 3.5deg\\
}

\begin{itemize}
\tightlist
\item
  Point the TMA to a random 3.5 deg combined offset in AZ and EL from
  {Pointing 8}⁠ at {-180}⁠, {15}⁠. Record the exact position of the
  offset in AZ and El
\end{itemize}

}
\hdashrule[0.5ex]{\textwidth}{1pt}{3mm}
  Expected Result \\
{\footnotesize
\begin{itemize}
\tightlist
\item
  The TMA reaches the commanded offset position.
\end{itemize}

}

\begin{tabular}{p{2cm}}
\toprule
Step 323  \\ \hline
\end{tabular}
 Description \\
{\footnotesize
\textbf{Move TMA to the 3. random distance of 3.5deg\\
}

\begin{itemize}
\tightlist
\item
  Point the TMA to a random 3.5 deg combined offset in AZ and EL from
  {Pointing 9}⁠ at {-90}⁠, {15}⁠. Record the exact position of the
  offset in AZ and El
\end{itemize}

}
\hdashrule[0.5ex]{\textwidth}{1pt}{3mm}
  Expected Result \\
{\footnotesize
\begin{itemize}
\tightlist
\item
  The TMA reaches the commanded offset position.
\end{itemize}

}

\begin{tabular}{p{2cm}}
\toprule
Step 324  \\ \hline
\end{tabular}
 Description \\
{\footnotesize
\textbf{Move TMA to the 3. random distance of 3.5deg\\
}

\begin{itemize}
\tightlist
\item
  Point the TMA to a random 3.5 deg combined offset in AZ and EL from
  {Pointing 10}⁠ at {-90}⁠, {45}⁠. Record the exact position of the
  offset in AZ and El
\end{itemize}

}
\hdashrule[0.5ex]{\textwidth}{1pt}{3mm}
  Expected Result \\
{\footnotesize
\begin{itemize}
\tightlist
\item
  The TMA reaches the commanded offset position.
\end{itemize}

}

\begin{tabular}{p{2cm}}
\toprule
Step 325  \\ \hline
\end{tabular}
 Description \\
{\footnotesize
\textbf{Move TMA to the 3. random distance of 3.5deg\\
}

\begin{itemize}
\tightlist
\item
  Point the TMA to a random 3.5 deg combined offset in AZ and EL from
  {Pointing 11}⁠ at {-90}⁠, {75}⁠. Record the exact position of the
  offset in AZ and El
\end{itemize}

}
\hdashrule[0.5ex]{\textwidth}{1pt}{3mm}
  Expected Result \\
{\footnotesize
\begin{itemize}
\tightlist
\item
  The TMA reaches the commanded offset position.
\end{itemize}

}

\begin{tabular}{p{2cm}}
\toprule
Step 326  \\ \hline
\end{tabular}
 Description \\
{\footnotesize
\textbf{Move TMA to the 3. random distance of 3.5deg\\
}

\begin{itemize}
\tightlist
\item
  Point the TMA to a random 3.5 deg combined offset in AZ and EL from
  {Pointing 12}⁠ at {-90}⁠, {86.5}⁠. Record the exact position of the
  offset in AZ and El
\end{itemize}

}
\hdashrule[0.5ex]{\textwidth}{1pt}{3mm}
  Expected Result \\
{\footnotesize
\begin{itemize}
\tightlist
\item
  The TMA reaches the commanded offset position.
\end{itemize}

}

\begin{tabular}{p{2cm}}
\toprule
Step 327  \\ \hline
\end{tabular}
 Description \\
{\footnotesize
\textbf{Move TMA to the 3. random distance of 3.5deg\\
}

\begin{itemize}
\tightlist
\item
  Point the TMA to a random 3.5 deg combined offset in AZ and EL from
  {Pointing 13}⁠ at {0}⁠, {86.5}⁠. Record the exact position of the
  offset in AZ and El
\end{itemize}

}
\hdashrule[0.5ex]{\textwidth}{1pt}{3mm}
  Expected Result \\
{\footnotesize
\begin{itemize}
\tightlist
\item
  The TMA reaches the commanded offset position.
\end{itemize}

}

\begin{tabular}{p{2cm}}
\toprule
Step 328  \\ \hline
\end{tabular}
 Description \\
{\footnotesize
\textbf{Move TMA to the 3. random distance of 3.5deg\\
}

\begin{itemize}
\tightlist
\item
  Point the TMA to a random 3.5 deg combined offset in AZ and EL from
  {Pointing 14}⁠ at {0}⁠, {75}⁠. Record the exact position of the offset
  in AZ and El
\end{itemize}

}
\hdashrule[0.5ex]{\textwidth}{1pt}{3mm}
  Expected Result \\
{\footnotesize
\begin{itemize}
\tightlist
\item
  The TMA reaches the commanded offset position.
\end{itemize}

}

\begin{tabular}{p{2cm}}
\toprule
Step 329  \\ \hline
\end{tabular}
 Description \\
{\footnotesize
\textbf{Move TMA to the 3. random distance of 3.5deg\\
}

\begin{itemize}
\tightlist
\item
  Point the TMA to a random 3.5 deg combined offset in AZ and EL from
  {Pointing 15}⁠ at {0}⁠, {45}⁠. Record the exact position of the offset
  in AZ and El
\end{itemize}

}
\hdashrule[0.5ex]{\textwidth}{1pt}{3mm}
  Expected Result \\
{\footnotesize
\begin{itemize}
\tightlist
\item
  The TMA reaches the commanded offset position.
\end{itemize}

}

\begin{tabular}{p{2cm}}
\toprule
Step 330  \\ \hline
\end{tabular}
 Description \\
{\footnotesize
\textbf{Move TMA to the 3. random distance of 3.5deg\\
}

\begin{itemize}
\tightlist
\item
  Point the TMA to a random 3.5 deg combined offset in AZ and EL from
  {Pointing 16}⁠ at {0}⁠, {15}⁠. Record the exact position of the offset
  in AZ and El
\end{itemize}

}
\hdashrule[0.5ex]{\textwidth}{1pt}{3mm}
  Expected Result \\
{\footnotesize
\begin{itemize}
\tightlist
\item
  The TMA reaches the commanded offset position.
\end{itemize}

}

\begin{tabular}{p{2cm}}
\toprule
Step 331  \\ \hline
\end{tabular}
 Description \\
{\footnotesize
\textbf{Move TMA to the 3. random distance of 3.5deg\\
}

\begin{itemize}
\tightlist
\item
  Point the TMA to a random 3.5 deg combined offset in AZ and EL from
  {Pointing 17}⁠ at {90}⁠, {15}⁠. Record the exact position of the
  offset in AZ and El
\end{itemize}

}
\hdashrule[0.5ex]{\textwidth}{1pt}{3mm}
  Expected Result \\
{\footnotesize
\begin{itemize}
\tightlist
\item
  The TMA reaches the commanded offset position.
\end{itemize}

}

\begin{tabular}{p{2cm}}
\toprule
Step 332  \\ \hline
\end{tabular}
 Description \\
{\footnotesize
\textbf{Move TMA to the 3. random distance of 3.5deg\\
}

\begin{itemize}
\tightlist
\item
  Point the TMA to a random 3.5 deg combined offset in AZ and EL from
  {Pointing 18}⁠ at {90}⁠, {45}⁠. Record the exact position of the
  offset in AZ and El
\end{itemize}

}
\hdashrule[0.5ex]{\textwidth}{1pt}{3mm}
  Expected Result \\
{\footnotesize
\begin{itemize}
\tightlist
\item
  The TMA reaches the commanded offset position.
\end{itemize}

}

\begin{tabular}{p{2cm}}
\toprule
Step 333  \\ \hline
\end{tabular}
 Description \\
{\footnotesize
\textbf{Move TMA to the 3. random distance of 3.5deg\\
}

\begin{itemize}
\tightlist
\item
  Point the TMA to a random 3.5 deg combined offset in AZ and EL from
  {Pointing 19}⁠ at {90}⁠, {75}⁠. Record the exact position of the
  offset in AZ and El
\end{itemize}

}
\hdashrule[0.5ex]{\textwidth}{1pt}{3mm}
  Expected Result \\
{\footnotesize
\begin{itemize}
\tightlist
\item
  The TMA reaches the commanded offset position.
\end{itemize}

}

\begin{tabular}{p{2cm}}
\toprule
Step 334  \\ \hline
\end{tabular}
 Description \\
{\footnotesize
\textbf{Move TMA to the 3. random distance of 3.5deg\\
}

\begin{itemize}
\tightlist
\item
  Point the TMA to a random 3.5 deg combined offset in AZ and EL from
  {Pointing 20}⁠ at {90}⁠, {86.5}⁠. Record the exact position of the
  offset in AZ and El
\end{itemize}

}
\hdashrule[0.5ex]{\textwidth}{1pt}{3mm}
  Expected Result \\
{\footnotesize
\begin{itemize}
\tightlist
\item
  The TMA reaches the commanded offset position.
\end{itemize}

}

\begin{tabular}{p{2cm}}
\toprule
Step 335  \\ \hline
\end{tabular}
 Description \\
{\footnotesize
\textbf{Move TMA to the 3. random distance of 3.5deg\\
}

\begin{itemize}
\tightlist
\item
  Point the TMA to a random 3.5 deg combined offset in AZ and EL from
  {Pointing 21}⁠ at {180}⁠, {86.5}⁠. Record the exact position of the
  offset in AZ and El
\end{itemize}

}
\hdashrule[0.5ex]{\textwidth}{1pt}{3mm}
  Expected Result \\
{\footnotesize
\begin{itemize}
\tightlist
\item
  The TMA reaches the commanded offset position.
\end{itemize}

}

\begin{tabular}{p{2cm}}
\toprule
Step 336  \\ \hline
\end{tabular}
 Description \\
{\footnotesize
\textbf{Move TMA to the 3. random distance of 3.5deg\\
}

\begin{itemize}
\tightlist
\item
  Point the TMA to a random 3.5 deg combined offset in AZ and EL from
  {Pointing 22}⁠ at {180}⁠, {75}⁠. Record the exact position of the
  offset in AZ and El
\end{itemize}

}
\hdashrule[0.5ex]{\textwidth}{1pt}{3mm}
  Expected Result \\
{\footnotesize
\begin{itemize}
\tightlist
\item
  The TMA reaches the commanded offset position.
\end{itemize}

}

\begin{tabular}{p{2cm}}
\toprule
Step 337  \\ \hline
\end{tabular}
 Description \\
{\footnotesize
\textbf{Move TMA to the 3. random distance of 3.5deg\\
}

\begin{itemize}
\tightlist
\item
  Point the TMA to a random 3.5 deg combined offset in AZ and EL from
  {Pointing 23}⁠ at {180}⁠, {45}⁠. Record the exact position of the
  offset in AZ and El
\end{itemize}

}
\hdashrule[0.5ex]{\textwidth}{1pt}{3mm}
  Expected Result \\
{\footnotesize
\begin{itemize}
\tightlist
\item
  The TMA reaches the commanded offset position.
\end{itemize}

}

\begin{tabular}{p{2cm}}
\toprule
Step 338  \\ \hline
\end{tabular}
 Description \\
{\footnotesize
\textbf{Move TMA to the 3. random distance of 3.5deg\\
}

\begin{itemize}
\tightlist
\item
  Point the TMA to a random 3.5 deg combined offset in AZ and EL from
  {Pointing 24}⁠ at {180}⁠, {15}⁠. Record the exact position of the
  offset in AZ and El
\end{itemize}

}
\hdashrule[0.5ex]{\textwidth}{1pt}{3mm}
  Expected Result \\
{\footnotesize
\begin{itemize}
\tightlist
\item
  The TMA reaches the commanded offset position.
\end{itemize}

}

\begin{tabular}{p{2cm}}
\toprule
Step 339  \\ \hline
\end{tabular}
 Description \\
{\footnotesize
\textbf{Move TMA to the 3. random distance of 3.5deg\\
}

\begin{itemize}
\tightlist
\item
  Point the TMA to a random 3.5 deg combined offset in AZ and EL from
  {Pointing 25}⁠ at {270}⁠, {15}⁠. Record the exact position of the
  offset in AZ and El
\end{itemize}

}
\hdashrule[0.5ex]{\textwidth}{1pt}{3mm}
  Expected Result \\
{\footnotesize
\begin{itemize}
\tightlist
\item
  The TMA reaches the commanded offset position.
\end{itemize}

}

\begin{tabular}{p{2cm}}
\toprule
Step 340  \\ \hline
\end{tabular}
 Description \\
{\footnotesize
\textbf{Move TMA to the 3. random distance of 3.5deg\\
}

\begin{itemize}
\tightlist
\item
  Point the TMA to a random 3.5 deg combined offset in AZ and EL from
  {Pointing 26}⁠ at {270}⁠, {45}⁠. Record the exact position of the
  offset in AZ and El
\end{itemize}

}
\hdashrule[0.5ex]{\textwidth}{1pt}{3mm}
  Expected Result \\
{\footnotesize
\begin{itemize}
\tightlist
\item
  The TMA reaches the commanded offset position.
\end{itemize}

}

\begin{tabular}{p{2cm}}
\toprule
Step 341  \\ \hline
\end{tabular}
 Description \\
{\footnotesize
\textbf{Move TMA to the 3. random distance of 3.5deg\\
}

\begin{itemize}
\tightlist
\item
  Point the TMA to a random 3.5 deg combined offset in AZ and EL from
  {Pointing 27}⁠ at {270}⁠, {75}⁠. Record the exact position of the
  offset in AZ and El
\end{itemize}

}
\hdashrule[0.5ex]{\textwidth}{1pt}{3mm}
  Expected Result \\
{\footnotesize
\begin{itemize}
\tightlist
\item
  The TMA reaches the commanded offset position.
\end{itemize}

}

\begin{tabular}{p{2cm}}
\toprule
Step 342  \\ \hline
\end{tabular}
 Description \\
{\footnotesize
\textbf{Move TMA to the 3. random distance of 3.5deg\\
}

\begin{itemize}
\tightlist
\item
  Point the TMA to a random 3.5 deg combined offset in AZ and EL from
  {Pointing 13}⁠ at {0}⁠, {86.5}⁠. Record the exact position of the
  offset in AZ and El
\end{itemize}

}
\hdashrule[0.5ex]{\textwidth}{1pt}{3mm}
  Expected Result \\
{\footnotesize
\begin{itemize}
\tightlist
\item
  The TMA reaches the commanded offset position.
\end{itemize}

}

\begin{tabular}{p{2cm}}
\toprule
Step 343  \\ \hline
\end{tabular}
 Description \\
{\footnotesize
\textbf{Move TMA to the 3. random distance of 3.5deg\\
}

\begin{itemize}
\tightlist
\item
  Point the TMA to a random 3.5 deg combined offset in AZ and EL from
  {Pointing 28}⁠ at {270}⁠, {86.5}⁠. Record the exact position of the
  offset in AZ and El
\end{itemize}

}
\hdashrule[0.5ex]{\textwidth}{1pt}{3mm}
  Expected Result \\
{\footnotesize
\begin{itemize}
\tightlist
\item
  The TMA reaches the commanded offset position.
\end{itemize}

}

\begin{tabular}{p{2cm}}
\toprule
Step 344  \\ \hline
\end{tabular}
 Description \\
{\footnotesize
\textbf{Move TMA to the 3. random distance of 3.5deg\\
}

\begin{itemize}
\tightlist
\item
  Point the TMA to a random 3.5 deg combined offset in AZ and EL from
  {Pointing 14}⁠ at {0}⁠, {75}⁠. Record the exact position of the offset
  in AZ and El
\end{itemize}

}
\hdashrule[0.5ex]{\textwidth}{1pt}{3mm}
  Expected Result \\
{\footnotesize
\begin{itemize}
\tightlist
\item
  The TMA reaches the commanded offset position.
\end{itemize}

}

\begin{tabular}{p{2cm}}
\toprule
Step 345  \\ \hline
\end{tabular}
 Description \\
{\footnotesize
\textbf{Move TMA to the 3. random distance of 3.5deg\\
}

\begin{itemize}
\tightlist
\item
  Point the TMA to a random 3.5 deg combined offset in AZ and EL from
  {Pointing 15}⁠ at {0}⁠, {45}⁠. Record the exact position of the offset
  in AZ and El
\end{itemize}

}
\hdashrule[0.5ex]{\textwidth}{1pt}{3mm}
  Expected Result \\
{\footnotesize
\begin{itemize}
\tightlist
\item
  The TMA reaches the commanded offset position.
\end{itemize}

}

\begin{tabular}{p{2cm}}
\toprule
Step 346  \\ \hline
\end{tabular}
 Description \\
{\footnotesize
\textbf{Move TMA to the 3. random distance of 3.5deg\\
}

\begin{itemize}
\tightlist
\item
  Point the TMA to a random 3.5 deg combined offset in AZ and EL from
  {Pointing 16}⁠ at {0}⁠, {15}⁠. Record the exact position of the offset
  in AZ and El
\end{itemize}

}
\hdashrule[0.5ex]{\textwidth}{1pt}{3mm}
  Expected Result \\
{\footnotesize
\begin{itemize}
\tightlist
\item
  The TMA reaches the commanded offset position.
\end{itemize}

}

\begin{tabular}{p{2cm}}
\toprule
Step 347  \\ \hline
\end{tabular}
 Description \\
{\footnotesize
\textbf{Move TMA to the 3. random distance of 3.5deg\\
}

\begin{itemize}
\tightlist
\item
  Point the TMA to a random 3.5 deg combined offset in AZ and EL from
  {Pointing 17}⁠ at {90}⁠, {15}⁠. Record the exact position of the
  offset in AZ and El
\end{itemize}

}
\hdashrule[0.5ex]{\textwidth}{1pt}{3mm}
  Expected Result \\
{\footnotesize
\begin{itemize}
\tightlist
\item
  The TMA reaches the commanded offset position.
\end{itemize}

}

\begin{tabular}{p{2cm}}
\toprule
Step 348  \\ \hline
\end{tabular}
 Description \\
{\footnotesize
\textbf{Move TMA to the 3. random distance of 3.5deg\\
}

\begin{itemize}
\tightlist
\item
  Point the TMA to a random 3.5 deg combined offset in AZ and EL from
  {Pointing 18}⁠ at {90}⁠, {45}⁠. Record the exact position of the
  offset in AZ and El
\end{itemize}

}
\hdashrule[0.5ex]{\textwidth}{1pt}{3mm}
  Expected Result \\
{\footnotesize
\begin{itemize}
\tightlist
\item
  The TMA reaches the commanded offset position.
\end{itemize}

}

\begin{tabular}{p{2cm}}
\toprule
Step 349  \\ \hline
\end{tabular}
 Description \\
{\footnotesize
\textbf{Move TMA to the 3. random distance of 3.5deg\\
}

\begin{itemize}
\tightlist
\item
  Point the TMA to a random 3.5 deg combined offset in AZ and EL from
  {Pointing 19}⁠ at {90}⁠, {75}⁠. Record the exact position of the
  offset in AZ and El
\end{itemize}

}
\hdashrule[0.5ex]{\textwidth}{1pt}{3mm}
  Expected Result \\
{\footnotesize
\begin{itemize}
\tightlist
\item
  The TMA reaches the commanded offset position.
\end{itemize}

}

\begin{tabular}{p{2cm}}
\toprule
Step 350  \\ \hline
\end{tabular}
 Description \\
{\footnotesize
\textbf{Move TMA to the 3. random distance of 3.5deg\\
}

\begin{itemize}
\tightlist
\item
  Point the TMA to a random 3.5 deg combined offset in AZ and EL from
  {Pointing 20}⁠ at {90}⁠, {86.5}⁠. Record the exact position of the
  offset in AZ and El
\end{itemize}

}
\hdashrule[0.5ex]{\textwidth}{1pt}{3mm}
  Expected Result \\
{\footnotesize
\begin{itemize}
\tightlist
\item
  The TMA reaches the commanded offset position.
\end{itemize}

}

\begin{tabular}{p{2cm}}
\toprule
Step 351  \\ \hline
\end{tabular}
 Description \\
{\footnotesize
\textbf{Move TMA to the 3. random distance of 3.5deg\\
}

\begin{itemize}
\tightlist
\item
  Point the TMA to a random 3.5 deg combined offset in AZ and EL from
  {Pointing 21}⁠ at {180}⁠, {86.5}⁠. Record the exact position of the
  offset in AZ and El
\end{itemize}

}
\hdashrule[0.5ex]{\textwidth}{1pt}{3mm}
  Expected Result \\
{\footnotesize
\begin{itemize}
\tightlist
\item
  The TMA reaches the commanded offset position.
\end{itemize}

}

\begin{tabular}{p{2cm}}
\toprule
Step 352  \\ \hline
\end{tabular}
 Description \\
{\footnotesize
\textbf{Move TMA to the 3. random distance of 3.5deg\\
}

\begin{itemize}
\tightlist
\item
  Point the TMA to a random 3.5 deg combined offset in AZ and EL from
  {Pointing 22}⁠ at {180}⁠, {75}⁠. Record the exact position of the
  offset in AZ and El
\end{itemize}

}
\hdashrule[0.5ex]{\textwidth}{1pt}{3mm}
  Expected Result \\
{\footnotesize
\begin{itemize}
\tightlist
\item
  The TMA reaches the commanded offset position.
\end{itemize}

}

\begin{tabular}{p{2cm}}
\toprule
Step 353  \\ \hline
\end{tabular}
 Description \\
{\footnotesize
\textbf{Move TMA to the 3. random distance of 3.5deg\\
}

\begin{itemize}
\tightlist
\item
  Point the TMA to a random 3.5 deg combined offset in AZ and EL from
  {Pointing 23}⁠ at {180}⁠, {45}⁠. Record the exact position of the
  offset in AZ and El
\end{itemize}

}
\hdashrule[0.5ex]{\textwidth}{1pt}{3mm}
  Expected Result \\
{\footnotesize
\begin{itemize}
\tightlist
\item
  The TMA reaches the commanded offset position.
\end{itemize}

}

\begin{tabular}{p{2cm}}
\toprule
Step 354  \\ \hline
\end{tabular}
 Description \\
{\footnotesize
\textbf{Move TMA to the 3. random distance of 3.5deg\\
}

\begin{itemize}
\tightlist
\item
  Point the TMA to a random 3.5 deg combined offset in AZ and EL from
  {Pointing 24}⁠ at {180}⁠, {15}⁠. Record the exact position of the
  offset in AZ and El
\end{itemize}

}
\hdashrule[0.5ex]{\textwidth}{1pt}{3mm}
  Expected Result \\
{\footnotesize
\begin{itemize}
\tightlist
\item
  The TMA reaches the commanded offset position.
\end{itemize}

}

\begin{tabular}{p{2cm}}
\toprule
Step 355  \\ \hline
\end{tabular}
 Description \\
{\footnotesize
\textbf{Move TMA to the 3. random distance of 3.5deg\\
}

\begin{itemize}
\tightlist
\item
  Point the TMA to a random 3.5 deg combined offset in AZ and EL from
  {Pointing 25}⁠ at {270}⁠, {15}⁠. Record the exact position of the
  offset in AZ and El
\end{itemize}

}
\hdashrule[0.5ex]{\textwidth}{1pt}{3mm}
  Expected Result \\
{\footnotesize
\begin{itemize}
\tightlist
\item
  The TMA reaches the commanded offset position.
\end{itemize}

}

\begin{tabular}{p{2cm}}
\toprule
Step 356  \\ \hline
\end{tabular}
 Description \\
{\footnotesize
\textbf{Move TMA to the 3. random distance of 3.5deg\\
}

\begin{itemize}
\tightlist
\item
  Point the TMA to a random 3.5 deg combined offset in AZ and EL from
  {Pointing 26}⁠ at {270}⁠, {45}⁠. Record the exact position of the
  offset in AZ and El
\end{itemize}

}
\hdashrule[0.5ex]{\textwidth}{1pt}{3mm}
  Expected Result \\
{\footnotesize
\begin{itemize}
\tightlist
\item
  The TMA reaches the commanded offset position.
\end{itemize}

}

\begin{tabular}{p{2cm}}
\toprule
Step 357  \\ \hline
\end{tabular}
 Description \\
{\footnotesize
\textbf{Move TMA to the 3. random distance of 3.5deg\\
}

\begin{itemize}
\tightlist
\item
  Point the TMA to a random 3.5 deg combined offset in AZ and EL from
  {Pointing 27}⁠ at {270}⁠, {75}⁠. Record the exact position of the
  offset in AZ and El
\end{itemize}

}
\hdashrule[0.5ex]{\textwidth}{1pt}{3mm}
  Expected Result \\
{\footnotesize
\begin{itemize}
\tightlist
\item
  The TMA reaches the commanded offset position.
\end{itemize}

}

\begin{tabular}{p{2cm}}
\toprule
Step 358  \\ \hline
\end{tabular}
 Description \\
{\footnotesize
\textbf{Move TMA to the 3. random distance of 3.5deg\\
}

\begin{itemize}
\tightlist
\item
  Point the TMA to a random 3.5 deg combined offset in AZ and EL from
  {Pointing 28}⁠ at {270}⁠, {86.5}⁠. Record the exact position of the
  offset in AZ and El
\end{itemize}

}
\hdashrule[0.5ex]{\textwidth}{1pt}{3mm}
  Expected Result \\
{\footnotesize
\begin{itemize}
\tightlist
\item
  The TMA reaches the commanded offset position.
\end{itemize}

}

\begin{tabular}{p{2cm}}
\toprule
Step 359  \\ \hline
\end{tabular}
 Description \\
{\footnotesize
\textbf{Find DIMM Object and DIMM Offset}\\

\begin{itemize}
\tightlist
\item
  While tracking, take a 10-sec exposure with the StarTracker.
\item
  Load the image into an image viewer.
\item
  Overlay the GAIA catalog.
\item
  Select a star brighter than XXX mag (bright enough for the DIMM).
\item
  Calculate the pixel offset between the StarTracker and the DIMM.
\item
  Transform the offset into AZ and EL offsets.
\end{itemize}

}
\hdashrule[0.5ex]{\textwidth}{1pt}{3mm}
  Expected Result \\
{\footnotesize
\begin{itemize}
\tightlist
\item
  An image was successfully taken with the StarTracker and is of
  sufficient quality.
\item
  AZ and EL offsets are available.
\end{itemize}

}

\begin{tabular}{p{2cm}}
\toprule
Step 360  \\ \hline
\end{tabular}
 Description \\
{\footnotesize
\textbf{Find DIMM Object and DIMM Offset}\\

\begin{itemize}
\tightlist
\item
  While tracking, take a 10-sec exposure with the StarTracker.
\item
  Load the image into an image viewer.
\item
  Overlay the GAIA catalog.
\item
  Select a star brighter than XXX mag (bright enough for the DIMM).
\item
  Calculate the pixel offset between the StarTracker and the DIMM.
\item
  Transform the offset into AZ and EL offsets.
\end{itemize}

}
\hdashrule[0.5ex]{\textwidth}{1pt}{3mm}
  Expected Result \\
{\footnotesize
\begin{itemize}
\tightlist
\item
  An image was successfully taken with the StarTracker and is of
  sufficient quality.
\item
  AZ and EL offsets are available.
\end{itemize}

}

\begin{tabular}{p{2cm}}
\toprule
Step 361  \\ \hline
\end{tabular}
 Description \\
{\footnotesize
\textbf{Find DIMM Object and DIMM Offset}\\

\begin{itemize}
\tightlist
\item
  While tracking, take a 10-sec exposure with the StarTracker.
\item
  Load the image into an image viewer.
\item
  Overlay the GAIA catalog.
\item
  Select a star brighter than XXX mag (bright enough for the DIMM).
\item
  Calculate the pixel offset between the StarTracker and the DIMM.
\item
  Transform the offset into AZ and EL offsets.
\end{itemize}

}
\hdashrule[0.5ex]{\textwidth}{1pt}{3mm}
  Expected Result \\
{\footnotesize
\begin{itemize}
\tightlist
\item
  An image was successfully taken with the StarTracker and is of
  sufficient quality.
\item
  AZ and EL offsets are available.
\end{itemize}

}

\begin{tabular}{p{2cm}}
\toprule
Step 362  \\ \hline
\end{tabular}
 Description \\
{\footnotesize
\textbf{Find DIMM Object and DIMM Offset}\\

\begin{itemize}
\tightlist
\item
  While tracking, take a 10-sec exposure with the StarTracker.
\item
  Load the image into an image viewer.
\item
  Overlay the GAIA catalog.
\item
  Select a star brighter than XXX mag (bright enough for the DIMM).
\item
  Calculate the pixel offset between the StarTracker and the DIMM.
\item
  Transform the offset into AZ and EL offsets.
\end{itemize}

}
\hdashrule[0.5ex]{\textwidth}{1pt}{3mm}
  Expected Result \\
{\footnotesize
\begin{itemize}
\tightlist
\item
  An image was successfully taken with the StarTracker and is of
  sufficient quality.
\item
  AZ and EL offsets are available.
\end{itemize}

}

\begin{tabular}{p{2cm}}
\toprule
Step 363  \\ \hline
\end{tabular}
 Description \\
{\footnotesize
\textbf{Find DIMM Object and DIMM Offset}\\

\begin{itemize}
\tightlist
\item
  While tracking, take a 10-sec exposure with the StarTracker.
\item
  Load the image into an image viewer.
\item
  Overlay the GAIA catalog.
\item
  Select a star brighter than XXX mag (bright enough for the DIMM).
\item
  Calculate the pixel offset between the StarTracker and the DIMM.
\item
  Transform the offset into AZ and EL offsets.
\end{itemize}

}
\hdashrule[0.5ex]{\textwidth}{1pt}{3mm}
  Expected Result \\
{\footnotesize
\begin{itemize}
\tightlist
\item
  An image was successfully taken with the StarTracker and is of
  sufficient quality.
\item
  AZ and EL offsets are available.
\end{itemize}

}

\begin{tabular}{p{2cm}}
\toprule
Step 364  \\ \hline
\end{tabular}
 Description \\
{\footnotesize
\textbf{Find DIMM Object and DIMM Offset}\\

\begin{itemize}
\tightlist
\item
  While tracking, take a 10-sec exposure with the StarTracker.
\item
  Load the image into an image viewer.
\item
  Overlay the GAIA catalog.
\item
  Select a star brighter than XXX mag (bright enough for the DIMM).
\item
  Calculate the pixel offset between the StarTracker and the DIMM.
\item
  Transform the offset into AZ and EL offsets.
\end{itemize}

}
\hdashrule[0.5ex]{\textwidth}{1pt}{3mm}
  Expected Result \\
{\footnotesize
\begin{itemize}
\tightlist
\item
  An image was successfully taken with the StarTracker and is of
  sufficient quality.
\item
  AZ and EL offsets are available.
\end{itemize}

}

\begin{tabular}{p{2cm}}
\toprule
Step 365  \\ \hline
\end{tabular}
 Description \\
{\footnotesize
\textbf{Find DIMM Object and DIMM Offset}\\

\begin{itemize}
\tightlist
\item
  While tracking, take a 10-sec exposure with the StarTracker.
\item
  Load the image into an image viewer.
\item
  Overlay the GAIA catalog.
\item
  Select a star brighter than XXX mag (bright enough for the DIMM).
\item
  Calculate the pixel offset between the StarTracker and the DIMM.
\item
  Transform the offset into AZ and EL offsets.
\end{itemize}

}
\hdashrule[0.5ex]{\textwidth}{1pt}{3mm}
  Expected Result \\
{\footnotesize
\begin{itemize}
\tightlist
\item
  An image was successfully taken with the StarTracker and is of
  sufficient quality.
\item
  AZ and EL offsets are available.
\end{itemize}

}

\begin{tabular}{p{2cm}}
\toprule
Step 366  \\ \hline
\end{tabular}
 Description \\
{\footnotesize
\textbf{Find DIMM Object and DIMM Offset}\\

\begin{itemize}
\tightlist
\item
  While tracking, take a 10-sec exposure with the StarTracker.
\item
  Load the image into an image viewer.
\item
  Overlay the GAIA catalog.
\item
  Select a star brighter than XXX mag (bright enough for the DIMM).
\item
  Calculate the pixel offset between the StarTracker and the DIMM.
\item
  Transform the offset into AZ and EL offsets.
\end{itemize}

}
\hdashrule[0.5ex]{\textwidth}{1pt}{3mm}
  Expected Result \\
{\footnotesize
\begin{itemize}
\tightlist
\item
  An image was successfully taken with the StarTracker and is of
  sufficient quality.
\item
  AZ and EL offsets are available.
\end{itemize}

}

\begin{tabular}{p{2cm}}
\toprule
Step 367  \\ \hline
\end{tabular}
 Description \\
{\footnotesize
\textbf{Find DIMM Object and DIMM Offset}\\

\begin{itemize}
\tightlist
\item
  While tracking, take a 10-sec exposure with the StarTracker.
\item
  Load the image into an image viewer.
\item
  Overlay the GAIA catalog.
\item
  Select a star brighter than XXX mag (bright enough for the DIMM).
\item
  Calculate the pixel offset between the StarTracker and the DIMM.
\item
  Transform the offset into AZ and EL offsets.
\end{itemize}

}
\hdashrule[0.5ex]{\textwidth}{1pt}{3mm}
  Expected Result \\
{\footnotesize
\begin{itemize}
\tightlist
\item
  An image was successfully taken with the StarTracker and is of
  sufficient quality.
\item
  AZ and EL offsets are available.
\end{itemize}

}

\begin{tabular}{p{2cm}}
\toprule
Step 368  \\ \hline
\end{tabular}
 Description \\
{\footnotesize
\textbf{Find DIMM Object and DIMM Offset}\\

\begin{itemize}
\tightlist
\item
  While tracking, take a 10-sec exposure with the StarTracker.
\item
  Load the image into an image viewer.
\item
  Overlay the GAIA catalog.
\item
  Select a star brighter than XXX mag (bright enough for the DIMM).
\item
  Calculate the pixel offset between the StarTracker and the DIMM.
\item
  Transform the offset into AZ and EL offsets.
\end{itemize}

}
\hdashrule[0.5ex]{\textwidth}{1pt}{3mm}
  Expected Result \\
{\footnotesize
\begin{itemize}
\tightlist
\item
  An image was successfully taken with the StarTracker and is of
  sufficient quality.
\item
  AZ and EL offsets are available.
\end{itemize}

}

\begin{tabular}{p{2cm}}
\toprule
Step 369  \\ \hline
\end{tabular}
 Description \\
{\footnotesize
\textbf{Find DIMM Object and DIMM Offset}\\

\begin{itemize}
\tightlist
\item
  While tracking, take a 10-sec exposure with the StarTracker.
\item
  Load the image into an image viewer.
\item
  Overlay the GAIA catalog.
\item
  Select a star brighter than XXX mag (bright enough for the DIMM).
\item
  Calculate the pixel offset between the StarTracker and the DIMM.
\item
  Transform the offset into AZ and EL offsets.
\end{itemize}

}
\hdashrule[0.5ex]{\textwidth}{1pt}{3mm}
  Expected Result \\
{\footnotesize
\begin{itemize}
\tightlist
\item
  An image was successfully taken with the StarTracker and is of
  sufficient quality.
\item
  AZ and EL offsets are available.
\end{itemize}

}

\begin{tabular}{p{2cm}}
\toprule
Step 370  \\ \hline
\end{tabular}
 Description \\
{\footnotesize
\textbf{Find DIMM Object and DIMM Offset}\\

\begin{itemize}
\tightlist
\item
  While tracking, take a 10-sec exposure with the StarTracker.
\item
  Load the image into an image viewer.
\item
  Overlay the GAIA catalog.
\item
  Select a star brighter than XXX mag (bright enough for the DIMM).
\item
  Calculate the pixel offset between the StarTracker and the DIMM.
\item
  Transform the offset into AZ and EL offsets.
\end{itemize}

}
\hdashrule[0.5ex]{\textwidth}{1pt}{3mm}
  Expected Result \\
{\footnotesize
\begin{itemize}
\tightlist
\item
  An image was successfully taken with the StarTracker and is of
  sufficient quality.
\item
  AZ and EL offsets are available.
\end{itemize}

}

\begin{tabular}{p{2cm}}
\toprule
Step 371  \\ \hline
\end{tabular}
 Description \\
{\footnotesize
\textbf{Find DIMM Object and DIMM Offset}\\

\begin{itemize}
\tightlist
\item
  While tracking, take a 10-sec exposure with the StarTracker.
\item
  Load the image into an image viewer.
\item
  Overlay the GAIA catalog.
\item
  Select a star brighter than XXX mag (bright enough for the DIMM).
\item
  Calculate the pixel offset between the StarTracker and the DIMM.
\item
  Transform the offset into AZ and EL offsets.
\end{itemize}

}
\hdashrule[0.5ex]{\textwidth}{1pt}{3mm}
  Expected Result \\
{\footnotesize
\begin{itemize}
\tightlist
\item
  An image was successfully taken with the StarTracker and is of
  sufficient quality.
\item
  AZ and EL offsets are available.
\end{itemize}

}

\begin{tabular}{p{2cm}}
\toprule
Step 372  \\ \hline
\end{tabular}
 Description \\
{\footnotesize
\textbf{Find DIMM Object and DIMM Offset}\\

\begin{itemize}
\tightlist
\item
  While tracking, take a 10-sec exposure with the StarTracker.
\item
  Load the image into an image viewer.
\item
  Overlay the GAIA catalog.
\item
  Select a star brighter than XXX mag (bright enough for the DIMM).
\item
  Calculate the pixel offset between the StarTracker and the DIMM.
\item
  Transform the offset into AZ and EL offsets.
\end{itemize}

}
\hdashrule[0.5ex]{\textwidth}{1pt}{3mm}
  Expected Result \\
{\footnotesize
\begin{itemize}
\tightlist
\item
  An image was successfully taken with the StarTracker and is of
  sufficient quality.
\item
  AZ and EL offsets are available.
\end{itemize}

}

\begin{tabular}{p{2cm}}
\toprule
Step 373  \\ \hline
\end{tabular}
 Description \\
{\footnotesize
\textbf{Find DIMM Object and DIMM Offset}\\

\begin{itemize}
\tightlist
\item
  While tracking, take a 10-sec exposure with the StarTracker.
\item
  Load the image into an image viewer.
\item
  Overlay the GAIA catalog.
\item
  Select a star brighter than XXX mag (bright enough for the DIMM).
\item
  Calculate the pixel offset between the StarTracker and the DIMM.
\item
  Transform the offset into AZ and EL offsets.
\end{itemize}

}
\hdashrule[0.5ex]{\textwidth}{1pt}{3mm}
  Expected Result \\
{\footnotesize
\begin{itemize}
\tightlist
\item
  An image was successfully taken with the StarTracker and is of
  sufficient quality.
\item
  AZ and EL offsets are available.
\end{itemize}

}

\begin{tabular}{p{2cm}}
\toprule
Step 374  \\ \hline
\end{tabular}
 Description \\
{\footnotesize
\textbf{Find DIMM Object and DIMM Offset}\\

\begin{itemize}
\tightlist
\item
  While tracking, take a 10-sec exposure with the StarTracker.
\item
  Load the image into an image viewer.
\item
  Overlay the GAIA catalog.
\item
  Select a star brighter than XXX mag (bright enough for the DIMM).
\item
  Calculate the pixel offset between the StarTracker and the DIMM.
\item
  Transform the offset into AZ and EL offsets.
\end{itemize}

}
\hdashrule[0.5ex]{\textwidth}{1pt}{3mm}
  Expected Result \\
{\footnotesize
\begin{itemize}
\tightlist
\item
  An image was successfully taken with the StarTracker and is of
  sufficient quality.
\item
  AZ and EL offsets are available.
\end{itemize}

}

\begin{tabular}{p{2cm}}
\toprule
Step 375  \\ \hline
\end{tabular}
 Description \\
{\footnotesize
\textbf{Find DIMM Object and DIMM Offset}\\

\begin{itemize}
\tightlist
\item
  While tracking, take a 10-sec exposure with the StarTracker.
\item
  Load the image into an image viewer.
\item
  Overlay the GAIA catalog.
\item
  Select a star brighter than XXX mag (bright enough for the DIMM).
\item
  Calculate the pixel offset between the StarTracker and the DIMM.
\item
  Transform the offset into AZ and EL offsets.
\end{itemize}

}
\hdashrule[0.5ex]{\textwidth}{1pt}{3mm}
  Expected Result \\
{\footnotesize
\begin{itemize}
\tightlist
\item
  An image was successfully taken with the StarTracker and is of
  sufficient quality.
\item
  AZ and EL offsets are available.
\end{itemize}

}

\begin{tabular}{p{2cm}}
\toprule
Step 376  \\ \hline
\end{tabular}
 Description \\
{\footnotesize
\textbf{Find DIMM Object and DIMM Offset}\\

\begin{itemize}
\tightlist
\item
  While tracking, take a 10-sec exposure with the StarTracker.
\item
  Load the image into an image viewer.
\item
  Overlay the GAIA catalog.
\item
  Select a star brighter than XXX mag (bright enough for the DIMM).
\item
  Calculate the pixel offset between the StarTracker and the DIMM.
\item
  Transform the offset into AZ and EL offsets.
\end{itemize}

}
\hdashrule[0.5ex]{\textwidth}{1pt}{3mm}
  Expected Result \\
{\footnotesize
\begin{itemize}
\tightlist
\item
  An image was successfully taken with the StarTracker and is of
  sufficient quality.
\item
  AZ and EL offsets are available.
\end{itemize}

}

\begin{tabular}{p{2cm}}
\toprule
Step 377  \\ \hline
\end{tabular}
 Description \\
{\footnotesize
\textbf{Find DIMM Object and DIMM Offset}\\

\begin{itemize}
\tightlist
\item
  While tracking, take a 10-sec exposure with the StarTracker.
\item
  Load the image into an image viewer.
\item
  Overlay the GAIA catalog.
\item
  Select a star brighter than XXX mag (bright enough for the DIMM).
\item
  Calculate the pixel offset between the StarTracker and the DIMM.
\item
  Transform the offset into AZ and EL offsets.
\end{itemize}

}
\hdashrule[0.5ex]{\textwidth}{1pt}{3mm}
  Expected Result \\
{\footnotesize
\begin{itemize}
\tightlist
\item
  An image was successfully taken with the StarTracker and is of
  sufficient quality.
\item
  AZ and EL offsets are available.
\end{itemize}

}

\begin{tabular}{p{2cm}}
\toprule
Step 378  \\ \hline
\end{tabular}
 Description \\
{\footnotesize
\textbf{Find DIMM Object and DIMM Offset}\\

\begin{itemize}
\tightlist
\item
  While tracking, take a 10-sec exposure with the StarTracker.
\item
  Load the image into an image viewer.
\item
  Overlay the GAIA catalog.
\item
  Select a star brighter than XXX mag (bright enough for the DIMM).
\item
  Calculate the pixel offset between the StarTracker and the DIMM.
\item
  Transform the offset into AZ and EL offsets.
\end{itemize}

}
\hdashrule[0.5ex]{\textwidth}{1pt}{3mm}
  Expected Result \\
{\footnotesize
\begin{itemize}
\tightlist
\item
  An image was successfully taken with the StarTracker and is of
  sufficient quality.
\item
  AZ and EL offsets are available.
\end{itemize}

}

\begin{tabular}{p{2cm}}
\toprule
Step 379  \\ \hline
\end{tabular}
 Description \\
{\footnotesize
\textbf{Find DIMM Object and DIMM Offset}\\

\begin{itemize}
\tightlist
\item
  While tracking, take a 10-sec exposure with the StarTracker.
\item
  Load the image into an image viewer.
\item
  Overlay the GAIA catalog.
\item
  Select a star brighter than XXX mag (bright enough for the DIMM).
\item
  Calculate the pixel offset between the StarTracker and the DIMM.
\item
  Transform the offset into AZ and EL offsets.
\end{itemize}

}
\hdashrule[0.5ex]{\textwidth}{1pt}{3mm}
  Expected Result \\
{\footnotesize
\begin{itemize}
\tightlist
\item
  An image was successfully taken with the StarTracker and is of
  sufficient quality.
\item
  AZ and EL offsets are available.
\end{itemize}

}

\begin{tabular}{p{2cm}}
\toprule
Step 380  \\ \hline
\end{tabular}
 Description \\
{\footnotesize
\textbf{Find DIMM Object and DIMM Offset}\\

\begin{itemize}
\tightlist
\item
  While tracking, take a 10-sec exposure with the StarTracker.
\item
  Load the image into an image viewer.
\item
  Overlay the GAIA catalog.
\item
  Select a star brighter than XXX mag (bright enough for the DIMM).
\item
  Calculate the pixel offset between the StarTracker and the DIMM.
\item
  Transform the offset into AZ and EL offsets.
\end{itemize}

}
\hdashrule[0.5ex]{\textwidth}{1pt}{3mm}
  Expected Result \\
{\footnotesize
\begin{itemize}
\tightlist
\item
  An image was successfully taken with the StarTracker and is of
  sufficient quality.
\item
  AZ and EL offsets are available.
\end{itemize}

}

\begin{tabular}{p{2cm}}
\toprule
Step 381  \\ \hline
\end{tabular}
 Description \\
{\footnotesize
\textbf{Find DIMM Object and DIMM Offset}\\

\begin{itemize}
\tightlist
\item
  While tracking, take a 10-sec exposure with the StarTracker.
\item
  Load the image into an image viewer.
\item
  Overlay the GAIA catalog.
\item
  Select a star brighter than XXX mag (bright enough for the DIMM).
\item
  Calculate the pixel offset between the StarTracker and the DIMM.
\item
  Transform the offset into AZ and EL offsets.
\end{itemize}

}
\hdashrule[0.5ex]{\textwidth}{1pt}{3mm}
  Expected Result \\
{\footnotesize
\begin{itemize}
\tightlist
\item
  An image was successfully taken with the StarTracker and is of
  sufficient quality.
\item
  AZ and EL offsets are available.
\end{itemize}

}

\begin{tabular}{p{2cm}}
\toprule
Step 382  \\ \hline
\end{tabular}
 Description \\
{\footnotesize
\textbf{Find DIMM Object and DIMM Offset}\\

\begin{itemize}
\tightlist
\item
  While tracking, take a 10-sec exposure with the StarTracker.
\item
  Load the image into an image viewer.
\item
  Overlay the GAIA catalog.
\item
  Select a star brighter than XXX mag (bright enough for the DIMM).
\item
  Calculate the pixel offset between the StarTracker and the DIMM.
\item
  Transform the offset into AZ and EL offsets.
\end{itemize}

}
\hdashrule[0.5ex]{\textwidth}{1pt}{3mm}
  Expected Result \\
{\footnotesize
\begin{itemize}
\tightlist
\item
  An image was successfully taken with the StarTracker and is of
  sufficient quality.
\item
  AZ and EL offsets are available.
\end{itemize}

}

\begin{tabular}{p{2cm}}
\toprule
Step 383  \\ \hline
\end{tabular}
 Description \\
{\footnotesize
\textbf{Find DIMM Object and DIMM Offset}\\

\begin{itemize}
\tightlist
\item
  While tracking, take a 10-sec exposure with the StarTracker.
\item
  Load the image into an image viewer.
\item
  Overlay the GAIA catalog.
\item
  Select a star brighter than XXX mag (bright enough for the DIMM).
\item
  Calculate the pixel offset between the StarTracker and the DIMM.
\item
  Transform the offset into AZ and EL offsets.
\end{itemize}

}
\hdashrule[0.5ex]{\textwidth}{1pt}{3mm}
  Expected Result \\
{\footnotesize
\begin{itemize}
\tightlist
\item
  An image was successfully taken with the StarTracker and is of
  sufficient quality.
\item
  AZ and EL offsets are available.
\end{itemize}

}

\begin{tabular}{p{2cm}}
\toprule
Step 384  \\ \hline
\end{tabular}
 Description \\
{\footnotesize
\textbf{Find DIMM Object and DIMM Offset}\\

\begin{itemize}
\tightlist
\item
  While tracking, take a 10-sec exposure with the StarTracker.
\item
  Load the image into an image viewer.
\item
  Overlay the GAIA catalog.
\item
  Select a star brighter than XXX mag (bright enough for the DIMM).
\item
  Calculate the pixel offset between the StarTracker and the DIMM.
\item
  Transform the offset into AZ and EL offsets.
\end{itemize}

}
\hdashrule[0.5ex]{\textwidth}{1pt}{3mm}
  Expected Result \\
{\footnotesize
\begin{itemize}
\tightlist
\item
  An image was successfully taken with the StarTracker and is of
  sufficient quality.
\item
  AZ and EL offsets are available.
\end{itemize}

}

\begin{tabular}{p{2cm}}
\toprule
Step 385  \\ \hline
\end{tabular}
 Description \\
{\footnotesize
\textbf{Find DIMM Object and DIMM Offset}\\

\begin{itemize}
\tightlist
\item
  While tracking, take a 10-sec exposure with the StarTracker.
\item
  Load the image into an image viewer.
\item
  Overlay the GAIA catalog.
\item
  Select a star brighter than XXX mag (bright enough for the DIMM).
\item
  Calculate the pixel offset between the StarTracker and the DIMM.
\item
  Transform the offset into AZ and EL offsets.
\end{itemize}

}
\hdashrule[0.5ex]{\textwidth}{1pt}{3mm}
  Expected Result \\
{\footnotesize
\begin{itemize}
\tightlist
\item
  An image was successfully taken with the StarTracker and is of
  sufficient quality.
\item
  AZ and EL offsets are available.
\end{itemize}

}

\begin{tabular}{p{2cm}}
\toprule
Step 386  \\ \hline
\end{tabular}
 Description \\
{\footnotesize
\textbf{Find DIMM Object and DIMM Offset}\\

\begin{itemize}
\tightlist
\item
  While tracking, take a 10-sec exposure with the StarTracker.
\item
  Load the image into an image viewer.
\item
  Overlay the GAIA catalog.
\item
  Select a star brighter than XXX mag (bright enough for the DIMM).
\item
  Calculate the pixel offset between the StarTracker and the DIMM.
\item
  Transform the offset into AZ and EL offsets.
\end{itemize}

}
\hdashrule[0.5ex]{\textwidth}{1pt}{3mm}
  Expected Result \\
{\footnotesize
\begin{itemize}
\tightlist
\item
  An image was successfully taken with the StarTracker and is of
  sufficient quality.
\item
  AZ and EL offsets are available.
\end{itemize}

}

\begin{tabular}{p{2cm}}
\toprule
Step 387  \\ \hline
\end{tabular}
 Description \\
{\footnotesize
\textbf{Find DIMM Object and DIMM Offset}\\

\begin{itemize}
\tightlist
\item
  While tracking, take a 10-sec exposure with the StarTracker.
\item
  Load the image into an image viewer.
\item
  Overlay the GAIA catalog.
\item
  Select a star brighter than XXX mag (bright enough for the DIMM).
\item
  Calculate the pixel offset between the StarTracker and the DIMM.
\item
  Transform the offset into AZ and EL offsets.
\end{itemize}

}
\hdashrule[0.5ex]{\textwidth}{1pt}{3mm}
  Expected Result \\
{\footnotesize
\begin{itemize}
\tightlist
\item
  An image was successfully taken with the StarTracker and is of
  sufficient quality.
\item
  AZ and EL offsets are available.
\end{itemize}

}

\begin{tabular}{p{2cm}}
\toprule
Step 388  \\ \hline
\end{tabular}
 Description \\
{\footnotesize
\textbf{Find DIMM Object and DIMM Offset}\\

\begin{itemize}
\tightlist
\item
  While tracking, take a 10-sec exposure with the StarTracker.
\item
  Load the image into an image viewer.
\item
  Overlay the GAIA catalog.
\item
  Select a star brighter than XXX mag (bright enough for the DIMM).
\item
  Calculate the pixel offset between the StarTracker and the DIMM.
\item
  Transform the offset into AZ and EL offsets.
\end{itemize}

}
\hdashrule[0.5ex]{\textwidth}{1pt}{3mm}
  Expected Result \\
{\footnotesize
\begin{itemize}
\tightlist
\item
  An image was successfully taken with the StarTracker and is of
  sufficient quality.
\item
  AZ and EL offsets are available.
\end{itemize}

}

\begin{tabular}{p{2cm}}
\toprule
Step 389  \\ \hline
\end{tabular}
 Description \\
{\footnotesize
\textbf{Find DIMM Object and DIMM Offset}\\

\begin{itemize}
\tightlist
\item
  While tracking, take a 10-sec exposure with the StarTracker.
\item
  Load the image into an image viewer.
\item
  Overlay the GAIA catalog.
\item
  Select a star brighter than XXX mag (bright enough for the DIMM).
\item
  Calculate the pixel offset between the StarTracker and the DIMM.
\item
  Transform the offset into AZ and EL offsets.
\end{itemize}

}
\hdashrule[0.5ex]{\textwidth}{1pt}{3mm}
  Expected Result \\
{\footnotesize
\begin{itemize}
\tightlist
\item
  An image was successfully taken with the StarTracker and is of
  sufficient quality.
\item
  AZ and EL offsets are available.
\end{itemize}

}

\begin{tabular}{p{2cm}}
\toprule
Step 390  \\ \hline
\end{tabular}
 Description \\
{\footnotesize
\textbf{Find DIMM Object and DIMM Offset}\\

\begin{itemize}
\tightlist
\item
  While tracking, take a 10-sec exposure with the StarTracker.
\item
  Load the image into an image viewer.
\item
  Overlay the GAIA catalog.
\item
  Select a star brighter than XXX mag (bright enough for the DIMM).
\item
  Calculate the pixel offset between the StarTracker and the DIMM.
\item
  Transform the offset into AZ and EL offsets.
\end{itemize}

}
\hdashrule[0.5ex]{\textwidth}{1pt}{3mm}
  Expected Result \\
{\footnotesize
\begin{itemize}
\tightlist
\item
  An image was successfully taken with the StarTracker and is of
  sufficient quality.
\item
  AZ and EL offsets are available.
\end{itemize}

}

\begin{tabular}{p{2cm}}
\toprule
Step 391  \\ \hline
\end{tabular}
 Description \\
{\footnotesize
\textbf{Find DIMM Object and DIMM Offset}\\

\begin{itemize}
\tightlist
\item
  While tracking, take a 10-sec exposure with the StarTracker.
\item
  Load the image into an image viewer.
\item
  Overlay the GAIA catalog.
\item
  Select a star brighter than XXX mag (bright enough for the DIMM).
\item
  Calculate the pixel offset between the StarTracker and the DIMM.
\item
  Transform the offset into AZ and EL offsets.
\end{itemize}

}
\hdashrule[0.5ex]{\textwidth}{1pt}{3mm}
  Expected Result \\
{\footnotesize
\begin{itemize}
\tightlist
\item
  An image was successfully taken with the StarTracker and is of
  sufficient quality.
\item
  AZ and EL offsets are available.
\end{itemize}

}

\begin{tabular}{p{2cm}}
\toprule
Step 392  \\ \hline
\end{tabular}
 Description \\
{\footnotesize
\textbf{Find DIMM Object and DIMM Offset}\\

\begin{itemize}
\tightlist
\item
  While tracking, take a 10-sec exposure with the StarTracker.
\item
  Load the image into an image viewer.
\item
  Overlay the GAIA catalog.
\item
  Select a star brighter than XXX mag (bright enough for the DIMM).
\item
  Calculate the pixel offset between the StarTracker and the DIMM.
\item
  Transform the offset into AZ and EL offsets.
\end{itemize}

}
\hdashrule[0.5ex]{\textwidth}{1pt}{3mm}
  Expected Result \\
{\footnotesize
\begin{itemize}
\tightlist
\item
  An image was successfully taken with the StarTracker and is of
  sufficient quality.
\item
  AZ and EL offsets are available.
\end{itemize}

}

\begin{tabular}{p{2cm}}
\toprule
Step 393  \\ \hline
\end{tabular}
 Description \\
{\footnotesize
\textbf{Find DIMM Object and DIMM Offset}\\

\begin{itemize}
\tightlist
\item
  While tracking, take a 10-sec exposure with the StarTracker.
\item
  Load the image into an image viewer.
\item
  Overlay the GAIA catalog.
\item
  Select a star brighter than XXX mag (bright enough for the DIMM).
\item
  Calculate the pixel offset between the StarTracker and the DIMM.
\item
  Transform the offset into AZ and EL offsets.
\end{itemize}

}
\hdashrule[0.5ex]{\textwidth}{1pt}{3mm}
  Expected Result \\
{\footnotesize
\begin{itemize}
\tightlist
\item
  An image was successfully taken with the StarTracker and is of
  sufficient quality.
\item
  AZ and EL offsets are available.
\end{itemize}

}

\begin{tabular}{p{2cm}}
\toprule
Step 394  \\ \hline
\end{tabular}
 Description \\
{\footnotesize
\textbf{Find DIMM Object and DIMM Offset}\\

\begin{itemize}
\tightlist
\item
  While tracking, take a 10-sec exposure with the StarTracker.
\item
  Load the image into an image viewer.
\item
  Overlay the GAIA catalog.
\item
  Select a star brighter than XXX mag (bright enough for the DIMM).
\item
  Calculate the pixel offset between the StarTracker and the DIMM.
\item
  Transform the offset into AZ and EL offsets.
\end{itemize}

}
\hdashrule[0.5ex]{\textwidth}{1pt}{3mm}
  Expected Result \\
{\footnotesize
\begin{itemize}
\tightlist
\item
  An image was successfully taken with the StarTracker and is of
  sufficient quality.
\item
  AZ and EL offsets are available.
\end{itemize}

}

\begin{tabular}{p{2cm}}
\toprule
Step 395  \\ \hline
\end{tabular}
 Description \\
{\footnotesize
\textbf{Find DIMM Object and DIMM Offset}\\

\begin{itemize}
\tightlist
\item
  While tracking, take a 10-sec exposure with the StarTracker.
\item
  Load the image into an image viewer.
\item
  Overlay the GAIA catalog.
\item
  Select a star brighter than XXX mag (bright enough for the DIMM).
\item
  Calculate the pixel offset between the StarTracker and the DIMM.
\item
  Transform the offset into AZ and EL offsets.
\end{itemize}

}
\hdashrule[0.5ex]{\textwidth}{1pt}{3mm}
  Expected Result \\
{\footnotesize
\begin{itemize}
\tightlist
\item
  An image was successfully taken with the StarTracker and is of
  sufficient quality.
\item
  AZ and EL offsets are available.
\end{itemize}

}

\begin{tabular}{p{2cm}}
\toprule
Step 396  \\ \hline
\end{tabular}
 Description \\
{\footnotesize
\textbf{Find DIMM Object and DIMM Offset}\\

\begin{itemize}
\tightlist
\item
  While tracking, take a 10-sec exposure with the StarTracker.
\item
  Load the image into an image viewer.
\item
  Overlay the GAIA catalog.
\item
  Select a star brighter than XXX mag (bright enough for the DIMM).
\item
  Calculate the pixel offset between the StarTracker and the DIMM.
\item
  Transform the offset into AZ and EL offsets.
\end{itemize}

}
\hdashrule[0.5ex]{\textwidth}{1pt}{3mm}
  Expected Result \\
{\footnotesize
\begin{itemize}
\tightlist
\item
  An image was successfully taken with the StarTracker and is of
  sufficient quality.
\item
  AZ and EL offsets are available.
\end{itemize}

}

\begin{tabular}{p{2cm}}
\toprule
Step 397  \\ \hline
\end{tabular}
 Description \\
{\footnotesize
\textbf{Find DIMM Object and DIMM Offset}\\

\begin{itemize}
\tightlist
\item
  While tracking, take a 10-sec exposure with the StarTracker.
\item
  Load the image into an image viewer.
\item
  Overlay the GAIA catalog.
\item
  Select a star brighter than XXX mag (bright enough for the DIMM).
\item
  Calculate the pixel offset between the StarTracker and the DIMM.
\item
  Transform the offset into AZ and EL offsets.
\end{itemize}

}
\hdashrule[0.5ex]{\textwidth}{1pt}{3mm}
  Expected Result \\
{\footnotesize
\begin{itemize}
\tightlist
\item
  An image was successfully taken with the StarTracker and is of
  sufficient quality.
\item
  AZ and EL offsets are available.
\end{itemize}

}

\begin{tabular}{p{2cm}}
\toprule
Step 398  \\ \hline
\end{tabular}
 Description \\
{\footnotesize
\textbf{Find DIMM Object and DIMM Offset}\\

\begin{itemize}
\tightlist
\item
  While tracking, take a 10-sec exposure with the StarTracker.
\item
  Load the image into an image viewer.
\item
  Overlay the GAIA catalog.
\item
  Select a star brighter than XXX mag (bright enough for the DIMM).
\item
  Calculate the pixel offset between the StarTracker and the DIMM.
\item
  Transform the offset into AZ and EL offsets.
\end{itemize}

}
\hdashrule[0.5ex]{\textwidth}{1pt}{3mm}
  Expected Result \\
{\footnotesize
\begin{itemize}
\tightlist
\item
  An image was successfully taken with the StarTracker and is of
  sufficient quality.
\item
  AZ and EL offsets are available.
\end{itemize}

}

\begin{tabular}{p{2cm}}
\toprule
Step 399  \\ \hline
\end{tabular}
 Description \\
{\footnotesize
\textbf{Find DIMM Object and DIMM Offset}\\

\begin{itemize}
\tightlist
\item
  While tracking, take a 10-sec exposure with the StarTracker.
\item
  Load the image into an image viewer.
\item
  Overlay the GAIA catalog.
\item
  Select a star brighter than XXX mag (bright enough for the DIMM).
\item
  Calculate the pixel offset between the StarTracker and the DIMM.
\item
  Transform the offset into AZ and EL offsets.
\end{itemize}

}
\hdashrule[0.5ex]{\textwidth}{1pt}{3mm}
  Expected Result \\
{\footnotesize
\begin{itemize}
\tightlist
\item
  An image was successfully taken with the StarTracker and is of
  sufficient quality.
\item
  AZ and EL offsets are available.
\end{itemize}

}

\begin{tabular}{p{2cm}}
\toprule
Step 400  \\ \hline
\end{tabular}
 Description \\
{\footnotesize
\textbf{Find DIMM Object and DIMM Offset}\\

\begin{itemize}
\tightlist
\item
  While tracking, take a 10-sec exposure with the StarTracker.
\item
  Load the image into an image viewer.
\item
  Overlay the GAIA catalog.
\item
  Select a star brighter than XXX mag (bright enough for the DIMM).
\item
  Calculate the pixel offset between the StarTracker and the DIMM.
\item
  Transform the offset into AZ and EL offsets.
\end{itemize}

}
\hdashrule[0.5ex]{\textwidth}{1pt}{3mm}
  Expected Result \\
{\footnotesize
\begin{itemize}
\tightlist
\item
  An image was successfully taken with the StarTracker and is of
  sufficient quality.
\item
  AZ and EL offsets are available.
\end{itemize}

}

\begin{tabular}{p{2cm}}
\toprule
Step 401  \\ \hline
\end{tabular}
 Description \\
{\footnotesize
\textbf{Find DIMM Object and DIMM Offset}\\

\begin{itemize}
\tightlist
\item
  While tracking, take a 10-sec exposure with the StarTracker.
\item
  Load the image into an image viewer.
\item
  Overlay the GAIA catalog.
\item
  Select a star brighter than XXX mag (bright enough for the DIMM).
\item
  Calculate the pixel offset between the StarTracker and the DIMM.
\item
  Transform the offset into AZ and EL offsets.
\end{itemize}

}
\hdashrule[0.5ex]{\textwidth}{1pt}{3mm}
  Expected Result \\
{\footnotesize
\begin{itemize}
\tightlist
\item
  An image was successfully taken with the StarTracker and is of
  sufficient quality.
\item
  AZ and EL offsets are available.
\end{itemize}

}

\begin{tabular}{p{2cm}}
\toprule
Step 402  \\ \hline
\end{tabular}
 Description \\
{\footnotesize
\textbf{Find DIMM Object and DIMM Offset}\\

\begin{itemize}
\tightlist
\item
  While tracking, take a 10-sec exposure with the StarTracker.
\item
  Load the image into an image viewer.
\item
  Overlay the GAIA catalog.
\item
  Select a star brighter than XXX mag (bright enough for the DIMM).
\item
  Calculate the pixel offset between the StarTracker and the DIMM.
\item
  Transform the offset into AZ and EL offsets.
\end{itemize}

}
\hdashrule[0.5ex]{\textwidth}{1pt}{3mm}
  Expected Result \\
{\footnotesize
\begin{itemize}
\tightlist
\item
  An image was successfully taken with the StarTracker and is of
  sufficient quality.
\item
  AZ and EL offsets are available.
\end{itemize}

}

\begin{tabular}{p{2cm}}
\toprule
Step 403  \\ \hline
\end{tabular}
 Description \\
{\footnotesize
\textbf{Move TMA to the DIMM position and \textbf{Take DIMM images}}\\

\begin{itemize}
\tightlist
\item
  Command the TMA to the DIMM position by applying the offsets
\item
  While tracking, take DIMM images with XXXs exposure time and inspect
  the quality.
\end{itemize}

}
\hdashrule[0.5ex]{\textwidth}{1pt}{3mm}
  Expected Result \\
{\footnotesize
\begin{itemize}
\tightlist
\item
  TMA reaches the DIMM position.
\item
  DIMM imaging quality is sufficient.
\end{itemize}

}

\begin{tabular}{p{2cm}}
\toprule
Step 404  \\ \hline
\end{tabular}
 Description \\
{\footnotesize
\textbf{Move TMA to the DIMM position and \textbf{Take DIMM images}}\\

\begin{itemize}
\tightlist
\item
  Command the TMA to the DIMM position by applying the offsets
\item
  While tracking, take DIMM images with XXXs exposure time and inspect
  the quality.
\end{itemize}

}
\hdashrule[0.5ex]{\textwidth}{1pt}{3mm}
  Expected Result \\
{\footnotesize
\begin{itemize}
\tightlist
\item
  TMA reaches the DIMM position.
\item
  DIMM imaging quality is sufficient.
\end{itemize}

}

\begin{tabular}{p{2cm}}
\toprule
Step 405  \\ \hline
\end{tabular}
 Description \\
{\footnotesize
\textbf{Move TMA to the DIMM position and \textbf{Take DIMM images}}\\

\begin{itemize}
\tightlist
\item
  Command the TMA to the DIMM position by applying the offsets
\item
  While tracking, take DIMM images with XXXs exposure time and inspect
  the quality.
\end{itemize}

}
\hdashrule[0.5ex]{\textwidth}{1pt}{3mm}
  Expected Result \\
{\footnotesize
\begin{itemize}
\tightlist
\item
  TMA reaches the DIMM position.
\item
  DIMM imaging quality is sufficient.
\end{itemize}

}

\begin{tabular}{p{2cm}}
\toprule
Step 406  \\ \hline
\end{tabular}
 Description \\
{\footnotesize
\textbf{Move TMA to the DIMM position and \textbf{Take DIMM images}}\\

\begin{itemize}
\tightlist
\item
  Command the TMA to the DIMM position by applying the offsets
\item
  While tracking, take DIMM images with XXXs exposure time and inspect
  the quality.
\end{itemize}

}
\hdashrule[0.5ex]{\textwidth}{1pt}{3mm}
  Expected Result \\
{\footnotesize
\begin{itemize}
\tightlist
\item
  TMA reaches the DIMM position.
\item
  DIMM imaging quality is sufficient.
\end{itemize}

}

\begin{tabular}{p{2cm}}
\toprule
Step 407  \\ \hline
\end{tabular}
 Description \\
{\footnotesize
\textbf{Move TMA to the DIMM position and \textbf{Take DIMM images}}\\

\begin{itemize}
\tightlist
\item
  Command the TMA to the DIMM position by applying the offsets
\item
  While tracking, take DIMM images with XXXs exposure time and inspect
  the quality.
\end{itemize}

}
\hdashrule[0.5ex]{\textwidth}{1pt}{3mm}
  Expected Result \\
{\footnotesize
\begin{itemize}
\tightlist
\item
  TMA reaches the DIMM position.
\item
  DIMM imaging quality is sufficient.
\end{itemize}

}

\begin{tabular}{p{2cm}}
\toprule
Step 408  \\ \hline
\end{tabular}
 Description \\
{\footnotesize
\textbf{Move TMA to the DIMM position and \textbf{Take DIMM images}}\\

\begin{itemize}
\tightlist
\item
  Command the TMA to the DIMM position by applying the offsets
\item
  While tracking, take DIMM images with XXXs exposure time and inspect
  the quality.
\end{itemize}

}
\hdashrule[0.5ex]{\textwidth}{1pt}{3mm}
  Expected Result \\
{\footnotesize
\begin{itemize}
\tightlist
\item
  TMA reaches the DIMM position.
\item
  DIMM imaging quality is sufficient.
\end{itemize}

}

\begin{tabular}{p{2cm}}
\toprule
Step 409  \\ \hline
\end{tabular}
 Description \\
{\footnotesize
\textbf{Move TMA to the DIMM position and \textbf{Take DIMM images}}\\

\begin{itemize}
\tightlist
\item
  Command the TMA to the DIMM position by applying the offsets
\item
  While tracking, take DIMM images with XXXs exposure time and inspect
  the quality.
\end{itemize}

}
\hdashrule[0.5ex]{\textwidth}{1pt}{3mm}
  Expected Result \\
{\footnotesize
\begin{itemize}
\tightlist
\item
  TMA reaches the DIMM position.
\item
  DIMM imaging quality is sufficient.
\end{itemize}

}

\begin{tabular}{p{2cm}}
\toprule
Step 410  \\ \hline
\end{tabular}
 Description \\
{\footnotesize
\textbf{Move TMA to the DIMM position and \textbf{Take DIMM images}}\\

\begin{itemize}
\tightlist
\item
  Command the TMA to the DIMM position by applying the offsets
\item
  While tracking, take DIMM images with XXXs exposure time and inspect
  the quality.
\end{itemize}

}
\hdashrule[0.5ex]{\textwidth}{1pt}{3mm}
  Expected Result \\
{\footnotesize
\begin{itemize}
\tightlist
\item
  TMA reaches the DIMM position.
\item
  DIMM imaging quality is sufficient.
\end{itemize}

}

\begin{tabular}{p{2cm}}
\toprule
Step 411  \\ \hline
\end{tabular}
 Description \\
{\footnotesize
\textbf{Move TMA to the DIMM position and \textbf{Take DIMM images}}\\

\begin{itemize}
\tightlist
\item
  Command the TMA to the DIMM position by applying the offsets
\item
  While tracking, take DIMM images with XXXs exposure time and inspect
  the quality.
\end{itemize}

}
\hdashrule[0.5ex]{\textwidth}{1pt}{3mm}
  Expected Result \\
{\footnotesize
\begin{itemize}
\tightlist
\item
  TMA reaches the DIMM position.
\item
  DIMM imaging quality is sufficient.
\end{itemize}

}

\begin{tabular}{p{2cm}}
\toprule
Step 412  \\ \hline
\end{tabular}
 Description \\
{\footnotesize
\textbf{Move TMA to the DIMM position and \textbf{Take DIMM images}}\\

\begin{itemize}
\tightlist
\item
  Command the TMA to the DIMM position by applying the offsets
\item
  While tracking, take DIMM images with XXXs exposure time and inspect
  the quality.
\end{itemize}

}
\hdashrule[0.5ex]{\textwidth}{1pt}{3mm}
  Expected Result \\
{\footnotesize
\begin{itemize}
\tightlist
\item
  TMA reaches the DIMM position.
\item
  DIMM imaging quality is sufficient.
\end{itemize}

}

\begin{tabular}{p{2cm}}
\toprule
Step 413  \\ \hline
\end{tabular}
 Description \\
{\footnotesize
\textbf{Move TMA to the DIMM position and \textbf{Take DIMM images}}\\

\begin{itemize}
\tightlist
\item
  Command the TMA to the DIMM position by applying the offsets
\item
  While tracking, take DIMM images with XXXs exposure time and inspect
  the quality.
\end{itemize}

}
\hdashrule[0.5ex]{\textwidth}{1pt}{3mm}
  Expected Result \\
{\footnotesize
\begin{itemize}
\tightlist
\item
  TMA reaches the DIMM position.
\item
  DIMM imaging quality is sufficient.
\end{itemize}

}

\begin{tabular}{p{2cm}}
\toprule
Step 414  \\ \hline
\end{tabular}
 Description \\
{\footnotesize
\textbf{Move TMA to the DIMM position and \textbf{Take DIMM images}}\\

\begin{itemize}
\tightlist
\item
  Command the TMA to the DIMM position by applying the offsets
\item
  While tracking, take DIMM images with XXXs exposure time and inspect
  the quality.
\end{itemize}

}
\hdashrule[0.5ex]{\textwidth}{1pt}{3mm}
  Expected Result \\
{\footnotesize
\begin{itemize}
\tightlist
\item
  TMA reaches the DIMM position.
\item
  DIMM imaging quality is sufficient.
\end{itemize}

}

\begin{tabular}{p{2cm}}
\toprule
Step 415  \\ \hline
\end{tabular}
 Description \\
{\footnotesize
\textbf{Move TMA to the DIMM position and \textbf{Take DIMM images}}\\

\begin{itemize}
\tightlist
\item
  Command the TMA to the DIMM position by applying the offsets
\item
  While tracking, take DIMM images with XXXs exposure time and inspect
  the quality.
\end{itemize}

}
\hdashrule[0.5ex]{\textwidth}{1pt}{3mm}
  Expected Result \\
{\footnotesize
\begin{itemize}
\tightlist
\item
  TMA reaches the DIMM position.
\item
  DIMM imaging quality is sufficient.
\end{itemize}

}

\begin{tabular}{p{2cm}}
\toprule
Step 416  \\ \hline
\end{tabular}
 Description \\
{\footnotesize
\textbf{Move TMA to the DIMM position and \textbf{Take DIMM images}}\\

\begin{itemize}
\tightlist
\item
  Command the TMA to the DIMM position by applying the offsets
\item
  While tracking, take DIMM images with XXXs exposure time and inspect
  the quality.
\end{itemize}

}
\hdashrule[0.5ex]{\textwidth}{1pt}{3mm}
  Expected Result \\
{\footnotesize
\begin{itemize}
\tightlist
\item
  TMA reaches the DIMM position.
\item
  DIMM imaging quality is sufficient.
\end{itemize}

}

\begin{tabular}{p{2cm}}
\toprule
Step 417  \\ \hline
\end{tabular}
 Description \\
{\footnotesize
\textbf{Move TMA to the DIMM position and \textbf{Take DIMM images}}\\

\begin{itemize}
\tightlist
\item
  Command the TMA to the DIMM position by applying the offsets
\item
  While tracking, take DIMM images with XXXs exposure time and inspect
  the quality.
\end{itemize}

}
\hdashrule[0.5ex]{\textwidth}{1pt}{3mm}
  Expected Result \\
{\footnotesize
\begin{itemize}
\tightlist
\item
  TMA reaches the DIMM position.
\item
  DIMM imaging quality is sufficient.
\end{itemize}

}

\begin{tabular}{p{2cm}}
\toprule
Step 418  \\ \hline
\end{tabular}
 Description \\
{\footnotesize
\textbf{Move TMA to the DIMM position and \textbf{Take DIMM images}}\\

\begin{itemize}
\tightlist
\item
  Command the TMA to the DIMM position by applying the offsets
\item
  While tracking, take DIMM images with XXXs exposure time and inspect
  the quality.
\end{itemize}

}
\hdashrule[0.5ex]{\textwidth}{1pt}{3mm}
  Expected Result \\
{\footnotesize
\begin{itemize}
\tightlist
\item
  TMA reaches the DIMM position.
\item
  DIMM imaging quality is sufficient.
\end{itemize}

}

\begin{tabular}{p{2cm}}
\toprule
Step 419  \\ \hline
\end{tabular}
 Description \\
{\footnotesize
\textbf{Move TMA to the DIMM position and \textbf{Take DIMM images}}\\

\begin{itemize}
\tightlist
\item
  Command the TMA to the DIMM position by applying the offsets
\item
  While tracking, take DIMM images with XXXs exposure time and inspect
  the quality.
\end{itemize}

}
\hdashrule[0.5ex]{\textwidth}{1pt}{3mm}
  Expected Result \\
{\footnotesize
\begin{itemize}
\tightlist
\item
  TMA reaches the DIMM position.
\item
  DIMM imaging quality is sufficient.
\end{itemize}

}

\begin{tabular}{p{2cm}}
\toprule
Step 420  \\ \hline
\end{tabular}
 Description \\
{\footnotesize
\textbf{Move TMA to the DIMM position and \textbf{Take DIMM images}}\\

\begin{itemize}
\tightlist
\item
  Command the TMA to the DIMM position by applying the offsets
\item
  While tracking, take DIMM images with XXXs exposure time and inspect
  the quality.
\end{itemize}

}
\hdashrule[0.5ex]{\textwidth}{1pt}{3mm}
  Expected Result \\
{\footnotesize
\begin{itemize}
\tightlist
\item
  TMA reaches the DIMM position.
\item
  DIMM imaging quality is sufficient.
\end{itemize}

}

\begin{tabular}{p{2cm}}
\toprule
Step 421  \\ \hline
\end{tabular}
 Description \\
{\footnotesize
\textbf{Move TMA to the DIMM position and \textbf{Take DIMM images}}\\

\begin{itemize}
\tightlist
\item
  Command the TMA to the DIMM position by applying the offsets
\item
  While tracking, take DIMM images with XXXs exposure time and inspect
  the quality.
\end{itemize}

}
\hdashrule[0.5ex]{\textwidth}{1pt}{3mm}
  Expected Result \\
{\footnotesize
\begin{itemize}
\tightlist
\item
  TMA reaches the DIMM position.
\item
  DIMM imaging quality is sufficient.
\end{itemize}

}

\begin{tabular}{p{2cm}}
\toprule
Step 422  \\ \hline
\end{tabular}
 Description \\
{\footnotesize
\textbf{Move TMA to the DIMM position and \textbf{Take DIMM images}}\\

\begin{itemize}
\tightlist
\item
  Command the TMA to the DIMM position by applying the offsets
\item
  While tracking, take DIMM images with XXXs exposure time and inspect
  the quality.
\end{itemize}

}
\hdashrule[0.5ex]{\textwidth}{1pt}{3mm}
  Expected Result \\
{\footnotesize
\begin{itemize}
\tightlist
\item
  TMA reaches the DIMM position.
\item
  DIMM imaging quality is sufficient.
\end{itemize}

}

\begin{tabular}{p{2cm}}
\toprule
Step 423  \\ \hline
\end{tabular}
 Description \\
{\footnotesize
\textbf{Move TMA to the DIMM position and \textbf{Take DIMM images}}\\

\begin{itemize}
\tightlist
\item
  Command the TMA to the DIMM position by applying the offsets
\item
  While tracking, take DIMM images with XXXs exposure time and inspect
  the quality.
\end{itemize}

}
\hdashrule[0.5ex]{\textwidth}{1pt}{3mm}
  Expected Result \\
{\footnotesize
\begin{itemize}
\tightlist
\item
  TMA reaches the DIMM position.
\item
  DIMM imaging quality is sufficient.
\end{itemize}

}

\begin{tabular}{p{2cm}}
\toprule
Step 424  \\ \hline
\end{tabular}
 Description \\
{\footnotesize
\textbf{Move TMA to the DIMM position and \textbf{Take DIMM images}}\\

\begin{itemize}
\tightlist
\item
  Command the TMA to the DIMM position by applying the offsets
\item
  While tracking, take DIMM images with XXXs exposure time and inspect
  the quality.
\end{itemize}

}
\hdashrule[0.5ex]{\textwidth}{1pt}{3mm}
  Expected Result \\
{\footnotesize
\begin{itemize}
\tightlist
\item
  TMA reaches the DIMM position.
\item
  DIMM imaging quality is sufficient.
\end{itemize}

}

\begin{tabular}{p{2cm}}
\toprule
Step 425  \\ \hline
\end{tabular}
 Description \\
{\footnotesize
\textbf{Move TMA to the DIMM position and \textbf{Take DIMM images}}\\

\begin{itemize}
\tightlist
\item
  Command the TMA to the DIMM position by applying the offsets
\item
  While tracking, take DIMM images with XXXs exposure time and inspect
  the quality.
\end{itemize}

}
\hdashrule[0.5ex]{\textwidth}{1pt}{3mm}
  Expected Result \\
{\footnotesize
\begin{itemize}
\tightlist
\item
  TMA reaches the DIMM position.
\item
  DIMM imaging quality is sufficient.
\end{itemize}

}

\begin{tabular}{p{2cm}}
\toprule
Step 426  \\ \hline
\end{tabular}
 Description \\
{\footnotesize
\textbf{Move TMA to the DIMM position and \textbf{Take DIMM images}}\\

\begin{itemize}
\tightlist
\item
  Command the TMA to the DIMM position by applying the offsets
\item
  While tracking, take DIMM images with XXXs exposure time and inspect
  the quality.
\end{itemize}

}
\hdashrule[0.5ex]{\textwidth}{1pt}{3mm}
  Expected Result \\
{\footnotesize
\begin{itemize}
\tightlist
\item
  TMA reaches the DIMM position.
\item
  DIMM imaging quality is sufficient.
\end{itemize}

}

\begin{tabular}{p{2cm}}
\toprule
Step 427  \\ \hline
\end{tabular}
 Description \\
{\footnotesize
\textbf{Move TMA to the DIMM position and \textbf{Take DIMM images}}\\

\begin{itemize}
\tightlist
\item
  Command the TMA to the DIMM position by applying the offsets
\item
  While tracking, take DIMM images with XXXs exposure time and inspect
  the quality.
\end{itemize}

}
\hdashrule[0.5ex]{\textwidth}{1pt}{3mm}
  Expected Result \\
{\footnotesize
\begin{itemize}
\tightlist
\item
  TMA reaches the DIMM position.
\item
  DIMM imaging quality is sufficient.
\end{itemize}

}

\begin{tabular}{p{2cm}}
\toprule
Step 428  \\ \hline
\end{tabular}
 Description \\
{\footnotesize
\textbf{Move TMA to the DIMM position and \textbf{Take DIMM images}}\\

\begin{itemize}
\tightlist
\item
  Command the TMA to the DIMM position by applying the offsets
\item
  While tracking, take DIMM images with XXXs exposure time and inspect
  the quality.
\end{itemize}

}
\hdashrule[0.5ex]{\textwidth}{1pt}{3mm}
  Expected Result \\
{\footnotesize
\begin{itemize}
\tightlist
\item
  TMA reaches the DIMM position.
\item
  DIMM imaging quality is sufficient.
\end{itemize}

}

\begin{tabular}{p{2cm}}
\toprule
Step 429  \\ \hline
\end{tabular}
 Description \\
{\footnotesize
\textbf{Move TMA to the DIMM position and \textbf{Take DIMM images}}\\

\begin{itemize}
\tightlist
\item
  Command the TMA to the DIMM position by applying the offsets
\item
  While tracking, take DIMM images with XXXs exposure time and inspect
  the quality.
\end{itemize}

}
\hdashrule[0.5ex]{\textwidth}{1pt}{3mm}
  Expected Result \\
{\footnotesize
\begin{itemize}
\tightlist
\item
  TMA reaches the DIMM position.
\item
  DIMM imaging quality is sufficient.
\end{itemize}

}

\begin{tabular}{p{2cm}}
\toprule
Step 430  \\ \hline
\end{tabular}
 Description \\
{\footnotesize
\textbf{Move TMA to the DIMM position and \textbf{Take DIMM images}}\\

\begin{itemize}
\tightlist
\item
  Command the TMA to the DIMM position by applying the offsets
\item
  While tracking, take DIMM images with XXXs exposure time and inspect
  the quality.
\end{itemize}

}
\hdashrule[0.5ex]{\textwidth}{1pt}{3mm}
  Expected Result \\
{\footnotesize
\begin{itemize}
\tightlist
\item
  TMA reaches the DIMM position.
\item
  DIMM imaging quality is sufficient.
\end{itemize}

}

\begin{tabular}{p{2cm}}
\toprule
Step 431  \\ \hline
\end{tabular}
 Description \\
{\footnotesize
\textbf{Move TMA to the DIMM position and \textbf{Take DIMM images}}\\

\begin{itemize}
\tightlist
\item
  Command the TMA to the DIMM position by applying the offsets
\item
  While tracking, take DIMM images with XXXs exposure time and inspect
  the quality.
\end{itemize}

}
\hdashrule[0.5ex]{\textwidth}{1pt}{3mm}
  Expected Result \\
{\footnotesize
\begin{itemize}
\tightlist
\item
  TMA reaches the DIMM position.
\item
  DIMM imaging quality is sufficient.
\end{itemize}

}

\begin{tabular}{p{2cm}}
\toprule
Step 432  \\ \hline
\end{tabular}
 Description \\
{\footnotesize
\textbf{Move TMA to the DIMM position and \textbf{Take DIMM images}}\\

\begin{itemize}
\tightlist
\item
  Command the TMA to the DIMM position by applying the offsets
\item
  While tracking, take DIMM images with XXXs exposure time and inspect
  the quality.
\end{itemize}

}
\hdashrule[0.5ex]{\textwidth}{1pt}{3mm}
  Expected Result \\
{\footnotesize
\begin{itemize}
\tightlist
\item
  TMA reaches the DIMM position.
\item
  DIMM imaging quality is sufficient.
\end{itemize}

}

\begin{tabular}{p{2cm}}
\toprule
Step 433  \\ \hline
\end{tabular}
 Description \\
{\footnotesize
\textbf{Move TMA to the DIMM position and \textbf{Take DIMM images}}\\

\begin{itemize}
\tightlist
\item
  Command the TMA to the DIMM position by applying the offsets
\item
  While tracking, take DIMM images with XXXs exposure time and inspect
  the quality.
\end{itemize}

}
\hdashrule[0.5ex]{\textwidth}{1pt}{3mm}
  Expected Result \\
{\footnotesize
\begin{itemize}
\tightlist
\item
  TMA reaches the DIMM position.
\item
  DIMM imaging quality is sufficient.
\end{itemize}

}

\begin{tabular}{p{2cm}}
\toprule
Step 434  \\ \hline
\end{tabular}
 Description \\
{\footnotesize
\textbf{Move TMA to the DIMM position and \textbf{Take DIMM images}}\\

\begin{itemize}
\tightlist
\item
  Command the TMA to the DIMM position by applying the offsets
\item
  While tracking, take DIMM images with XXXs exposure time and inspect
  the quality.
\end{itemize}

}
\hdashrule[0.5ex]{\textwidth}{1pt}{3mm}
  Expected Result \\
{\footnotesize
\begin{itemize}
\tightlist
\item
  TMA reaches the DIMM position.
\item
  DIMM imaging quality is sufficient.
\end{itemize}

}

\begin{tabular}{p{2cm}}
\toprule
Step 435  \\ \hline
\end{tabular}
 Description \\
{\footnotesize
\textbf{Move TMA to the DIMM position and \textbf{Take DIMM images}}\\

\begin{itemize}
\tightlist
\item
  Command the TMA to the DIMM position by applying the offsets
\item
  While tracking, take DIMM images with XXXs exposure time and inspect
  the quality.
\end{itemize}

}
\hdashrule[0.5ex]{\textwidth}{1pt}{3mm}
  Expected Result \\
{\footnotesize
\begin{itemize}
\tightlist
\item
  TMA reaches the DIMM position.
\item
  DIMM imaging quality is sufficient.
\end{itemize}

}

\begin{tabular}{p{2cm}}
\toprule
Step 436  \\ \hline
\end{tabular}
 Description \\
{\footnotesize
\textbf{Move TMA to the DIMM position and \textbf{Take DIMM images}}\\

\begin{itemize}
\tightlist
\item
  Command the TMA to the DIMM position by applying the offsets
\item
  While tracking, take DIMM images with XXXs exposure time and inspect
  the quality.
\end{itemize}

}
\hdashrule[0.5ex]{\textwidth}{1pt}{3mm}
  Expected Result \\
{\footnotesize
\begin{itemize}
\tightlist
\item
  TMA reaches the DIMM position.
\item
  DIMM imaging quality is sufficient.
\end{itemize}

}

\begin{tabular}{p{2cm}}
\toprule
Step 437  \\ \hline
\end{tabular}
 Description \\
{\footnotesize
\textbf{Move TMA to the DIMM position and \textbf{Take DIMM images}}\\

\begin{itemize}
\tightlist
\item
  Command the TMA to the DIMM position by applying the offsets
\item
  While tracking, take DIMM images with XXXs exposure time and inspect
  the quality.
\end{itemize}

}
\hdashrule[0.5ex]{\textwidth}{1pt}{3mm}
  Expected Result \\
{\footnotesize
\begin{itemize}
\tightlist
\item
  TMA reaches the DIMM position.
\item
  DIMM imaging quality is sufficient.
\end{itemize}

}

\begin{tabular}{p{2cm}}
\toprule
Step 438  \\ \hline
\end{tabular}
 Description \\
{\footnotesize
\textbf{Move TMA to the DIMM position and \textbf{Take DIMM images}}\\

\begin{itemize}
\tightlist
\item
  Command the TMA to the DIMM position by applying the offsets
\item
  While tracking, take DIMM images with XXXs exposure time and inspect
  the quality.
\end{itemize}

}
\hdashrule[0.5ex]{\textwidth}{1pt}{3mm}
  Expected Result \\
{\footnotesize
\begin{itemize}
\tightlist
\item
  TMA reaches the DIMM position.
\item
  DIMM imaging quality is sufficient.
\end{itemize}

}

\begin{tabular}{p{2cm}}
\toprule
Step 439  \\ \hline
\end{tabular}
 Description \\
{\footnotesize
\textbf{Move TMA to the DIMM position and \textbf{Take DIMM images}}\\

\begin{itemize}
\tightlist
\item
  Command the TMA to the DIMM position by applying the offsets
\item
  While tracking, take DIMM images with XXXs exposure time and inspect
  the quality.
\end{itemize}

}
\hdashrule[0.5ex]{\textwidth}{1pt}{3mm}
  Expected Result \\
{\footnotesize
\begin{itemize}
\tightlist
\item
  TMA reaches the DIMM position.
\item
  DIMM imaging quality is sufficient.
\end{itemize}

}

\begin{tabular}{p{2cm}}
\toprule
Step 440  \\ \hline
\end{tabular}
 Description \\
{\footnotesize
\textbf{Move TMA to the DIMM position and \textbf{Take DIMM images}}\\

\begin{itemize}
\tightlist
\item
  Command the TMA to the DIMM position by applying the offsets
\item
  While tracking, take DIMM images with XXXs exposure time and inspect
  the quality.
\end{itemize}

}
\hdashrule[0.5ex]{\textwidth}{1pt}{3mm}
  Expected Result \\
{\footnotesize
\begin{itemize}
\tightlist
\item
  TMA reaches the DIMM position.
\item
  DIMM imaging quality is sufficient.
\end{itemize}

}

\begin{tabular}{p{2cm}}
\toprule
Step 441  \\ \hline
\end{tabular}
 Description \\
{\footnotesize
\textbf{Move TMA to the DIMM position and \textbf{Take DIMM images}}\\

\begin{itemize}
\tightlist
\item
  Command the TMA to the DIMM position by applying the offsets
\item
  While tracking, take DIMM images with XXXs exposure time and inspect
  the quality.
\end{itemize}

}
\hdashrule[0.5ex]{\textwidth}{1pt}{3mm}
  Expected Result \\
{\footnotesize
\begin{itemize}
\tightlist
\item
  TMA reaches the DIMM position.
\item
  DIMM imaging quality is sufficient.
\end{itemize}

}

\begin{tabular}{p{2cm}}
\toprule
Step 442  \\ \hline
\end{tabular}
 Description \\
{\footnotesize
\textbf{Move TMA to the DIMM position and \textbf{Take DIMM images}}\\

\begin{itemize}
\tightlist
\item
  Command the TMA to the DIMM position by applying the offsets
\item
  While tracking, take DIMM images with XXXs exposure time and inspect
  the quality.
\end{itemize}

}
\hdashrule[0.5ex]{\textwidth}{1pt}{3mm}
  Expected Result \\
{\footnotesize
\begin{itemize}
\tightlist
\item
  TMA reaches the DIMM position.
\item
  DIMM imaging quality is sufficient.
\end{itemize}

}

\begin{tabular}{p{2cm}}
\toprule
Step 443  \\ \hline
\end{tabular}
 Description \\
{\footnotesize
\textbf{Move TMA to the DIMM position and \textbf{Take DIMM images}}\\

\begin{itemize}
\tightlist
\item
  Command the TMA to the DIMM position by applying the offsets
\item
  While tracking, take DIMM images with XXXs exposure time and inspect
  the quality.
\end{itemize}

}
\hdashrule[0.5ex]{\textwidth}{1pt}{3mm}
  Expected Result \\
{\footnotesize
\begin{itemize}
\tightlist
\item
  TMA reaches the DIMM position.
\item
  DIMM imaging quality is sufficient.
\end{itemize}

}

\begin{tabular}{p{2cm}}
\toprule
Step 444  \\ \hline
\end{tabular}
 Description \\
{\footnotesize
\textbf{Move TMA to the DIMM position and \textbf{Take DIMM images}}\\

\begin{itemize}
\tightlist
\item
  Command the TMA to the DIMM position by applying the offsets
\item
  While tracking, take DIMM images with XXXs exposure time and inspect
  the quality.
\end{itemize}

}
\hdashrule[0.5ex]{\textwidth}{1pt}{3mm}
  Expected Result \\
{\footnotesize
\begin{itemize}
\tightlist
\item
  TMA reaches the DIMM position.
\item
  DIMM imaging quality is sufficient.
\end{itemize}

}

\begin{tabular}{p{2cm}}
\toprule
Step 445  \\ \hline
\end{tabular}
 Description \\
{\footnotesize
\textbf{Move TMA to the DIMM position and \textbf{Take DIMM images}}\\

\begin{itemize}
\tightlist
\item
  Command the TMA to the DIMM position by applying the offsets
\item
  While tracking, take DIMM images with XXXs exposure time and inspect
  the quality.
\end{itemize}

}
\hdashrule[0.5ex]{\textwidth}{1pt}{3mm}
  Expected Result \\
{\footnotesize
\begin{itemize}
\tightlist
\item
  TMA reaches the DIMM position.
\item
  DIMM imaging quality is sufficient.
\end{itemize}

}

\begin{tabular}{p{2cm}}
\toprule
Step 446  \\ \hline
\end{tabular}
 Description \\
{\footnotesize
\textbf{Move TMA to the DIMM position and \textbf{Take DIMM images}}\\

\begin{itemize}
\tightlist
\item
  Command the TMA to the DIMM position by applying the offsets
\item
  While tracking, take DIMM images with XXXs exposure time and inspect
  the quality.
\end{itemize}

}
\hdashrule[0.5ex]{\textwidth}{1pt}{3mm}
  Expected Result \\
{\footnotesize
\begin{itemize}
\tightlist
\item
  TMA reaches the DIMM position.
\item
  DIMM imaging quality is sufficient.
\end{itemize}

}

\begin{tabular}{p{2cm}}
\toprule
Step 447  \\ \hline
\end{tabular}
 Description \\
{\footnotesize
\textbf{Point the TMA to (Az, El)-pattern position + DIMM pattern offset
\textbf{~and take DIMM images}\\
}

\begin{itemize}
\tightlist
\item
  Point the TMA back to {Pointing 1}⁠ at {-270}⁠ + DIMM offset, {15}⁠ +
  DIMM offset.
\item
  While tracking, take DIMM images with XXXs exposure time and inspect
  the quality.
\end{itemize}

}
\hdashrule[0.5ex]{\textwidth}{1pt}{3mm}
  Expected Result \\
{\footnotesize
\begin{itemize}
\tightlist
\item
  TMA reaches the position
\item
  DIMM image quality is sufficient
\end{itemize}

}

\begin{tabular}{p{2cm}}
\toprule
Step 448  \\ \hline
\end{tabular}
 Description \\
{\footnotesize
\textbf{Point the TMA to (Az, El)-pattern position + DIMM pattern offset
\textbf{~and take DIMM images}\\
}

\begin{itemize}
\tightlist
\item
  Point the TMA back to {Pointing 2}⁠ at {-270}⁠ + DIMM offset, {45}⁠ +
  DIMM offset.
\item
  While tracking, take DIMM images with XXXs exposure time and inspect
  the quality.
\end{itemize}

}
\hdashrule[0.5ex]{\textwidth}{1pt}{3mm}
  Expected Result \\
{\footnotesize
\begin{itemize}
\tightlist
\item
  TMA reaches the position
\item
  DIMM image quality is sufficient
\end{itemize}

}

\begin{tabular}{p{2cm}}
\toprule
Step 449  \\ \hline
\end{tabular}
 Description \\
{\footnotesize
\textbf{Point the TMA to (Az, El)-pattern position + DIMM pattern offset
\textbf{~and take DIMM images}\\
}

\begin{itemize}
\tightlist
\item
  Point the TMA back to {Pointing 3}⁠ at {-270}⁠ + DIMM offset, {75}⁠ +
  DIMM offset.
\item
  While tracking, take DIMM images with XXXs exposure time and inspect
  the quality.
\end{itemize}

}
\hdashrule[0.5ex]{\textwidth}{1pt}{3mm}
  Expected Result \\
{\footnotesize
\begin{itemize}
\tightlist
\item
  TMA reaches the position
\item
  DIMM image quality is sufficient
\end{itemize}

}

\begin{tabular}{p{2cm}}
\toprule
Step 450  \\ \hline
\end{tabular}
 Description \\
{\footnotesize
\textbf{Point the TMA to (Az, El)-pattern position + DIMM pattern offset
\textbf{~and take DIMM images}\\
}

\begin{itemize}
\tightlist
\item
  Point the TMA back to {Pointing 4}⁠ at {-270}⁠ + DIMM offset, {86.5}⁠
  + DIMM offset.
\item
  While tracking, take DIMM images with XXXs exposure time and inspect
  the quality.
\end{itemize}

}
\hdashrule[0.5ex]{\textwidth}{1pt}{3mm}
  Expected Result \\
{\footnotesize
\begin{itemize}
\tightlist
\item
  TMA reaches the position
\item
  DIMM image quality is sufficient
\end{itemize}

}

\begin{tabular}{p{2cm}}
\toprule
Step 451  \\ \hline
\end{tabular}
 Description \\
{\footnotesize
\textbf{Point the TMA to (Az, El)-pattern position + DIMM pattern offset
\textbf{~and take DIMM images}\\
}

\begin{itemize}
\tightlist
\item
  Point the TMA back to {Pointing 5}⁠ at {-180}⁠ + DIMM offset, {86.5}⁠
  + DIMM offset.
\item
  While tracking, take DIMM images with XXXs exposure time and inspect
  the quality.
\end{itemize}

}
\hdashrule[0.5ex]{\textwidth}{1pt}{3mm}
  Expected Result \\
{\footnotesize
\begin{itemize}
\tightlist
\item
  TMA reaches the position
\item
  DIMM image quality is sufficient
\end{itemize}

}

\begin{tabular}{p{2cm}}
\toprule
Step 452  \\ \hline
\end{tabular}
 Description \\
{\footnotesize
\textbf{Point the TMA to (Az, El)-pattern position + DIMM pattern offset
\textbf{~and take DIMM images}\\
}

\begin{itemize}
\tightlist
\item
  Point the TMA back to {Pointing 6}⁠ at {-180}⁠ + DIMM offset, {75}⁠ +
  DIMM offset.
\item
  While tracking, take DIMM images with XXXs exposure time and inspect
  the quality.
\end{itemize}

}
\hdashrule[0.5ex]{\textwidth}{1pt}{3mm}
  Expected Result \\
{\footnotesize
\begin{itemize}
\tightlist
\item
  TMA reaches the position
\item
  DIMM image quality is sufficient
\end{itemize}

}

\begin{tabular}{p{2cm}}
\toprule
Step 453  \\ \hline
\end{tabular}
 Description \\
{\footnotesize
\textbf{Point the TMA to (Az, El)-pattern position + DIMM pattern offset
\textbf{~and take DIMM images}\\
}

\begin{itemize}
\tightlist
\item
  Point the TMA back to {Pointing 7}⁠ at {-180}⁠ + DIMM offset, {45}⁠ +
  DIMM offset.
\item
  While tracking, take DIMM images with XXXs exposure time and inspect
  the quality.
\end{itemize}

}
\hdashrule[0.5ex]{\textwidth}{1pt}{3mm}
  Expected Result \\
{\footnotesize
\begin{itemize}
\tightlist
\item
  TMA reaches the position
\item
  DIMM image quality is sufficient
\end{itemize}

}

\begin{tabular}{p{2cm}}
\toprule
Step 454  \\ \hline
\end{tabular}
 Description \\
{\footnotesize
\textbf{Point the TMA to (Az, El)-pattern position + DIMM pattern offset
\textbf{~and take DIMM images}\\
}

\begin{itemize}
\tightlist
\item
  Point the TMA back to {Pointing 8}⁠ at {-180}⁠ + DIMM offset, {15}⁠ +
  DIMM offset.
\item
  While tracking, take DIMM images with XXXs exposure time and inspect
  the quality.
\end{itemize}

}
\hdashrule[0.5ex]{\textwidth}{1pt}{3mm}
  Expected Result \\
{\footnotesize
\begin{itemize}
\tightlist
\item
  TMA reaches the position
\item
  DIMM image quality is sufficient
\end{itemize}

}

\begin{tabular}{p{2cm}}
\toprule
Step 455  \\ \hline
\end{tabular}
 Description \\
{\footnotesize
\textbf{Point the TMA to (Az, El)-pattern position + DIMM pattern offset
\textbf{~and take DIMM images}\\
}

\begin{itemize}
\tightlist
\item
  Point the TMA back to {Pointing 9}⁠ at {-90}⁠ + DIMM offset, {15}⁠ +
  DIMM offset.
\item
  While tracking, take DIMM images with XXXs exposure time and inspect
  the quality.
\end{itemize}

}
\hdashrule[0.5ex]{\textwidth}{1pt}{3mm}
  Expected Result \\
{\footnotesize
\begin{itemize}
\tightlist
\item
  TMA reaches the position
\item
  DIMM image quality is sufficient
\end{itemize}

}

\begin{tabular}{p{2cm}}
\toprule
Step 456  \\ \hline
\end{tabular}
 Description \\
{\footnotesize
\textbf{Point the TMA to (Az, El)-pattern position + DIMM pattern offset
\textbf{~and take DIMM images}\\
}

\begin{itemize}
\tightlist
\item
  Point the TMA back to {Pointing 10}⁠ at {-90}⁠ + DIMM offset, {45}⁠ +
  DIMM offset.
\item
  While tracking, take DIMM images with XXXs exposure time and inspect
  the quality.
\end{itemize}

}
\hdashrule[0.5ex]{\textwidth}{1pt}{3mm}
  Expected Result \\
{\footnotesize
\begin{itemize}
\tightlist
\item
  TMA reaches the position
\item
  DIMM image quality is sufficient
\end{itemize}

}

\begin{tabular}{p{2cm}}
\toprule
Step 457  \\ \hline
\end{tabular}
 Description \\
{\footnotesize
\textbf{Point the TMA to (Az, El)-pattern position + DIMM pattern offset
\textbf{~and take DIMM images}\\
}

\begin{itemize}
\tightlist
\item
  Point the TMA back to {Pointing 11}⁠ at {-90}⁠ + DIMM offset, {75}⁠ +
  DIMM offset.
\item
  While tracking, take DIMM images with XXXs exposure time and inspect
  the quality.
\end{itemize}

}
\hdashrule[0.5ex]{\textwidth}{1pt}{3mm}
  Expected Result \\
{\footnotesize
\begin{itemize}
\tightlist
\item
  TMA reaches the position
\item
  DIMM image quality is sufficient
\end{itemize}

}

\begin{tabular}{p{2cm}}
\toprule
Step 458  \\ \hline
\end{tabular}
 Description \\
{\footnotesize
\textbf{Point the TMA to (Az, El)-pattern position + DIMM pattern offset
\textbf{~and take DIMM images}\\
}

\begin{itemize}
\tightlist
\item
  Point the TMA back to {Pointing 12}⁠ at {-90}⁠ + DIMM offset, {86.5}⁠
  + DIMM offset.
\item
  While tracking, take DIMM images with XXXs exposure time and inspect
  the quality.
\end{itemize}

}
\hdashrule[0.5ex]{\textwidth}{1pt}{3mm}
  Expected Result \\
{\footnotesize
\begin{itemize}
\tightlist
\item
  TMA reaches the position
\item
  DIMM image quality is sufficient
\end{itemize}

}

\begin{tabular}{p{2cm}}
\toprule
Step 459  \\ \hline
\end{tabular}
 Description \\
{\footnotesize
\textbf{Point the TMA to (Az, El)-pattern position + DIMM pattern offset
\textbf{~and take DIMM images}\\
}

\begin{itemize}
\tightlist
\item
  Point the TMA back to {Pointing 13}⁠ at {0}⁠ + DIMM offset, {86.5}⁠ +
  DIMM offset.
\item
  While tracking, take DIMM images with XXXs exposure time and inspect
  the quality.
\end{itemize}

}
\hdashrule[0.5ex]{\textwidth}{1pt}{3mm}
  Expected Result \\
{\footnotesize
\begin{itemize}
\tightlist
\item
  TMA reaches the position
\item
  DIMM image quality is sufficient
\end{itemize}

}

\begin{tabular}{p{2cm}}
\toprule
Step 460  \\ \hline
\end{tabular}
 Description \\
{\footnotesize
\textbf{Point the TMA to (Az, El)-pattern position + DIMM pattern offset
\textbf{~and take DIMM images}\\
}

\begin{itemize}
\tightlist
\item
  Point the TMA back to {Pointing 14}⁠ at {0}⁠ + DIMM offset, {75}⁠ +
  DIMM offset.
\item
  While tracking, take DIMM images with XXXs exposure time and inspect
  the quality.
\end{itemize}

}
\hdashrule[0.5ex]{\textwidth}{1pt}{3mm}
  Expected Result \\
{\footnotesize
\begin{itemize}
\tightlist
\item
  TMA reaches the position
\item
  DIMM image quality is sufficient
\end{itemize}

}

\begin{tabular}{p{2cm}}
\toprule
Step 461  \\ \hline
\end{tabular}
 Description \\
{\footnotesize
\textbf{Point the TMA to (Az, El)-pattern position + DIMM pattern offset
\textbf{~and take DIMM images}\\
}

\begin{itemize}
\tightlist
\item
  Point the TMA back to {Pointing 15}⁠ at {0}⁠ + DIMM offset, {45}⁠ +
  DIMM offset.
\item
  While tracking, take DIMM images with XXXs exposure time and inspect
  the quality.
\end{itemize}

}
\hdashrule[0.5ex]{\textwidth}{1pt}{3mm}
  Expected Result \\
{\footnotesize
\begin{itemize}
\tightlist
\item
  TMA reaches the position
\item
  DIMM image quality is sufficient
\end{itemize}

}

\begin{tabular}{p{2cm}}
\toprule
Step 462  \\ \hline
\end{tabular}
 Description \\
{\footnotesize
\textbf{Point the TMA to (Az, El)-pattern position + DIMM pattern offset
\textbf{~and take DIMM images}\\
}

\begin{itemize}
\tightlist
\item
  Point the TMA back to {Pointing 16}⁠ at {0}⁠ + DIMM offset, {15}⁠ +
  DIMM offset.
\item
  While tracking, take DIMM images with XXXs exposure time and inspect
  the quality.
\end{itemize}

}
\hdashrule[0.5ex]{\textwidth}{1pt}{3mm}
  Expected Result \\
{\footnotesize
\begin{itemize}
\tightlist
\item
  TMA reaches the position
\item
  DIMM image quality is sufficient
\end{itemize}

}

\begin{tabular}{p{2cm}}
\toprule
Step 463  \\ \hline
\end{tabular}
 Description \\
{\footnotesize
\textbf{Point the TMA to (Az, El)-pattern position + DIMM pattern offset
\textbf{~and take DIMM images}\\
}

\begin{itemize}
\tightlist
\item
  Point the TMA back to {Pointing 17}⁠ at {90}⁠ + DIMM offset, {15}⁠ +
  DIMM offset.
\item
  While tracking, take DIMM images with XXXs exposure time and inspect
  the quality.
\end{itemize}

}
\hdashrule[0.5ex]{\textwidth}{1pt}{3mm}
  Expected Result \\
{\footnotesize
\begin{itemize}
\tightlist
\item
  TMA reaches the position
\item
  DIMM image quality is sufficient
\end{itemize}

}

\begin{tabular}{p{2cm}}
\toprule
Step 464  \\ \hline
\end{tabular}
 Description \\
{\footnotesize
\textbf{Point the TMA to (Az, El)-pattern position + DIMM pattern offset
\textbf{~and take DIMM images}\\
}

\begin{itemize}
\tightlist
\item
  Point the TMA back to {Pointing 18}⁠ at {90}⁠ + DIMM offset, {45}⁠ +
  DIMM offset.
\item
  While tracking, take DIMM images with XXXs exposure time and inspect
  the quality.
\end{itemize}

}
\hdashrule[0.5ex]{\textwidth}{1pt}{3mm}
  Expected Result \\
{\footnotesize
\begin{itemize}
\tightlist
\item
  TMA reaches the position
\item
  DIMM image quality is sufficient
\end{itemize}

}

\begin{tabular}{p{2cm}}
\toprule
Step 465  \\ \hline
\end{tabular}
 Description \\
{\footnotesize
\textbf{Point the TMA to (Az, El)-pattern position + DIMM pattern offset
\textbf{~and take DIMM images}\\
}

\begin{itemize}
\tightlist
\item
  Point the TMA back to {Pointing 19}⁠ at {90}⁠ + DIMM offset, {75}⁠ +
  DIMM offset.
\item
  While tracking, take DIMM images with XXXs exposure time and inspect
  the quality.
\end{itemize}

}
\hdashrule[0.5ex]{\textwidth}{1pt}{3mm}
  Expected Result \\
{\footnotesize
\begin{itemize}
\tightlist
\item
  TMA reaches the position
\item
  DIMM image quality is sufficient
\end{itemize}

}

\begin{tabular}{p{2cm}}
\toprule
Step 466  \\ \hline
\end{tabular}
 Description \\
{\footnotesize
\textbf{Point the TMA to (Az, El)-pattern position + DIMM pattern offset
\textbf{~and take DIMM images}\\
}

\begin{itemize}
\tightlist
\item
  Point the TMA back to {Pointing 20}⁠ at {90}⁠ + DIMM offset, {86.5}⁠ +
  DIMM offset.
\item
  While tracking, take DIMM images with XXXs exposure time and inspect
  the quality.
\end{itemize}

}
\hdashrule[0.5ex]{\textwidth}{1pt}{3mm}
  Expected Result \\
{\footnotesize
\begin{itemize}
\tightlist
\item
  TMA reaches the position
\item
  DIMM image quality is sufficient
\end{itemize}

}

\begin{tabular}{p{2cm}}
\toprule
Step 467  \\ \hline
\end{tabular}
 Description \\
{\footnotesize
\textbf{Point the TMA to (Az, El)-pattern position + DIMM pattern offset
\textbf{~and take DIMM images}\\
}

\begin{itemize}
\tightlist
\item
  Point the TMA back to {Pointing 21}⁠ at {180}⁠ + DIMM offset, {86.5}⁠
  + DIMM offset.
\item
  While tracking, take DIMM images with XXXs exposure time and inspect
  the quality.
\end{itemize}

}
\hdashrule[0.5ex]{\textwidth}{1pt}{3mm}
  Expected Result \\
{\footnotesize
\begin{itemize}
\tightlist
\item
  TMA reaches the position
\item
  DIMM image quality is sufficient
\end{itemize}

}

\begin{tabular}{p{2cm}}
\toprule
Step 468  \\ \hline
\end{tabular}
 Description \\
{\footnotesize
\textbf{Point the TMA to (Az, El)-pattern position + DIMM pattern offset
\textbf{~and take DIMM images}\\
}

\begin{itemize}
\tightlist
\item
  Point the TMA back to {Pointing 22}⁠ at {180}⁠ + DIMM offset, {75}⁠ +
  DIMM offset.
\item
  While tracking, take DIMM images with XXXs exposure time and inspect
  the quality.
\end{itemize}

}
\hdashrule[0.5ex]{\textwidth}{1pt}{3mm}
  Expected Result \\
{\footnotesize
\begin{itemize}
\tightlist
\item
  TMA reaches the position
\item
  DIMM image quality is sufficient
\end{itemize}

}

\begin{tabular}{p{2cm}}
\toprule
Step 469  \\ \hline
\end{tabular}
 Description \\
{\footnotesize
\textbf{Point the TMA to (Az, El)-pattern position + DIMM pattern offset
\textbf{~and take DIMM images}\\
}

\begin{itemize}
\tightlist
\item
  Point the TMA back to {Pointing 23}⁠ at {180}⁠ + DIMM offset, {45}⁠ +
  DIMM offset.
\item
  While tracking, take DIMM images with XXXs exposure time and inspect
  the quality.
\end{itemize}

}
\hdashrule[0.5ex]{\textwidth}{1pt}{3mm}
  Expected Result \\
{\footnotesize
\begin{itemize}
\tightlist
\item
  TMA reaches the position
\item
  DIMM image quality is sufficient
\end{itemize}

}

\begin{tabular}{p{2cm}}
\toprule
Step 470  \\ \hline
\end{tabular}
 Description \\
{\footnotesize
\textbf{Point the TMA to (Az, El)-pattern position + DIMM pattern offset
\textbf{~and take DIMM images}\\
}

\begin{itemize}
\tightlist
\item
  Point the TMA back to {Pointing 24}⁠ at {180}⁠ + DIMM offset, {15}⁠ +
  DIMM offset.
\item
  While tracking, take DIMM images with XXXs exposure time and inspect
  the quality.
\end{itemize}

}
\hdashrule[0.5ex]{\textwidth}{1pt}{3mm}
  Expected Result \\
{\footnotesize
\begin{itemize}
\tightlist
\item
  TMA reaches the position
\item
  DIMM image quality is sufficient
\end{itemize}

}

\begin{tabular}{p{2cm}}
\toprule
Step 471  \\ \hline
\end{tabular}
 Description \\
{\footnotesize
\textbf{Point the TMA to (Az, El)-pattern position + DIMM pattern offset
\textbf{~and take DIMM images}\\
}

\begin{itemize}
\tightlist
\item
  Point the TMA back to {Pointing 25}⁠ at {270}⁠ + DIMM offset, {15}⁠ +
  DIMM offset.
\item
  While tracking, take DIMM images with XXXs exposure time and inspect
  the quality.
\end{itemize}

}
\hdashrule[0.5ex]{\textwidth}{1pt}{3mm}
  Expected Result \\
{\footnotesize
\begin{itemize}
\tightlist
\item
  TMA reaches the position
\item
  DIMM image quality is sufficient
\end{itemize}

}

\begin{tabular}{p{2cm}}
\toprule
Step 472  \\ \hline
\end{tabular}
 Description \\
{\footnotesize
\textbf{Point the TMA to (Az, El)-pattern position + DIMM pattern offset
\textbf{~and take DIMM images}\\
}

\begin{itemize}
\tightlist
\item
  Point the TMA back to {Pointing 26}⁠ at {270}⁠ + DIMM offset, {45}⁠ +
  DIMM offset.
\item
  While tracking, take DIMM images with XXXs exposure time and inspect
  the quality.
\end{itemize}

}
\hdashrule[0.5ex]{\textwidth}{1pt}{3mm}
  Expected Result \\
{\footnotesize
\begin{itemize}
\tightlist
\item
  TMA reaches the position
\item
  DIMM image quality is sufficient
\end{itemize}

}

\begin{tabular}{p{2cm}}
\toprule
Step 473  \\ \hline
\end{tabular}
 Description \\
{\footnotesize
\textbf{Point the TMA to (Az, El)-pattern position + DIMM pattern offset
\textbf{~and take DIMM images}\\
}

\begin{itemize}
\tightlist
\item
  Point the TMA back to {Pointing 27}⁠ at {270}⁠ + DIMM offset, {75}⁠ +
  DIMM offset.
\item
  While tracking, take DIMM images with XXXs exposure time and inspect
  the quality.
\end{itemize}

}
\hdashrule[0.5ex]{\textwidth}{1pt}{3mm}
  Expected Result \\
{\footnotesize
\begin{itemize}
\tightlist
\item
  TMA reaches the position
\item
  DIMM image quality is sufficient
\end{itemize}

}

\begin{tabular}{p{2cm}}
\toprule
Step 474  \\ \hline
\end{tabular}
 Description \\
{\footnotesize
\textbf{Point the TMA to (Az, El)-pattern position + DIMM pattern offset
\textbf{~and take DIMM images}\\
}

\begin{itemize}
\tightlist
\item
  Point the TMA back to {Pointing 13}⁠ at {0}⁠ + DIMM offset, {86.5}⁠ +
  DIMM offset.
\item
  While tracking, take DIMM images with XXXs exposure time and inspect
  the quality.
\end{itemize}

}
\hdashrule[0.5ex]{\textwidth}{1pt}{3mm}
  Expected Result \\
{\footnotesize
\begin{itemize}
\tightlist
\item
  TMA reaches the position
\item
  DIMM image quality is sufficient
\end{itemize}

}

\begin{tabular}{p{2cm}}
\toprule
Step 475  \\ \hline
\end{tabular}
 Description \\
{\footnotesize
\textbf{Point the TMA to (Az, El)-pattern position + DIMM pattern offset
\textbf{~and take DIMM images}\\
}

\begin{itemize}
\tightlist
\item
  Point the TMA back to {Pointing 28}⁠ at {270}⁠ + DIMM offset, {86.5}⁠
  + DIMM offset.
\item
  While tracking, take DIMM images with XXXs exposure time and inspect
  the quality.
\end{itemize}

}
\hdashrule[0.5ex]{\textwidth}{1pt}{3mm}
  Expected Result \\
{\footnotesize
\begin{itemize}
\tightlist
\item
  TMA reaches the position
\item
  DIMM image quality is sufficient
\end{itemize}

}

\begin{tabular}{p{2cm}}
\toprule
Step 476  \\ \hline
\end{tabular}
 Description \\
{\footnotesize
\textbf{Point the TMA to (Az, El)-pattern position + DIMM pattern offset
\textbf{~and take DIMM images}\\
}

\begin{itemize}
\tightlist
\item
  Point the TMA back to {Pointing 14}⁠ at {0}⁠ + DIMM offset, {75}⁠ +
  DIMM offset.
\item
  While tracking, take DIMM images with XXXs exposure time and inspect
  the quality.
\end{itemize}

}
\hdashrule[0.5ex]{\textwidth}{1pt}{3mm}
  Expected Result \\
{\footnotesize
\begin{itemize}
\tightlist
\item
  TMA reaches the position
\item
  DIMM image quality is sufficient
\end{itemize}

}

\begin{tabular}{p{2cm}}
\toprule
Step 477  \\ \hline
\end{tabular}
 Description \\
{\footnotesize
\textbf{Point the TMA to (Az, El)-pattern position + DIMM pattern offset
\textbf{~and take DIMM images}\\
}

\begin{itemize}
\tightlist
\item
  Point the TMA back to {Pointing 15}⁠ at {0}⁠ + DIMM offset, {45}⁠ +
  DIMM offset.
\item
  While tracking, take DIMM images with XXXs exposure time and inspect
  the quality.
\end{itemize}

}
\hdashrule[0.5ex]{\textwidth}{1pt}{3mm}
  Expected Result \\
{\footnotesize
\begin{itemize}
\tightlist
\item
  TMA reaches the position
\item
  DIMM image quality is sufficient
\end{itemize}

}

\begin{tabular}{p{2cm}}
\toprule
Step 478  \\ \hline
\end{tabular}
 Description \\
{\footnotesize
\textbf{Point the TMA to (Az, El)-pattern position + DIMM pattern offset
\textbf{~and take DIMM images}\\
}

\begin{itemize}
\tightlist
\item
  Point the TMA back to {Pointing 16}⁠ at {0}⁠ + DIMM offset, {15}⁠ +
  DIMM offset.
\item
  While tracking, take DIMM images with XXXs exposure time and inspect
  the quality.
\end{itemize}

}
\hdashrule[0.5ex]{\textwidth}{1pt}{3mm}
  Expected Result \\
{\footnotesize
\begin{itemize}
\tightlist
\item
  TMA reaches the position
\item
  DIMM image quality is sufficient
\end{itemize}

}

\begin{tabular}{p{2cm}}
\toprule
Step 479  \\ \hline
\end{tabular}
 Description \\
{\footnotesize
\textbf{Point the TMA to (Az, El)-pattern position + DIMM pattern offset
\textbf{~and take DIMM images}\\
}

\begin{itemize}
\tightlist
\item
  Point the TMA back to {Pointing 17}⁠ at {90}⁠ + DIMM offset, {15}⁠ +
  DIMM offset.
\item
  While tracking, take DIMM images with XXXs exposure time and inspect
  the quality.
\end{itemize}

}
\hdashrule[0.5ex]{\textwidth}{1pt}{3mm}
  Expected Result \\
{\footnotesize
\begin{itemize}
\tightlist
\item
  TMA reaches the position
\item
  DIMM image quality is sufficient
\end{itemize}

}

\begin{tabular}{p{2cm}}
\toprule
Step 480  \\ \hline
\end{tabular}
 Description \\
{\footnotesize
\textbf{Point the TMA to (Az, El)-pattern position + DIMM pattern offset
\textbf{~and take DIMM images}\\
}

\begin{itemize}
\tightlist
\item
  Point the TMA back to {Pointing 18}⁠ at {90}⁠ + DIMM offset, {45}⁠ +
  DIMM offset.
\item
  While tracking, take DIMM images with XXXs exposure time and inspect
  the quality.
\end{itemize}

}
\hdashrule[0.5ex]{\textwidth}{1pt}{3mm}
  Expected Result \\
{\footnotesize
\begin{itemize}
\tightlist
\item
  TMA reaches the position
\item
  DIMM image quality is sufficient
\end{itemize}

}

\begin{tabular}{p{2cm}}
\toprule
Step 481  \\ \hline
\end{tabular}
 Description \\
{\footnotesize
\textbf{Point the TMA to (Az, El)-pattern position + DIMM pattern offset
\textbf{~and take DIMM images}\\
}

\begin{itemize}
\tightlist
\item
  Point the TMA back to {Pointing 19}⁠ at {90}⁠ + DIMM offset, {75}⁠ +
  DIMM offset.
\item
  While tracking, take DIMM images with XXXs exposure time and inspect
  the quality.
\end{itemize}

}
\hdashrule[0.5ex]{\textwidth}{1pt}{3mm}
  Expected Result \\
{\footnotesize
\begin{itemize}
\tightlist
\item
  TMA reaches the position
\item
  DIMM image quality is sufficient
\end{itemize}

}

\begin{tabular}{p{2cm}}
\toprule
Step 482  \\ \hline
\end{tabular}
 Description \\
{\footnotesize
\textbf{Point the TMA to (Az, El)-pattern position + DIMM pattern offset
\textbf{~and take DIMM images}\\
}

\begin{itemize}
\tightlist
\item
  Point the TMA back to {Pointing 20}⁠ at {90}⁠ + DIMM offset, {86.5}⁠ +
  DIMM offset.
\item
  While tracking, take DIMM images with XXXs exposure time and inspect
  the quality.
\end{itemize}

}
\hdashrule[0.5ex]{\textwidth}{1pt}{3mm}
  Expected Result \\
{\footnotesize
\begin{itemize}
\tightlist
\item
  TMA reaches the position
\item
  DIMM image quality is sufficient
\end{itemize}

}

\begin{tabular}{p{2cm}}
\toprule
Step 483  \\ \hline
\end{tabular}
 Description \\
{\footnotesize
\textbf{Point the TMA to (Az, El)-pattern position + DIMM pattern offset
\textbf{~and take DIMM images}\\
}

\begin{itemize}
\tightlist
\item
  Point the TMA back to {Pointing 21}⁠ at {180}⁠ + DIMM offset, {86.5}⁠
  + DIMM offset.
\item
  While tracking, take DIMM images with XXXs exposure time and inspect
  the quality.
\end{itemize}

}
\hdashrule[0.5ex]{\textwidth}{1pt}{3mm}
  Expected Result \\
{\footnotesize
\begin{itemize}
\tightlist
\item
  TMA reaches the position
\item
  DIMM image quality is sufficient
\end{itemize}

}

\begin{tabular}{p{2cm}}
\toprule
Step 484  \\ \hline
\end{tabular}
 Description \\
{\footnotesize
\textbf{Point the TMA to (Az, El)-pattern position + DIMM pattern offset
\textbf{~and take DIMM images}\\
}

\begin{itemize}
\tightlist
\item
  Point the TMA back to {Pointing 22}⁠ at {180}⁠ + DIMM offset, {75}⁠ +
  DIMM offset.
\item
  While tracking, take DIMM images with XXXs exposure time and inspect
  the quality.
\end{itemize}

}
\hdashrule[0.5ex]{\textwidth}{1pt}{3mm}
  Expected Result \\
{\footnotesize
\begin{itemize}
\tightlist
\item
  TMA reaches the position
\item
  DIMM image quality is sufficient
\end{itemize}

}

\begin{tabular}{p{2cm}}
\toprule
Step 485  \\ \hline
\end{tabular}
 Description \\
{\footnotesize
\textbf{Point the TMA to (Az, El)-pattern position + DIMM pattern offset
\textbf{~and take DIMM images}\\
}

\begin{itemize}
\tightlist
\item
  Point the TMA back to {Pointing 23}⁠ at {180}⁠ + DIMM offset, {45}⁠ +
  DIMM offset.
\item
  While tracking, take DIMM images with XXXs exposure time and inspect
  the quality.
\end{itemize}

}
\hdashrule[0.5ex]{\textwidth}{1pt}{3mm}
  Expected Result \\
{\footnotesize
\begin{itemize}
\tightlist
\item
  TMA reaches the position
\item
  DIMM image quality is sufficient
\end{itemize}

}

\begin{tabular}{p{2cm}}
\toprule
Step 486  \\ \hline
\end{tabular}
 Description \\
{\footnotesize
\textbf{Point the TMA to (Az, El)-pattern position + DIMM pattern offset
\textbf{~and take DIMM images}\\
}

\begin{itemize}
\tightlist
\item
  Point the TMA back to {Pointing 24}⁠ at {180}⁠ + DIMM offset, {15}⁠ +
  DIMM offset.
\item
  While tracking, take DIMM images with XXXs exposure time and inspect
  the quality.
\end{itemize}

}
\hdashrule[0.5ex]{\textwidth}{1pt}{3mm}
  Expected Result \\
{\footnotesize
\begin{itemize}
\tightlist
\item
  TMA reaches the position
\item
  DIMM image quality is sufficient
\end{itemize}

}

\begin{tabular}{p{2cm}}
\toprule
Step 487  \\ \hline
\end{tabular}
 Description \\
{\footnotesize
\textbf{Point the TMA to (Az, El)-pattern position + DIMM pattern offset
\textbf{~and take DIMM images}\\
}

\begin{itemize}
\tightlist
\item
  Point the TMA back to {Pointing 25}⁠ at {270}⁠ + DIMM offset, {15}⁠ +
  DIMM offset.
\item
  While tracking, take DIMM images with XXXs exposure time and inspect
  the quality.
\end{itemize}

}
\hdashrule[0.5ex]{\textwidth}{1pt}{3mm}
  Expected Result \\
{\footnotesize
\begin{itemize}
\tightlist
\item
  TMA reaches the position
\item
  DIMM image quality is sufficient
\end{itemize}

}

\begin{tabular}{p{2cm}}
\toprule
Step 488  \\ \hline
\end{tabular}
 Description \\
{\footnotesize
\textbf{Point the TMA to (Az, El)-pattern position + DIMM pattern offset
\textbf{~and take DIMM images}\\
}

\begin{itemize}
\tightlist
\item
  Point the TMA back to {Pointing 26}⁠ at {270}⁠ + DIMM offset, {45}⁠ +
  DIMM offset.
\item
  While tracking, take DIMM images with XXXs exposure time and inspect
  the quality.
\end{itemize}

}
\hdashrule[0.5ex]{\textwidth}{1pt}{3mm}
  Expected Result \\
{\footnotesize
\begin{itemize}
\tightlist
\item
  TMA reaches the position
\item
  DIMM image quality is sufficient
\end{itemize}

}

\begin{tabular}{p{2cm}}
\toprule
Step 489  \\ \hline
\end{tabular}
 Description \\
{\footnotesize
\textbf{Point the TMA to (Az, El)-pattern position + DIMM pattern offset
\textbf{~and take DIMM images}\\
}

\begin{itemize}
\tightlist
\item
  Point the TMA back to {Pointing 27}⁠ at {270}⁠ + DIMM offset, {75}⁠ +
  DIMM offset.
\item
  While tracking, take DIMM images with XXXs exposure time and inspect
  the quality.
\end{itemize}

}
\hdashrule[0.5ex]{\textwidth}{1pt}{3mm}
  Expected Result \\
{\footnotesize
\begin{itemize}
\tightlist
\item
  TMA reaches the position
\item
  DIMM image quality is sufficient
\end{itemize}

}

\begin{tabular}{p{2cm}}
\toprule
Step 490  \\ \hline
\end{tabular}
 Description \\
{\footnotesize
\textbf{Point the TMA to (Az, El)-pattern position + DIMM pattern offset
\textbf{~and take DIMM images}\\
}

\begin{itemize}
\tightlist
\item
  Point the TMA back to {Pointing 28}⁠ at {270}⁠ + DIMM offset, {86.5}⁠
  + DIMM offset.
\item
  While tracking, take DIMM images with XXXs exposure time and inspect
  the quality.
\end{itemize}

}
\hdashrule[0.5ex]{\textwidth}{1pt}{3mm}
  Expected Result \\
{\footnotesize
\begin{itemize}
\tightlist
\item
  TMA reaches the position
\item
  DIMM image quality is sufficient
\end{itemize}

}

\begin{tabular}{p{2cm}}
\toprule
Step 491  \\ \hline
\end{tabular}
 Description \\
{\footnotesize
\textbf{Move TMA to the 4. random distance of 3.5deg\\
}

\begin{itemize}
\tightlist
\item
  Point the TMA to a random 3.5 deg combined offset in AZ and EL from
  {Pointing 1}⁠ at {-270}⁠, {15}⁠. Record the exact position of the
  offset in AZ and El
\end{itemize}

}
\hdashrule[0.5ex]{\textwidth}{1pt}{3mm}
  Expected Result \\
{\footnotesize
\begin{itemize}
\tightlist
\item
  The TMA reaches the commanded offset position.
\end{itemize}

}

\begin{tabular}{p{2cm}}
\toprule
Step 492  \\ \hline
\end{tabular}
 Description \\
{\footnotesize
\textbf{Move TMA to the 4. random distance of 3.5deg\\
}

\begin{itemize}
\tightlist
\item
  Point the TMA to a random 3.5 deg combined offset in AZ and EL from
  {Pointing 2}⁠ at {-270}⁠, {45}⁠. Record the exact position of the
  offset in AZ and El
\end{itemize}

}
\hdashrule[0.5ex]{\textwidth}{1pt}{3mm}
  Expected Result \\
{\footnotesize
\begin{itemize}
\tightlist
\item
  The TMA reaches the commanded offset position.
\end{itemize}

}

\begin{tabular}{p{2cm}}
\toprule
Step 493  \\ \hline
\end{tabular}
 Description \\
{\footnotesize
\textbf{Move TMA to the 4. random distance of 3.5deg\\
}

\begin{itemize}
\tightlist
\item
  Point the TMA to a random 3.5 deg combined offset in AZ and EL from
  {Pointing 3}⁠ at {-270}⁠, {75}⁠. Record the exact position of the
  offset in AZ and El
\end{itemize}

}
\hdashrule[0.5ex]{\textwidth}{1pt}{3mm}
  Expected Result \\
{\footnotesize
\begin{itemize}
\tightlist
\item
  The TMA reaches the commanded offset position.
\end{itemize}

}

\begin{tabular}{p{2cm}}
\toprule
Step 494  \\ \hline
\end{tabular}
 Description \\
{\footnotesize
\textbf{Move TMA to the 4. random distance of 3.5deg\\
}

\begin{itemize}
\tightlist
\item
  Point the TMA to a random 3.5 deg combined offset in AZ and EL from
  {Pointing 4}⁠ at {-270}⁠, {86.5}⁠. Record the exact position of the
  offset in AZ and El
\end{itemize}

}
\hdashrule[0.5ex]{\textwidth}{1pt}{3mm}
  Expected Result \\
{\footnotesize
\begin{itemize}
\tightlist
\item
  The TMA reaches the commanded offset position.
\end{itemize}

}

\begin{tabular}{p{2cm}}
\toprule
Step 495  \\ \hline
\end{tabular}
 Description \\
{\footnotesize
\textbf{Move TMA to the 4. random distance of 3.5deg\\
}

\begin{itemize}
\tightlist
\item
  Point the TMA to a random 3.5 deg combined offset in AZ and EL from
  {Pointing 5}⁠ at {-180}⁠, {86.5}⁠. Record the exact position of the
  offset in AZ and El
\end{itemize}

}
\hdashrule[0.5ex]{\textwidth}{1pt}{3mm}
  Expected Result \\
{\footnotesize
\begin{itemize}
\tightlist
\item
  The TMA reaches the commanded offset position.
\end{itemize}

}

\begin{tabular}{p{2cm}}
\toprule
Step 496  \\ \hline
\end{tabular}
 Description \\
{\footnotesize
\textbf{Move TMA to the 4. random distance of 3.5deg\\
}

\begin{itemize}
\tightlist
\item
  Point the TMA to a random 3.5 deg combined offset in AZ and EL from
  {Pointing 6}⁠ at {-180}⁠, {75}⁠. Record the exact position of the
  offset in AZ and El
\end{itemize}

}
\hdashrule[0.5ex]{\textwidth}{1pt}{3mm}
  Expected Result \\
{\footnotesize
\begin{itemize}
\tightlist
\item
  The TMA reaches the commanded offset position.
\end{itemize}

}

\begin{tabular}{p{2cm}}
\toprule
Step 497  \\ \hline
\end{tabular}
 Description \\
{\footnotesize
\textbf{Move TMA to the 4. random distance of 3.5deg\\
}

\begin{itemize}
\tightlist
\item
  Point the TMA to a random 3.5 deg combined offset in AZ and EL from
  {Pointing 7}⁠ at {-180}⁠, {45}⁠. Record the exact position of the
  offset in AZ and El
\end{itemize}

}
\hdashrule[0.5ex]{\textwidth}{1pt}{3mm}
  Expected Result \\
{\footnotesize
\begin{itemize}
\tightlist
\item
  The TMA reaches the commanded offset position.
\end{itemize}

}

\begin{tabular}{p{2cm}}
\toprule
Step 498  \\ \hline
\end{tabular}
 Description \\
{\footnotesize
\textbf{Move TMA to the 4. random distance of 3.5deg\\
}

\begin{itemize}
\tightlist
\item
  Point the TMA to a random 3.5 deg combined offset in AZ and EL from
  {Pointing 8}⁠ at {-180}⁠, {15}⁠. Record the exact position of the
  offset in AZ and El
\end{itemize}

}
\hdashrule[0.5ex]{\textwidth}{1pt}{3mm}
  Expected Result \\
{\footnotesize
\begin{itemize}
\tightlist
\item
  The TMA reaches the commanded offset position.
\end{itemize}

}

\begin{tabular}{p{2cm}}
\toprule
Step 499  \\ \hline
\end{tabular}
 Description \\
{\footnotesize
\textbf{Move TMA to the 4. random distance of 3.5deg\\
}

\begin{itemize}
\tightlist
\item
  Point the TMA to a random 3.5 deg combined offset in AZ and EL from
  {Pointing 9}⁠ at {-90}⁠, {15}⁠. Record the exact position of the
  offset in AZ and El
\end{itemize}

}
\hdashrule[0.5ex]{\textwidth}{1pt}{3mm}
  Expected Result \\
{\footnotesize
\begin{itemize}
\tightlist
\item
  The TMA reaches the commanded offset position.
\end{itemize}

}

\begin{tabular}{p{2cm}}
\toprule
Step 500  \\ \hline
\end{tabular}
 Description \\
{\footnotesize
\textbf{Move TMA to the 4. random distance of 3.5deg\\
}

\begin{itemize}
\tightlist
\item
  Point the TMA to a random 3.5 deg combined offset in AZ and EL from
  {Pointing 10}⁠ at {-90}⁠, {45}⁠. Record the exact position of the
  offset in AZ and El
\end{itemize}

}
\hdashrule[0.5ex]{\textwidth}{1pt}{3mm}
  Expected Result \\
{\footnotesize
\begin{itemize}
\tightlist
\item
  The TMA reaches the commanded offset position.
\end{itemize}

}

\begin{tabular}{p{2cm}}
\toprule
Step 501  \\ \hline
\end{tabular}
 Description \\
{\footnotesize
\textbf{Move TMA to the 4. random distance of 3.5deg\\
}

\begin{itemize}
\tightlist
\item
  Point the TMA to a random 3.5 deg combined offset in AZ and EL from
  {Pointing 11}⁠ at {-90}⁠, {75}⁠. Record the exact position of the
  offset in AZ and El
\end{itemize}

}
\hdashrule[0.5ex]{\textwidth}{1pt}{3mm}
  Expected Result \\
{\footnotesize
\begin{itemize}
\tightlist
\item
  The TMA reaches the commanded offset position.
\end{itemize}

}

\begin{tabular}{p{2cm}}
\toprule
Step 502  \\ \hline
\end{tabular}
 Description \\
{\footnotesize
\textbf{Move TMA to the 4. random distance of 3.5deg\\
}

\begin{itemize}
\tightlist
\item
  Point the TMA to a random 3.5 deg combined offset in AZ and EL from
  {Pointing 12}⁠ at {-90}⁠, {86.5}⁠. Record the exact position of the
  offset in AZ and El
\end{itemize}

}
\hdashrule[0.5ex]{\textwidth}{1pt}{3mm}
  Expected Result \\
{\footnotesize
\begin{itemize}
\tightlist
\item
  The TMA reaches the commanded offset position.
\end{itemize}

}

\begin{tabular}{p{2cm}}
\toprule
Step 503  \\ \hline
\end{tabular}
 Description \\
{\footnotesize
\textbf{Move TMA to the 4. random distance of 3.5deg\\
}

\begin{itemize}
\tightlist
\item
  Point the TMA to a random 3.5 deg combined offset in AZ and EL from
  {Pointing 13}⁠ at {0}⁠, {86.5}⁠. Record the exact position of the
  offset in AZ and El
\end{itemize}

}
\hdashrule[0.5ex]{\textwidth}{1pt}{3mm}
  Expected Result \\
{\footnotesize
\begin{itemize}
\tightlist
\item
  The TMA reaches the commanded offset position.
\end{itemize}

}

\begin{tabular}{p{2cm}}
\toprule
Step 504  \\ \hline
\end{tabular}
 Description \\
{\footnotesize
\textbf{Move TMA to the 4. random distance of 3.5deg\\
}

\begin{itemize}
\tightlist
\item
  Point the TMA to a random 3.5 deg combined offset in AZ and EL from
  {Pointing 14}⁠ at {0}⁠, {75}⁠. Record the exact position of the offset
  in AZ and El
\end{itemize}

}
\hdashrule[0.5ex]{\textwidth}{1pt}{3mm}
  Expected Result \\
{\footnotesize
\begin{itemize}
\tightlist
\item
  The TMA reaches the commanded offset position.
\end{itemize}

}

\begin{tabular}{p{2cm}}
\toprule
Step 505  \\ \hline
\end{tabular}
 Description \\
{\footnotesize
\textbf{Move TMA to the 4. random distance of 3.5deg\\
}

\begin{itemize}
\tightlist
\item
  Point the TMA to a random 3.5 deg combined offset in AZ and EL from
  {Pointing 15}⁠ at {0}⁠, {45}⁠. Record the exact position of the offset
  in AZ and El
\end{itemize}

}
\hdashrule[0.5ex]{\textwidth}{1pt}{3mm}
  Expected Result \\
{\footnotesize
\begin{itemize}
\tightlist
\item
  The TMA reaches the commanded offset position.
\end{itemize}

}

\begin{tabular}{p{2cm}}
\toprule
Step 506  \\ \hline
\end{tabular}
 Description \\
{\footnotesize
\textbf{Move TMA to the 4. random distance of 3.5deg\\
}

\begin{itemize}
\tightlist
\item
  Point the TMA to a random 3.5 deg combined offset in AZ and EL from
  {Pointing 16}⁠ at {0}⁠, {15}⁠. Record the exact position of the offset
  in AZ and El
\end{itemize}

}
\hdashrule[0.5ex]{\textwidth}{1pt}{3mm}
  Expected Result \\
{\footnotesize
\begin{itemize}
\tightlist
\item
  The TMA reaches the commanded offset position.
\end{itemize}

}

\begin{tabular}{p{2cm}}
\toprule
Step 507  \\ \hline
\end{tabular}
 Description \\
{\footnotesize
\textbf{Move TMA to the 4. random distance of 3.5deg\\
}

\begin{itemize}
\tightlist
\item
  Point the TMA to a random 3.5 deg combined offset in AZ and EL from
  {Pointing 17}⁠ at {90}⁠, {15}⁠. Record the exact position of the
  offset in AZ and El
\end{itemize}

}
\hdashrule[0.5ex]{\textwidth}{1pt}{3mm}
  Expected Result \\
{\footnotesize
\begin{itemize}
\tightlist
\item
  The TMA reaches the commanded offset position.
\end{itemize}

}

\begin{tabular}{p{2cm}}
\toprule
Step 508  \\ \hline
\end{tabular}
 Description \\
{\footnotesize
\textbf{Move TMA to the 4. random distance of 3.5deg\\
}

\begin{itemize}
\tightlist
\item
  Point the TMA to a random 3.5 deg combined offset in AZ and EL from
  {Pointing 18}⁠ at {90}⁠, {45}⁠. Record the exact position of the
  offset in AZ and El
\end{itemize}

}
\hdashrule[0.5ex]{\textwidth}{1pt}{3mm}
  Expected Result \\
{\footnotesize
\begin{itemize}
\tightlist
\item
  The TMA reaches the commanded offset position.
\end{itemize}

}

\begin{tabular}{p{2cm}}
\toprule
Step 509  \\ \hline
\end{tabular}
 Description \\
{\footnotesize
\textbf{Move TMA to the 4. random distance of 3.5deg\\
}

\begin{itemize}
\tightlist
\item
  Point the TMA to a random 3.5 deg combined offset in AZ and EL from
  {Pointing 19}⁠ at {90}⁠, {75}⁠. Record the exact position of the
  offset in AZ and El
\end{itemize}

}
\hdashrule[0.5ex]{\textwidth}{1pt}{3mm}
  Expected Result \\
{\footnotesize
\begin{itemize}
\tightlist
\item
  The TMA reaches the commanded offset position.
\end{itemize}

}

\begin{tabular}{p{2cm}}
\toprule
Step 510  \\ \hline
\end{tabular}
 Description \\
{\footnotesize
\textbf{Move TMA to the 4. random distance of 3.5deg\\
}

\begin{itemize}
\tightlist
\item
  Point the TMA to a random 3.5 deg combined offset in AZ and EL from
  {Pointing 20}⁠ at {90}⁠, {86.5}⁠. Record the exact position of the
  offset in AZ and El
\end{itemize}

}
\hdashrule[0.5ex]{\textwidth}{1pt}{3mm}
  Expected Result \\
{\footnotesize
\begin{itemize}
\tightlist
\item
  The TMA reaches the commanded offset position.
\end{itemize}

}

\begin{tabular}{p{2cm}}
\toprule
Step 511  \\ \hline
\end{tabular}
 Description \\
{\footnotesize
\textbf{Move TMA to the 4. random distance of 3.5deg\\
}

\begin{itemize}
\tightlist
\item
  Point the TMA to a random 3.5 deg combined offset in AZ and EL from
  {Pointing 21}⁠ at {180}⁠, {86.5}⁠. Record the exact position of the
  offset in AZ and El
\end{itemize}

}
\hdashrule[0.5ex]{\textwidth}{1pt}{3mm}
  Expected Result \\
{\footnotesize
\begin{itemize}
\tightlist
\item
  The TMA reaches the commanded offset position.
\end{itemize}

}

\begin{tabular}{p{2cm}}
\toprule
Step 512  \\ \hline
\end{tabular}
 Description \\
{\footnotesize
\textbf{Move TMA to the 4. random distance of 3.5deg\\
}

\begin{itemize}
\tightlist
\item
  Point the TMA to a random 3.5 deg combined offset in AZ and EL from
  {Pointing 22}⁠ at {180}⁠, {75}⁠. Record the exact position of the
  offset in AZ and El
\end{itemize}

}
\hdashrule[0.5ex]{\textwidth}{1pt}{3mm}
  Expected Result \\
{\footnotesize
\begin{itemize}
\tightlist
\item
  The TMA reaches the commanded offset position.
\end{itemize}

}

\begin{tabular}{p{2cm}}
\toprule
Step 513  \\ \hline
\end{tabular}
 Description \\
{\footnotesize
\textbf{Move TMA to the 4. random distance of 3.5deg\\
}

\begin{itemize}
\tightlist
\item
  Point the TMA to a random 3.5 deg combined offset in AZ and EL from
  {Pointing 23}⁠ at {180}⁠, {45}⁠. Record the exact position of the
  offset in AZ and El
\end{itemize}

}
\hdashrule[0.5ex]{\textwidth}{1pt}{3mm}
  Expected Result \\
{\footnotesize
\begin{itemize}
\tightlist
\item
  The TMA reaches the commanded offset position.
\end{itemize}

}

\begin{tabular}{p{2cm}}
\toprule
Step 514  \\ \hline
\end{tabular}
 Description \\
{\footnotesize
\textbf{Move TMA to the 4. random distance of 3.5deg\\
}

\begin{itemize}
\tightlist
\item
  Point the TMA to a random 3.5 deg combined offset in AZ and EL from
  {Pointing 24}⁠ at {180}⁠, {15}⁠. Record the exact position of the
  offset in AZ and El
\end{itemize}

}
\hdashrule[0.5ex]{\textwidth}{1pt}{3mm}
  Expected Result \\
{\footnotesize
\begin{itemize}
\tightlist
\item
  The TMA reaches the commanded offset position.
\end{itemize}

}

\begin{tabular}{p{2cm}}
\toprule
Step 515  \\ \hline
\end{tabular}
 Description \\
{\footnotesize
\textbf{Move TMA to the 4. random distance of 3.5deg\\
}

\begin{itemize}
\tightlist
\item
  Point the TMA to a random 3.5 deg combined offset in AZ and EL from
  {Pointing 25}⁠ at {270}⁠, {15}⁠. Record the exact position of the
  offset in AZ and El
\end{itemize}

}
\hdashrule[0.5ex]{\textwidth}{1pt}{3mm}
  Expected Result \\
{\footnotesize
\begin{itemize}
\tightlist
\item
  The TMA reaches the commanded offset position.
\end{itemize}

}

\begin{tabular}{p{2cm}}
\toprule
Step 516  \\ \hline
\end{tabular}
 Description \\
{\footnotesize
\textbf{Move TMA to the 4. random distance of 3.5deg\\
}

\begin{itemize}
\tightlist
\item
  Point the TMA to a random 3.5 deg combined offset in AZ and EL from
  {Pointing 26}⁠ at {270}⁠, {45}⁠. Record the exact position of the
  offset in AZ and El
\end{itemize}

}
\hdashrule[0.5ex]{\textwidth}{1pt}{3mm}
  Expected Result \\
{\footnotesize
\begin{itemize}
\tightlist
\item
  The TMA reaches the commanded offset position.
\end{itemize}

}

\begin{tabular}{p{2cm}}
\toprule
Step 517  \\ \hline
\end{tabular}
 Description \\
{\footnotesize
\textbf{Move TMA to the 4. random distance of 3.5deg\\
}

\begin{itemize}
\tightlist
\item
  Point the TMA to a random 3.5 deg combined offset in AZ and EL from
  {Pointing 27}⁠ at {270}⁠, {75}⁠. Record the exact position of the
  offset in AZ and El
\end{itemize}

}
\hdashrule[0.5ex]{\textwidth}{1pt}{3mm}
  Expected Result \\
{\footnotesize
\begin{itemize}
\tightlist
\item
  The TMA reaches the commanded offset position.
\end{itemize}

}

\begin{tabular}{p{2cm}}
\toprule
Step 518  \\ \hline
\end{tabular}
 Description \\
{\footnotesize
\textbf{Move TMA to the 4. random distance of 3.5deg\\
}

\begin{itemize}
\tightlist
\item
  Point the TMA to a random 3.5 deg combined offset in AZ and EL from
  {Pointing 13}⁠ at {0}⁠, {86.5}⁠. Record the exact position of the
  offset in AZ and El
\end{itemize}

}
\hdashrule[0.5ex]{\textwidth}{1pt}{3mm}
  Expected Result \\
{\footnotesize
\begin{itemize}
\tightlist
\item
  The TMA reaches the commanded offset position.
\end{itemize}

}

\begin{tabular}{p{2cm}}
\toprule
Step 519  \\ \hline
\end{tabular}
 Description \\
{\footnotesize
\textbf{Move TMA to the 4. random distance of 3.5deg\\
}

\begin{itemize}
\tightlist
\item
  Point the TMA to a random 3.5 deg combined offset in AZ and EL from
  {Pointing 28}⁠ at {270}⁠, {86.5}⁠. Record the exact position of the
  offset in AZ and El
\end{itemize}

}
\hdashrule[0.5ex]{\textwidth}{1pt}{3mm}
  Expected Result \\
{\footnotesize
\begin{itemize}
\tightlist
\item
  The TMA reaches the commanded offset position.
\end{itemize}

}

\begin{tabular}{p{2cm}}
\toprule
Step 520  \\ \hline
\end{tabular}
 Description \\
{\footnotesize
\textbf{Move TMA to the 4. random distance of 3.5deg\\
}

\begin{itemize}
\tightlist
\item
  Point the TMA to a random 3.5 deg combined offset in AZ and EL from
  {Pointing 14}⁠ at {0}⁠, {75}⁠. Record the exact position of the offset
  in AZ and El
\end{itemize}

}
\hdashrule[0.5ex]{\textwidth}{1pt}{3mm}
  Expected Result \\
{\footnotesize
\begin{itemize}
\tightlist
\item
  The TMA reaches the commanded offset position.
\end{itemize}

}

\begin{tabular}{p{2cm}}
\toprule
Step 521  \\ \hline
\end{tabular}
 Description \\
{\footnotesize
\textbf{Move TMA to the 4. random distance of 3.5deg\\
}

\begin{itemize}
\tightlist
\item
  Point the TMA to a random 3.5 deg combined offset in AZ and EL from
  {Pointing 15}⁠ at {0}⁠, {45}⁠. Record the exact position of the offset
  in AZ and El
\end{itemize}

}
\hdashrule[0.5ex]{\textwidth}{1pt}{3mm}
  Expected Result \\
{\footnotesize
\begin{itemize}
\tightlist
\item
  The TMA reaches the commanded offset position.
\end{itemize}

}

\begin{tabular}{p{2cm}}
\toprule
Step 522  \\ \hline
\end{tabular}
 Description \\
{\footnotesize
\textbf{Move TMA to the 4. random distance of 3.5deg\\
}

\begin{itemize}
\tightlist
\item
  Point the TMA to a random 3.5 deg combined offset in AZ and EL from
  {Pointing 16}⁠ at {0}⁠, {15}⁠. Record the exact position of the offset
  in AZ and El
\end{itemize}

}
\hdashrule[0.5ex]{\textwidth}{1pt}{3mm}
  Expected Result \\
{\footnotesize
\begin{itemize}
\tightlist
\item
  The TMA reaches the commanded offset position.
\end{itemize}

}

\begin{tabular}{p{2cm}}
\toprule
Step 523  \\ \hline
\end{tabular}
 Description \\
{\footnotesize
\textbf{Move TMA to the 4. random distance of 3.5deg\\
}

\begin{itemize}
\tightlist
\item
  Point the TMA to a random 3.5 deg combined offset in AZ and EL from
  {Pointing 17}⁠ at {90}⁠, {15}⁠. Record the exact position of the
  offset in AZ and El
\end{itemize}

}
\hdashrule[0.5ex]{\textwidth}{1pt}{3mm}
  Expected Result \\
{\footnotesize
\begin{itemize}
\tightlist
\item
  The TMA reaches the commanded offset position.
\end{itemize}

}

\begin{tabular}{p{2cm}}
\toprule
Step 524  \\ \hline
\end{tabular}
 Description \\
{\footnotesize
\textbf{Move TMA to the 4. random distance of 3.5deg\\
}

\begin{itemize}
\tightlist
\item
  Point the TMA to a random 3.5 deg combined offset in AZ and EL from
  {Pointing 18}⁠ at {90}⁠, {45}⁠. Record the exact position of the
  offset in AZ and El
\end{itemize}

}
\hdashrule[0.5ex]{\textwidth}{1pt}{3mm}
  Expected Result \\
{\footnotesize
\begin{itemize}
\tightlist
\item
  The TMA reaches the commanded offset position.
\end{itemize}

}

\begin{tabular}{p{2cm}}
\toprule
Step 525  \\ \hline
\end{tabular}
 Description \\
{\footnotesize
\textbf{Move TMA to the 4. random distance of 3.5deg\\
}

\begin{itemize}
\tightlist
\item
  Point the TMA to a random 3.5 deg combined offset in AZ and EL from
  {Pointing 19}⁠ at {90}⁠, {75}⁠. Record the exact position of the
  offset in AZ and El
\end{itemize}

}
\hdashrule[0.5ex]{\textwidth}{1pt}{3mm}
  Expected Result \\
{\footnotesize
\begin{itemize}
\tightlist
\item
  The TMA reaches the commanded offset position.
\end{itemize}

}

\begin{tabular}{p{2cm}}
\toprule
Step 526  \\ \hline
\end{tabular}
 Description \\
{\footnotesize
\textbf{Move TMA to the 4. random distance of 3.5deg\\
}

\begin{itemize}
\tightlist
\item
  Point the TMA to a random 3.5 deg combined offset in AZ and EL from
  {Pointing 20}⁠ at {90}⁠, {86.5}⁠. Record the exact position of the
  offset in AZ and El
\end{itemize}

}
\hdashrule[0.5ex]{\textwidth}{1pt}{3mm}
  Expected Result \\
{\footnotesize
\begin{itemize}
\tightlist
\item
  The TMA reaches the commanded offset position.
\end{itemize}

}

\begin{tabular}{p{2cm}}
\toprule
Step 527  \\ \hline
\end{tabular}
 Description \\
{\footnotesize
\textbf{Move TMA to the 4. random distance of 3.5deg\\
}

\begin{itemize}
\tightlist
\item
  Point the TMA to a random 3.5 deg combined offset in AZ and EL from
  {Pointing 21}⁠ at {180}⁠, {86.5}⁠. Record the exact position of the
  offset in AZ and El
\end{itemize}

}
\hdashrule[0.5ex]{\textwidth}{1pt}{3mm}
  Expected Result \\
{\footnotesize
\begin{itemize}
\tightlist
\item
  The TMA reaches the commanded offset position.
\end{itemize}

}

\begin{tabular}{p{2cm}}
\toprule
Step 528  \\ \hline
\end{tabular}
 Description \\
{\footnotesize
\textbf{Move TMA to the 4. random distance of 3.5deg\\
}

\begin{itemize}
\tightlist
\item
  Point the TMA to a random 3.5 deg combined offset in AZ and EL from
  {Pointing 22}⁠ at {180}⁠, {75}⁠. Record the exact position of the
  offset in AZ and El
\end{itemize}

}
\hdashrule[0.5ex]{\textwidth}{1pt}{3mm}
  Expected Result \\
{\footnotesize
\begin{itemize}
\tightlist
\item
  The TMA reaches the commanded offset position.
\end{itemize}

}

\begin{tabular}{p{2cm}}
\toprule
Step 529  \\ \hline
\end{tabular}
 Description \\
{\footnotesize
\textbf{Move TMA to the 4. random distance of 3.5deg\\
}

\begin{itemize}
\tightlist
\item
  Point the TMA to a random 3.5 deg combined offset in AZ and EL from
  {Pointing 23}⁠ at {180}⁠, {45}⁠. Record the exact position of the
  offset in AZ and El
\end{itemize}

}
\hdashrule[0.5ex]{\textwidth}{1pt}{3mm}
  Expected Result \\
{\footnotesize
\begin{itemize}
\tightlist
\item
  The TMA reaches the commanded offset position.
\end{itemize}

}

\begin{tabular}{p{2cm}}
\toprule
Step 530  \\ \hline
\end{tabular}
 Description \\
{\footnotesize
\textbf{Move TMA to the 4. random distance of 3.5deg\\
}

\begin{itemize}
\tightlist
\item
  Point the TMA to a random 3.5 deg combined offset in AZ and EL from
  {Pointing 24}⁠ at {180}⁠, {15}⁠. Record the exact position of the
  offset in AZ and El
\end{itemize}

}
\hdashrule[0.5ex]{\textwidth}{1pt}{3mm}
  Expected Result \\
{\footnotesize
\begin{itemize}
\tightlist
\item
  The TMA reaches the commanded offset position.
\end{itemize}

}

\begin{tabular}{p{2cm}}
\toprule
Step 531  \\ \hline
\end{tabular}
 Description \\
{\footnotesize
\textbf{Move TMA to the 4. random distance of 3.5deg\\
}

\begin{itemize}
\tightlist
\item
  Point the TMA to a random 3.5 deg combined offset in AZ and EL from
  {Pointing 25}⁠ at {270}⁠, {15}⁠. Record the exact position of the
  offset in AZ and El
\end{itemize}

}
\hdashrule[0.5ex]{\textwidth}{1pt}{3mm}
  Expected Result \\
{\footnotesize
\begin{itemize}
\tightlist
\item
  The TMA reaches the commanded offset position.
\end{itemize}

}

\begin{tabular}{p{2cm}}
\toprule
Step 532  \\ \hline
\end{tabular}
 Description \\
{\footnotesize
\textbf{Move TMA to the 4. random distance of 3.5deg\\
}

\begin{itemize}
\tightlist
\item
  Point the TMA to a random 3.5 deg combined offset in AZ and EL from
  {Pointing 26}⁠ at {270}⁠, {45}⁠. Record the exact position of the
  offset in AZ and El
\end{itemize}

}
\hdashrule[0.5ex]{\textwidth}{1pt}{3mm}
  Expected Result \\
{\footnotesize
\begin{itemize}
\tightlist
\item
  The TMA reaches the commanded offset position.
\end{itemize}

}

\begin{tabular}{p{2cm}}
\toprule
Step 533  \\ \hline
\end{tabular}
 Description \\
{\footnotesize
\textbf{Move TMA to the 4. random distance of 3.5deg\\
}

\begin{itemize}
\tightlist
\item
  Point the TMA to a random 3.5 deg combined offset in AZ and EL from
  {Pointing 27}⁠ at {270}⁠, {75}⁠. Record the exact position of the
  offset in AZ and El
\end{itemize}

}
\hdashrule[0.5ex]{\textwidth}{1pt}{3mm}
  Expected Result \\
{\footnotesize
\begin{itemize}
\tightlist
\item
  The TMA reaches the commanded offset position.
\end{itemize}

}

\begin{tabular}{p{2cm}}
\toprule
Step 534  \\ \hline
\end{tabular}
 Description \\
{\footnotesize
\textbf{Move TMA to the 4. random distance of 3.5deg\\
}

\begin{itemize}
\tightlist
\item
  Point the TMA to a random 3.5 deg combined offset in AZ and EL from
  {Pointing 28}⁠ at {270}⁠, {86.5}⁠. Record the exact position of the
  offset in AZ and El
\end{itemize}

}
\hdashrule[0.5ex]{\textwidth}{1pt}{3mm}
  Expected Result \\
{\footnotesize
\begin{itemize}
\tightlist
\item
  The TMA reaches the commanded offset position.
\end{itemize}

}

\begin{tabular}{p{2cm}}
\toprule
Step 535  \\ \hline
\end{tabular}
 Description \\
{\footnotesize
Wait for the Dome to reach the commanded position.

}
\hdashrule[0.5ex]{\textwidth}{1pt}{3mm}
  Expected Result \\
{\footnotesize
The \emph{MTDome\_logevent\_azMotion} and
\emph{MTDome\_logevent\_elMotion} inPosition parameter = true.

}

\begin{tabular}{p{2cm}}
\toprule
Step 536  \\ \hline
\end{tabular}
 Description \\
{\footnotesize
\textbf{Find DIMM Object and DIMM Offset}\\

\begin{itemize}
\tightlist
\item
  While tracking, take a 10-sec exposure with the StarTracker.
\item
  Load the image into an image viewer.
\item
  Overlay the GAIA catalog.
\item
  Select a star brighter than XXX mag (bright enough for the DIMM).
\item
  Calculate the pixel offset between the StarTracker and the DIMM.
\item
  Transform the offset into AZ and EL offsets.
\end{itemize}

}
\hdashrule[0.5ex]{\textwidth}{1pt}{3mm}
  Expected Result \\
{\footnotesize
\begin{itemize}
\tightlist
\item
  An image was successfully taken with the StarTracker and is of
  sufficient quality.
\item
  AZ and EL offsets are available.
\end{itemize}

}

\begin{tabular}{p{2cm}}
\toprule
Step 537  \\ \hline
\end{tabular}
 Description \\
{\footnotesize
\textbf{Find DIMM Object and DIMM Offset}\\

\begin{itemize}
\tightlist
\item
  While tracking, take a 10-sec exposure with the StarTracker.
\item
  Load the image into an image viewer.
\item
  Overlay the GAIA catalog.
\item
  Select a star brighter than XXX mag (bright enough for the DIMM).
\item
  Calculate the pixel offset between the StarTracker and the DIMM.
\item
  Transform the offset into AZ and EL offsets.
\end{itemize}

}
\hdashrule[0.5ex]{\textwidth}{1pt}{3mm}
  Expected Result \\
{\footnotesize
\begin{itemize}
\tightlist
\item
  An image was successfully taken with the StarTracker and is of
  sufficient quality.
\item
  AZ and EL offsets are available.
\end{itemize}

}

\begin{tabular}{p{2cm}}
\toprule
Step 538  \\ \hline
\end{tabular}
 Description \\
{\footnotesize
\textbf{Find DIMM Object and DIMM Offset}\\

\begin{itemize}
\tightlist
\item
  While tracking, take a 10-sec exposure with the StarTracker.
\item
  Load the image into an image viewer.
\item
  Overlay the GAIA catalog.
\item
  Select a star brighter than XXX mag (bright enough for the DIMM).
\item
  Calculate the pixel offset between the StarTracker and the DIMM.
\item
  Transform the offset into AZ and EL offsets.
\end{itemize}

}
\hdashrule[0.5ex]{\textwidth}{1pt}{3mm}
  Expected Result \\
{\footnotesize
\begin{itemize}
\tightlist
\item
  An image was successfully taken with the StarTracker and is of
  sufficient quality.
\item
  AZ and EL offsets are available.
\end{itemize}

}

\begin{tabular}{p{2cm}}
\toprule
Step 539  \\ \hline
\end{tabular}
 Description \\
{\footnotesize
\textbf{Find DIMM Object and DIMM Offset}\\

\begin{itemize}
\tightlist
\item
  While tracking, take a 10-sec exposure with the StarTracker.
\item
  Load the image into an image viewer.
\item
  Overlay the GAIA catalog.
\item
  Select a star brighter than XXX mag (bright enough for the DIMM).
\item
  Calculate the pixel offset between the StarTracker and the DIMM.
\item
  Transform the offset into AZ and EL offsets.
\end{itemize}

}
\hdashrule[0.5ex]{\textwidth}{1pt}{3mm}
  Expected Result \\
{\footnotesize
\begin{itemize}
\tightlist
\item
  An image was successfully taken with the StarTracker and is of
  sufficient quality.
\item
  AZ and EL offsets are available.
\end{itemize}

}

\begin{tabular}{p{2cm}}
\toprule
Step 540  \\ \hline
\end{tabular}
 Description \\
{\footnotesize
\textbf{Find DIMM Object and DIMM Offset}\\

\begin{itemize}
\tightlist
\item
  While tracking, take a 10-sec exposure with the StarTracker.
\item
  Load the image into an image viewer.
\item
  Overlay the GAIA catalog.
\item
  Select a star brighter than XXX mag (bright enough for the DIMM).
\item
  Calculate the pixel offset between the StarTracker and the DIMM.
\item
  Transform the offset into AZ and EL offsets.
\end{itemize}

}
\hdashrule[0.5ex]{\textwidth}{1pt}{3mm}
  Expected Result \\
{\footnotesize
\begin{itemize}
\tightlist
\item
  An image was successfully taken with the StarTracker and is of
  sufficient quality.
\item
  AZ and EL offsets are available.
\end{itemize}

}

\begin{tabular}{p{2cm}}
\toprule
Step 541  \\ \hline
\end{tabular}
 Description \\
{\footnotesize
\textbf{Find DIMM Object and DIMM Offset}\\

\begin{itemize}
\tightlist
\item
  While tracking, take a 10-sec exposure with the StarTracker.
\item
  Load the image into an image viewer.
\item
  Overlay the GAIA catalog.
\item
  Select a star brighter than XXX mag (bright enough for the DIMM).
\item
  Calculate the pixel offset between the StarTracker and the DIMM.
\item
  Transform the offset into AZ and EL offsets.
\end{itemize}

}
\hdashrule[0.5ex]{\textwidth}{1pt}{3mm}
  Expected Result \\
{\footnotesize
\begin{itemize}
\tightlist
\item
  An image was successfully taken with the StarTracker and is of
  sufficient quality.
\item
  AZ and EL offsets are available.
\end{itemize}

}

\begin{tabular}{p{2cm}}
\toprule
Step 542  \\ \hline
\end{tabular}
 Description \\
{\footnotesize
\textbf{Find DIMM Object and DIMM Offset}\\

\begin{itemize}
\tightlist
\item
  While tracking, take a 10-sec exposure with the StarTracker.
\item
  Load the image into an image viewer.
\item
  Overlay the GAIA catalog.
\item
  Select a star brighter than XXX mag (bright enough for the DIMM).
\item
  Calculate the pixel offset between the StarTracker and the DIMM.
\item
  Transform the offset into AZ and EL offsets.
\end{itemize}

}
\hdashrule[0.5ex]{\textwidth}{1pt}{3mm}
  Expected Result \\
{\footnotesize
\begin{itemize}
\tightlist
\item
  An image was successfully taken with the StarTracker and is of
  sufficient quality.
\item
  AZ and EL offsets are available.
\end{itemize}

}

\begin{tabular}{p{2cm}}
\toprule
Step 543  \\ \hline
\end{tabular}
 Description \\
{\footnotesize
\textbf{Find DIMM Object and DIMM Offset}\\

\begin{itemize}
\tightlist
\item
  While tracking, take a 10-sec exposure with the StarTracker.
\item
  Load the image into an image viewer.
\item
  Overlay the GAIA catalog.
\item
  Select a star brighter than XXX mag (bright enough for the DIMM).
\item
  Calculate the pixel offset between the StarTracker and the DIMM.
\item
  Transform the offset into AZ and EL offsets.
\end{itemize}

}
\hdashrule[0.5ex]{\textwidth}{1pt}{3mm}
  Expected Result \\
{\footnotesize
\begin{itemize}
\tightlist
\item
  An image was successfully taken with the StarTracker and is of
  sufficient quality.
\item
  AZ and EL offsets are available.
\end{itemize}

}

\begin{tabular}{p{2cm}}
\toprule
Step 544  \\ \hline
\end{tabular}
 Description \\
{\footnotesize
\textbf{Find DIMM Object and DIMM Offset}\\

\begin{itemize}
\tightlist
\item
  While tracking, take a 10-sec exposure with the StarTracker.
\item
  Load the image into an image viewer.
\item
  Overlay the GAIA catalog.
\item
  Select a star brighter than XXX mag (bright enough for the DIMM).
\item
  Calculate the pixel offset between the StarTracker and the DIMM.
\item
  Transform the offset into AZ and EL offsets.
\end{itemize}

}
\hdashrule[0.5ex]{\textwidth}{1pt}{3mm}
  Expected Result \\
{\footnotesize
\begin{itemize}
\tightlist
\item
  An image was successfully taken with the StarTracker and is of
  sufficient quality.
\item
  AZ and EL offsets are available.
\end{itemize}

}

\begin{tabular}{p{2cm}}
\toprule
Step 545  \\ \hline
\end{tabular}
 Description \\
{\footnotesize
\textbf{Find DIMM Object and DIMM Offset}\\

\begin{itemize}
\tightlist
\item
  While tracking, take a 10-sec exposure with the StarTracker.
\item
  Load the image into an image viewer.
\item
  Overlay the GAIA catalog.
\item
  Select a star brighter than XXX mag (bright enough for the DIMM).
\item
  Calculate the pixel offset between the StarTracker and the DIMM.
\item
  Transform the offset into AZ and EL offsets.
\end{itemize}

}
\hdashrule[0.5ex]{\textwidth}{1pt}{3mm}
  Expected Result \\
{\footnotesize
\begin{itemize}
\tightlist
\item
  An image was successfully taken with the StarTracker and is of
  sufficient quality.
\item
  AZ and EL offsets are available.
\end{itemize}

}

\begin{tabular}{p{2cm}}
\toprule
Step 546  \\ \hline
\end{tabular}
 Description \\
{\footnotesize
\textbf{Find DIMM Object and DIMM Offset}\\

\begin{itemize}
\tightlist
\item
  While tracking, take a 10-sec exposure with the StarTracker.
\item
  Load the image into an image viewer.
\item
  Overlay the GAIA catalog.
\item
  Select a star brighter than XXX mag (bright enough for the DIMM).
\item
  Calculate the pixel offset between the StarTracker and the DIMM.
\item
  Transform the offset into AZ and EL offsets.
\end{itemize}

}
\hdashrule[0.5ex]{\textwidth}{1pt}{3mm}
  Expected Result \\
{\footnotesize
\begin{itemize}
\tightlist
\item
  An image was successfully taken with the StarTracker and is of
  sufficient quality.
\item
  AZ and EL offsets are available.
\end{itemize}

}

\begin{tabular}{p{2cm}}
\toprule
Step 547  \\ \hline
\end{tabular}
 Description \\
{\footnotesize
\textbf{Find DIMM Object and DIMM Offset}\\

\begin{itemize}
\tightlist
\item
  While tracking, take a 10-sec exposure with the StarTracker.
\item
  Load the image into an image viewer.
\item
  Overlay the GAIA catalog.
\item
  Select a star brighter than XXX mag (bright enough for the DIMM).
\item
  Calculate the pixel offset between the StarTracker and the DIMM.
\item
  Transform the offset into AZ and EL offsets.
\end{itemize}

}
\hdashrule[0.5ex]{\textwidth}{1pt}{3mm}
  Expected Result \\
{\footnotesize
\begin{itemize}
\tightlist
\item
  An image was successfully taken with the StarTracker and is of
  sufficient quality.
\item
  AZ and EL offsets are available.
\end{itemize}

}

\begin{tabular}{p{2cm}}
\toprule
Step 548  \\ \hline
\end{tabular}
 Description \\
{\footnotesize
\textbf{Find DIMM Object and DIMM Offset}\\

\begin{itemize}
\tightlist
\item
  While tracking, take a 10-sec exposure with the StarTracker.
\item
  Load the image into an image viewer.
\item
  Overlay the GAIA catalog.
\item
  Select a star brighter than XXX mag (bright enough for the DIMM).
\item
  Calculate the pixel offset between the StarTracker and the DIMM.
\item
  Transform the offset into AZ and EL offsets.
\end{itemize}

}
\hdashrule[0.5ex]{\textwidth}{1pt}{3mm}
  Expected Result \\
{\footnotesize
\begin{itemize}
\tightlist
\item
  An image was successfully taken with the StarTracker and is of
  sufficient quality.
\item
  AZ and EL offsets are available.
\end{itemize}

}

\begin{tabular}{p{2cm}}
\toprule
Step 549  \\ \hline
\end{tabular}
 Description \\
{\footnotesize
\textbf{Find DIMM Object and DIMM Offset}\\

\begin{itemize}
\tightlist
\item
  While tracking, take a 10-sec exposure with the StarTracker.
\item
  Load the image into an image viewer.
\item
  Overlay the GAIA catalog.
\item
  Select a star brighter than XXX mag (bright enough for the DIMM).
\item
  Calculate the pixel offset between the StarTracker and the DIMM.
\item
  Transform the offset into AZ and EL offsets.
\end{itemize}

}
\hdashrule[0.5ex]{\textwidth}{1pt}{3mm}
  Expected Result \\
{\footnotesize
\begin{itemize}
\tightlist
\item
  An image was successfully taken with the StarTracker and is of
  sufficient quality.
\item
  AZ and EL offsets are available.
\end{itemize}

}

\begin{tabular}{p{2cm}}
\toprule
Step 550  \\ \hline
\end{tabular}
 Description \\
{\footnotesize
\textbf{Find DIMM Object and DIMM Offset}\\

\begin{itemize}
\tightlist
\item
  While tracking, take a 10-sec exposure with the StarTracker.
\item
  Load the image into an image viewer.
\item
  Overlay the GAIA catalog.
\item
  Select a star brighter than XXX mag (bright enough for the DIMM).
\item
  Calculate the pixel offset between the StarTracker and the DIMM.
\item
  Transform the offset into AZ and EL offsets.
\end{itemize}

}
\hdashrule[0.5ex]{\textwidth}{1pt}{3mm}
  Expected Result \\
{\footnotesize
\begin{itemize}
\tightlist
\item
  An image was successfully taken with the StarTracker and is of
  sufficient quality.
\item
  AZ and EL offsets are available.
\end{itemize}

}

\begin{tabular}{p{2cm}}
\toprule
Step 551  \\ \hline
\end{tabular}
 Description \\
{\footnotesize
\textbf{Find DIMM Object and DIMM Offset}\\

\begin{itemize}
\tightlist
\item
  While tracking, take a 10-sec exposure with the StarTracker.
\item
  Load the image into an image viewer.
\item
  Overlay the GAIA catalog.
\item
  Select a star brighter than XXX mag (bright enough for the DIMM).
\item
  Calculate the pixel offset between the StarTracker and the DIMM.
\item
  Transform the offset into AZ and EL offsets.
\end{itemize}

}
\hdashrule[0.5ex]{\textwidth}{1pt}{3mm}
  Expected Result \\
{\footnotesize
\begin{itemize}
\tightlist
\item
  An image was successfully taken with the StarTracker and is of
  sufficient quality.
\item
  AZ and EL offsets are available.
\end{itemize}

}

\begin{tabular}{p{2cm}}
\toprule
Step 552  \\ \hline
\end{tabular}
 Description \\
{\footnotesize
\textbf{Find DIMM Object and DIMM Offset}\\

\begin{itemize}
\tightlist
\item
  While tracking, take a 10-sec exposure with the StarTracker.
\item
  Load the image into an image viewer.
\item
  Overlay the GAIA catalog.
\item
  Select a star brighter than XXX mag (bright enough for the DIMM).
\item
  Calculate the pixel offset between the StarTracker and the DIMM.
\item
  Transform the offset into AZ and EL offsets.
\end{itemize}

}
\hdashrule[0.5ex]{\textwidth}{1pt}{3mm}
  Expected Result \\
{\footnotesize
\begin{itemize}
\tightlist
\item
  An image was successfully taken with the StarTracker and is of
  sufficient quality.
\item
  AZ and EL offsets are available.
\end{itemize}

}

\begin{tabular}{p{2cm}}
\toprule
Step 553  \\ \hline
\end{tabular}
 Description \\
{\footnotesize
\textbf{Find DIMM Object and DIMM Offset}\\

\begin{itemize}
\tightlist
\item
  While tracking, take a 10-sec exposure with the StarTracker.
\item
  Load the image into an image viewer.
\item
  Overlay the GAIA catalog.
\item
  Select a star brighter than XXX mag (bright enough for the DIMM).
\item
  Calculate the pixel offset between the StarTracker and the DIMM.
\item
  Transform the offset into AZ and EL offsets.
\end{itemize}

}
\hdashrule[0.5ex]{\textwidth}{1pt}{3mm}
  Expected Result \\
{\footnotesize
\begin{itemize}
\tightlist
\item
  An image was successfully taken with the StarTracker and is of
  sufficient quality.
\item
  AZ and EL offsets are available.
\end{itemize}

}

\begin{tabular}{p{2cm}}
\toprule
Step 554  \\ \hline
\end{tabular}
 Description \\
{\footnotesize
\textbf{Find DIMM Object and DIMM Offset}\\

\begin{itemize}
\tightlist
\item
  While tracking, take a 10-sec exposure with the StarTracker.
\item
  Load the image into an image viewer.
\item
  Overlay the GAIA catalog.
\item
  Select a star brighter than XXX mag (bright enough for the DIMM).
\item
  Calculate the pixel offset between the StarTracker and the DIMM.
\item
  Transform the offset into AZ and EL offsets.
\end{itemize}

}
\hdashrule[0.5ex]{\textwidth}{1pt}{3mm}
  Expected Result \\
{\footnotesize
\begin{itemize}
\tightlist
\item
  An image was successfully taken with the StarTracker and is of
  sufficient quality.
\item
  AZ and EL offsets are available.
\end{itemize}

}

\begin{tabular}{p{2cm}}
\toprule
Step 555  \\ \hline
\end{tabular}
 Description \\
{\footnotesize
\textbf{Find DIMM Object and DIMM Offset}\\

\begin{itemize}
\tightlist
\item
  While tracking, take a 10-sec exposure with the StarTracker.
\item
  Load the image into an image viewer.
\item
  Overlay the GAIA catalog.
\item
  Select a star brighter than XXX mag (bright enough for the DIMM).
\item
  Calculate the pixel offset between the StarTracker and the DIMM.
\item
  Transform the offset into AZ and EL offsets.
\end{itemize}

}
\hdashrule[0.5ex]{\textwidth}{1pt}{3mm}
  Expected Result \\
{\footnotesize
\begin{itemize}
\tightlist
\item
  An image was successfully taken with the StarTracker and is of
  sufficient quality.
\item
  AZ and EL offsets are available.
\end{itemize}

}

\begin{tabular}{p{2cm}}
\toprule
Step 556  \\ \hline
\end{tabular}
 Description \\
{\footnotesize
\textbf{Find DIMM Object and DIMM Offset}\\

\begin{itemize}
\tightlist
\item
  While tracking, take a 10-sec exposure with the StarTracker.
\item
  Load the image into an image viewer.
\item
  Overlay the GAIA catalog.
\item
  Select a star brighter than XXX mag (bright enough for the DIMM).
\item
  Calculate the pixel offset between the StarTracker and the DIMM.
\item
  Transform the offset into AZ and EL offsets.
\end{itemize}

}
\hdashrule[0.5ex]{\textwidth}{1pt}{3mm}
  Expected Result \\
{\footnotesize
\begin{itemize}
\tightlist
\item
  An image was successfully taken with the StarTracker and is of
  sufficient quality.
\item
  AZ and EL offsets are available.
\end{itemize}

}

\begin{tabular}{p{2cm}}
\toprule
Step 557  \\ \hline
\end{tabular}
 Description \\
{\footnotesize
\textbf{Find DIMM Object and DIMM Offset}\\

\begin{itemize}
\tightlist
\item
  While tracking, take a 10-sec exposure with the StarTracker.
\item
  Load the image into an image viewer.
\item
  Overlay the GAIA catalog.
\item
  Select a star brighter than XXX mag (bright enough for the DIMM).
\item
  Calculate the pixel offset between the StarTracker and the DIMM.
\item
  Transform the offset into AZ and EL offsets.
\end{itemize}

}
\hdashrule[0.5ex]{\textwidth}{1pt}{3mm}
  Expected Result \\
{\footnotesize
\begin{itemize}
\tightlist
\item
  An image was successfully taken with the StarTracker and is of
  sufficient quality.
\item
  AZ and EL offsets are available.
\end{itemize}

}

\begin{tabular}{p{2cm}}
\toprule
Step 558  \\ \hline
\end{tabular}
 Description \\
{\footnotesize
\textbf{Find DIMM Object and DIMM Offset}\\

\begin{itemize}
\tightlist
\item
  While tracking, take a 10-sec exposure with the StarTracker.
\item
  Load the image into an image viewer.
\item
  Overlay the GAIA catalog.
\item
  Select a star brighter than XXX mag (bright enough for the DIMM).
\item
  Calculate the pixel offset between the StarTracker and the DIMM.
\item
  Transform the offset into AZ and EL offsets.
\end{itemize}

}
\hdashrule[0.5ex]{\textwidth}{1pt}{3mm}
  Expected Result \\
{\footnotesize
\begin{itemize}
\tightlist
\item
  An image was successfully taken with the StarTracker and is of
  sufficient quality.
\item
  AZ and EL offsets are available.
\end{itemize}

}

\begin{tabular}{p{2cm}}
\toprule
Step 559  \\ \hline
\end{tabular}
 Description \\
{\footnotesize
\textbf{Find DIMM Object and DIMM Offset}\\

\begin{itemize}
\tightlist
\item
  While tracking, take a 10-sec exposure with the StarTracker.
\item
  Load the image into an image viewer.
\item
  Overlay the GAIA catalog.
\item
  Select a star brighter than XXX mag (bright enough for the DIMM).
\item
  Calculate the pixel offset between the StarTracker and the DIMM.
\item
  Transform the offset into AZ and EL offsets.
\end{itemize}

}
\hdashrule[0.5ex]{\textwidth}{1pt}{3mm}
  Expected Result \\
{\footnotesize
\begin{itemize}
\tightlist
\item
  An image was successfully taken with the StarTracker and is of
  sufficient quality.
\item
  AZ and EL offsets are available.
\end{itemize}

}

\begin{tabular}{p{2cm}}
\toprule
Step 560  \\ \hline
\end{tabular}
 Description \\
{\footnotesize
\textbf{Find DIMM Object and DIMM Offset}\\

\begin{itemize}
\tightlist
\item
  While tracking, take a 10-sec exposure with the StarTracker.
\item
  Load the image into an image viewer.
\item
  Overlay the GAIA catalog.
\item
  Select a star brighter than XXX mag (bright enough for the DIMM).
\item
  Calculate the pixel offset between the StarTracker and the DIMM.
\item
  Transform the offset into AZ and EL offsets.
\end{itemize}

}
\hdashrule[0.5ex]{\textwidth}{1pt}{3mm}
  Expected Result \\
{\footnotesize
\begin{itemize}
\tightlist
\item
  An image was successfully taken with the StarTracker and is of
  sufficient quality.
\item
  AZ and EL offsets are available.
\end{itemize}

}

\begin{tabular}{p{2cm}}
\toprule
Step 561  \\ \hline
\end{tabular}
 Description \\
{\footnotesize
\textbf{Find DIMM Object and DIMM Offset}\\

\begin{itemize}
\tightlist
\item
  While tracking, take a 10-sec exposure with the StarTracker.
\item
  Load the image into an image viewer.
\item
  Overlay the GAIA catalog.
\item
  Select a star brighter than XXX mag (bright enough for the DIMM).
\item
  Calculate the pixel offset between the StarTracker and the DIMM.
\item
  Transform the offset into AZ and EL offsets.
\end{itemize}

}
\hdashrule[0.5ex]{\textwidth}{1pt}{3mm}
  Expected Result \\
{\footnotesize
\begin{itemize}
\tightlist
\item
  An image was successfully taken with the StarTracker and is of
  sufficient quality.
\item
  AZ and EL offsets are available.
\end{itemize}

}

\begin{tabular}{p{2cm}}
\toprule
Step 562  \\ \hline
\end{tabular}
 Description \\
{\footnotesize
\textbf{Find DIMM Object and DIMM Offset}\\

\begin{itemize}
\tightlist
\item
  While tracking, take a 10-sec exposure with the StarTracker.
\item
  Load the image into an image viewer.
\item
  Overlay the GAIA catalog.
\item
  Select a star brighter than XXX mag (bright enough for the DIMM).
\item
  Calculate the pixel offset between the StarTracker and the DIMM.
\item
  Transform the offset into AZ and EL offsets.
\end{itemize}

}
\hdashrule[0.5ex]{\textwidth}{1pt}{3mm}
  Expected Result \\
{\footnotesize
\begin{itemize}
\tightlist
\item
  An image was successfully taken with the StarTracker and is of
  sufficient quality.
\item
  AZ and EL offsets are available.
\end{itemize}

}

\begin{tabular}{p{2cm}}
\toprule
Step 563  \\ \hline
\end{tabular}
 Description \\
{\footnotesize
\textbf{Find DIMM Object and DIMM Offset}\\

\begin{itemize}
\tightlist
\item
  While tracking, take a 10-sec exposure with the StarTracker.
\item
  Load the image into an image viewer.
\item
  Overlay the GAIA catalog.
\item
  Select a star brighter than XXX mag (bright enough for the DIMM).
\item
  Calculate the pixel offset between the StarTracker and the DIMM.
\item
  Transform the offset into AZ and EL offsets.
\end{itemize}

}
\hdashrule[0.5ex]{\textwidth}{1pt}{3mm}
  Expected Result \\
{\footnotesize
\begin{itemize}
\tightlist
\item
  An image was successfully taken with the StarTracker and is of
  sufficient quality.
\item
  AZ and EL offsets are available.
\end{itemize}

}

\begin{tabular}{p{2cm}}
\toprule
Step 564  \\ \hline
\end{tabular}
 Description \\
{\footnotesize
\textbf{Find DIMM Object and DIMM Offset}\\

\begin{itemize}
\tightlist
\item
  While tracking, take a 10-sec exposure with the StarTracker.
\item
  Load the image into an image viewer.
\item
  Overlay the GAIA catalog.
\item
  Select a star brighter than XXX mag (bright enough for the DIMM).
\item
  Calculate the pixel offset between the StarTracker and the DIMM.
\item
  Transform the offset into AZ and EL offsets.
\end{itemize}

}
\hdashrule[0.5ex]{\textwidth}{1pt}{3mm}
  Expected Result \\
{\footnotesize
\begin{itemize}
\tightlist
\item
  An image was successfully taken with the StarTracker and is of
  sufficient quality.
\item
  AZ and EL offsets are available.
\end{itemize}

}

\begin{tabular}{p{2cm}}
\toprule
Step 565  \\ \hline
\end{tabular}
 Description \\
{\footnotesize
\textbf{Find DIMM Object and DIMM Offset}\\

\begin{itemize}
\tightlist
\item
  While tracking, take a 10-sec exposure with the StarTracker.
\item
  Load the image into an image viewer.
\item
  Overlay the GAIA catalog.
\item
  Select a star brighter than XXX mag (bright enough for the DIMM).
\item
  Calculate the pixel offset between the StarTracker and the DIMM.
\item
  Transform the offset into AZ and EL offsets.
\end{itemize}

}
\hdashrule[0.5ex]{\textwidth}{1pt}{3mm}
  Expected Result \\
{\footnotesize
\begin{itemize}
\tightlist
\item
  An image was successfully taken with the StarTracker and is of
  sufficient quality.
\item
  AZ and EL offsets are available.
\end{itemize}

}

\begin{tabular}{p{2cm}}
\toprule
Step 566  \\ \hline
\end{tabular}
 Description \\
{\footnotesize
\textbf{Find DIMM Object and DIMM Offset}\\

\begin{itemize}
\tightlist
\item
  While tracking, take a 10-sec exposure with the StarTracker.
\item
  Load the image into an image viewer.
\item
  Overlay the GAIA catalog.
\item
  Select a star brighter than XXX mag (bright enough for the DIMM).
\item
  Calculate the pixel offset between the StarTracker and the DIMM.
\item
  Transform the offset into AZ and EL offsets.
\end{itemize}

}
\hdashrule[0.5ex]{\textwidth}{1pt}{3mm}
  Expected Result \\
{\footnotesize
\begin{itemize}
\tightlist
\item
  An image was successfully taken with the StarTracker and is of
  sufficient quality.
\item
  AZ and EL offsets are available.
\end{itemize}

}

\begin{tabular}{p{2cm}}
\toprule
Step 567  \\ \hline
\end{tabular}
 Description \\
{\footnotesize
\textbf{Find DIMM Object and DIMM Offset}\\

\begin{itemize}
\tightlist
\item
  While tracking, take a 10-sec exposure with the StarTracker.
\item
  Load the image into an image viewer.
\item
  Overlay the GAIA catalog.
\item
  Select a star brighter than XXX mag (bright enough for the DIMM).
\item
  Calculate the pixel offset between the StarTracker and the DIMM.
\item
  Transform the offset into AZ and EL offsets.
\end{itemize}

}
\hdashrule[0.5ex]{\textwidth}{1pt}{3mm}
  Expected Result \\
{\footnotesize
\begin{itemize}
\tightlist
\item
  An image was successfully taken with the StarTracker and is of
  sufficient quality.
\item
  AZ and EL offsets are available.
\end{itemize}

}

\begin{tabular}{p{2cm}}
\toprule
Step 568  \\ \hline
\end{tabular}
 Description \\
{\footnotesize
\textbf{Find DIMM Object and DIMM Offset}\\

\begin{itemize}
\tightlist
\item
  While tracking, take a 10-sec exposure with the StarTracker.
\item
  Load the image into an image viewer.
\item
  Overlay the GAIA catalog.
\item
  Select a star brighter than XXX mag (bright enough for the DIMM).
\item
  Calculate the pixel offset between the StarTracker and the DIMM.
\item
  Transform the offset into AZ and EL offsets.
\end{itemize}

}
\hdashrule[0.5ex]{\textwidth}{1pt}{3mm}
  Expected Result \\
{\footnotesize
\begin{itemize}
\tightlist
\item
  An image was successfully taken with the StarTracker and is of
  sufficient quality.
\item
  AZ and EL offsets are available.
\end{itemize}

}

\begin{tabular}{p{2cm}}
\toprule
Step 569  \\ \hline
\end{tabular}
 Description \\
{\footnotesize
\textbf{Find DIMM Object and DIMM Offset}\\

\begin{itemize}
\tightlist
\item
  While tracking, take a 10-sec exposure with the StarTracker.
\item
  Load the image into an image viewer.
\item
  Overlay the GAIA catalog.
\item
  Select a star brighter than XXX mag (bright enough for the DIMM).
\item
  Calculate the pixel offset between the StarTracker and the DIMM.
\item
  Transform the offset into AZ and EL offsets.
\end{itemize}

}
\hdashrule[0.5ex]{\textwidth}{1pt}{3mm}
  Expected Result \\
{\footnotesize
\begin{itemize}
\tightlist
\item
  An image was successfully taken with the StarTracker and is of
  sufficient quality.
\item
  AZ and EL offsets are available.
\end{itemize}

}

\begin{tabular}{p{2cm}}
\toprule
Step 570  \\ \hline
\end{tabular}
 Description \\
{\footnotesize
\textbf{Find DIMM Object and DIMM Offset}\\

\begin{itemize}
\tightlist
\item
  While tracking, take a 10-sec exposure with the StarTracker.
\item
  Load the image into an image viewer.
\item
  Overlay the GAIA catalog.
\item
  Select a star brighter than XXX mag (bright enough for the DIMM).
\item
  Calculate the pixel offset between the StarTracker and the DIMM.
\item
  Transform the offset into AZ and EL offsets.
\end{itemize}

}
\hdashrule[0.5ex]{\textwidth}{1pt}{3mm}
  Expected Result \\
{\footnotesize
\begin{itemize}
\tightlist
\item
  An image was successfully taken with the StarTracker and is of
  sufficient quality.
\item
  AZ and EL offsets are available.
\end{itemize}

}

\begin{tabular}{p{2cm}}
\toprule
Step 571  \\ \hline
\end{tabular}
 Description \\
{\footnotesize
\textbf{Find DIMM Object and DIMM Offset}\\

\begin{itemize}
\tightlist
\item
  While tracking, take a 10-sec exposure with the StarTracker.
\item
  Load the image into an image viewer.
\item
  Overlay the GAIA catalog.
\item
  Select a star brighter than XXX mag (bright enough for the DIMM).
\item
  Calculate the pixel offset between the StarTracker and the DIMM.
\item
  Transform the offset into AZ and EL offsets.
\end{itemize}

}
\hdashrule[0.5ex]{\textwidth}{1pt}{3mm}
  Expected Result \\
{\footnotesize
\begin{itemize}
\tightlist
\item
  An image was successfully taken with the StarTracker and is of
  sufficient quality.
\item
  AZ and EL offsets are available.
\end{itemize}

}

\begin{tabular}{p{2cm}}
\toprule
Step 572  \\ \hline
\end{tabular}
 Description \\
{\footnotesize
\textbf{Find DIMM Object and DIMM Offset}\\

\begin{itemize}
\tightlist
\item
  While tracking, take a 10-sec exposure with the StarTracker.
\item
  Load the image into an image viewer.
\item
  Overlay the GAIA catalog.
\item
  Select a star brighter than XXX mag (bright enough for the DIMM).
\item
  Calculate the pixel offset between the StarTracker and the DIMM.
\item
  Transform the offset into AZ and EL offsets.
\end{itemize}

}
\hdashrule[0.5ex]{\textwidth}{1pt}{3mm}
  Expected Result \\
{\footnotesize
\begin{itemize}
\tightlist
\item
  An image was successfully taken with the StarTracker and is of
  sufficient quality.
\item
  AZ and EL offsets are available.
\end{itemize}

}

\begin{tabular}{p{2cm}}
\toprule
Step 573  \\ \hline
\end{tabular}
 Description \\
{\footnotesize
\textbf{Find DIMM Object and DIMM Offset}\\

\begin{itemize}
\tightlist
\item
  While tracking, take a 10-sec exposure with the StarTracker.
\item
  Load the image into an image viewer.
\item
  Overlay the GAIA catalog.
\item
  Select a star brighter than XXX mag (bright enough for the DIMM).
\item
  Calculate the pixel offset between the StarTracker and the DIMM.
\item
  Transform the offset into AZ and EL offsets.
\end{itemize}

}
\hdashrule[0.5ex]{\textwidth}{1pt}{3mm}
  Expected Result \\
{\footnotesize
\begin{itemize}
\tightlist
\item
  An image was successfully taken with the StarTracker and is of
  sufficient quality.
\item
  AZ and EL offsets are available.
\end{itemize}

}

\begin{tabular}{p{2cm}}
\toprule
Step 574  \\ \hline
\end{tabular}
 Description \\
{\footnotesize
\textbf{Find DIMM Object and DIMM Offset}\\

\begin{itemize}
\tightlist
\item
  While tracking, take a 10-sec exposure with the StarTracker.
\item
  Load the image into an image viewer.
\item
  Overlay the GAIA catalog.
\item
  Select a star brighter than XXX mag (bright enough for the DIMM).
\item
  Calculate the pixel offset between the StarTracker and the DIMM.
\item
  Transform the offset into AZ and EL offsets.
\end{itemize}

}
\hdashrule[0.5ex]{\textwidth}{1pt}{3mm}
  Expected Result \\
{\footnotesize
\begin{itemize}
\tightlist
\item
  An image was successfully taken with the StarTracker and is of
  sufficient quality.
\item
  AZ and EL offsets are available.
\end{itemize}

}

\begin{tabular}{p{2cm}}
\toprule
Step 575  \\ \hline
\end{tabular}
 Description \\
{\footnotesize
\textbf{Find DIMM Object and DIMM Offset}\\

\begin{itemize}
\tightlist
\item
  While tracking, take a 10-sec exposure with the StarTracker.
\item
  Load the image into an image viewer.
\item
  Overlay the GAIA catalog.
\item
  Select a star brighter than XXX mag (bright enough for the DIMM).
\item
  Calculate the pixel offset between the StarTracker and the DIMM.
\item
  Transform the offset into AZ and EL offsets.
\end{itemize}

}
\hdashrule[0.5ex]{\textwidth}{1pt}{3mm}
  Expected Result \\
{\footnotesize
\begin{itemize}
\tightlist
\item
  An image was successfully taken with the StarTracker and is of
  sufficient quality.
\item
  AZ and EL offsets are available.
\end{itemize}

}

\begin{tabular}{p{2cm}}
\toprule
Step 576  \\ \hline
\end{tabular}
 Description \\
{\footnotesize
\textbf{Find DIMM Object and DIMM Offset}\\

\begin{itemize}
\tightlist
\item
  While tracking, take a 10-sec exposure with the StarTracker.
\item
  Load the image into an image viewer.
\item
  Overlay the GAIA catalog.
\item
  Select a star brighter than XXX mag (bright enough for the DIMM).
\item
  Calculate the pixel offset between the StarTracker and the DIMM.
\item
  Transform the offset into AZ and EL offsets.
\end{itemize}

}
\hdashrule[0.5ex]{\textwidth}{1pt}{3mm}
  Expected Result \\
{\footnotesize
\begin{itemize}
\tightlist
\item
  An image was successfully taken with the StarTracker and is of
  sufficient quality.
\item
  AZ and EL offsets are available.
\end{itemize}

}

\begin{tabular}{p{2cm}}
\toprule
Step 577  \\ \hline
\end{tabular}
 Description \\
{\footnotesize
\textbf{Find DIMM Object and DIMM Offset}\\

\begin{itemize}
\tightlist
\item
  While tracking, take a 10-sec exposure with the StarTracker.
\item
  Load the image into an image viewer.
\item
  Overlay the GAIA catalog.
\item
  Select a star brighter than XXX mag (bright enough for the DIMM).
\item
  Calculate the pixel offset between the StarTracker and the DIMM.
\item
  Transform the offset into AZ and EL offsets.
\end{itemize}

}
\hdashrule[0.5ex]{\textwidth}{1pt}{3mm}
  Expected Result \\
{\footnotesize
\begin{itemize}
\tightlist
\item
  An image was successfully taken with the StarTracker and is of
  sufficient quality.
\item
  AZ and EL offsets are available.
\end{itemize}

}

\begin{tabular}{p{2cm}}
\toprule
Step 578  \\ \hline
\end{tabular}
 Description \\
{\footnotesize
\textbf{Find DIMM Object and DIMM Offset}\\

\begin{itemize}
\tightlist
\item
  While tracking, take a 10-sec exposure with the StarTracker.
\item
  Load the image into an image viewer.
\item
  Overlay the GAIA catalog.
\item
  Select a star brighter than XXX mag (bright enough for the DIMM).
\item
  Calculate the pixel offset between the StarTracker and the DIMM.
\item
  Transform the offset into AZ and EL offsets.
\end{itemize}

}
\hdashrule[0.5ex]{\textwidth}{1pt}{3mm}
  Expected Result \\
{\footnotesize
\begin{itemize}
\tightlist
\item
  An image was successfully taken with the StarTracker and is of
  sufficient quality.
\item
  AZ and EL offsets are available.
\end{itemize}

}

\begin{tabular}{p{2cm}}
\toprule
Step 579  \\ \hline
\end{tabular}
 Description \\
{\footnotesize
\textbf{Find DIMM Object and DIMM Offset}\\

\begin{itemize}
\tightlist
\item
  While tracking, take a 10-sec exposure with the StarTracker.
\item
  Load the image into an image viewer.
\item
  Overlay the GAIA catalog.
\item
  Select a star brighter than XXX mag (bright enough for the DIMM).
\item
  Calculate the pixel offset between the StarTracker and the DIMM.
\item
  Transform the offset into AZ and EL offsets.
\end{itemize}

}
\hdashrule[0.5ex]{\textwidth}{1pt}{3mm}
  Expected Result \\
{\footnotesize
\begin{itemize}
\tightlist
\item
  An image was successfully taken with the StarTracker and is of
  sufficient quality.
\item
  AZ and EL offsets are available.
\end{itemize}

}

\begin{tabular}{p{2cm}}
\toprule
Step 580  \\ \hline
\end{tabular}
 Description \\
{\footnotesize
\textbf{Move TMA to the DIMM position and \textbf{Take DIMM images}}\\

\begin{itemize}
\tightlist
\item
  Command the TMA to the DIMM position by applying the offsets
\item
  While tracking, take DIMM images with XXXs exposure time and inspect
  the quality.
\end{itemize}

}
\hdashrule[0.5ex]{\textwidth}{1pt}{3mm}
  Expected Result \\
{\footnotesize
\begin{itemize}
\tightlist
\item
  TMA reaches the DIMM position.
\item
  DIMM imaging quality is sufficient.
\end{itemize}

}

\begin{tabular}{p{2cm}}
\toprule
Step 581  \\ \hline
\end{tabular}
 Description \\
{\footnotesize
\textbf{Move TMA to the DIMM position and \textbf{Take DIMM images}}\\

\begin{itemize}
\tightlist
\item
  Command the TMA to the DIMM position by applying the offsets
\item
  While tracking, take DIMM images with XXXs exposure time and inspect
  the quality.
\end{itemize}

}
\hdashrule[0.5ex]{\textwidth}{1pt}{3mm}
  Expected Result \\
{\footnotesize
\begin{itemize}
\tightlist
\item
  TMA reaches the DIMM position.
\item
  DIMM imaging quality is sufficient.
\end{itemize}

}

\begin{tabular}{p{2cm}}
\toprule
Step 582  \\ \hline
\end{tabular}
 Description \\
{\footnotesize
\textbf{Move TMA to the DIMM position and \textbf{Take DIMM images}}\\

\begin{itemize}
\tightlist
\item
  Command the TMA to the DIMM position by applying the offsets
\item
  While tracking, take DIMM images with XXXs exposure time and inspect
  the quality.
\end{itemize}

}
\hdashrule[0.5ex]{\textwidth}{1pt}{3mm}
  Expected Result \\
{\footnotesize
\begin{itemize}
\tightlist
\item
  TMA reaches the DIMM position.
\item
  DIMM imaging quality is sufficient.
\end{itemize}

}

\begin{tabular}{p{2cm}}
\toprule
Step 583  \\ \hline
\end{tabular}
 Description \\
{\footnotesize
\textbf{Move TMA to the DIMM position and \textbf{Take DIMM images}}\\

\begin{itemize}
\tightlist
\item
  Command the TMA to the DIMM position by applying the offsets
\item
  While tracking, take DIMM images with XXXs exposure time and inspect
  the quality.
\end{itemize}

}
\hdashrule[0.5ex]{\textwidth}{1pt}{3mm}
  Expected Result \\
{\footnotesize
\begin{itemize}
\tightlist
\item
  TMA reaches the DIMM position.
\item
  DIMM imaging quality is sufficient.
\end{itemize}

}

\begin{tabular}{p{2cm}}
\toprule
Step 584  \\ \hline
\end{tabular}
 Description \\
{\footnotesize
\textbf{Move TMA to the DIMM position and \textbf{Take DIMM images}}\\

\begin{itemize}
\tightlist
\item
  Command the TMA to the DIMM position by applying the offsets
\item
  While tracking, take DIMM images with XXXs exposure time and inspect
  the quality.
\end{itemize}

}
\hdashrule[0.5ex]{\textwidth}{1pt}{3mm}
  Expected Result \\
{\footnotesize
\begin{itemize}
\tightlist
\item
  TMA reaches the DIMM position.
\item
  DIMM imaging quality is sufficient.
\end{itemize}

}

\begin{tabular}{p{2cm}}
\toprule
Step 585  \\ \hline
\end{tabular}
 Description \\
{\footnotesize
\textbf{Move TMA to the DIMM position and \textbf{Take DIMM images}}\\

\begin{itemize}
\tightlist
\item
  Command the TMA to the DIMM position by applying the offsets
\item
  While tracking, take DIMM images with XXXs exposure time and inspect
  the quality.
\end{itemize}

}
\hdashrule[0.5ex]{\textwidth}{1pt}{3mm}
  Expected Result \\
{\footnotesize
\begin{itemize}
\tightlist
\item
  TMA reaches the DIMM position.
\item
  DIMM imaging quality is sufficient.
\end{itemize}

}

\begin{tabular}{p{2cm}}
\toprule
Step 586  \\ \hline
\end{tabular}
 Description \\
{\footnotesize
\textbf{Move TMA to the DIMM position and \textbf{Take DIMM images}}\\

\begin{itemize}
\tightlist
\item
  Command the TMA to the DIMM position by applying the offsets
\item
  While tracking, take DIMM images with XXXs exposure time and inspect
  the quality.
\end{itemize}

}
\hdashrule[0.5ex]{\textwidth}{1pt}{3mm}
  Expected Result \\
{\footnotesize
\begin{itemize}
\tightlist
\item
  TMA reaches the DIMM position.
\item
  DIMM imaging quality is sufficient.
\end{itemize}

}

\begin{tabular}{p{2cm}}
\toprule
Step 587  \\ \hline
\end{tabular}
 Description \\
{\footnotesize
\textbf{Move TMA to the DIMM position and \textbf{Take DIMM images}}\\

\begin{itemize}
\tightlist
\item
  Command the TMA to the DIMM position by applying the offsets
\item
  While tracking, take DIMM images with XXXs exposure time and inspect
  the quality.
\end{itemize}

}
\hdashrule[0.5ex]{\textwidth}{1pt}{3mm}
  Expected Result \\
{\footnotesize
\begin{itemize}
\tightlist
\item
  TMA reaches the DIMM position.
\item
  DIMM imaging quality is sufficient.
\end{itemize}

}

\begin{tabular}{p{2cm}}
\toprule
Step 588  \\ \hline
\end{tabular}
 Description \\
{\footnotesize
\textbf{Move TMA to the DIMM position and \textbf{Take DIMM images}}\\

\begin{itemize}
\tightlist
\item
  Command the TMA to the DIMM position by applying the offsets
\item
  While tracking, take DIMM images with XXXs exposure time and inspect
  the quality.
\end{itemize}

}
\hdashrule[0.5ex]{\textwidth}{1pt}{3mm}
  Expected Result \\
{\footnotesize
\begin{itemize}
\tightlist
\item
  TMA reaches the DIMM position.
\item
  DIMM imaging quality is sufficient.
\end{itemize}

}

\begin{tabular}{p{2cm}}
\toprule
Step 589  \\ \hline
\end{tabular}
 Description \\
{\footnotesize
\textbf{Move TMA to the DIMM position and \textbf{Take DIMM images}}\\

\begin{itemize}
\tightlist
\item
  Command the TMA to the DIMM position by applying the offsets
\item
  While tracking, take DIMM images with XXXs exposure time and inspect
  the quality.
\end{itemize}

}
\hdashrule[0.5ex]{\textwidth}{1pt}{3mm}
  Expected Result \\
{\footnotesize
\begin{itemize}
\tightlist
\item
  TMA reaches the DIMM position.
\item
  DIMM imaging quality is sufficient.
\end{itemize}

}

\begin{tabular}{p{2cm}}
\toprule
Step 590  \\ \hline
\end{tabular}
 Description \\
{\footnotesize
\textbf{Move TMA to the DIMM position and \textbf{Take DIMM images}}\\

\begin{itemize}
\tightlist
\item
  Command the TMA to the DIMM position by applying the offsets
\item
  While tracking, take DIMM images with XXXs exposure time and inspect
  the quality.
\end{itemize}

}
\hdashrule[0.5ex]{\textwidth}{1pt}{3mm}
  Expected Result \\
{\footnotesize
\begin{itemize}
\tightlist
\item
  TMA reaches the DIMM position.
\item
  DIMM imaging quality is sufficient.
\end{itemize}

}

\begin{tabular}{p{2cm}}
\toprule
Step 591  \\ \hline
\end{tabular}
 Description \\
{\footnotesize
\textbf{Move TMA to the DIMM position and \textbf{Take DIMM images}}\\

\begin{itemize}
\tightlist
\item
  Command the TMA to the DIMM position by applying the offsets
\item
  While tracking, take DIMM images with XXXs exposure time and inspect
  the quality.
\end{itemize}

}
\hdashrule[0.5ex]{\textwidth}{1pt}{3mm}
  Expected Result \\
{\footnotesize
\begin{itemize}
\tightlist
\item
  TMA reaches the DIMM position.
\item
  DIMM imaging quality is sufficient.
\end{itemize}

}

\begin{tabular}{p{2cm}}
\toprule
Step 592  \\ \hline
\end{tabular}
 Description \\
{\footnotesize
\textbf{Move TMA to the DIMM position and \textbf{Take DIMM images}}\\

\begin{itemize}
\tightlist
\item
  Command the TMA to the DIMM position by applying the offsets
\item
  While tracking, take DIMM images with XXXs exposure time and inspect
  the quality.
\end{itemize}

}
\hdashrule[0.5ex]{\textwidth}{1pt}{3mm}
  Expected Result \\
{\footnotesize
\begin{itemize}
\tightlist
\item
  TMA reaches the DIMM position.
\item
  DIMM imaging quality is sufficient.
\end{itemize}

}

\begin{tabular}{p{2cm}}
\toprule
Step 593  \\ \hline
\end{tabular}
 Description \\
{\footnotesize
\textbf{Move TMA to the DIMM position and \textbf{Take DIMM images}}\\

\begin{itemize}
\tightlist
\item
  Command the TMA to the DIMM position by applying the offsets
\item
  While tracking, take DIMM images with XXXs exposure time and inspect
  the quality.
\end{itemize}

}
\hdashrule[0.5ex]{\textwidth}{1pt}{3mm}
  Expected Result \\
{\footnotesize
\begin{itemize}
\tightlist
\item
  TMA reaches the DIMM position.
\item
  DIMM imaging quality is sufficient.
\end{itemize}

}

\begin{tabular}{p{2cm}}
\toprule
Step 594  \\ \hline
\end{tabular}
 Description \\
{\footnotesize
\textbf{Move TMA to the DIMM position and \textbf{Take DIMM images}}\\

\begin{itemize}
\tightlist
\item
  Command the TMA to the DIMM position by applying the offsets
\item
  While tracking, take DIMM images with XXXs exposure time and inspect
  the quality.
\end{itemize}

}
\hdashrule[0.5ex]{\textwidth}{1pt}{3mm}
  Expected Result \\
{\footnotesize
\begin{itemize}
\tightlist
\item
  TMA reaches the DIMM position.
\item
  DIMM imaging quality is sufficient.
\end{itemize}

}

\begin{tabular}{p{2cm}}
\toprule
Step 595  \\ \hline
\end{tabular}
 Description \\
{\footnotesize
\textbf{Move TMA to the DIMM position and \textbf{Take DIMM images}}\\

\begin{itemize}
\tightlist
\item
  Command the TMA to the DIMM position by applying the offsets
\item
  While tracking, take DIMM images with XXXs exposure time and inspect
  the quality.
\end{itemize}

}
\hdashrule[0.5ex]{\textwidth}{1pt}{3mm}
  Expected Result \\
{\footnotesize
\begin{itemize}
\tightlist
\item
  TMA reaches the DIMM position.
\item
  DIMM imaging quality is sufficient.
\end{itemize}

}

\begin{tabular}{p{2cm}}
\toprule
Step 596  \\ \hline
\end{tabular}
 Description \\
{\footnotesize
\textbf{Move TMA to the DIMM position and \textbf{Take DIMM images}}\\

\begin{itemize}
\tightlist
\item
  Command the TMA to the DIMM position by applying the offsets
\item
  While tracking, take DIMM images with XXXs exposure time and inspect
  the quality.
\end{itemize}

}
\hdashrule[0.5ex]{\textwidth}{1pt}{3mm}
  Expected Result \\
{\footnotesize
\begin{itemize}
\tightlist
\item
  TMA reaches the DIMM position.
\item
  DIMM imaging quality is sufficient.
\end{itemize}

}

\begin{tabular}{p{2cm}}
\toprule
Step 597  \\ \hline
\end{tabular}
 Description \\
{\footnotesize
\textbf{Move TMA to the DIMM position and \textbf{Take DIMM images}}\\

\begin{itemize}
\tightlist
\item
  Command the TMA to the DIMM position by applying the offsets
\item
  While tracking, take DIMM images with XXXs exposure time and inspect
  the quality.
\end{itemize}

}
\hdashrule[0.5ex]{\textwidth}{1pt}{3mm}
  Expected Result \\
{\footnotesize
\begin{itemize}
\tightlist
\item
  TMA reaches the DIMM position.
\item
  DIMM imaging quality is sufficient.
\end{itemize}

}

\begin{tabular}{p{2cm}}
\toprule
Step 598  \\ \hline
\end{tabular}
 Description \\
{\footnotesize
\textbf{Move TMA to the DIMM position and \textbf{Take DIMM images}}\\

\begin{itemize}
\tightlist
\item
  Command the TMA to the DIMM position by applying the offsets
\item
  While tracking, take DIMM images with XXXs exposure time and inspect
  the quality.
\end{itemize}

}
\hdashrule[0.5ex]{\textwidth}{1pt}{3mm}
  Expected Result \\
{\footnotesize
\begin{itemize}
\tightlist
\item
  TMA reaches the DIMM position.
\item
  DIMM imaging quality is sufficient.
\end{itemize}

}

\begin{tabular}{p{2cm}}
\toprule
Step 599  \\ \hline
\end{tabular}
 Description \\
{\footnotesize
\textbf{Move TMA to the DIMM position and \textbf{Take DIMM images}}\\

\begin{itemize}
\tightlist
\item
  Command the TMA to the DIMM position by applying the offsets
\item
  While tracking, take DIMM images with XXXs exposure time and inspect
  the quality.
\end{itemize}

}
\hdashrule[0.5ex]{\textwidth}{1pt}{3mm}
  Expected Result \\
{\footnotesize
\begin{itemize}
\tightlist
\item
  TMA reaches the DIMM position.
\item
  DIMM imaging quality is sufficient.
\end{itemize}

}

\begin{tabular}{p{2cm}}
\toprule
Step 600  \\ \hline
\end{tabular}
 Description \\
{\footnotesize
\textbf{Move TMA to the DIMM position and \textbf{Take DIMM images}}\\

\begin{itemize}
\tightlist
\item
  Command the TMA to the DIMM position by applying the offsets
\item
  While tracking, take DIMM images with XXXs exposure time and inspect
  the quality.
\end{itemize}

}
\hdashrule[0.5ex]{\textwidth}{1pt}{3mm}
  Expected Result \\
{\footnotesize
\begin{itemize}
\tightlist
\item
  TMA reaches the DIMM position.
\item
  DIMM imaging quality is sufficient.
\end{itemize}

}

\begin{tabular}{p{2cm}}
\toprule
Step 601  \\ \hline
\end{tabular}
 Description \\
{\footnotesize
\textbf{Move TMA to the DIMM position and \textbf{Take DIMM images}}\\

\begin{itemize}
\tightlist
\item
  Command the TMA to the DIMM position by applying the offsets
\item
  While tracking, take DIMM images with XXXs exposure time and inspect
  the quality.
\end{itemize}

}
\hdashrule[0.5ex]{\textwidth}{1pt}{3mm}
  Expected Result \\
{\footnotesize
\begin{itemize}
\tightlist
\item
  TMA reaches the DIMM position.
\item
  DIMM imaging quality is sufficient.
\end{itemize}

}

\begin{tabular}{p{2cm}}
\toprule
Step 602  \\ \hline
\end{tabular}
 Description \\
{\footnotesize
\textbf{Move TMA to the DIMM position and \textbf{Take DIMM images}}\\

\begin{itemize}
\tightlist
\item
  Command the TMA to the DIMM position by applying the offsets
\item
  While tracking, take DIMM images with XXXs exposure time and inspect
  the quality.
\end{itemize}

}
\hdashrule[0.5ex]{\textwidth}{1pt}{3mm}
  Expected Result \\
{\footnotesize
\begin{itemize}
\tightlist
\item
  TMA reaches the DIMM position.
\item
  DIMM imaging quality is sufficient.
\end{itemize}

}

\begin{tabular}{p{2cm}}
\toprule
Step 603  \\ \hline
\end{tabular}
 Description \\
{\footnotesize
\textbf{Move TMA to the DIMM position and \textbf{Take DIMM images}}\\

\begin{itemize}
\tightlist
\item
  Command the TMA to the DIMM position by applying the offsets
\item
  While tracking, take DIMM images with XXXs exposure time and inspect
  the quality.
\end{itemize}

}
\hdashrule[0.5ex]{\textwidth}{1pt}{3mm}
  Expected Result \\
{\footnotesize
\begin{itemize}
\tightlist
\item
  TMA reaches the DIMM position.
\item
  DIMM imaging quality is sufficient.
\end{itemize}

}

\begin{tabular}{p{2cm}}
\toprule
Step 604  \\ \hline
\end{tabular}
 Description \\
{\footnotesize
\textbf{Move TMA to the DIMM position and \textbf{Take DIMM images}}\\

\begin{itemize}
\tightlist
\item
  Command the TMA to the DIMM position by applying the offsets
\item
  While tracking, take DIMM images with XXXs exposure time and inspect
  the quality.
\end{itemize}

}
\hdashrule[0.5ex]{\textwidth}{1pt}{3mm}
  Expected Result \\
{\footnotesize
\begin{itemize}
\tightlist
\item
  TMA reaches the DIMM position.
\item
  DIMM imaging quality is sufficient.
\end{itemize}

}

\begin{tabular}{p{2cm}}
\toprule
Step 605  \\ \hline
\end{tabular}
 Description \\
{\footnotesize
\textbf{Move TMA to the DIMM position and \textbf{Take DIMM images}}\\

\begin{itemize}
\tightlist
\item
  Command the TMA to the DIMM position by applying the offsets
\item
  While tracking, take DIMM images with XXXs exposure time and inspect
  the quality.
\end{itemize}

}
\hdashrule[0.5ex]{\textwidth}{1pt}{3mm}
  Expected Result \\
{\footnotesize
\begin{itemize}
\tightlist
\item
  TMA reaches the DIMM position.
\item
  DIMM imaging quality is sufficient.
\end{itemize}

}

\begin{tabular}{p{2cm}}
\toprule
Step 606  \\ \hline
\end{tabular}
 Description \\
{\footnotesize
\textbf{Move TMA to the DIMM position and \textbf{Take DIMM images}}\\

\begin{itemize}
\tightlist
\item
  Command the TMA to the DIMM position by applying the offsets
\item
  While tracking, take DIMM images with XXXs exposure time and inspect
  the quality.
\end{itemize}

}
\hdashrule[0.5ex]{\textwidth}{1pt}{3mm}
  Expected Result \\
{\footnotesize
\begin{itemize}
\tightlist
\item
  TMA reaches the DIMM position.
\item
  DIMM imaging quality is sufficient.
\end{itemize}

}

\begin{tabular}{p{2cm}}
\toprule
Step 607  \\ \hline
\end{tabular}
 Description \\
{\footnotesize
\textbf{Move TMA to the DIMM position and \textbf{Take DIMM images}}\\

\begin{itemize}
\tightlist
\item
  Command the TMA to the DIMM position by applying the offsets
\item
  While tracking, take DIMM images with XXXs exposure time and inspect
  the quality.
\end{itemize}

}
\hdashrule[0.5ex]{\textwidth}{1pt}{3mm}
  Expected Result \\
{\footnotesize
\begin{itemize}
\tightlist
\item
  TMA reaches the DIMM position.
\item
  DIMM imaging quality is sufficient.
\end{itemize}

}

\begin{tabular}{p{2cm}}
\toprule
Step 608  \\ \hline
\end{tabular}
 Description \\
{\footnotesize
\textbf{Move TMA to the DIMM position and \textbf{Take DIMM images}}\\

\begin{itemize}
\tightlist
\item
  Command the TMA to the DIMM position by applying the offsets
\item
  While tracking, take DIMM images with XXXs exposure time and inspect
  the quality.
\end{itemize}

}
\hdashrule[0.5ex]{\textwidth}{1pt}{3mm}
  Expected Result \\
{\footnotesize
\begin{itemize}
\tightlist
\item
  TMA reaches the DIMM position.
\item
  DIMM imaging quality is sufficient.
\end{itemize}

}

\begin{tabular}{p{2cm}}
\toprule
Step 609  \\ \hline
\end{tabular}
 Description \\
{\footnotesize
\textbf{Move TMA to the DIMM position and \textbf{Take DIMM images}}\\

\begin{itemize}
\tightlist
\item
  Command the TMA to the DIMM position by applying the offsets
\item
  While tracking, take DIMM images with XXXs exposure time and inspect
  the quality.
\end{itemize}

}
\hdashrule[0.5ex]{\textwidth}{1pt}{3mm}
  Expected Result \\
{\footnotesize
\begin{itemize}
\tightlist
\item
  TMA reaches the DIMM position.
\item
  DIMM imaging quality is sufficient.
\end{itemize}

}

\begin{tabular}{p{2cm}}
\toprule
Step 610  \\ \hline
\end{tabular}
 Description \\
{\footnotesize
\textbf{Move TMA to the DIMM position and \textbf{Take DIMM images}}\\

\begin{itemize}
\tightlist
\item
  Command the TMA to the DIMM position by applying the offsets
\item
  While tracking, take DIMM images with XXXs exposure time and inspect
  the quality.
\end{itemize}

}
\hdashrule[0.5ex]{\textwidth}{1pt}{3mm}
  Expected Result \\
{\footnotesize
\begin{itemize}
\tightlist
\item
  TMA reaches the DIMM position.
\item
  DIMM imaging quality is sufficient.
\end{itemize}

}

\begin{tabular}{p{2cm}}
\toprule
Step 611  \\ \hline
\end{tabular}
 Description \\
{\footnotesize
\textbf{Move TMA to the DIMM position and \textbf{Take DIMM images}}\\

\begin{itemize}
\tightlist
\item
  Command the TMA to the DIMM position by applying the offsets
\item
  While tracking, take DIMM images with XXXs exposure time and inspect
  the quality.
\end{itemize}

}
\hdashrule[0.5ex]{\textwidth}{1pt}{3mm}
  Expected Result \\
{\footnotesize
\begin{itemize}
\tightlist
\item
  TMA reaches the DIMM position.
\item
  DIMM imaging quality is sufficient.
\end{itemize}

}

\begin{tabular}{p{2cm}}
\toprule
Step 612  \\ \hline
\end{tabular}
 Description \\
{\footnotesize
\textbf{Move TMA to the DIMM position and \textbf{Take DIMM images}}\\

\begin{itemize}
\tightlist
\item
  Command the TMA to the DIMM position by applying the offsets
\item
  While tracking, take DIMM images with XXXs exposure time and inspect
  the quality.
\end{itemize}

}
\hdashrule[0.5ex]{\textwidth}{1pt}{3mm}
  Expected Result \\
{\footnotesize
\begin{itemize}
\tightlist
\item
  TMA reaches the DIMM position.
\item
  DIMM imaging quality is sufficient.
\end{itemize}

}

\begin{tabular}{p{2cm}}
\toprule
Step 613  \\ \hline
\end{tabular}
 Description \\
{\footnotesize
\textbf{Move TMA to the DIMM position and \textbf{Take DIMM images}}\\

\begin{itemize}
\tightlist
\item
  Command the TMA to the DIMM position by applying the offsets
\item
  While tracking, take DIMM images with XXXs exposure time and inspect
  the quality.
\end{itemize}

}
\hdashrule[0.5ex]{\textwidth}{1pt}{3mm}
  Expected Result \\
{\footnotesize
\begin{itemize}
\tightlist
\item
  TMA reaches the DIMM position.
\item
  DIMM imaging quality is sufficient.
\end{itemize}

}

\begin{tabular}{p{2cm}}
\toprule
Step 614  \\ \hline
\end{tabular}
 Description \\
{\footnotesize
\textbf{Move TMA to the DIMM position and \textbf{Take DIMM images}}\\

\begin{itemize}
\tightlist
\item
  Command the TMA to the DIMM position by applying the offsets
\item
  While tracking, take DIMM images with XXXs exposure time and inspect
  the quality.
\end{itemize}

}
\hdashrule[0.5ex]{\textwidth}{1pt}{3mm}
  Expected Result \\
{\footnotesize
\begin{itemize}
\tightlist
\item
  TMA reaches the DIMM position.
\item
  DIMM imaging quality is sufficient.
\end{itemize}

}

\begin{tabular}{p{2cm}}
\toprule
Step 615  \\ \hline
\end{tabular}
 Description \\
{\footnotesize
\textbf{Move TMA to the DIMM position and \textbf{Take DIMM images}}\\

\begin{itemize}
\tightlist
\item
  Command the TMA to the DIMM position by applying the offsets
\item
  While tracking, take DIMM images with XXXs exposure time and inspect
  the quality.
\end{itemize}

}
\hdashrule[0.5ex]{\textwidth}{1pt}{3mm}
  Expected Result \\
{\footnotesize
\begin{itemize}
\tightlist
\item
  TMA reaches the DIMM position.
\item
  DIMM imaging quality is sufficient.
\end{itemize}

}

\begin{tabular}{p{2cm}}
\toprule
Step 616  \\ \hline
\end{tabular}
 Description \\
{\footnotesize
\textbf{Move TMA to the DIMM position and \textbf{Take DIMM images}}\\

\begin{itemize}
\tightlist
\item
  Command the TMA to the DIMM position by applying the offsets
\item
  While tracking, take DIMM images with XXXs exposure time and inspect
  the quality.
\end{itemize}

}
\hdashrule[0.5ex]{\textwidth}{1pt}{3mm}
  Expected Result \\
{\footnotesize
\begin{itemize}
\tightlist
\item
  TMA reaches the DIMM position.
\item
  DIMM imaging quality is sufficient.
\end{itemize}

}

\begin{tabular}{p{2cm}}
\toprule
Step 617  \\ \hline
\end{tabular}
 Description \\
{\footnotesize
\textbf{Move TMA to the DIMM position and \textbf{Take DIMM images}}\\

\begin{itemize}
\tightlist
\item
  Command the TMA to the DIMM position by applying the offsets
\item
  While tracking, take DIMM images with XXXs exposure time and inspect
  the quality.
\end{itemize}

}
\hdashrule[0.5ex]{\textwidth}{1pt}{3mm}
  Expected Result \\
{\footnotesize
\begin{itemize}
\tightlist
\item
  TMA reaches the DIMM position.
\item
  DIMM imaging quality is sufficient.
\end{itemize}

}

\begin{tabular}{p{2cm}}
\toprule
Step 618  \\ \hline
\end{tabular}
 Description \\
{\footnotesize
\textbf{Move TMA to the DIMM position and \textbf{Take DIMM images}}\\

\begin{itemize}
\tightlist
\item
  Command the TMA to the DIMM position by applying the offsets
\item
  While tracking, take DIMM images with XXXs exposure time and inspect
  the quality.
\end{itemize}

}
\hdashrule[0.5ex]{\textwidth}{1pt}{3mm}
  Expected Result \\
{\footnotesize
\begin{itemize}
\tightlist
\item
  TMA reaches the DIMM position.
\item
  DIMM imaging quality is sufficient.
\end{itemize}

}

\begin{tabular}{p{2cm}}
\toprule
Step 619  \\ \hline
\end{tabular}
 Description \\
{\footnotesize
\textbf{Move TMA to the DIMM position and \textbf{Take DIMM images}}\\

\begin{itemize}
\tightlist
\item
  Command the TMA to the DIMM position by applying the offsets
\item
  While tracking, take DIMM images with XXXs exposure time and inspect
  the quality.
\end{itemize}

}
\hdashrule[0.5ex]{\textwidth}{1pt}{3mm}
  Expected Result \\
{\footnotesize
\begin{itemize}
\tightlist
\item
  TMA reaches the DIMM position.
\item
  DIMM imaging quality is sufficient.
\end{itemize}

}

\begin{tabular}{p{2cm}}
\toprule
Step 620  \\ \hline
\end{tabular}
 Description \\
{\footnotesize
\textbf{Move TMA to the DIMM position and \textbf{Take DIMM images}}\\

\begin{itemize}
\tightlist
\item
  Command the TMA to the DIMM position by applying the offsets
\item
  While tracking, take DIMM images with XXXs exposure time and inspect
  the quality.
\end{itemize}

}
\hdashrule[0.5ex]{\textwidth}{1pt}{3mm}
  Expected Result \\
{\footnotesize
\begin{itemize}
\tightlist
\item
  TMA reaches the DIMM position.
\item
  DIMM imaging quality is sufficient.
\end{itemize}

}

\begin{tabular}{p{2cm}}
\toprule
Step 621  \\ \hline
\end{tabular}
 Description \\
{\footnotesize
\textbf{Move TMA to the DIMM position and \textbf{Take DIMM images}}\\

\begin{itemize}
\tightlist
\item
  Command the TMA to the DIMM position by applying the offsets
\item
  While tracking, take DIMM images with XXXs exposure time and inspect
  the quality.
\end{itemize}

}
\hdashrule[0.5ex]{\textwidth}{1pt}{3mm}
  Expected Result \\
{\footnotesize
\begin{itemize}
\tightlist
\item
  TMA reaches the DIMM position.
\item
  DIMM imaging quality is sufficient.
\end{itemize}

}

\begin{tabular}{p{2cm}}
\toprule
Step 622  \\ \hline
\end{tabular}
 Description \\
{\footnotesize
\textbf{Move TMA to the DIMM position and \textbf{Take DIMM images}}\\

\begin{itemize}
\tightlist
\item
  Command the TMA to the DIMM position by applying the offsets
\item
  While tracking, take DIMM images with XXXs exposure time and inspect
  the quality.
\end{itemize}

}
\hdashrule[0.5ex]{\textwidth}{1pt}{3mm}
  Expected Result \\
{\footnotesize
\begin{itemize}
\tightlist
\item
  TMA reaches the DIMM position.
\item
  DIMM imaging quality is sufficient.
\end{itemize}

}

\begin{tabular}{p{2cm}}
\toprule
Step 623  \\ \hline
\end{tabular}
 Description \\
{\footnotesize
\textbf{Move TMA to the DIMM position and \textbf{Take DIMM images}}\\

\begin{itemize}
\tightlist
\item
  Command the TMA to the DIMM position by applying the offsets
\item
  While tracking, take DIMM images with XXXs exposure time and inspect
  the quality.
\end{itemize}

}
\hdashrule[0.5ex]{\textwidth}{1pt}{3mm}
  Expected Result \\
{\footnotesize
\begin{itemize}
\tightlist
\item
  TMA reaches the DIMM position.
\item
  DIMM imaging quality is sufficient.
\end{itemize}

}

\begin{tabular}{p{2cm}}
\toprule
Step 624  \\ \hline
\end{tabular}
 Description \\
{\footnotesize
\textbf{Point the TMA to (Az, El)-pattern position + DIMM pattern
offset\textbf{~and take DIMM images}\\
}

\begin{itemize}
\tightlist
\item
  Point the TMA back to {Pointing 1}⁠ at {-270}⁠ + DIMM offset, {15}⁠ +
  DIMM offset.
\item
  While tracking, take DIMM images with XXXs exposure time and inspect
  the quality.
\end{itemize}

}
\hdashrule[0.5ex]{\textwidth}{1pt}{3mm}
  Expected Result \\
{\footnotesize
\begin{itemize}
\tightlist
\item
  TMA reaches the position
\item
  DIMM image quality is sufficient
\end{itemize}

}

\begin{tabular}{p{2cm}}
\toprule
Step 625  \\ \hline
\end{tabular}
 Description \\
{\footnotesize
\textbf{Point the TMA to (Az, El)-pattern position + DIMM pattern
offset\textbf{~and take DIMM images}\\
}

\begin{itemize}
\tightlist
\item
  Point the TMA back to {Pointing 2}⁠ at {-270}⁠ + DIMM offset, {45}⁠ +
  DIMM offset.
\item
  While tracking, take DIMM images with XXXs exposure time and inspect
  the quality.
\end{itemize}

}
\hdashrule[0.5ex]{\textwidth}{1pt}{3mm}
  Expected Result \\
{\footnotesize
\begin{itemize}
\tightlist
\item
  TMA reaches the position
\item
  DIMM image quality is sufficient
\end{itemize}

}

\begin{tabular}{p{2cm}}
\toprule
Step 626  \\ \hline
\end{tabular}
 Description \\
{\footnotesize
\textbf{Point the TMA to (Az, El)-pattern position + DIMM pattern
offset\textbf{~and take DIMM images}\\
}

\begin{itemize}
\tightlist
\item
  Point the TMA back to {Pointing 3}⁠ at {-270}⁠ + DIMM offset, {75}⁠ +
  DIMM offset.
\item
  While tracking, take DIMM images with XXXs exposure time and inspect
  the quality.
\end{itemize}

}
\hdashrule[0.5ex]{\textwidth}{1pt}{3mm}
  Expected Result \\
{\footnotesize
\begin{itemize}
\tightlist
\item
  TMA reaches the position
\item
  DIMM image quality is sufficient
\end{itemize}

}

\begin{tabular}{p{2cm}}
\toprule
Step 627  \\ \hline
\end{tabular}
 Description \\
{\footnotesize
\textbf{Point the TMA to (Az, El)-pattern position + DIMM pattern
offset\textbf{~and take DIMM images}\\
}

\begin{itemize}
\tightlist
\item
  Point the TMA back to {Pointing 4}⁠ at {-270}⁠ + DIMM offset, {86.5}⁠
  + DIMM offset.
\item
  While tracking, take DIMM images with XXXs exposure time and inspect
  the quality.
\end{itemize}

}
\hdashrule[0.5ex]{\textwidth}{1pt}{3mm}
  Expected Result \\
{\footnotesize
\begin{itemize}
\tightlist
\item
  TMA reaches the position
\item
  DIMM image quality is sufficient
\end{itemize}

}

\begin{tabular}{p{2cm}}
\toprule
Step 628  \\ \hline
\end{tabular}
 Description \\
{\footnotesize
\textbf{Point the TMA to (Az, El)-pattern position + DIMM pattern
offset\textbf{~and take DIMM images}\\
}

\begin{itemize}
\tightlist
\item
  Point the TMA back to {Pointing 5}⁠ at {-180}⁠ + DIMM offset, {86.5}⁠
  + DIMM offset.
\item
  While tracking, take DIMM images with XXXs exposure time and inspect
  the quality.
\end{itemize}

}
\hdashrule[0.5ex]{\textwidth}{1pt}{3mm}
  Expected Result \\
{\footnotesize
\begin{itemize}
\tightlist
\item
  TMA reaches the position
\item
  DIMM image quality is sufficient
\end{itemize}

}

\begin{tabular}{p{2cm}}
\toprule
Step 629  \\ \hline
\end{tabular}
 Description \\
{\footnotesize
\textbf{Point the TMA to (Az, El)-pattern position + DIMM pattern
offset\textbf{~and take DIMM images}\\
}

\begin{itemize}
\tightlist
\item
  Point the TMA back to {Pointing 6}⁠ at {-180}⁠ + DIMM offset, {75}⁠ +
  DIMM offset.
\item
  While tracking, take DIMM images with XXXs exposure time and inspect
  the quality.
\end{itemize}

}
\hdashrule[0.5ex]{\textwidth}{1pt}{3mm}
  Expected Result \\
{\footnotesize
\begin{itemize}
\tightlist
\item
  TMA reaches the position
\item
  DIMM image quality is sufficient
\end{itemize}

}

\begin{tabular}{p{2cm}}
\toprule
Step 630  \\ \hline
\end{tabular}
 Description \\
{\footnotesize
\textbf{Point the TMA to (Az, El)-pattern position + DIMM pattern
offset\textbf{~and take DIMM images}\\
}

\begin{itemize}
\tightlist
\item
  Point the TMA back to {Pointing 7}⁠ at {-180}⁠ + DIMM offset, {45}⁠ +
  DIMM offset.
\item
  While tracking, take DIMM images with XXXs exposure time and inspect
  the quality.
\end{itemize}

}
\hdashrule[0.5ex]{\textwidth}{1pt}{3mm}
  Expected Result \\
{\footnotesize
\begin{itemize}
\tightlist
\item
  TMA reaches the position
\item
  DIMM image quality is sufficient
\end{itemize}

}

\begin{tabular}{p{2cm}}
\toprule
Step 631  \\ \hline
\end{tabular}
 Description \\
{\footnotesize
\textbf{Point the TMA to (Az, El)-pattern position + DIMM pattern
offset\textbf{~and take DIMM images}\\
}

\begin{itemize}
\tightlist
\item
  Point the TMA back to {Pointing 8}⁠ at {-180}⁠ + DIMM offset, {15}⁠ +
  DIMM offset.
\item
  While tracking, take DIMM images with XXXs exposure time and inspect
  the quality.
\end{itemize}

}
\hdashrule[0.5ex]{\textwidth}{1pt}{3mm}
  Expected Result \\
{\footnotesize
\begin{itemize}
\tightlist
\item
  TMA reaches the position
\item
  DIMM image quality is sufficient
\end{itemize}

}

\begin{tabular}{p{2cm}}
\toprule
Step 632  \\ \hline
\end{tabular}
 Description \\
{\footnotesize
\textbf{Point the TMA to (Az, El)-pattern position + DIMM pattern
offset\textbf{~and take DIMM images}\\
}

\begin{itemize}
\tightlist
\item
  Point the TMA back to {Pointing 9}⁠ at {-90}⁠ + DIMM offset, {15}⁠ +
  DIMM offset.
\item
  While tracking, take DIMM images with XXXs exposure time and inspect
  the quality.
\end{itemize}

}
\hdashrule[0.5ex]{\textwidth}{1pt}{3mm}
  Expected Result \\
{\footnotesize
\begin{itemize}
\tightlist
\item
  TMA reaches the position
\item
  DIMM image quality is sufficient
\end{itemize}

}

\begin{tabular}{p{2cm}}
\toprule
Step 633  \\ \hline
\end{tabular}
 Description \\
{\footnotesize
\textbf{Point the TMA to (Az, El)-pattern position + DIMM pattern
offset\textbf{~and take DIMM images}\\
}

\begin{itemize}
\tightlist
\item
  Point the TMA back to {Pointing 10}⁠ at {-90}⁠ + DIMM offset, {45}⁠ +
  DIMM offset.
\item
  While tracking, take DIMM images with XXXs exposure time and inspect
  the quality.
\end{itemize}

}
\hdashrule[0.5ex]{\textwidth}{1pt}{3mm}
  Expected Result \\
{\footnotesize
\begin{itemize}
\tightlist
\item
  TMA reaches the position
\item
  DIMM image quality is sufficient
\end{itemize}

}

\begin{tabular}{p{2cm}}
\toprule
Step 634  \\ \hline
\end{tabular}
 Description \\
{\footnotesize
\textbf{Point the TMA to (Az, El)-pattern position + DIMM pattern
offset\textbf{~and take DIMM images}\\
}

\begin{itemize}
\tightlist
\item
  Point the TMA back to {Pointing 11}⁠ at {-90}⁠ + DIMM offset, {75}⁠ +
  DIMM offset.
\item
  While tracking, take DIMM images with XXXs exposure time and inspect
  the quality.
\end{itemize}

}
\hdashrule[0.5ex]{\textwidth}{1pt}{3mm}
  Expected Result \\
{\footnotesize
\begin{itemize}
\tightlist
\item
  TMA reaches the position
\item
  DIMM image quality is sufficient
\end{itemize}

}

\begin{tabular}{p{2cm}}
\toprule
Step 635  \\ \hline
\end{tabular}
 Description \\
{\footnotesize
\textbf{Point the TMA to (Az, El)-pattern position + DIMM pattern
offset\textbf{~and take DIMM images}\\
}

\begin{itemize}
\tightlist
\item
  Point the TMA back to {Pointing 12}⁠ at {-90}⁠ + DIMM offset, {86.5}⁠
  + DIMM offset.
\item
  While tracking, take DIMM images with XXXs exposure time and inspect
  the quality.
\end{itemize}

}
\hdashrule[0.5ex]{\textwidth}{1pt}{3mm}
  Expected Result \\
{\footnotesize
\begin{itemize}
\tightlist
\item
  TMA reaches the position
\item
  DIMM image quality is sufficient
\end{itemize}

}

\begin{tabular}{p{2cm}}
\toprule
Step 636  \\ \hline
\end{tabular}
 Description \\
{\footnotesize
\textbf{Point the TMA to (Az, El)-pattern position + DIMM pattern
offset\textbf{~and take DIMM images}\\
}

\begin{itemize}
\tightlist
\item
  Point the TMA back to {Pointing 13}⁠ at {0}⁠ + DIMM offset, {86.5}⁠ +
  DIMM offset.
\item
  While tracking, take DIMM images with XXXs exposure time and inspect
  the quality.
\end{itemize}

}
\hdashrule[0.5ex]{\textwidth}{1pt}{3mm}
  Expected Result \\
{\footnotesize
\begin{itemize}
\tightlist
\item
  TMA reaches the position
\item
  DIMM image quality is sufficient
\end{itemize}

}

\begin{tabular}{p{2cm}}
\toprule
Step 637  \\ \hline
\end{tabular}
 Description \\
{\footnotesize
\textbf{Point the TMA to (Az, El)-pattern position + DIMM pattern
offset\textbf{~and take DIMM images}\\
}

\begin{itemize}
\tightlist
\item
  Point the TMA back to {Pointing 14}⁠ at {0}⁠ + DIMM offset, {75}⁠ +
  DIMM offset.
\item
  While tracking, take DIMM images with XXXs exposure time and inspect
  the quality.
\end{itemize}

}
\hdashrule[0.5ex]{\textwidth}{1pt}{3mm}
  Expected Result \\
{\footnotesize
\begin{itemize}
\tightlist
\item
  TMA reaches the position
\item
  DIMM image quality is sufficient
\end{itemize}

}

\begin{tabular}{p{2cm}}
\toprule
Step 638  \\ \hline
\end{tabular}
 Description \\
{\footnotesize
\textbf{Point the TMA to (Az, El)-pattern position + DIMM pattern
offset\textbf{~and take DIMM images}\\
}

\begin{itemize}
\tightlist
\item
  Point the TMA back to {Pointing 15}⁠ at {0}⁠ + DIMM offset, {45}⁠ +
  DIMM offset.
\item
  While tracking, take DIMM images with XXXs exposure time and inspect
  the quality.
\end{itemize}

}
\hdashrule[0.5ex]{\textwidth}{1pt}{3mm}
  Expected Result \\
{\footnotesize
\begin{itemize}
\tightlist
\item
  TMA reaches the position
\item
  DIMM image quality is sufficient
\end{itemize}

}

\begin{tabular}{p{2cm}}
\toprule
Step 639  \\ \hline
\end{tabular}
 Description \\
{\footnotesize
\textbf{Point the TMA to (Az, El)-pattern position + DIMM pattern
offset\textbf{~and take DIMM images}\\
}

\begin{itemize}
\tightlist
\item
  Point the TMA back to {Pointing 16}⁠ at {0}⁠ + DIMM offset, {15}⁠ +
  DIMM offset.
\item
  While tracking, take DIMM images with XXXs exposure time and inspect
  the quality.
\end{itemize}

}
\hdashrule[0.5ex]{\textwidth}{1pt}{3mm}
  Expected Result \\
{\footnotesize
\begin{itemize}
\tightlist
\item
  TMA reaches the position
\item
  DIMM image quality is sufficient
\end{itemize}

}

\begin{tabular}{p{2cm}}
\toprule
Step 640  \\ \hline
\end{tabular}
 Description \\
{\footnotesize
\textbf{Point the TMA to (Az, El)-pattern position + DIMM pattern
offset\textbf{~and take DIMM images}\\
}

\begin{itemize}
\tightlist
\item
  Point the TMA back to {Pointing 17}⁠ at {90}⁠ + DIMM offset, {15}⁠ +
  DIMM offset.
\item
  While tracking, take DIMM images with XXXs exposure time and inspect
  the quality.
\end{itemize}

}
\hdashrule[0.5ex]{\textwidth}{1pt}{3mm}
  Expected Result \\
{\footnotesize
\begin{itemize}
\tightlist
\item
  TMA reaches the position
\item
  DIMM image quality is sufficient
\end{itemize}

}

\begin{tabular}{p{2cm}}
\toprule
Step 641  \\ \hline
\end{tabular}
 Description \\
{\footnotesize
\textbf{Point the TMA to (Az, El)-pattern position + DIMM pattern
offset\textbf{~and take DIMM images}\\
}

\begin{itemize}
\tightlist
\item
  Point the TMA back to {Pointing 18}⁠ at {90}⁠ + DIMM offset, {45}⁠ +
  DIMM offset.
\item
  While tracking, take DIMM images with XXXs exposure time and inspect
  the quality.
\end{itemize}

}
\hdashrule[0.5ex]{\textwidth}{1pt}{3mm}
  Expected Result \\
{\footnotesize
\begin{itemize}
\tightlist
\item
  TMA reaches the position
\item
  DIMM image quality is sufficient
\end{itemize}

}

\begin{tabular}{p{2cm}}
\toprule
Step 642  \\ \hline
\end{tabular}
 Description \\
{\footnotesize
\textbf{Point the TMA to (Az, El)-pattern position + DIMM pattern
offset\textbf{~and take DIMM images}\\
}

\begin{itemize}
\tightlist
\item
  Point the TMA back to {Pointing 19}⁠ at {90}⁠ + DIMM offset, {75}⁠ +
  DIMM offset.
\item
  While tracking, take DIMM images with XXXs exposure time and inspect
  the quality.
\end{itemize}

}
\hdashrule[0.5ex]{\textwidth}{1pt}{3mm}
  Expected Result \\
{\footnotesize
\begin{itemize}
\tightlist
\item
  TMA reaches the position
\item
  DIMM image quality is sufficient
\end{itemize}

}

\begin{tabular}{p{2cm}}
\toprule
Step 643  \\ \hline
\end{tabular}
 Description \\
{\footnotesize
\textbf{Point the TMA to (Az, El)-pattern position + DIMM pattern
offset\textbf{~and take DIMM images}\\
}

\begin{itemize}
\tightlist
\item
  Point the TMA back to {Pointing 20}⁠ at {90}⁠ + DIMM offset, {86.5}⁠ +
  DIMM offset.
\item
  While tracking, take DIMM images with XXXs exposure time and inspect
  the quality.
\end{itemize}

}
\hdashrule[0.5ex]{\textwidth}{1pt}{3mm}
  Expected Result \\
{\footnotesize
\begin{itemize}
\tightlist
\item
  TMA reaches the position
\item
  DIMM image quality is sufficient
\end{itemize}

}

\begin{tabular}{p{2cm}}
\toprule
Step 644  \\ \hline
\end{tabular}
 Description \\
{\footnotesize
\textbf{Point the TMA to (Az, El)-pattern position + DIMM pattern
offset\textbf{~and take DIMM images}\\
}

\begin{itemize}
\tightlist
\item
  Point the TMA back to {Pointing 21}⁠ at {180}⁠ + DIMM offset, {86.5}⁠
  + DIMM offset.
\item
  While tracking, take DIMM images with XXXs exposure time and inspect
  the quality.
\end{itemize}

}
\hdashrule[0.5ex]{\textwidth}{1pt}{3mm}
  Expected Result \\
{\footnotesize
\begin{itemize}
\tightlist
\item
  TMA reaches the position
\item
  DIMM image quality is sufficient
\end{itemize}

}

\begin{tabular}{p{2cm}}
\toprule
Step 645  \\ \hline
\end{tabular}
 Description \\
{\footnotesize
\textbf{Point the TMA to (Az, El)-pattern position + DIMM pattern
offset\textbf{~and take DIMM images}\\
}

\begin{itemize}
\tightlist
\item
  Point the TMA back to {Pointing 22}⁠ at {180}⁠ + DIMM offset, {75}⁠ +
  DIMM offset.
\item
  While tracking, take DIMM images with XXXs exposure time and inspect
  the quality.
\end{itemize}

}
\hdashrule[0.5ex]{\textwidth}{1pt}{3mm}
  Expected Result \\
{\footnotesize
\begin{itemize}
\tightlist
\item
  TMA reaches the position
\item
  DIMM image quality is sufficient
\end{itemize}

}

\begin{tabular}{p{2cm}}
\toprule
Step 646  \\ \hline
\end{tabular}
 Description \\
{\footnotesize
\textbf{Point the TMA to (Az, El)-pattern position + DIMM pattern
offset\textbf{~and take DIMM images}\\
}

\begin{itemize}
\tightlist
\item
  Point the TMA back to {Pointing 23}⁠ at {180}⁠ + DIMM offset, {45}⁠ +
  DIMM offset.
\item
  While tracking, take DIMM images with XXXs exposure time and inspect
  the quality.
\end{itemize}

}
\hdashrule[0.5ex]{\textwidth}{1pt}{3mm}
  Expected Result \\
{\footnotesize
\begin{itemize}
\tightlist
\item
  TMA reaches the position
\item
  DIMM image quality is sufficient
\end{itemize}

}

\begin{tabular}{p{2cm}}
\toprule
Step 647  \\ \hline
\end{tabular}
 Description \\
{\footnotesize
\textbf{Point the TMA to (Az, El)-pattern position + DIMM pattern
offset\textbf{~and take DIMM images}\\
}

\begin{itemize}
\tightlist
\item
  Point the TMA back to {Pointing 24}⁠ at {180}⁠ + DIMM offset, {15}⁠ +
  DIMM offset.
\item
  While tracking, take DIMM images with XXXs exposure time and inspect
  the quality.
\end{itemize}

}
\hdashrule[0.5ex]{\textwidth}{1pt}{3mm}
  Expected Result \\
{\footnotesize
\begin{itemize}
\tightlist
\item
  TMA reaches the position
\item
  DIMM image quality is sufficient
\end{itemize}

}

\begin{tabular}{p{2cm}}
\toprule
Step 648  \\ \hline
\end{tabular}
 Description \\
{\footnotesize
\textbf{Point the TMA to (Az, El)-pattern position + DIMM pattern
offset\textbf{~and take DIMM images}\\
}

\begin{itemize}
\tightlist
\item
  Point the TMA back to {Pointing 25}⁠ at {270}⁠ + DIMM offset, {15}⁠ +
  DIMM offset.
\item
  While tracking, take DIMM images with XXXs exposure time and inspect
  the quality.
\end{itemize}

}
\hdashrule[0.5ex]{\textwidth}{1pt}{3mm}
  Expected Result \\
{\footnotesize
\begin{itemize}
\tightlist
\item
  TMA reaches the position
\item
  DIMM image quality is sufficient
\end{itemize}

}

\begin{tabular}{p{2cm}}
\toprule
Step 649  \\ \hline
\end{tabular}
 Description \\
{\footnotesize
\textbf{Point the TMA to (Az, El)-pattern position + DIMM pattern
offset\textbf{~and take DIMM images}\\
}

\begin{itemize}
\tightlist
\item
  Point the TMA back to {Pointing 26}⁠ at {270}⁠ + DIMM offset, {45}⁠ +
  DIMM offset.
\item
  While tracking, take DIMM images with XXXs exposure time and inspect
  the quality.
\end{itemize}

}
\hdashrule[0.5ex]{\textwidth}{1pt}{3mm}
  Expected Result \\
{\footnotesize
\begin{itemize}
\tightlist
\item
  TMA reaches the position
\item
  DIMM image quality is sufficient
\end{itemize}

}

\begin{tabular}{p{2cm}}
\toprule
Step 650  \\ \hline
\end{tabular}
 Description \\
{\footnotesize
\textbf{Point the TMA to (Az, El)-pattern position + DIMM pattern
offset\textbf{~and take DIMM images}\\
}

\begin{itemize}
\tightlist
\item
  Point the TMA back to {Pointing 27}⁠ at {270}⁠ + DIMM offset, {75}⁠ +
  DIMM offset.
\item
  While tracking, take DIMM images with XXXs exposure time and inspect
  the quality.
\end{itemize}

}
\hdashrule[0.5ex]{\textwidth}{1pt}{3mm}
  Expected Result \\
{\footnotesize
\begin{itemize}
\tightlist
\item
  TMA reaches the position
\item
  DIMM image quality is sufficient
\end{itemize}

}

\begin{tabular}{p{2cm}}
\toprule
Step 651  \\ \hline
\end{tabular}
 Description \\
{\footnotesize
\textbf{Point the TMA to (Az, El)-pattern position + DIMM pattern
offset\textbf{~and take DIMM images}\\
}

\begin{itemize}
\tightlist
\item
  Point the TMA back to {Pointing 13}⁠ at {0}⁠ + DIMM offset, {86.5}⁠ +
  DIMM offset.
\item
  While tracking, take DIMM images with XXXs exposure time and inspect
  the quality.
\end{itemize}

}
\hdashrule[0.5ex]{\textwidth}{1pt}{3mm}
  Expected Result \\
{\footnotesize
\begin{itemize}
\tightlist
\item
  TMA reaches the position
\item
  DIMM image quality is sufficient
\end{itemize}

}

\begin{tabular}{p{2cm}}
\toprule
Step 652  \\ \hline
\end{tabular}
 Description \\
{\footnotesize
\textbf{Point the TMA to (Az, El)-pattern position + DIMM pattern
offset\textbf{~and take DIMM images}\\
}

\begin{itemize}
\tightlist
\item
  Point the TMA back to {Pointing 28}⁠ at {270}⁠ + DIMM offset, {86.5}⁠
  + DIMM offset.
\item
  While tracking, take DIMM images with XXXs exposure time and inspect
  the quality.
\end{itemize}

}
\hdashrule[0.5ex]{\textwidth}{1pt}{3mm}
  Expected Result \\
{\footnotesize
\begin{itemize}
\tightlist
\item
  TMA reaches the position
\item
  DIMM image quality is sufficient
\end{itemize}

}

\begin{tabular}{p{2cm}}
\toprule
Step 653  \\ \hline
\end{tabular}
 Description \\
{\footnotesize
\textbf{Point the TMA to (Az, El)-pattern position + DIMM pattern
offset\textbf{~and take DIMM images}\\
}

\begin{itemize}
\tightlist
\item
  Point the TMA back to {Pointing 14}⁠ at {0}⁠ + DIMM offset, {75}⁠ +
  DIMM offset.
\item
  While tracking, take DIMM images with XXXs exposure time and inspect
  the quality.
\end{itemize}

}
\hdashrule[0.5ex]{\textwidth}{1pt}{3mm}
  Expected Result \\
{\footnotesize
\begin{itemize}
\tightlist
\item
  TMA reaches the position
\item
  DIMM image quality is sufficient
\end{itemize}

}

\begin{tabular}{p{2cm}}
\toprule
Step 654  \\ \hline
\end{tabular}
 Description \\
{\footnotesize
\textbf{Point the TMA to (Az, El)-pattern position + DIMM pattern
offset\textbf{~and take DIMM images}\\
}

\begin{itemize}
\tightlist
\item
  Point the TMA back to {Pointing 15}⁠ at {0}⁠ + DIMM offset, {45}⁠ +
  DIMM offset.
\item
  While tracking, take DIMM images with XXXs exposure time and inspect
  the quality.
\end{itemize}

}
\hdashrule[0.5ex]{\textwidth}{1pt}{3mm}
  Expected Result \\
{\footnotesize
\begin{itemize}
\tightlist
\item
  TMA reaches the position
\item
  DIMM image quality is sufficient
\end{itemize}

}

\begin{tabular}{p{2cm}}
\toprule
Step 655  \\ \hline
\end{tabular}
 Description \\
{\footnotesize
\textbf{Point the TMA to (Az, El)-pattern position + DIMM pattern
offset\textbf{~and take DIMM images}\\
}

\begin{itemize}
\tightlist
\item
  Point the TMA back to {Pointing 16}⁠ at {0}⁠ + DIMM offset, {15}⁠ +
  DIMM offset.
\item
  While tracking, take DIMM images with XXXs exposure time and inspect
  the quality.
\end{itemize}

}
\hdashrule[0.5ex]{\textwidth}{1pt}{3mm}
  Expected Result \\
{\footnotesize
\begin{itemize}
\tightlist
\item
  TMA reaches the position
\item
  DIMM image quality is sufficient
\end{itemize}

}

\begin{tabular}{p{2cm}}
\toprule
Step 656  \\ \hline
\end{tabular}
 Description \\
{\footnotesize
\textbf{Point the TMA to (Az, El)-pattern position + DIMM pattern
offset\textbf{~and take DIMM images}\\
}

\begin{itemize}
\tightlist
\item
  Point the TMA back to {Pointing 17}⁠ at {90}⁠ + DIMM offset, {15}⁠ +
  DIMM offset.
\item
  While tracking, take DIMM images with XXXs exposure time and inspect
  the quality.
\end{itemize}

}
\hdashrule[0.5ex]{\textwidth}{1pt}{3mm}
  Expected Result \\
{\footnotesize
\begin{itemize}
\tightlist
\item
  TMA reaches the position
\item
  DIMM image quality is sufficient
\end{itemize}

}

\begin{tabular}{p{2cm}}
\toprule
Step 657  \\ \hline
\end{tabular}
 Description \\
{\footnotesize
\textbf{Point the TMA to (Az, El)-pattern position + DIMM pattern
offset\textbf{~and take DIMM images}\\
}

\begin{itemize}
\tightlist
\item
  Point the TMA back to {Pointing 18}⁠ at {90}⁠ + DIMM offset, {45}⁠ +
  DIMM offset.
\item
  While tracking, take DIMM images with XXXs exposure time and inspect
  the quality.
\end{itemize}

}
\hdashrule[0.5ex]{\textwidth}{1pt}{3mm}
  Expected Result \\
{\footnotesize
\begin{itemize}
\tightlist
\item
  TMA reaches the position
\item
  DIMM image quality is sufficient
\end{itemize}

}

\begin{tabular}{p{2cm}}
\toprule
Step 658  \\ \hline
\end{tabular}
 Description \\
{\footnotesize
\textbf{Point the TMA to (Az, El)-pattern position + DIMM pattern
offset\textbf{~and take DIMM images}\\
}

\begin{itemize}
\tightlist
\item
  Point the TMA back to {Pointing 19}⁠ at {90}⁠ + DIMM offset, {75}⁠ +
  DIMM offset.
\item
  While tracking, take DIMM images with XXXs exposure time and inspect
  the quality.
\end{itemize}

}
\hdashrule[0.5ex]{\textwidth}{1pt}{3mm}
  Expected Result \\
{\footnotesize
\begin{itemize}
\tightlist
\item
  TMA reaches the position
\item
  DIMM image quality is sufficient
\end{itemize}

}

\begin{tabular}{p{2cm}}
\toprule
Step 659  \\ \hline
\end{tabular}
 Description \\
{\footnotesize
\textbf{Point the TMA to (Az, El)-pattern position + DIMM pattern
offset\textbf{~and take DIMM images}\\
}

\begin{itemize}
\tightlist
\item
  Point the TMA back to {Pointing 20}⁠ at {90}⁠ + DIMM offset, {86.5}⁠ +
  DIMM offset.
\item
  While tracking, take DIMM images with XXXs exposure time and inspect
  the quality.
\end{itemize}

}
\hdashrule[0.5ex]{\textwidth}{1pt}{3mm}
  Expected Result \\
{\footnotesize
\begin{itemize}
\tightlist
\item
  TMA reaches the position
\item
  DIMM image quality is sufficient
\end{itemize}

}

\begin{tabular}{p{2cm}}
\toprule
Step 660  \\ \hline
\end{tabular}
 Description \\
{\footnotesize
\textbf{Point the TMA to (Az, El)-pattern position + DIMM pattern
offset\textbf{~and take DIMM images}\\
}

\begin{itemize}
\tightlist
\item
  Point the TMA back to {Pointing 21}⁠ at {180}⁠ + DIMM offset, {86.5}⁠
  + DIMM offset.
\item
  While tracking, take DIMM images with XXXs exposure time and inspect
  the quality.
\end{itemize}

}
\hdashrule[0.5ex]{\textwidth}{1pt}{3mm}
  Expected Result \\
{\footnotesize
\begin{itemize}
\tightlist
\item
  TMA reaches the position
\item
  DIMM image quality is sufficient
\end{itemize}

}

\begin{tabular}{p{2cm}}
\toprule
Step 661  \\ \hline
\end{tabular}
 Description \\
{\footnotesize
\textbf{Point the TMA to (Az, El)-pattern position + DIMM pattern
offset\textbf{~and take DIMM images}\\
}

\begin{itemize}
\tightlist
\item
  Point the TMA back to {Pointing 22}⁠ at {180}⁠ + DIMM offset, {75}⁠ +
  DIMM offset.
\item
  While tracking, take DIMM images with XXXs exposure time and inspect
  the quality.
\end{itemize}

}
\hdashrule[0.5ex]{\textwidth}{1pt}{3mm}
  Expected Result \\
{\footnotesize
\begin{itemize}
\tightlist
\item
  TMA reaches the position
\item
  DIMM image quality is sufficient
\end{itemize}

}

\begin{tabular}{p{2cm}}
\toprule
Step 662  \\ \hline
\end{tabular}
 Description \\
{\footnotesize
\textbf{Point the TMA to (Az, El)-pattern position + DIMM pattern
offset\textbf{~and take DIMM images}\\
}

\begin{itemize}
\tightlist
\item
  Point the TMA back to {Pointing 23}⁠ at {180}⁠ + DIMM offset, {45}⁠ +
  DIMM offset.
\item
  While tracking, take DIMM images with XXXs exposure time and inspect
  the quality.
\end{itemize}

}
\hdashrule[0.5ex]{\textwidth}{1pt}{3mm}
  Expected Result \\
{\footnotesize
\begin{itemize}
\tightlist
\item
  TMA reaches the position
\item
  DIMM image quality is sufficient
\end{itemize}

}

\begin{tabular}{p{2cm}}
\toprule
Step 663  \\ \hline
\end{tabular}
 Description \\
{\footnotesize
\textbf{Point the TMA to (Az, El)-pattern position + DIMM pattern
offset\textbf{~and take DIMM images}\\
}

\begin{itemize}
\tightlist
\item
  Point the TMA back to {Pointing 24}⁠ at {180}⁠ + DIMM offset, {15}⁠ +
  DIMM offset.
\item
  While tracking, take DIMM images with XXXs exposure time and inspect
  the quality.
\end{itemize}

}
\hdashrule[0.5ex]{\textwidth}{1pt}{3mm}
  Expected Result \\
{\footnotesize
\begin{itemize}
\tightlist
\item
  TMA reaches the position
\item
  DIMM image quality is sufficient
\end{itemize}

}

\begin{tabular}{p{2cm}}
\toprule
Step 664  \\ \hline
\end{tabular}
 Description \\
{\footnotesize
\textbf{Point the TMA to (Az, El)-pattern position + DIMM pattern
offset\textbf{~and take DIMM images}\\
}

\begin{itemize}
\tightlist
\item
  Point the TMA back to {Pointing 25}⁠ at {270}⁠ + DIMM offset, {15}⁠ +
  DIMM offset.
\item
  While tracking, take DIMM images with XXXs exposure time and inspect
  the quality.
\end{itemize}

}
\hdashrule[0.5ex]{\textwidth}{1pt}{3mm}
  Expected Result \\
{\footnotesize
\begin{itemize}
\tightlist
\item
  TMA reaches the position
\item
  DIMM image quality is sufficient
\end{itemize}

}

\begin{tabular}{p{2cm}}
\toprule
Step 665  \\ \hline
\end{tabular}
 Description \\
{\footnotesize
\textbf{Point the TMA to (Az, El)-pattern position + DIMM pattern
offset\textbf{~and take DIMM images}\\
}

\begin{itemize}
\tightlist
\item
  Point the TMA back to {Pointing 26}⁠ at {270}⁠ + DIMM offset, {45}⁠ +
  DIMM offset.
\item
  While tracking, take DIMM images with XXXs exposure time and inspect
  the quality.
\end{itemize}

}
\hdashrule[0.5ex]{\textwidth}{1pt}{3mm}
  Expected Result \\
{\footnotesize
\begin{itemize}
\tightlist
\item
  TMA reaches the position
\item
  DIMM image quality is sufficient
\end{itemize}

}

\begin{tabular}{p{2cm}}
\toprule
Step 666  \\ \hline
\end{tabular}
 Description \\
{\footnotesize
\textbf{Point the TMA to (Az, El)-pattern position + DIMM pattern
offset\textbf{~and take DIMM images}\\
}

\begin{itemize}
\tightlist
\item
  Point the TMA back to {Pointing 27}⁠ at {270}⁠ + DIMM offset, {75}⁠ +
  DIMM offset.
\item
  While tracking, take DIMM images with XXXs exposure time and inspect
  the quality.
\end{itemize}

}
\hdashrule[0.5ex]{\textwidth}{1pt}{3mm}
  Expected Result \\
{\footnotesize
\begin{itemize}
\tightlist
\item
  TMA reaches the position
\item
  DIMM image quality is sufficient
\end{itemize}

}

\begin{tabular}{p{2cm}}
\toprule
Step 667  \\ \hline
\end{tabular}
 Description \\
{\footnotesize
\textbf{Point the TMA to (Az, El)-pattern position + DIMM pattern
offset\textbf{~and take DIMM images}\\
}

\begin{itemize}
\tightlist
\item
  Point the TMA back to {Pointing 28}⁠ at {270}⁠ + DIMM offset, {86.5}⁠
  + DIMM offset.
\item
  While tracking, take DIMM images with XXXs exposure time and inspect
  the quality.
\end{itemize}

}
\hdashrule[0.5ex]{\textwidth}{1pt}{3mm}
  Expected Result \\
{\footnotesize
\begin{itemize}
\tightlist
\item
  TMA reaches the position
\item
  DIMM image quality is sufficient
\end{itemize}

}

\begin{tabular}{p{2cm}}
\toprule
Step 668  \\ \hline
\end{tabular}
 Description \\
{\footnotesize
\textbf{Move TMA to the 5. random distance of 3.5deg\\
}

\begin{itemize}
\tightlist
\item
  Point the TMA to a random 3.5 deg combined offset in AZ and EL from
  {Pointing 1}⁠ at {-270}⁠, {15}⁠. Record the exact position of the
  offset in AZ and El
\end{itemize}

}
\hdashrule[0.5ex]{\textwidth}{1pt}{3mm}
  Expected Result \\
{\footnotesize
\begin{itemize}
\tightlist
\item
  The TMA reaches the commanded offset position.
\end{itemize}

}

\begin{tabular}{p{2cm}}
\toprule
Step 669  \\ \hline
\end{tabular}
 Description \\
{\footnotesize
\textbf{Move TMA to the 5. random distance of 3.5deg\\
}

\begin{itemize}
\tightlist
\item
  Point the TMA to a random 3.5 deg combined offset in AZ and EL from
  {Pointing 2}⁠ at {-270}⁠, {45}⁠. Record the exact position of the
  offset in AZ and El
\end{itemize}

}
\hdashrule[0.5ex]{\textwidth}{1pt}{3mm}
  Expected Result \\
{\footnotesize
\begin{itemize}
\tightlist
\item
  The TMA reaches the commanded offset position.
\end{itemize}

}

\begin{tabular}{p{2cm}}
\toprule
Step 670  \\ \hline
\end{tabular}
 Description \\
{\footnotesize
\textbf{Move TMA to the 5. random distance of 3.5deg\\
}

\begin{itemize}
\tightlist
\item
  Point the TMA to a random 3.5 deg combined offset in AZ and EL from
  {Pointing 3}⁠ at {-270}⁠, {75}⁠. Record the exact position of the
  offset in AZ and El
\end{itemize}

}
\hdashrule[0.5ex]{\textwidth}{1pt}{3mm}
  Expected Result \\
{\footnotesize
\begin{itemize}
\tightlist
\item
  The TMA reaches the commanded offset position.
\end{itemize}

}

\begin{tabular}{p{2cm}}
\toprule
Step 671  \\ \hline
\end{tabular}
 Description \\
{\footnotesize
\textbf{Move TMA to the 5. random distance of 3.5deg\\
}

\begin{itemize}
\tightlist
\item
  Point the TMA to a random 3.5 deg combined offset in AZ and EL from
  {Pointing 4}⁠ at {-270}⁠, {86.5}⁠. Record the exact position of the
  offset in AZ and El
\end{itemize}

}
\hdashrule[0.5ex]{\textwidth}{1pt}{3mm}
  Expected Result \\
{\footnotesize
\begin{itemize}
\tightlist
\item
  The TMA reaches the commanded offset position.
\end{itemize}

}

\begin{tabular}{p{2cm}}
\toprule
Step 672  \\ \hline
\end{tabular}
 Description \\
{\footnotesize
\textbf{Move TMA to the 5. random distance of 3.5deg\\
}

\begin{itemize}
\tightlist
\item
  Point the TMA to a random 3.5 deg combined offset in AZ and EL from
  {Pointing 5}⁠ at {-180}⁠, {86.5}⁠. Record the exact position of the
  offset in AZ and El
\end{itemize}

}
\hdashrule[0.5ex]{\textwidth}{1pt}{3mm}
  Expected Result \\
{\footnotesize
\begin{itemize}
\tightlist
\item
  The TMA reaches the commanded offset position.
\end{itemize}

}

\begin{tabular}{p{2cm}}
\toprule
Step 673  \\ \hline
\end{tabular}
 Description \\
{\footnotesize
\textbf{Move TMA to the 5. random distance of 3.5deg\\
}

\begin{itemize}
\tightlist
\item
  Point the TMA to a random 3.5 deg combined offset in AZ and EL from
  {Pointing 6}⁠ at {-180}⁠, {75}⁠. Record the exact position of the
  offset in AZ and El
\end{itemize}

}
\hdashrule[0.5ex]{\textwidth}{1pt}{3mm}
  Expected Result \\
{\footnotesize
\begin{itemize}
\tightlist
\item
  The TMA reaches the commanded offset position.
\end{itemize}

}

\begin{tabular}{p{2cm}}
\toprule
Step 674  \\ \hline
\end{tabular}
 Description \\
{\footnotesize
\textbf{Move TMA to the 5. random distance of 3.5deg\\
}

\begin{itemize}
\tightlist
\item
  Point the TMA to a random 3.5 deg combined offset in AZ and EL from
  {Pointing 7}⁠ at {-180}⁠, {45}⁠. Record the exact position of the
  offset in AZ and El
\end{itemize}

}
\hdashrule[0.5ex]{\textwidth}{1pt}{3mm}
  Expected Result \\
{\footnotesize
\begin{itemize}
\tightlist
\item
  The TMA reaches the commanded offset position.
\end{itemize}

}

\begin{tabular}{p{2cm}}
\toprule
Step 675  \\ \hline
\end{tabular}
 Description \\
{\footnotesize
\textbf{Move TMA to the 5. random distance of 3.5deg\\
}

\begin{itemize}
\tightlist
\item
  Point the TMA to a random 3.5 deg combined offset in AZ and EL from
  {Pointing 8}⁠ at {-180}⁠, {15}⁠. Record the exact position of the
  offset in AZ and El
\end{itemize}

}
\hdashrule[0.5ex]{\textwidth}{1pt}{3mm}
  Expected Result \\
{\footnotesize
\begin{itemize}
\tightlist
\item
  The TMA reaches the commanded offset position.
\end{itemize}

}

\begin{tabular}{p{2cm}}
\toprule
Step 676  \\ \hline
\end{tabular}
 Description \\
{\footnotesize
\textbf{Move TMA to the 5. random distance of 3.5deg\\
}

\begin{itemize}
\tightlist
\item
  Point the TMA to a random 3.5 deg combined offset in AZ and EL from
  {Pointing 9}⁠ at {-90}⁠, {15}⁠. Record the exact position of the
  offset in AZ and El
\end{itemize}

}
\hdashrule[0.5ex]{\textwidth}{1pt}{3mm}
  Expected Result \\
{\footnotesize
\begin{itemize}
\tightlist
\item
  The TMA reaches the commanded offset position.
\end{itemize}

}

\begin{tabular}{p{2cm}}
\toprule
Step 677  \\ \hline
\end{tabular}
 Description \\
{\footnotesize
\textbf{Move TMA to the 5. random distance of 3.5deg\\
}

\begin{itemize}
\tightlist
\item
  Point the TMA to a random 3.5 deg combined offset in AZ and EL from
  {Pointing 10}⁠ at {-90}⁠, {45}⁠. Record the exact position of the
  offset in AZ and El
\end{itemize}

}
\hdashrule[0.5ex]{\textwidth}{1pt}{3mm}
  Expected Result \\
{\footnotesize
\begin{itemize}
\tightlist
\item
  The TMA reaches the commanded offset position.
\end{itemize}

}

\begin{tabular}{p{2cm}}
\toprule
Step 678  \\ \hline
\end{tabular}
 Description \\
{\footnotesize
\textbf{Move TMA to the 5. random distance of 3.5deg\\
}

\begin{itemize}
\tightlist
\item
  Point the TMA to a random 3.5 deg combined offset in AZ and EL from
  {Pointing 11}⁠ at {-90}⁠, {75}⁠. Record the exact position of the
  offset in AZ and El
\end{itemize}

}
\hdashrule[0.5ex]{\textwidth}{1pt}{3mm}
  Expected Result \\
{\footnotesize
\begin{itemize}
\tightlist
\item
  The TMA reaches the commanded offset position.
\end{itemize}

}

\begin{tabular}{p{2cm}}
\toprule
Step 679  \\ \hline
\end{tabular}
 Description \\
{\footnotesize
\textbf{Move TMA to the 5. random distance of 3.5deg\\
}

\begin{itemize}
\tightlist
\item
  Point the TMA to a random 3.5 deg combined offset in AZ and EL from
  {Pointing 12}⁠ at {-90}⁠, {86.5}⁠. Record the exact position of the
  offset in AZ and El
\end{itemize}

}
\hdashrule[0.5ex]{\textwidth}{1pt}{3mm}
  Expected Result \\
{\footnotesize
\begin{itemize}
\tightlist
\item
  The TMA reaches the commanded offset position.
\end{itemize}

}

\begin{tabular}{p{2cm}}
\toprule
Step 680  \\ \hline
\end{tabular}
 Description \\
{\footnotesize
\textbf{Move TMA to the 5. random distance of 3.5deg\\
}

\begin{itemize}
\tightlist
\item
  Point the TMA to a random 3.5 deg combined offset in AZ and EL from
  {Pointing 13}⁠ at {0}⁠, {86.5}⁠. Record the exact position of the
  offset in AZ and El
\end{itemize}

}
\hdashrule[0.5ex]{\textwidth}{1pt}{3mm}
  Expected Result \\
{\footnotesize
\begin{itemize}
\tightlist
\item
  The TMA reaches the commanded offset position.
\end{itemize}

}

\begin{tabular}{p{2cm}}
\toprule
Step 681  \\ \hline
\end{tabular}
 Description \\
{\footnotesize
\textbf{Move TMA to the 5. random distance of 3.5deg\\
}

\begin{itemize}
\tightlist
\item
  Point the TMA to a random 3.5 deg combined offset in AZ and EL from
  {Pointing 14}⁠ at {0}⁠, {75}⁠. Record the exact position of the offset
  in AZ and El
\end{itemize}

}
\hdashrule[0.5ex]{\textwidth}{1pt}{3mm}
  Expected Result \\
{\footnotesize
\begin{itemize}
\tightlist
\item
  The TMA reaches the commanded offset position.
\end{itemize}

}

\begin{tabular}{p{2cm}}
\toprule
Step 682  \\ \hline
\end{tabular}
 Description \\
{\footnotesize
\textbf{Move TMA to the 5. random distance of 3.5deg\\
}

\begin{itemize}
\tightlist
\item
  Point the TMA to a random 3.5 deg combined offset in AZ and EL from
  {Pointing 15}⁠ at {0}⁠, {45}⁠. Record the exact position of the offset
  in AZ and El
\end{itemize}

}
\hdashrule[0.5ex]{\textwidth}{1pt}{3mm}
  Expected Result \\
{\footnotesize
\begin{itemize}
\tightlist
\item
  The TMA reaches the commanded offset position.
\end{itemize}

}

\begin{tabular}{p{2cm}}
\toprule
Step 683  \\ \hline
\end{tabular}
 Description \\
{\footnotesize
\textbf{Move TMA to the 5. random distance of 3.5deg\\
}

\begin{itemize}
\tightlist
\item
  Point the TMA to a random 3.5 deg combined offset in AZ and EL from
  {Pointing 16}⁠ at {0}⁠, {15}⁠. Record the exact position of the offset
  in AZ and El
\end{itemize}

}
\hdashrule[0.5ex]{\textwidth}{1pt}{3mm}
  Expected Result \\
{\footnotesize
\begin{itemize}
\tightlist
\item
  The TMA reaches the commanded offset position.
\end{itemize}

}

\begin{tabular}{p{2cm}}
\toprule
Step 684  \\ \hline
\end{tabular}
 Description \\
{\footnotesize
\textbf{Move TMA to the 5. random distance of 3.5deg\\
}

\begin{itemize}
\tightlist
\item
  Point the TMA to a random 3.5 deg combined offset in AZ and EL from
  {Pointing 17}⁠ at {90}⁠, {15}⁠. Record the exact position of the
  offset in AZ and El
\end{itemize}

}
\hdashrule[0.5ex]{\textwidth}{1pt}{3mm}
  Expected Result \\
{\footnotesize
\begin{itemize}
\tightlist
\item
  The TMA reaches the commanded offset position.
\end{itemize}

}

\begin{tabular}{p{2cm}}
\toprule
Step 685  \\ \hline
\end{tabular}
 Description \\
{\footnotesize
\textbf{Move TMA to the 5. random distance of 3.5deg\\
}

\begin{itemize}
\tightlist
\item
  Point the TMA to a random 3.5 deg combined offset in AZ and EL from
  {Pointing 18}⁠ at {90}⁠, {45}⁠. Record the exact position of the
  offset in AZ and El
\end{itemize}

}
\hdashrule[0.5ex]{\textwidth}{1pt}{3mm}
  Expected Result \\
{\footnotesize
\begin{itemize}
\tightlist
\item
  The TMA reaches the commanded offset position.
\end{itemize}

}

\begin{tabular}{p{2cm}}
\toprule
Step 686  \\ \hline
\end{tabular}
 Description \\
{\footnotesize
\textbf{Move TMA to the 5. random distance of 3.5deg\\
}

\begin{itemize}
\tightlist
\item
  Point the TMA to a random 3.5 deg combined offset in AZ and EL from
  {Pointing 19}⁠ at {90}⁠, {75}⁠. Record the exact position of the
  offset in AZ and El
\end{itemize}

}
\hdashrule[0.5ex]{\textwidth}{1pt}{3mm}
  Expected Result \\
{\footnotesize
\begin{itemize}
\tightlist
\item
  The TMA reaches the commanded offset position.
\end{itemize}

}

\begin{tabular}{p{2cm}}
\toprule
Step 687  \\ \hline
\end{tabular}
 Description \\
{\footnotesize
\textbf{Move TMA to the 5. random distance of 3.5deg\\
}

\begin{itemize}
\tightlist
\item
  Point the TMA to a random 3.5 deg combined offset in AZ and EL from
  {Pointing 20}⁠ at {90}⁠, {86.5}⁠. Record the exact position of the
  offset in AZ and El
\end{itemize}

}
\hdashrule[0.5ex]{\textwidth}{1pt}{3mm}
  Expected Result \\
{\footnotesize
\begin{itemize}
\tightlist
\item
  The TMA reaches the commanded offset position.
\end{itemize}

}

\begin{tabular}{p{2cm}}
\toprule
Step 688  \\ \hline
\end{tabular}
 Description \\
{\footnotesize
\textbf{Move TMA to the 5. random distance of 3.5deg\\
}

\begin{itemize}
\tightlist
\item
  Point the TMA to a random 3.5 deg combined offset in AZ and EL from
  {Pointing 21}⁠ at {180}⁠, {86.5}⁠. Record the exact position of the
  offset in AZ and El
\end{itemize}

}
\hdashrule[0.5ex]{\textwidth}{1pt}{3mm}
  Expected Result \\
{\footnotesize
\begin{itemize}
\tightlist
\item
  The TMA reaches the commanded offset position.
\end{itemize}

}

\begin{tabular}{p{2cm}}
\toprule
Step 689  \\ \hline
\end{tabular}
 Description \\
{\footnotesize
\textbf{Move TMA to the 5. random distance of 3.5deg\\
}

\begin{itemize}
\tightlist
\item
  Point the TMA to a random 3.5 deg combined offset in AZ and EL from
  {Pointing 22}⁠ at {180}⁠, {75}⁠. Record the exact position of the
  offset in AZ and El
\end{itemize}

}
\hdashrule[0.5ex]{\textwidth}{1pt}{3mm}
  Expected Result \\
{\footnotesize
\begin{itemize}
\tightlist
\item
  The TMA reaches the commanded offset position.
\end{itemize}

}

\begin{tabular}{p{2cm}}
\toprule
Step 690  \\ \hline
\end{tabular}
 Description \\
{\footnotesize
\textbf{Move TMA to the 5. random distance of 3.5deg\\
}

\begin{itemize}
\tightlist
\item
  Point the TMA to a random 3.5 deg combined offset in AZ and EL from
  {Pointing 23}⁠ at {180}⁠, {45}⁠. Record the exact position of the
  offset in AZ and El
\end{itemize}

}
\hdashrule[0.5ex]{\textwidth}{1pt}{3mm}
  Expected Result \\
{\footnotesize
\begin{itemize}
\tightlist
\item
  The TMA reaches the commanded offset position.
\end{itemize}

}

\begin{tabular}{p{2cm}}
\toprule
Step 691  \\ \hline
\end{tabular}
 Description \\
{\footnotesize
\textbf{Move TMA to the 5. random distance of 3.5deg\\
}

\begin{itemize}
\tightlist
\item
  Point the TMA to a random 3.5 deg combined offset in AZ and EL from
  {Pointing 24}⁠ at {180}⁠, {15}⁠. Record the exact position of the
  offset in AZ and El
\end{itemize}

}
\hdashrule[0.5ex]{\textwidth}{1pt}{3mm}
  Expected Result \\
{\footnotesize
\begin{itemize}
\tightlist
\item
  The TMA reaches the commanded offset position.
\end{itemize}

}

\begin{tabular}{p{2cm}}
\toprule
Step 692  \\ \hline
\end{tabular}
 Description \\
{\footnotesize
\textbf{Move TMA to the 5. random distance of 3.5deg\\
}

\begin{itemize}
\tightlist
\item
  Point the TMA to a random 3.5 deg combined offset in AZ and EL from
  {Pointing 25}⁠ at {270}⁠, {15}⁠. Record the exact position of the
  offset in AZ and El
\end{itemize}

}
\hdashrule[0.5ex]{\textwidth}{1pt}{3mm}
  Expected Result \\
{\footnotesize
\begin{itemize}
\tightlist
\item
  The TMA reaches the commanded offset position.
\end{itemize}

}

\begin{tabular}{p{2cm}}
\toprule
Step 693  \\ \hline
\end{tabular}
 Description \\
{\footnotesize
\textbf{Move TMA to the 5. random distance of 3.5deg\\
}

\begin{itemize}
\tightlist
\item
  Point the TMA to a random 3.5 deg combined offset in AZ and EL from
  {Pointing 26}⁠ at {270}⁠, {45}⁠. Record the exact position of the
  offset in AZ and El
\end{itemize}

}
\hdashrule[0.5ex]{\textwidth}{1pt}{3mm}
  Expected Result \\
{\footnotesize
\begin{itemize}
\tightlist
\item
  The TMA reaches the commanded offset position.
\end{itemize}

}

\begin{tabular}{p{2cm}}
\toprule
Step 694  \\ \hline
\end{tabular}
 Description \\
{\footnotesize
\textbf{Move TMA to the 5. random distance of 3.5deg\\
}

\begin{itemize}
\tightlist
\item
  Point the TMA to a random 3.5 deg combined offset in AZ and EL from
  {Pointing 27}⁠ at {270}⁠, {75}⁠. Record the exact position of the
  offset in AZ and El
\end{itemize}

}
\hdashrule[0.5ex]{\textwidth}{1pt}{3mm}
  Expected Result \\
{\footnotesize
\begin{itemize}
\tightlist
\item
  The TMA reaches the commanded offset position.
\end{itemize}

}

\begin{tabular}{p{2cm}}
\toprule
Step 695  \\ \hline
\end{tabular}
 Description \\
{\footnotesize
\textbf{Move TMA to the 5. random distance of 3.5deg\\
}

\begin{itemize}
\tightlist
\item
  Point the TMA to a random 3.5 deg combined offset in AZ and EL from
  {Pointing 13}⁠ at {0}⁠, {86.5}⁠. Record the exact position of the
  offset in AZ and El
\end{itemize}

}
\hdashrule[0.5ex]{\textwidth}{1pt}{3mm}
  Expected Result \\
{\footnotesize
\begin{itemize}
\tightlist
\item
  The TMA reaches the commanded offset position.
\end{itemize}

}

\begin{tabular}{p{2cm}}
\toprule
Step 696  \\ \hline
\end{tabular}
 Description \\
{\footnotesize
\textbf{Move TMA to the 5. random distance of 3.5deg\\
}

\begin{itemize}
\tightlist
\item
  Point the TMA to a random 3.5 deg combined offset in AZ and EL from
  {Pointing 28}⁠ at {270}⁠, {86.5}⁠. Record the exact position of the
  offset in AZ and El
\end{itemize}

}
\hdashrule[0.5ex]{\textwidth}{1pt}{3mm}
  Expected Result \\
{\footnotesize
\begin{itemize}
\tightlist
\item
  The TMA reaches the commanded offset position.
\end{itemize}

}

\begin{tabular}{p{2cm}}
\toprule
Step 697  \\ \hline
\end{tabular}
 Description \\
{\footnotesize
\textbf{Move TMA to the 5. random distance of 3.5deg\\
}

\begin{itemize}
\tightlist
\item
  Point the TMA to a random 3.5 deg combined offset in AZ and EL from
  {Pointing 14}⁠ at {0}⁠, {75}⁠. Record the exact position of the offset
  in AZ and El
\end{itemize}

}
\hdashrule[0.5ex]{\textwidth}{1pt}{3mm}
  Expected Result \\
{\footnotesize
\begin{itemize}
\tightlist
\item
  The TMA reaches the commanded offset position.
\end{itemize}

}

\begin{tabular}{p{2cm}}
\toprule
Step 698  \\ \hline
\end{tabular}
 Description \\
{\footnotesize
\textbf{Move TMA to the 5. random distance of 3.5deg\\
}

\begin{itemize}
\tightlist
\item
  Point the TMA to a random 3.5 deg combined offset in AZ and EL from
  {Pointing 15}⁠ at {0}⁠, {45}⁠. Record the exact position of the offset
  in AZ and El
\end{itemize}

}
\hdashrule[0.5ex]{\textwidth}{1pt}{3mm}
  Expected Result \\
{\footnotesize
\begin{itemize}
\tightlist
\item
  The TMA reaches the commanded offset position.
\end{itemize}

}

\begin{tabular}{p{2cm}}
\toprule
Step 699  \\ \hline
\end{tabular}
 Description \\
{\footnotesize
\textbf{Move TMA to the 5. random distance of 3.5deg\\
}

\begin{itemize}
\tightlist
\item
  Point the TMA to a random 3.5 deg combined offset in AZ and EL from
  {Pointing 16}⁠ at {0}⁠, {15}⁠. Record the exact position of the offset
  in AZ and El
\end{itemize}

}
\hdashrule[0.5ex]{\textwidth}{1pt}{3mm}
  Expected Result \\
{\footnotesize
\begin{itemize}
\tightlist
\item
  The TMA reaches the commanded offset position.
\end{itemize}

}

\begin{tabular}{p{2cm}}
\toprule
Step 700  \\ \hline
\end{tabular}
 Description \\
{\footnotesize
\textbf{Move TMA to the 5. random distance of 3.5deg\\
}

\begin{itemize}
\tightlist
\item
  Point the TMA to a random 3.5 deg combined offset in AZ and EL from
  {Pointing 17}⁠ at {90}⁠, {15}⁠. Record the exact position of the
  offset in AZ and El
\end{itemize}

}
\hdashrule[0.5ex]{\textwidth}{1pt}{3mm}
  Expected Result \\
{\footnotesize
\begin{itemize}
\tightlist
\item
  The TMA reaches the commanded offset position.
\end{itemize}

}

\begin{tabular}{p{2cm}}
\toprule
Step 701  \\ \hline
\end{tabular}
 Description \\
{\footnotesize
\textbf{Move TMA to the 5. random distance of 3.5deg\\
}

\begin{itemize}
\tightlist
\item
  Point the TMA to a random 3.5 deg combined offset in AZ and EL from
  {Pointing 18}⁠ at {90}⁠, {45}⁠. Record the exact position of the
  offset in AZ and El
\end{itemize}

}
\hdashrule[0.5ex]{\textwidth}{1pt}{3mm}
  Expected Result \\
{\footnotesize
\begin{itemize}
\tightlist
\item
  The TMA reaches the commanded offset position.
\end{itemize}

}

\begin{tabular}{p{2cm}}
\toprule
Step 702  \\ \hline
\end{tabular}
 Description \\
{\footnotesize
\textbf{Move TMA to the 5. random distance of 3.5deg\\
}

\begin{itemize}
\tightlist
\item
  Point the TMA to a random 3.5 deg combined offset in AZ and EL from
  {Pointing 19}⁠ at {90}⁠, {75}⁠. Record the exact position of the
  offset in AZ and El
\end{itemize}

}
\hdashrule[0.5ex]{\textwidth}{1pt}{3mm}
  Expected Result \\
{\footnotesize
\begin{itemize}
\tightlist
\item
  The TMA reaches the commanded offset position.
\end{itemize}

}

\begin{tabular}{p{2cm}}
\toprule
Step 703  \\ \hline
\end{tabular}
 Description \\
{\footnotesize
\textbf{Move TMA to the 5. random distance of 3.5deg\\
}

\begin{itemize}
\tightlist
\item
  Point the TMA to a random 3.5 deg combined offset in AZ and EL from
  {Pointing 20}⁠ at {90}⁠, {86.5}⁠. Record the exact position of the
  offset in AZ and El
\end{itemize}

}
\hdashrule[0.5ex]{\textwidth}{1pt}{3mm}
  Expected Result \\
{\footnotesize
\begin{itemize}
\tightlist
\item
  The TMA reaches the commanded offset position.
\end{itemize}

}

\begin{tabular}{p{2cm}}
\toprule
Step 704  \\ \hline
\end{tabular}
 Description \\
{\footnotesize
\textbf{Move TMA to the 5. random distance of 3.5deg\\
}

\begin{itemize}
\tightlist
\item
  Point the TMA to a random 3.5 deg combined offset in AZ and EL from
  {Pointing 21}⁠ at {180}⁠, {86.5}⁠. Record the exact position of the
  offset in AZ and El
\end{itemize}

}
\hdashrule[0.5ex]{\textwidth}{1pt}{3mm}
  Expected Result \\
{\footnotesize
\begin{itemize}
\tightlist
\item
  The TMA reaches the commanded offset position.
\end{itemize}

}

\begin{tabular}{p{2cm}}
\toprule
Step 705  \\ \hline
\end{tabular}
 Description \\
{\footnotesize
\textbf{Move TMA to the 5. random distance of 3.5deg\\
}

\begin{itemize}
\tightlist
\item
  Point the TMA to a random 3.5 deg combined offset in AZ and EL from
  {Pointing 22}⁠ at {180}⁠, {75}⁠. Record the exact position of the
  offset in AZ and El
\end{itemize}

}
\hdashrule[0.5ex]{\textwidth}{1pt}{3mm}
  Expected Result \\
{\footnotesize
\begin{itemize}
\tightlist
\item
  The TMA reaches the commanded offset position.
\end{itemize}

}

\begin{tabular}{p{2cm}}
\toprule
Step 706  \\ \hline
\end{tabular}
 Description \\
{\footnotesize
\textbf{Move TMA to the 5. random distance of 3.5deg\\
}

\begin{itemize}
\tightlist
\item
  Point the TMA to a random 3.5 deg combined offset in AZ and EL from
  {Pointing 23}⁠ at {180}⁠, {45}⁠. Record the exact position of the
  offset in AZ and El
\end{itemize}

}
\hdashrule[0.5ex]{\textwidth}{1pt}{3mm}
  Expected Result \\
{\footnotesize
\begin{itemize}
\tightlist
\item
  The TMA reaches the commanded offset position.
\end{itemize}

}

\begin{tabular}{p{2cm}}
\toprule
Step 707  \\ \hline
\end{tabular}
 Description \\
{\footnotesize
\textbf{Move TMA to the 5. random distance of 3.5deg\\
}

\begin{itemize}
\tightlist
\item
  Point the TMA to a random 3.5 deg combined offset in AZ and EL from
  {Pointing 24}⁠ at {180}⁠, {15}⁠. Record the exact position of the
  offset in AZ and El
\end{itemize}

}
\hdashrule[0.5ex]{\textwidth}{1pt}{3mm}
  Expected Result \\
{\footnotesize
\begin{itemize}
\tightlist
\item
  The TMA reaches the commanded offset position.
\end{itemize}

}

\begin{tabular}{p{2cm}}
\toprule
Step 708  \\ \hline
\end{tabular}
 Description \\
{\footnotesize
\textbf{Move TMA to the 5. random distance of 3.5deg\\
}

\begin{itemize}
\tightlist
\item
  Point the TMA to a random 3.5 deg combined offset in AZ and EL from
  {Pointing 25}⁠ at {270}⁠, {15}⁠. Record the exact position of the
  offset in AZ and El
\end{itemize}

}
\hdashrule[0.5ex]{\textwidth}{1pt}{3mm}
  Expected Result \\
{\footnotesize
\begin{itemize}
\tightlist
\item
  The TMA reaches the commanded offset position.
\end{itemize}

}

\begin{tabular}{p{2cm}}
\toprule
Step 709  \\ \hline
\end{tabular}
 Description \\
{\footnotesize
\textbf{Move TMA to the 5. random distance of 3.5deg\\
}

\begin{itemize}
\tightlist
\item
  Point the TMA to a random 3.5 deg combined offset in AZ and EL from
  {Pointing 26}⁠ at {270}⁠, {45}⁠. Record the exact position of the
  offset in AZ and El
\end{itemize}

}
\hdashrule[0.5ex]{\textwidth}{1pt}{3mm}
  Expected Result \\
{\footnotesize
\begin{itemize}
\tightlist
\item
  The TMA reaches the commanded offset position.
\end{itemize}

}

\begin{tabular}{p{2cm}}
\toprule
Step 710  \\ \hline
\end{tabular}
 Description \\
{\footnotesize
\textbf{Move TMA to the 5. random distance of 3.5deg\\
}

\begin{itemize}
\tightlist
\item
  Point the TMA to a random 3.5 deg combined offset in AZ and EL from
  {Pointing 27}⁠ at {270}⁠, {75}⁠. Record the exact position of the
  offset in AZ and El
\end{itemize}

}
\hdashrule[0.5ex]{\textwidth}{1pt}{3mm}
  Expected Result \\
{\footnotesize
\begin{itemize}
\tightlist
\item
  The TMA reaches the commanded offset position.
\end{itemize}

}

\begin{tabular}{p{2cm}}
\toprule
Step 711  \\ \hline
\end{tabular}
 Description \\
{\footnotesize
\textbf{Move TMA to the 5. random distance of 3.5deg\\
}

\begin{itemize}
\tightlist
\item
  Point the TMA to a random 3.5 deg combined offset in AZ and EL from
  {Pointing 28}⁠ at {270}⁠, {86.5}⁠. Record the exact position of the
  offset in AZ and El
\end{itemize}

}
\hdashrule[0.5ex]{\textwidth}{1pt}{3mm}
  Expected Result \\
{\footnotesize
\begin{itemize}
\tightlist
\item
  The TMA reaches the commanded offset position.
\end{itemize}

}

\begin{tabular}{p{2cm}}
\toprule
Step 712  \\ \hline
\end{tabular}
 Description \\
{\footnotesize
\textbf{Find DIMM Object and DIMM Offset}\\

\begin{itemize}
\tightlist
\item
  While tracking, take a 10-sec exposure with the StarTracker.
\item
  Load the image into an image viewer.
\item
  Overlay the GAIA catalog.
\item
  Select a star brighter than XXX mag (bright enough for the DIMM).
\item
  Calculate the pixel offset between the StarTracker and the DIMM.
\item
  Transform the offset into AZ and EL offsets.
\end{itemize}

}
\hdashrule[0.5ex]{\textwidth}{1pt}{3mm}
  Expected Result \\
{\footnotesize
\begin{itemize}
\tightlist
\item
  An image was successfully taken with the StarTracker and is of
  sufficient quality.
\item
  AZ and EL offsets are available.
\end{itemize}

}

\begin{tabular}{p{2cm}}
\toprule
Step 713  \\ \hline
\end{tabular}
 Description \\
{\footnotesize
\textbf{Find DIMM Object and DIMM Offset}\\

\begin{itemize}
\tightlist
\item
  While tracking, take a 10-sec exposure with the StarTracker.
\item
  Load the image into an image viewer.
\item
  Overlay the GAIA catalog.
\item
  Select a star brighter than XXX mag (bright enough for the DIMM).
\item
  Calculate the pixel offset between the StarTracker and the DIMM.
\item
  Transform the offset into AZ and EL offsets.
\end{itemize}

}
\hdashrule[0.5ex]{\textwidth}{1pt}{3mm}
  Expected Result \\
{\footnotesize
\begin{itemize}
\tightlist
\item
  An image was successfully taken with the StarTracker and is of
  sufficient quality.
\item
  AZ and EL offsets are available.
\end{itemize}

}

\begin{tabular}{p{2cm}}
\toprule
Step 714  \\ \hline
\end{tabular}
 Description \\
{\footnotesize
\textbf{Find DIMM Object and DIMM Offset}\\

\begin{itemize}
\tightlist
\item
  While tracking, take a 10-sec exposure with the StarTracker.
\item
  Load the image into an image viewer.
\item
  Overlay the GAIA catalog.
\item
  Select a star brighter than XXX mag (bright enough for the DIMM).
\item
  Calculate the pixel offset between the StarTracker and the DIMM.
\item
  Transform the offset into AZ and EL offsets.
\end{itemize}

}
\hdashrule[0.5ex]{\textwidth}{1pt}{3mm}
  Expected Result \\
{\footnotesize
\begin{itemize}
\tightlist
\item
  An image was successfully taken with the StarTracker and is of
  sufficient quality.
\item
  AZ and EL offsets are available.
\end{itemize}

}

\begin{tabular}{p{2cm}}
\toprule
Step 715  \\ \hline
\end{tabular}
 Description \\
{\footnotesize
\textbf{Find DIMM Object and DIMM Offset}\\

\begin{itemize}
\tightlist
\item
  While tracking, take a 10-sec exposure with the StarTracker.
\item
  Load the image into an image viewer.
\item
  Overlay the GAIA catalog.
\item
  Select a star brighter than XXX mag (bright enough for the DIMM).
\item
  Calculate the pixel offset between the StarTracker and the DIMM.
\item
  Transform the offset into AZ and EL offsets.
\end{itemize}

}
\hdashrule[0.5ex]{\textwidth}{1pt}{3mm}
  Expected Result \\
{\footnotesize
\begin{itemize}
\tightlist
\item
  An image was successfully taken with the StarTracker and is of
  sufficient quality.
\item
  AZ and EL offsets are available.
\end{itemize}

}

\begin{tabular}{p{2cm}}
\toprule
Step 716  \\ \hline
\end{tabular}
 Description \\
{\footnotesize
\textbf{Find DIMM Object and DIMM Offset}\\

\begin{itemize}
\tightlist
\item
  While tracking, take a 10-sec exposure with the StarTracker.
\item
  Load the image into an image viewer.
\item
  Overlay the GAIA catalog.
\item
  Select a star brighter than XXX mag (bright enough for the DIMM).
\item
  Calculate the pixel offset between the StarTracker and the DIMM.
\item
  Transform the offset into AZ and EL offsets.
\end{itemize}

}
\hdashrule[0.5ex]{\textwidth}{1pt}{3mm}
  Expected Result \\
{\footnotesize
\begin{itemize}
\tightlist
\item
  An image was successfully taken with the StarTracker and is of
  sufficient quality.
\item
  AZ and EL offsets are available.
\end{itemize}

}

\begin{tabular}{p{2cm}}
\toprule
Step 717  \\ \hline
\end{tabular}
 Description \\
{\footnotesize
\textbf{Find DIMM Object and DIMM Offset}\\

\begin{itemize}
\tightlist
\item
  While tracking, take a 10-sec exposure with the StarTracker.
\item
  Load the image into an image viewer.
\item
  Overlay the GAIA catalog.
\item
  Select a star brighter than XXX mag (bright enough for the DIMM).
\item
  Calculate the pixel offset between the StarTracker and the DIMM.
\item
  Transform the offset into AZ and EL offsets.
\end{itemize}

}
\hdashrule[0.5ex]{\textwidth}{1pt}{3mm}
  Expected Result \\
{\footnotesize
\begin{itemize}
\tightlist
\item
  An image was successfully taken with the StarTracker and is of
  sufficient quality.
\item
  AZ and EL offsets are available.
\end{itemize}

}

\begin{tabular}{p{2cm}}
\toprule
Step 718  \\ \hline
\end{tabular}
 Description \\
{\footnotesize
\textbf{Find DIMM Object and DIMM Offset}\\

\begin{itemize}
\tightlist
\item
  While tracking, take a 10-sec exposure with the StarTracker.
\item
  Load the image into an image viewer.
\item
  Overlay the GAIA catalog.
\item
  Select a star brighter than XXX mag (bright enough for the DIMM).
\item
  Calculate the pixel offset between the StarTracker and the DIMM.
\item
  Transform the offset into AZ and EL offsets.
\end{itemize}

}
\hdashrule[0.5ex]{\textwidth}{1pt}{3mm}
  Expected Result \\
{\footnotesize
\begin{itemize}
\tightlist
\item
  An image was successfully taken with the StarTracker and is of
  sufficient quality.
\item
  AZ and EL offsets are available.
\end{itemize}

}

\begin{tabular}{p{2cm}}
\toprule
Step 719  \\ \hline
\end{tabular}
 Description \\
{\footnotesize
\textbf{Find DIMM Object and DIMM Offset}\\

\begin{itemize}
\tightlist
\item
  While tracking, take a 10-sec exposure with the StarTracker.
\item
  Load the image into an image viewer.
\item
  Overlay the GAIA catalog.
\item
  Select a star brighter than XXX mag (bright enough for the DIMM).
\item
  Calculate the pixel offset between the StarTracker and the DIMM.
\item
  Transform the offset into AZ and EL offsets.
\end{itemize}

}
\hdashrule[0.5ex]{\textwidth}{1pt}{3mm}
  Expected Result \\
{\footnotesize
\begin{itemize}
\tightlist
\item
  An image was successfully taken with the StarTracker and is of
  sufficient quality.
\item
  AZ and EL offsets are available.
\end{itemize}

}

\begin{tabular}{p{2cm}}
\toprule
Step 720  \\ \hline
\end{tabular}
 Description \\
{\footnotesize
\textbf{Find DIMM Object and DIMM Offset}\\

\begin{itemize}
\tightlist
\item
  While tracking, take a 10-sec exposure with the StarTracker.
\item
  Load the image into an image viewer.
\item
  Overlay the GAIA catalog.
\item
  Select a star brighter than XXX mag (bright enough for the DIMM).
\item
  Calculate the pixel offset between the StarTracker and the DIMM.
\item
  Transform the offset into AZ and EL offsets.
\end{itemize}

}
\hdashrule[0.5ex]{\textwidth}{1pt}{3mm}
  Expected Result \\
{\footnotesize
\begin{itemize}
\tightlist
\item
  An image was successfully taken with the StarTracker and is of
  sufficient quality.
\item
  AZ and EL offsets are available.
\end{itemize}

}

\begin{tabular}{p{2cm}}
\toprule
Step 721  \\ \hline
\end{tabular}
 Description \\
{\footnotesize
\textbf{Find DIMM Object and DIMM Offset}\\

\begin{itemize}
\tightlist
\item
  While tracking, take a 10-sec exposure with the StarTracker.
\item
  Load the image into an image viewer.
\item
  Overlay the GAIA catalog.
\item
  Select a star brighter than XXX mag (bright enough for the DIMM).
\item
  Calculate the pixel offset between the StarTracker and the DIMM.
\item
  Transform the offset into AZ and EL offsets.
\end{itemize}

}
\hdashrule[0.5ex]{\textwidth}{1pt}{3mm}
  Expected Result \\
{\footnotesize
\begin{itemize}
\tightlist
\item
  An image was successfully taken with the StarTracker and is of
  sufficient quality.
\item
  AZ and EL offsets are available.
\end{itemize}

}

\begin{tabular}{p{2cm}}
\toprule
Step 722  \\ \hline
\end{tabular}
 Description \\
{\footnotesize
\textbf{Find DIMM Object and DIMM Offset}\\

\begin{itemize}
\tightlist
\item
  While tracking, take a 10-sec exposure with the StarTracker.
\item
  Load the image into an image viewer.
\item
  Overlay the GAIA catalog.
\item
  Select a star brighter than XXX mag (bright enough for the DIMM).
\item
  Calculate the pixel offset between the StarTracker and the DIMM.
\item
  Transform the offset into AZ and EL offsets.
\end{itemize}

}
\hdashrule[0.5ex]{\textwidth}{1pt}{3mm}
  Expected Result \\
{\footnotesize
\begin{itemize}
\tightlist
\item
  An image was successfully taken with the StarTracker and is of
  sufficient quality.
\item
  AZ and EL offsets are available.
\end{itemize}

}

\begin{tabular}{p{2cm}}
\toprule
Step 723  \\ \hline
\end{tabular}
 Description \\
{\footnotesize
\textbf{Find DIMM Object and DIMM Offset}\\

\begin{itemize}
\tightlist
\item
  While tracking, take a 10-sec exposure with the StarTracker.
\item
  Load the image into an image viewer.
\item
  Overlay the GAIA catalog.
\item
  Select a star brighter than XXX mag (bright enough for the DIMM).
\item
  Calculate the pixel offset between the StarTracker and the DIMM.
\item
  Transform the offset into AZ and EL offsets.
\end{itemize}

}
\hdashrule[0.5ex]{\textwidth}{1pt}{3mm}
  Expected Result \\
{\footnotesize
\begin{itemize}
\tightlist
\item
  An image was successfully taken with the StarTracker and is of
  sufficient quality.
\item
  AZ and EL offsets are available.
\end{itemize}

}

\begin{tabular}{p{2cm}}
\toprule
Step 724  \\ \hline
\end{tabular}
 Description \\
{\footnotesize
\textbf{Find DIMM Object and DIMM Offset}\\

\begin{itemize}
\tightlist
\item
  While tracking, take a 10-sec exposure with the StarTracker.
\item
  Load the image into an image viewer.
\item
  Overlay the GAIA catalog.
\item
  Select a star brighter than XXX mag (bright enough for the DIMM).
\item
  Calculate the pixel offset between the StarTracker and the DIMM.
\item
  Transform the offset into AZ and EL offsets.
\end{itemize}

}
\hdashrule[0.5ex]{\textwidth}{1pt}{3mm}
  Expected Result \\
{\footnotesize
\begin{itemize}
\tightlist
\item
  An image was successfully taken with the StarTracker and is of
  sufficient quality.
\item
  AZ and EL offsets are available.
\end{itemize}

}

\begin{tabular}{p{2cm}}
\toprule
Step 725  \\ \hline
\end{tabular}
 Description \\
{\footnotesize
\textbf{Find DIMM Object and DIMM Offset}\\

\begin{itemize}
\tightlist
\item
  While tracking, take a 10-sec exposure with the StarTracker.
\item
  Load the image into an image viewer.
\item
  Overlay the GAIA catalog.
\item
  Select a star brighter than XXX mag (bright enough for the DIMM).
\item
  Calculate the pixel offset between the StarTracker and the DIMM.
\item
  Transform the offset into AZ and EL offsets.
\end{itemize}

}
\hdashrule[0.5ex]{\textwidth}{1pt}{3mm}
  Expected Result \\
{\footnotesize
\begin{itemize}
\tightlist
\item
  An image was successfully taken with the StarTracker and is of
  sufficient quality.
\item
  AZ and EL offsets are available.
\end{itemize}

}

\begin{tabular}{p{2cm}}
\toprule
Step 726  \\ \hline
\end{tabular}
 Description \\
{\footnotesize
\textbf{Find DIMM Object and DIMM Offset}\\

\begin{itemize}
\tightlist
\item
  While tracking, take a 10-sec exposure with the StarTracker.
\item
  Load the image into an image viewer.
\item
  Overlay the GAIA catalog.
\item
  Select a star brighter than XXX mag (bright enough for the DIMM).
\item
  Calculate the pixel offset between the StarTracker and the DIMM.
\item
  Transform the offset into AZ and EL offsets.
\end{itemize}

}
\hdashrule[0.5ex]{\textwidth}{1pt}{3mm}
  Expected Result \\
{\footnotesize
\begin{itemize}
\tightlist
\item
  An image was successfully taken with the StarTracker and is of
  sufficient quality.
\item
  AZ and EL offsets are available.
\end{itemize}

}

\begin{tabular}{p{2cm}}
\toprule
Step 727  \\ \hline
\end{tabular}
 Description \\
{\footnotesize
\textbf{Find DIMM Object and DIMM Offset}\\

\begin{itemize}
\tightlist
\item
  While tracking, take a 10-sec exposure with the StarTracker.
\item
  Load the image into an image viewer.
\item
  Overlay the GAIA catalog.
\item
  Select a star brighter than XXX mag (bright enough for the DIMM).
\item
  Calculate the pixel offset between the StarTracker and the DIMM.
\item
  Transform the offset into AZ and EL offsets.
\end{itemize}

}
\hdashrule[0.5ex]{\textwidth}{1pt}{3mm}
  Expected Result \\
{\footnotesize
\begin{itemize}
\tightlist
\item
  An image was successfully taken with the StarTracker and is of
  sufficient quality.
\item
  AZ and EL offsets are available.
\end{itemize}

}

\begin{tabular}{p{2cm}}
\toprule
Step 728  \\ \hline
\end{tabular}
 Description \\
{\footnotesize
\textbf{Find DIMM Object and DIMM Offset}\\

\begin{itemize}
\tightlist
\item
  While tracking, take a 10-sec exposure with the StarTracker.
\item
  Load the image into an image viewer.
\item
  Overlay the GAIA catalog.
\item
  Select a star brighter than XXX mag (bright enough for the DIMM).
\item
  Calculate the pixel offset between the StarTracker and the DIMM.
\item
  Transform the offset into AZ and EL offsets.
\end{itemize}

}
\hdashrule[0.5ex]{\textwidth}{1pt}{3mm}
  Expected Result \\
{\footnotesize
\begin{itemize}
\tightlist
\item
  An image was successfully taken with the StarTracker and is of
  sufficient quality.
\item
  AZ and EL offsets are available.
\end{itemize}

}

\begin{tabular}{p{2cm}}
\toprule
Step 729  \\ \hline
\end{tabular}
 Description \\
{\footnotesize
\textbf{Find DIMM Object and DIMM Offset}\\

\begin{itemize}
\tightlist
\item
  While tracking, take a 10-sec exposure with the StarTracker.
\item
  Load the image into an image viewer.
\item
  Overlay the GAIA catalog.
\item
  Select a star brighter than XXX mag (bright enough for the DIMM).
\item
  Calculate the pixel offset between the StarTracker and the DIMM.
\item
  Transform the offset into AZ and EL offsets.
\end{itemize}

}
\hdashrule[0.5ex]{\textwidth}{1pt}{3mm}
  Expected Result \\
{\footnotesize
\begin{itemize}
\tightlist
\item
  An image was successfully taken with the StarTracker and is of
  sufficient quality.
\item
  AZ and EL offsets are available.
\end{itemize}

}

\begin{tabular}{p{2cm}}
\toprule
Step 730  \\ \hline
\end{tabular}
 Description \\
{\footnotesize
\textbf{Find DIMM Object and DIMM Offset}\\

\begin{itemize}
\tightlist
\item
  While tracking, take a 10-sec exposure with the StarTracker.
\item
  Load the image into an image viewer.
\item
  Overlay the GAIA catalog.
\item
  Select a star brighter than XXX mag (bright enough for the DIMM).
\item
  Calculate the pixel offset between the StarTracker and the DIMM.
\item
  Transform the offset into AZ and EL offsets.
\end{itemize}

}
\hdashrule[0.5ex]{\textwidth}{1pt}{3mm}
  Expected Result \\
{\footnotesize
\begin{itemize}
\tightlist
\item
  An image was successfully taken with the StarTracker and is of
  sufficient quality.
\item
  AZ and EL offsets are available.
\end{itemize}

}

\begin{tabular}{p{2cm}}
\toprule
Step 731  \\ \hline
\end{tabular}
 Description \\
{\footnotesize
\textbf{Find DIMM Object and DIMM Offset}\\

\begin{itemize}
\tightlist
\item
  While tracking, take a 10-sec exposure with the StarTracker.
\item
  Load the image into an image viewer.
\item
  Overlay the GAIA catalog.
\item
  Select a star brighter than XXX mag (bright enough for the DIMM).
\item
  Calculate the pixel offset between the StarTracker and the DIMM.
\item
  Transform the offset into AZ and EL offsets.
\end{itemize}

}
\hdashrule[0.5ex]{\textwidth}{1pt}{3mm}
  Expected Result \\
{\footnotesize
\begin{itemize}
\tightlist
\item
  An image was successfully taken with the StarTracker and is of
  sufficient quality.
\item
  AZ and EL offsets are available.
\end{itemize}

}

\begin{tabular}{p{2cm}}
\toprule
Step 732  \\ \hline
\end{tabular}
 Description \\
{\footnotesize
\textbf{Find DIMM Object and DIMM Offset}\\

\begin{itemize}
\tightlist
\item
  While tracking, take a 10-sec exposure with the StarTracker.
\item
  Load the image into an image viewer.
\item
  Overlay the GAIA catalog.
\item
  Select a star brighter than XXX mag (bright enough for the DIMM).
\item
  Calculate the pixel offset between the StarTracker and the DIMM.
\item
  Transform the offset into AZ and EL offsets.
\end{itemize}

}
\hdashrule[0.5ex]{\textwidth}{1pt}{3mm}
  Expected Result \\
{\footnotesize
\begin{itemize}
\tightlist
\item
  An image was successfully taken with the StarTracker and is of
  sufficient quality.
\item
  AZ and EL offsets are available.
\end{itemize}

}

\begin{tabular}{p{2cm}}
\toprule
Step 733  \\ \hline
\end{tabular}
 Description \\
{\footnotesize
\textbf{Find DIMM Object and DIMM Offset}\\

\begin{itemize}
\tightlist
\item
  While tracking, take a 10-sec exposure with the StarTracker.
\item
  Load the image into an image viewer.
\item
  Overlay the GAIA catalog.
\item
  Select a star brighter than XXX mag (bright enough for the DIMM).
\item
  Calculate the pixel offset between the StarTracker and the DIMM.
\item
  Transform the offset into AZ and EL offsets.
\end{itemize}

}
\hdashrule[0.5ex]{\textwidth}{1pt}{3mm}
  Expected Result \\
{\footnotesize
\begin{itemize}
\tightlist
\item
  An image was successfully taken with the StarTracker and is of
  sufficient quality.
\item
  AZ and EL offsets are available.
\end{itemize}

}

\begin{tabular}{p{2cm}}
\toprule
Step 734  \\ \hline
\end{tabular}
 Description \\
{\footnotesize
\textbf{Find DIMM Object and DIMM Offset}\\

\begin{itemize}
\tightlist
\item
  While tracking, take a 10-sec exposure with the StarTracker.
\item
  Load the image into an image viewer.
\item
  Overlay the GAIA catalog.
\item
  Select a star brighter than XXX mag (bright enough for the DIMM).
\item
  Calculate the pixel offset between the StarTracker and the DIMM.
\item
  Transform the offset into AZ and EL offsets.
\end{itemize}

}
\hdashrule[0.5ex]{\textwidth}{1pt}{3mm}
  Expected Result \\
{\footnotesize
\begin{itemize}
\tightlist
\item
  An image was successfully taken with the StarTracker and is of
  sufficient quality.
\item
  AZ and EL offsets are available.
\end{itemize}

}

\begin{tabular}{p{2cm}}
\toprule
Step 735  \\ \hline
\end{tabular}
 Description \\
{\footnotesize
\textbf{Find DIMM Object and DIMM Offset}\\

\begin{itemize}
\tightlist
\item
  While tracking, take a 10-sec exposure with the StarTracker.
\item
  Load the image into an image viewer.
\item
  Overlay the GAIA catalog.
\item
  Select a star brighter than XXX mag (bright enough for the DIMM).
\item
  Calculate the pixel offset between the StarTracker and the DIMM.
\item
  Transform the offset into AZ and EL offsets.
\end{itemize}

}
\hdashrule[0.5ex]{\textwidth}{1pt}{3mm}
  Expected Result \\
{\footnotesize
\begin{itemize}
\tightlist
\item
  An image was successfully taken with the StarTracker and is of
  sufficient quality.
\item
  AZ and EL offsets are available.
\end{itemize}

}

\begin{tabular}{p{2cm}}
\toprule
Step 736  \\ \hline
\end{tabular}
 Description \\
{\footnotesize
\textbf{Find DIMM Object and DIMM Offset}\\

\begin{itemize}
\tightlist
\item
  While tracking, take a 10-sec exposure with the StarTracker.
\item
  Load the image into an image viewer.
\item
  Overlay the GAIA catalog.
\item
  Select a star brighter than XXX mag (bright enough for the DIMM).
\item
  Calculate the pixel offset between the StarTracker and the DIMM.
\item
  Transform the offset into AZ and EL offsets.
\end{itemize}

}
\hdashrule[0.5ex]{\textwidth}{1pt}{3mm}
  Expected Result \\
{\footnotesize
\begin{itemize}
\tightlist
\item
  An image was successfully taken with the StarTracker and is of
  sufficient quality.
\item
  AZ and EL offsets are available.
\end{itemize}

}

\begin{tabular}{p{2cm}}
\toprule
Step 737  \\ \hline
\end{tabular}
 Description \\
{\footnotesize
\textbf{Find DIMM Object and DIMM Offset}\\

\begin{itemize}
\tightlist
\item
  While tracking, take a 10-sec exposure with the StarTracker.
\item
  Load the image into an image viewer.
\item
  Overlay the GAIA catalog.
\item
  Select a star brighter than XXX mag (bright enough for the DIMM).
\item
  Calculate the pixel offset between the StarTracker and the DIMM.
\item
  Transform the offset into AZ and EL offsets.
\end{itemize}

}
\hdashrule[0.5ex]{\textwidth}{1pt}{3mm}
  Expected Result \\
{\footnotesize
\begin{itemize}
\tightlist
\item
  An image was successfully taken with the StarTracker and is of
  sufficient quality.
\item
  AZ and EL offsets are available.
\end{itemize}

}

\begin{tabular}{p{2cm}}
\toprule
Step 738  \\ \hline
\end{tabular}
 Description \\
{\footnotesize
\textbf{Find DIMM Object and DIMM Offset}\\

\begin{itemize}
\tightlist
\item
  While tracking, take a 10-sec exposure with the StarTracker.
\item
  Load the image into an image viewer.
\item
  Overlay the GAIA catalog.
\item
  Select a star brighter than XXX mag (bright enough for the DIMM).
\item
  Calculate the pixel offset between the StarTracker and the DIMM.
\item
  Transform the offset into AZ and EL offsets.
\end{itemize}

}
\hdashrule[0.5ex]{\textwidth}{1pt}{3mm}
  Expected Result \\
{\footnotesize
\begin{itemize}
\tightlist
\item
  An image was successfully taken with the StarTracker and is of
  sufficient quality.
\item
  AZ and EL offsets are available.
\end{itemize}

}

\begin{tabular}{p{2cm}}
\toprule
Step 739  \\ \hline
\end{tabular}
 Description \\
{\footnotesize
\textbf{Find DIMM Object and DIMM Offset}\\

\begin{itemize}
\tightlist
\item
  While tracking, take a 10-sec exposure with the StarTracker.
\item
  Load the image into an image viewer.
\item
  Overlay the GAIA catalog.
\item
  Select a star brighter than XXX mag (bright enough for the DIMM).
\item
  Calculate the pixel offset between the StarTracker and the DIMM.
\item
  Transform the offset into AZ and EL offsets.
\end{itemize}

}
\hdashrule[0.5ex]{\textwidth}{1pt}{3mm}
  Expected Result \\
{\footnotesize
\begin{itemize}
\tightlist
\item
  An image was successfully taken with the StarTracker and is of
  sufficient quality.
\item
  AZ and EL offsets are available.
\end{itemize}

}

\begin{tabular}{p{2cm}}
\toprule
Step 740  \\ \hline
\end{tabular}
 Description \\
{\footnotesize
\textbf{Find DIMM Object and DIMM Offset}\\

\begin{itemize}
\tightlist
\item
  While tracking, take a 10-sec exposure with the StarTracker.
\item
  Load the image into an image viewer.
\item
  Overlay the GAIA catalog.
\item
  Select a star brighter than XXX mag (bright enough for the DIMM).
\item
  Calculate the pixel offset between the StarTracker and the DIMM.
\item
  Transform the offset into AZ and EL offsets.
\end{itemize}

}
\hdashrule[0.5ex]{\textwidth}{1pt}{3mm}
  Expected Result \\
{\footnotesize
\begin{itemize}
\tightlist
\item
  An image was successfully taken with the StarTracker and is of
  sufficient quality.
\item
  AZ and EL offsets are available.
\end{itemize}

}

\begin{tabular}{p{2cm}}
\toprule
Step 741  \\ \hline
\end{tabular}
 Description \\
{\footnotesize
\textbf{Find DIMM Object and DIMM Offset}\\

\begin{itemize}
\tightlist
\item
  While tracking, take a 10-sec exposure with the StarTracker.
\item
  Load the image into an image viewer.
\item
  Overlay the GAIA catalog.
\item
  Select a star brighter than XXX mag (bright enough for the DIMM).
\item
  Calculate the pixel offset between the StarTracker and the DIMM.
\item
  Transform the offset into AZ and EL offsets.
\end{itemize}

}
\hdashrule[0.5ex]{\textwidth}{1pt}{3mm}
  Expected Result \\
{\footnotesize
\begin{itemize}
\tightlist
\item
  An image was successfully taken with the StarTracker and is of
  sufficient quality.
\item
  AZ and EL offsets are available.
\end{itemize}

}

\begin{tabular}{p{2cm}}
\toprule
Step 742  \\ \hline
\end{tabular}
 Description \\
{\footnotesize
\textbf{Find DIMM Object and DIMM Offset}\\

\begin{itemize}
\tightlist
\item
  While tracking, take a 10-sec exposure with the StarTracker.
\item
  Load the image into an image viewer.
\item
  Overlay the GAIA catalog.
\item
  Select a star brighter than XXX mag (bright enough for the DIMM).
\item
  Calculate the pixel offset between the StarTracker and the DIMM.
\item
  Transform the offset into AZ and EL offsets.
\end{itemize}

}
\hdashrule[0.5ex]{\textwidth}{1pt}{3mm}
  Expected Result \\
{\footnotesize
\begin{itemize}
\tightlist
\item
  An image was successfully taken with the StarTracker and is of
  sufficient quality.
\item
  AZ and EL offsets are available.
\end{itemize}

}

\begin{tabular}{p{2cm}}
\toprule
Step 743  \\ \hline
\end{tabular}
 Description \\
{\footnotesize
\textbf{Find DIMM Object and DIMM Offset}\\

\begin{itemize}
\tightlist
\item
  While tracking, take a 10-sec exposure with the StarTracker.
\item
  Load the image into an image viewer.
\item
  Overlay the GAIA catalog.
\item
  Select a star brighter than XXX mag (bright enough for the DIMM).
\item
  Calculate the pixel offset between the StarTracker and the DIMM.
\item
  Transform the offset into AZ and EL offsets.
\end{itemize}

}
\hdashrule[0.5ex]{\textwidth}{1pt}{3mm}
  Expected Result \\
{\footnotesize
\begin{itemize}
\tightlist
\item
  An image was successfully taken with the StarTracker and is of
  sufficient quality.
\item
  AZ and EL offsets are available.
\end{itemize}

}

\begin{tabular}{p{2cm}}
\toprule
Step 744  \\ \hline
\end{tabular}
 Description \\
{\footnotesize
\textbf{Find DIMM Object and DIMM Offset}\\

\begin{itemize}
\tightlist
\item
  While tracking, take a 10-sec exposure with the StarTracker.
\item
  Load the image into an image viewer.
\item
  Overlay the GAIA catalog.
\item
  Select a star brighter than XXX mag (bright enough for the DIMM).
\item
  Calculate the pixel offset between the StarTracker and the DIMM.
\item
  Transform the offset into AZ and EL offsets.
\end{itemize}

}
\hdashrule[0.5ex]{\textwidth}{1pt}{3mm}
  Expected Result \\
{\footnotesize
\begin{itemize}
\tightlist
\item
  An image was successfully taken with the StarTracker and is of
  sufficient quality.
\item
  AZ and EL offsets are available.
\end{itemize}

}

\begin{tabular}{p{2cm}}
\toprule
Step 745  \\ \hline
\end{tabular}
 Description \\
{\footnotesize
\textbf{Find DIMM Object and DIMM Offset}\\

\begin{itemize}
\tightlist
\item
  While tracking, take a 10-sec exposure with the StarTracker.
\item
  Load the image into an image viewer.
\item
  Overlay the GAIA catalog.
\item
  Select a star brighter than XXX mag (bright enough for the DIMM).
\item
  Calculate the pixel offset between the StarTracker and the DIMM.
\item
  Transform the offset into AZ and EL offsets.
\end{itemize}

}
\hdashrule[0.5ex]{\textwidth}{1pt}{3mm}
  Expected Result \\
{\footnotesize
\begin{itemize}
\tightlist
\item
  An image was successfully taken with the StarTracker and is of
  sufficient quality.
\item
  AZ and EL offsets are available.
\end{itemize}

}

\begin{tabular}{p{2cm}}
\toprule
Step 746  \\ \hline
\end{tabular}
 Description \\
{\footnotesize
\textbf{Find DIMM Object and DIMM Offset}\\

\begin{itemize}
\tightlist
\item
  While tracking, take a 10-sec exposure with the StarTracker.
\item
  Load the image into an image viewer.
\item
  Overlay the GAIA catalog.
\item
  Select a star brighter than XXX mag (bright enough for the DIMM).
\item
  Calculate the pixel offset between the StarTracker and the DIMM.
\item
  Transform the offset into AZ and EL offsets.
\end{itemize}

}
\hdashrule[0.5ex]{\textwidth}{1pt}{3mm}
  Expected Result \\
{\footnotesize
\begin{itemize}
\tightlist
\item
  An image was successfully taken with the StarTracker and is of
  sufficient quality.
\item
  AZ and EL offsets are available.
\end{itemize}

}

\begin{tabular}{p{2cm}}
\toprule
Step 747  \\ \hline
\end{tabular}
 Description \\
{\footnotesize
\textbf{Find DIMM Object and DIMM Offset}\\

\begin{itemize}
\tightlist
\item
  While tracking, take a 10-sec exposure with the StarTracker.
\item
  Load the image into an image viewer.
\item
  Overlay the GAIA catalog.
\item
  Select a star brighter than XXX mag (bright enough for the DIMM).
\item
  Calculate the pixel offset between the StarTracker and the DIMM.
\item
  Transform the offset into AZ and EL offsets.
\end{itemize}

}
\hdashrule[0.5ex]{\textwidth}{1pt}{3mm}
  Expected Result \\
{\footnotesize
\begin{itemize}
\tightlist
\item
  An image was successfully taken with the StarTracker and is of
  sufficient quality.
\item
  AZ and EL offsets are available.
\end{itemize}

}

\begin{tabular}{p{2cm}}
\toprule
Step 748  \\ \hline
\end{tabular}
 Description \\
{\footnotesize
\textbf{Find DIMM Object and DIMM Offset}\\

\begin{itemize}
\tightlist
\item
  While tracking, take a 10-sec exposure with the StarTracker.
\item
  Load the image into an image viewer.
\item
  Overlay the GAIA catalog.
\item
  Select a star brighter than XXX mag (bright enough for the DIMM).
\item
  Calculate the pixel offset between the StarTracker and the DIMM.
\item
  Transform the offset into AZ and EL offsets.
\end{itemize}

}
\hdashrule[0.5ex]{\textwidth}{1pt}{3mm}
  Expected Result \\
{\footnotesize
\begin{itemize}
\tightlist
\item
  An image was successfully taken with the StarTracker and is of
  sufficient quality.
\item
  AZ and EL offsets are available.
\end{itemize}

}

\begin{tabular}{p{2cm}}
\toprule
Step 749  \\ \hline
\end{tabular}
 Description \\
{\footnotesize
\textbf{Find DIMM Object and DIMM Offset}\\

\begin{itemize}
\tightlist
\item
  While tracking, take a 10-sec exposure with the StarTracker.
\item
  Load the image into an image viewer.
\item
  Overlay the GAIA catalog.
\item
  Select a star brighter than XXX mag (bright enough for the DIMM).
\item
  Calculate the pixel offset between the StarTracker and the DIMM.
\item
  Transform the offset into AZ and EL offsets.
\end{itemize}

}
\hdashrule[0.5ex]{\textwidth}{1pt}{3mm}
  Expected Result \\
{\footnotesize
\begin{itemize}
\tightlist
\item
  An image was successfully taken with the StarTracker and is of
  sufficient quality.
\item
  AZ and EL offsets are available.
\end{itemize}

}

\begin{tabular}{p{2cm}}
\toprule
Step 750  \\ \hline
\end{tabular}
 Description \\
{\footnotesize
\textbf{Find DIMM Object and DIMM Offset}\\

\begin{itemize}
\tightlist
\item
  While tracking, take a 10-sec exposure with the StarTracker.
\item
  Load the image into an image viewer.
\item
  Overlay the GAIA catalog.
\item
  Select a star brighter than XXX mag (bright enough for the DIMM).
\item
  Calculate the pixel offset between the StarTracker and the DIMM.
\item
  Transform the offset into AZ and EL offsets.
\end{itemize}

}
\hdashrule[0.5ex]{\textwidth}{1pt}{3mm}
  Expected Result \\
{\footnotesize
\begin{itemize}
\tightlist
\item
  An image was successfully taken with the StarTracker and is of
  sufficient quality.
\item
  AZ and EL offsets are available.
\end{itemize}

}

\begin{tabular}{p{2cm}}
\toprule
Step 751  \\ \hline
\end{tabular}
 Description \\
{\footnotesize
\textbf{Find DIMM Object and DIMM Offset}\\

\begin{itemize}
\tightlist
\item
  While tracking, take a 10-sec exposure with the StarTracker.
\item
  Load the image into an image viewer.
\item
  Overlay the GAIA catalog.
\item
  Select a star brighter than XXX mag (bright enough for the DIMM).
\item
  Calculate the pixel offset between the StarTracker and the DIMM.
\item
  Transform the offset into AZ and EL offsets.
\end{itemize}

}
\hdashrule[0.5ex]{\textwidth}{1pt}{3mm}
  Expected Result \\
{\footnotesize
\begin{itemize}
\tightlist
\item
  An image was successfully taken with the StarTracker and is of
  sufficient quality.
\item
  AZ and EL offsets are available.
\end{itemize}

}

\begin{tabular}{p{2cm}}
\toprule
Step 752  \\ \hline
\end{tabular}
 Description \\
{\footnotesize
\textbf{Find DIMM Object and DIMM Offset}\\

\begin{itemize}
\tightlist
\item
  While tracking, take a 10-sec exposure with the StarTracker.
\item
  Load the image into an image viewer.
\item
  Overlay the GAIA catalog.
\item
  Select a star brighter than XXX mag (bright enough for the DIMM).
\item
  Calculate the pixel offset between the StarTracker and the DIMM.
\item
  Transform the offset into AZ and EL offsets.
\end{itemize}

}
\hdashrule[0.5ex]{\textwidth}{1pt}{3mm}
  Expected Result \\
{\footnotesize
\begin{itemize}
\tightlist
\item
  An image was successfully taken with the StarTracker and is of
  sufficient quality.
\item
  AZ and EL offsets are available.
\end{itemize}

}

\begin{tabular}{p{2cm}}
\toprule
Step 753  \\ \hline
\end{tabular}
 Description \\
{\footnotesize
\textbf{Find DIMM Object and DIMM Offset}\\

\begin{itemize}
\tightlist
\item
  While tracking, take a 10-sec exposure with the StarTracker.
\item
  Load the image into an image viewer.
\item
  Overlay the GAIA catalog.
\item
  Select a star brighter than XXX mag (bright enough for the DIMM).
\item
  Calculate the pixel offset between the StarTracker and the DIMM.
\item
  Transform the offset into AZ and EL offsets.
\end{itemize}

}
\hdashrule[0.5ex]{\textwidth}{1pt}{3mm}
  Expected Result \\
{\footnotesize
\begin{itemize}
\tightlist
\item
  An image was successfully taken with the StarTracker and is of
  sufficient quality.
\item
  AZ and EL offsets are available.
\end{itemize}

}

\begin{tabular}{p{2cm}}
\toprule
Step 754  \\ \hline
\end{tabular}
 Description \\
{\footnotesize
\textbf{Find DIMM Object and DIMM Offset}\\

\begin{itemize}
\tightlist
\item
  While tracking, take a 10-sec exposure with the StarTracker.
\item
  Load the image into an image viewer.
\item
  Overlay the GAIA catalog.
\item
  Select a star brighter than XXX mag (bright enough for the DIMM).
\item
  Calculate the pixel offset between the StarTracker and the DIMM.
\item
  Transform the offset into AZ and EL offsets.
\end{itemize}

}
\hdashrule[0.5ex]{\textwidth}{1pt}{3mm}
  Expected Result \\
{\footnotesize
\begin{itemize}
\tightlist
\item
  An image was successfully taken with the StarTracker and is of
  sufficient quality.
\item
  AZ and EL offsets are available.
\end{itemize}

}

\begin{tabular}{p{2cm}}
\toprule
Step 755  \\ \hline
\end{tabular}
 Description \\
{\footnotesize
\textbf{Find DIMM Object and DIMM Offset}\\

\begin{itemize}
\tightlist
\item
  While tracking, take a 10-sec exposure with the StarTracker.
\item
  Load the image into an image viewer.
\item
  Overlay the GAIA catalog.
\item
  Select a star brighter than XXX mag (bright enough for the DIMM).
\item
  Calculate the pixel offset between the StarTracker and the DIMM.
\item
  Transform the offset into AZ and EL offsets.
\end{itemize}

}
\hdashrule[0.5ex]{\textwidth}{1pt}{3mm}
  Expected Result \\
{\footnotesize
\begin{itemize}
\tightlist
\item
  An image was successfully taken with the StarTracker and is of
  sufficient quality.
\item
  AZ and EL offsets are available.
\end{itemize}

}

\begin{tabular}{p{2cm}}
\toprule
Step 756  \\ \hline
\end{tabular}
 Description \\
{\footnotesize
\textbf{Find DIMM Object and DIMM Offset}\\

\begin{itemize}
\tightlist
\item
  While tracking, take a 10-sec exposure with the StarTracker.
\item
  Load the image into an image viewer.
\item
  Overlay the GAIA catalog.
\item
  Select a star brighter than XXX mag (bright enough for the DIMM).
\item
  Calculate the pixel offset between the StarTracker and the DIMM.
\item
  Transform the offset into AZ and EL offsets.
\end{itemize}

}
\hdashrule[0.5ex]{\textwidth}{1pt}{3mm}
  Expected Result \\
{\footnotesize
\begin{itemize}
\tightlist
\item
  An image was successfully taken with the StarTracker and is of
  sufficient quality.
\item
  AZ and EL offsets are available.
\end{itemize}

}

\begin{tabular}{p{2cm}}
\toprule
Step 757  \\ \hline
\end{tabular}
 Description \\
{\footnotesize
\textbf{Move TMA to the DIMM position and \textbf{Take DIMM images}}\\

\begin{itemize}
\tightlist
\item
  Command the TMA to the DIMM position by applying the offsets
\item
  While tracking, take DIMM images with XXXs exposure time and inspect
  the quality.
\end{itemize}

}
\hdashrule[0.5ex]{\textwidth}{1pt}{3mm}
  Expected Result \\
{\footnotesize
\begin{itemize}
\tightlist
\item
  TMA reaches the DIMM position.
\item
  DIMM imaging quality is sufficient.
\end{itemize}

}

\begin{tabular}{p{2cm}}
\toprule
Step 758  \\ \hline
\end{tabular}
 Description \\
{\footnotesize
\textbf{Move TMA to the DIMM position and \textbf{Take DIMM images}}\\

\begin{itemize}
\tightlist
\item
  Command the TMA to the DIMM position by applying the offsets
\item
  While tracking, take DIMM images with XXXs exposure time and inspect
  the quality.
\end{itemize}

}
\hdashrule[0.5ex]{\textwidth}{1pt}{3mm}
  Expected Result \\
{\footnotesize
\begin{itemize}
\tightlist
\item
  TMA reaches the DIMM position.
\item
  DIMM imaging quality is sufficient.
\end{itemize}

}

\begin{tabular}{p{2cm}}
\toprule
Step 759  \\ \hline
\end{tabular}
 Description \\
{\footnotesize
\textbf{Move TMA to the DIMM position and \textbf{Take DIMM images}}\\

\begin{itemize}
\tightlist
\item
  Command the TMA to the DIMM position by applying the offsets
\item
  While tracking, take DIMM images with XXXs exposure time and inspect
  the quality.
\end{itemize}

}
\hdashrule[0.5ex]{\textwidth}{1pt}{3mm}
  Expected Result \\
{\footnotesize
\begin{itemize}
\tightlist
\item
  TMA reaches the DIMM position.
\item
  DIMM imaging quality is sufficient.
\end{itemize}

}

\begin{tabular}{p{2cm}}
\toprule
Step 760  \\ \hline
\end{tabular}
 Description \\
{\footnotesize
\textbf{Move TMA to the DIMM position and \textbf{Take DIMM images}}\\

\begin{itemize}
\tightlist
\item
  Command the TMA to the DIMM position by applying the offsets
\item
  While tracking, take DIMM images with XXXs exposure time and inspect
  the quality.
\end{itemize}

}
\hdashrule[0.5ex]{\textwidth}{1pt}{3mm}
  Expected Result \\
{\footnotesize
\begin{itemize}
\tightlist
\item
  TMA reaches the DIMM position.
\item
  DIMM imaging quality is sufficient.
\end{itemize}

}

\begin{tabular}{p{2cm}}
\toprule
Step 761  \\ \hline
\end{tabular}
 Description \\
{\footnotesize
\textbf{Move TMA to the DIMM position and \textbf{Take DIMM images}}\\

\begin{itemize}
\tightlist
\item
  Command the TMA to the DIMM position by applying the offsets
\item
  While tracking, take DIMM images with XXXs exposure time and inspect
  the quality.
\end{itemize}

}
\hdashrule[0.5ex]{\textwidth}{1pt}{3mm}
  Expected Result \\
{\footnotesize
\begin{itemize}
\tightlist
\item
  TMA reaches the DIMM position.
\item
  DIMM imaging quality is sufficient.
\end{itemize}

}

\begin{tabular}{p{2cm}}
\toprule
Step 762  \\ \hline
\end{tabular}
 Description \\
{\footnotesize
\textbf{Move TMA to the DIMM position and \textbf{Take DIMM images}}\\

\begin{itemize}
\tightlist
\item
  Command the TMA to the DIMM position by applying the offsets
\item
  While tracking, take DIMM images with XXXs exposure time and inspect
  the quality.
\end{itemize}

}
\hdashrule[0.5ex]{\textwidth}{1pt}{3mm}
  Expected Result \\
{\footnotesize
\begin{itemize}
\tightlist
\item
  TMA reaches the DIMM position.
\item
  DIMM imaging quality is sufficient.
\end{itemize}

}

\begin{tabular}{p{2cm}}
\toprule
Step 763  \\ \hline
\end{tabular}
 Description \\
{\footnotesize
\textbf{Move TMA to the DIMM position and \textbf{Take DIMM images}}\\

\begin{itemize}
\tightlist
\item
  Command the TMA to the DIMM position by applying the offsets
\item
  While tracking, take DIMM images with XXXs exposure time and inspect
  the quality.
\end{itemize}

}
\hdashrule[0.5ex]{\textwidth}{1pt}{3mm}
  Expected Result \\
{\footnotesize
\begin{itemize}
\tightlist
\item
  TMA reaches the DIMM position.
\item
  DIMM imaging quality is sufficient.
\end{itemize}

}

\begin{tabular}{p{2cm}}
\toprule
Step 764  \\ \hline
\end{tabular}
 Description \\
{\footnotesize
\textbf{Move TMA to the DIMM position and \textbf{Take DIMM images}}\\

\begin{itemize}
\tightlist
\item
  Command the TMA to the DIMM position by applying the offsets
\item
  While tracking, take DIMM images with XXXs exposure time and inspect
  the quality.
\end{itemize}

}
\hdashrule[0.5ex]{\textwidth}{1pt}{3mm}
  Expected Result \\
{\footnotesize
\begin{itemize}
\tightlist
\item
  TMA reaches the DIMM position.
\item
  DIMM imaging quality is sufficient.
\end{itemize}

}

\begin{tabular}{p{2cm}}
\toprule
Step 765  \\ \hline
\end{tabular}
 Description \\
{\footnotesize
\textbf{Move TMA to the DIMM position and \textbf{Take DIMM images}}\\

\begin{itemize}
\tightlist
\item
  Command the TMA to the DIMM position by applying the offsets
\item
  While tracking, take DIMM images with XXXs exposure time and inspect
  the quality.
\end{itemize}

}
\hdashrule[0.5ex]{\textwidth}{1pt}{3mm}
  Expected Result \\
{\footnotesize
\begin{itemize}
\tightlist
\item
  TMA reaches the DIMM position.
\item
  DIMM imaging quality is sufficient.
\end{itemize}

}

\begin{tabular}{p{2cm}}
\toprule
Step 766  \\ \hline
\end{tabular}
 Description \\
{\footnotesize
\textbf{Move TMA to the DIMM position and \textbf{Take DIMM images}}\\

\begin{itemize}
\tightlist
\item
  Command the TMA to the DIMM position by applying the offsets
\item
  While tracking, take DIMM images with XXXs exposure time and inspect
  the quality.
\end{itemize}

}
\hdashrule[0.5ex]{\textwidth}{1pt}{3mm}
  Expected Result \\
{\footnotesize
\begin{itemize}
\tightlist
\item
  TMA reaches the DIMM position.
\item
  DIMM imaging quality is sufficient.
\end{itemize}

}

\begin{tabular}{p{2cm}}
\toprule
Step 767  \\ \hline
\end{tabular}
 Description \\
{\footnotesize
\textbf{Move TMA to the DIMM position and \textbf{Take DIMM images}}\\

\begin{itemize}
\tightlist
\item
  Command the TMA to the DIMM position by applying the offsets
\item
  While tracking, take DIMM images with XXXs exposure time and inspect
  the quality.
\end{itemize}

}
\hdashrule[0.5ex]{\textwidth}{1pt}{3mm}
  Expected Result \\
{\footnotesize
\begin{itemize}
\tightlist
\item
  TMA reaches the DIMM position.
\item
  DIMM imaging quality is sufficient.
\end{itemize}

}

\begin{tabular}{p{2cm}}
\toprule
Step 768  \\ \hline
\end{tabular}
 Description \\
{\footnotesize
\textbf{Move TMA to the DIMM position and \textbf{Take DIMM images}}\\

\begin{itemize}
\tightlist
\item
  Command the TMA to the DIMM position by applying the offsets
\item
  While tracking, take DIMM images with XXXs exposure time and inspect
  the quality.
\end{itemize}

}
\hdashrule[0.5ex]{\textwidth}{1pt}{3mm}
  Expected Result \\
{\footnotesize
\begin{itemize}
\tightlist
\item
  TMA reaches the DIMM position.
\item
  DIMM imaging quality is sufficient.
\end{itemize}

}

\begin{tabular}{p{2cm}}
\toprule
Step 769  \\ \hline
\end{tabular}
 Description \\
{\footnotesize
\textbf{Move TMA to the DIMM position and \textbf{Take DIMM images}}\\

\begin{itemize}
\tightlist
\item
  Command the TMA to the DIMM position by applying the offsets
\item
  While tracking, take DIMM images with XXXs exposure time and inspect
  the quality.
\end{itemize}

}
\hdashrule[0.5ex]{\textwidth}{1pt}{3mm}
  Expected Result \\
{\footnotesize
\begin{itemize}
\tightlist
\item
  TMA reaches the DIMM position.
\item
  DIMM imaging quality is sufficient.
\end{itemize}

}

\begin{tabular}{p{2cm}}
\toprule
Step 770  \\ \hline
\end{tabular}
 Description \\
{\footnotesize
\textbf{Move TMA to the DIMM position and \textbf{Take DIMM images}}\\

\begin{itemize}
\tightlist
\item
  Command the TMA to the DIMM position by applying the offsets
\item
  While tracking, take DIMM images with XXXs exposure time and inspect
  the quality.
\end{itemize}

}
\hdashrule[0.5ex]{\textwidth}{1pt}{3mm}
  Expected Result \\
{\footnotesize
\begin{itemize}
\tightlist
\item
  TMA reaches the DIMM position.
\item
  DIMM imaging quality is sufficient.
\end{itemize}

}

\begin{tabular}{p{2cm}}
\toprule
Step 771  \\ \hline
\end{tabular}
 Description \\
{\footnotesize
\textbf{Move TMA to the DIMM position and \textbf{Take DIMM images}}\\

\begin{itemize}
\tightlist
\item
  Command the TMA to the DIMM position by applying the offsets
\item
  While tracking, take DIMM images with XXXs exposure time and inspect
  the quality.
\end{itemize}

}
\hdashrule[0.5ex]{\textwidth}{1pt}{3mm}
  Expected Result \\
{\footnotesize
\begin{itemize}
\tightlist
\item
  TMA reaches the DIMM position.
\item
  DIMM imaging quality is sufficient.
\end{itemize}

}

\begin{tabular}{p{2cm}}
\toprule
Step 772  \\ \hline
\end{tabular}
 Description \\
{\footnotesize
\textbf{Move TMA to the DIMM position and \textbf{Take DIMM images}}\\

\begin{itemize}
\tightlist
\item
  Command the TMA to the DIMM position by applying the offsets
\item
  While tracking, take DIMM images with XXXs exposure time and inspect
  the quality.
\end{itemize}

}
\hdashrule[0.5ex]{\textwidth}{1pt}{3mm}
  Expected Result \\
{\footnotesize
\begin{itemize}
\tightlist
\item
  TMA reaches the DIMM position.
\item
  DIMM imaging quality is sufficient.
\end{itemize}

}

\begin{tabular}{p{2cm}}
\toprule
Step 773  \\ \hline
\end{tabular}
 Description \\
{\footnotesize
\textbf{Move TMA to the DIMM position and \textbf{Take DIMM images}}\\

\begin{itemize}
\tightlist
\item
  Command the TMA to the DIMM position by applying the offsets
\item
  While tracking, take DIMM images with XXXs exposure time and inspect
  the quality.
\end{itemize}

}
\hdashrule[0.5ex]{\textwidth}{1pt}{3mm}
  Expected Result \\
{\footnotesize
\begin{itemize}
\tightlist
\item
  TMA reaches the DIMM position.
\item
  DIMM imaging quality is sufficient.
\end{itemize}

}

\begin{tabular}{p{2cm}}
\toprule
Step 774  \\ \hline
\end{tabular}
 Description \\
{\footnotesize
\textbf{Move TMA to the DIMM position and \textbf{Take DIMM images}}\\

\begin{itemize}
\tightlist
\item
  Command the TMA to the DIMM position by applying the offsets
\item
  While tracking, take DIMM images with XXXs exposure time and inspect
  the quality.
\end{itemize}

}
\hdashrule[0.5ex]{\textwidth}{1pt}{3mm}
  Expected Result \\
{\footnotesize
\begin{itemize}
\tightlist
\item
  TMA reaches the DIMM position.
\item
  DIMM imaging quality is sufficient.
\end{itemize}

}

\begin{tabular}{p{2cm}}
\toprule
Step 775  \\ \hline
\end{tabular}
 Description \\
{\footnotesize
\textbf{Move TMA to the DIMM position and \textbf{Take DIMM images}}\\

\begin{itemize}
\tightlist
\item
  Command the TMA to the DIMM position by applying the offsets
\item
  While tracking, take DIMM images with XXXs exposure time and inspect
  the quality.
\end{itemize}

}
\hdashrule[0.5ex]{\textwidth}{1pt}{3mm}
  Expected Result \\
{\footnotesize
\begin{itemize}
\tightlist
\item
  TMA reaches the DIMM position.
\item
  DIMM imaging quality is sufficient.
\end{itemize}

}

\begin{tabular}{p{2cm}}
\toprule
Step 776  \\ \hline
\end{tabular}
 Description \\
{\footnotesize
\textbf{Move TMA to the DIMM position and \textbf{Take DIMM images}}\\

\begin{itemize}
\tightlist
\item
  Command the TMA to the DIMM position by applying the offsets
\item
  While tracking, take DIMM images with XXXs exposure time and inspect
  the quality.
\end{itemize}

}
\hdashrule[0.5ex]{\textwidth}{1pt}{3mm}
  Expected Result \\
{\footnotesize
\begin{itemize}
\tightlist
\item
  TMA reaches the DIMM position.
\item
  DIMM imaging quality is sufficient.
\end{itemize}

}

\begin{tabular}{p{2cm}}
\toprule
Step 777  \\ \hline
\end{tabular}
 Description \\
{\footnotesize
\textbf{Move TMA to the DIMM position and \textbf{Take DIMM images}}\\

\begin{itemize}
\tightlist
\item
  Command the TMA to the DIMM position by applying the offsets
\item
  While tracking, take DIMM images with XXXs exposure time and inspect
  the quality.
\end{itemize}

}
\hdashrule[0.5ex]{\textwidth}{1pt}{3mm}
  Expected Result \\
{\footnotesize
\begin{itemize}
\tightlist
\item
  TMA reaches the DIMM position.
\item
  DIMM imaging quality is sufficient.
\end{itemize}

}

\begin{tabular}{p{2cm}}
\toprule
Step 778  \\ \hline
\end{tabular}
 Description \\
{\footnotesize
\textbf{Move TMA to the DIMM position and \textbf{Take DIMM images}}\\

\begin{itemize}
\tightlist
\item
  Command the TMA to the DIMM position by applying the offsets
\item
  While tracking, take DIMM images with XXXs exposure time and inspect
  the quality.
\end{itemize}

}
\hdashrule[0.5ex]{\textwidth}{1pt}{3mm}
  Expected Result \\
{\footnotesize
\begin{itemize}
\tightlist
\item
  TMA reaches the DIMM position.
\item
  DIMM imaging quality is sufficient.
\end{itemize}

}

\begin{tabular}{p{2cm}}
\toprule
Step 779  \\ \hline
\end{tabular}
 Description \\
{\footnotesize
\textbf{Move TMA to the DIMM position and \textbf{Take DIMM images}}\\

\begin{itemize}
\tightlist
\item
  Command the TMA to the DIMM position by applying the offsets
\item
  While tracking, take DIMM images with XXXs exposure time and inspect
  the quality.
\end{itemize}

}
\hdashrule[0.5ex]{\textwidth}{1pt}{3mm}
  Expected Result \\
{\footnotesize
\begin{itemize}
\tightlist
\item
  TMA reaches the DIMM position.
\item
  DIMM imaging quality is sufficient.
\end{itemize}

}

\begin{tabular}{p{2cm}}
\toprule
Step 780  \\ \hline
\end{tabular}
 Description \\
{\footnotesize
\textbf{Move TMA to the DIMM position and \textbf{Take DIMM images}}\\

\begin{itemize}
\tightlist
\item
  Command the TMA to the DIMM position by applying the offsets
\item
  While tracking, take DIMM images with XXXs exposure time and inspect
  the quality.
\end{itemize}

}
\hdashrule[0.5ex]{\textwidth}{1pt}{3mm}
  Expected Result \\
{\footnotesize
\begin{itemize}
\tightlist
\item
  TMA reaches the DIMM position.
\item
  DIMM imaging quality is sufficient.
\end{itemize}

}

\begin{tabular}{p{2cm}}
\toprule
Step 781  \\ \hline
\end{tabular}
 Description \\
{\footnotesize
\textbf{Move TMA to the DIMM position and \textbf{Take DIMM images}}\\

\begin{itemize}
\tightlist
\item
  Command the TMA to the DIMM position by applying the offsets
\item
  While tracking, take DIMM images with XXXs exposure time and inspect
  the quality.
\end{itemize}

}
\hdashrule[0.5ex]{\textwidth}{1pt}{3mm}
  Expected Result \\
{\footnotesize
\begin{itemize}
\tightlist
\item
  TMA reaches the DIMM position.
\item
  DIMM imaging quality is sufficient.
\end{itemize}

}

\begin{tabular}{p{2cm}}
\toprule
Step 782  \\ \hline
\end{tabular}
 Description \\
{\footnotesize
\textbf{Move TMA to the DIMM position and \textbf{Take DIMM images}}\\

\begin{itemize}
\tightlist
\item
  Command the TMA to the DIMM position by applying the offsets
\item
  While tracking, take DIMM images with XXXs exposure time and inspect
  the quality.
\end{itemize}

}
\hdashrule[0.5ex]{\textwidth}{1pt}{3mm}
  Expected Result \\
{\footnotesize
\begin{itemize}
\tightlist
\item
  TMA reaches the DIMM position.
\item
  DIMM imaging quality is sufficient.
\end{itemize}

}

\begin{tabular}{p{2cm}}
\toprule
Step 783  \\ \hline
\end{tabular}
 Description \\
{\footnotesize
\textbf{Move TMA to the DIMM position and \textbf{Take DIMM images}}\\

\begin{itemize}
\tightlist
\item
  Command the TMA to the DIMM position by applying the offsets
\item
  While tracking, take DIMM images with XXXs exposure time and inspect
  the quality.
\end{itemize}

}
\hdashrule[0.5ex]{\textwidth}{1pt}{3mm}
  Expected Result \\
{\footnotesize
\begin{itemize}
\tightlist
\item
  TMA reaches the DIMM position.
\item
  DIMM imaging quality is sufficient.
\end{itemize}

}

\begin{tabular}{p{2cm}}
\toprule
Step 784  \\ \hline
\end{tabular}
 Description \\
{\footnotesize
\textbf{Move TMA to the DIMM position and \textbf{Take DIMM images}}\\

\begin{itemize}
\tightlist
\item
  Command the TMA to the DIMM position by applying the offsets
\item
  While tracking, take DIMM images with XXXs exposure time and inspect
  the quality.
\end{itemize}

}
\hdashrule[0.5ex]{\textwidth}{1pt}{3mm}
  Expected Result \\
{\footnotesize
\begin{itemize}
\tightlist
\item
  TMA reaches the DIMM position.
\item
  DIMM imaging quality is sufficient.
\end{itemize}

}

\begin{tabular}{p{2cm}}
\toprule
Step 785  \\ \hline
\end{tabular}
 Description \\
{\footnotesize
\textbf{Move TMA to the DIMM position and \textbf{Take DIMM images}}\\

\begin{itemize}
\tightlist
\item
  Command the TMA to the DIMM position by applying the offsets
\item
  While tracking, take DIMM images with XXXs exposure time and inspect
  the quality.
\end{itemize}

}
\hdashrule[0.5ex]{\textwidth}{1pt}{3mm}
  Expected Result \\
{\footnotesize
\begin{itemize}
\tightlist
\item
  TMA reaches the DIMM position.
\item
  DIMM imaging quality is sufficient.
\end{itemize}

}

\begin{tabular}{p{2cm}}
\toprule
Step 786  \\ \hline
\end{tabular}
 Description \\
{\footnotesize
\textbf{Move TMA to the DIMM position and \textbf{Take DIMM images}}\\

\begin{itemize}
\tightlist
\item
  Command the TMA to the DIMM position by applying the offsets
\item
  While tracking, take DIMM images with XXXs exposure time and inspect
  the quality.
\end{itemize}

}
\hdashrule[0.5ex]{\textwidth}{1pt}{3mm}
  Expected Result \\
{\footnotesize
\begin{itemize}
\tightlist
\item
  TMA reaches the DIMM position.
\item
  DIMM imaging quality is sufficient.
\end{itemize}

}

\begin{tabular}{p{2cm}}
\toprule
Step 787  \\ \hline
\end{tabular}
 Description \\
{\footnotesize
\textbf{Move TMA to the DIMM position and \textbf{Take DIMM images}}\\

\begin{itemize}
\tightlist
\item
  Command the TMA to the DIMM position by applying the offsets
\item
  While tracking, take DIMM images with XXXs exposure time and inspect
  the quality.
\end{itemize}

}
\hdashrule[0.5ex]{\textwidth}{1pt}{3mm}
  Expected Result \\
{\footnotesize
\begin{itemize}
\tightlist
\item
  TMA reaches the DIMM position.
\item
  DIMM imaging quality is sufficient.
\end{itemize}

}

\begin{tabular}{p{2cm}}
\toprule
Step 788  \\ \hline
\end{tabular}
 Description \\
{\footnotesize
\textbf{Move TMA to the DIMM position and \textbf{Take DIMM images}}\\

\begin{itemize}
\tightlist
\item
  Command the TMA to the DIMM position by applying the offsets
\item
  While tracking, take DIMM images with XXXs exposure time and inspect
  the quality.
\end{itemize}

}
\hdashrule[0.5ex]{\textwidth}{1pt}{3mm}
  Expected Result \\
{\footnotesize
\begin{itemize}
\tightlist
\item
  TMA reaches the DIMM position.
\item
  DIMM imaging quality is sufficient.
\end{itemize}

}

\begin{tabular}{p{2cm}}
\toprule
Step 789  \\ \hline
\end{tabular}
 Description \\
{\footnotesize
\textbf{Move TMA to the DIMM position and \textbf{Take DIMM images}}\\

\begin{itemize}
\tightlist
\item
  Command the TMA to the DIMM position by applying the offsets
\item
  While tracking, take DIMM images with XXXs exposure time and inspect
  the quality.
\end{itemize}

}
\hdashrule[0.5ex]{\textwidth}{1pt}{3mm}
  Expected Result \\
{\footnotesize
\begin{itemize}
\tightlist
\item
  TMA reaches the DIMM position.
\item
  DIMM imaging quality is sufficient.
\end{itemize}

}

\begin{tabular}{p{2cm}}
\toprule
Step 790  \\ \hline
\end{tabular}
 Description \\
{\footnotesize
\textbf{Move TMA to the DIMM position and \textbf{Take DIMM images}}\\

\begin{itemize}
\tightlist
\item
  Command the TMA to the DIMM position by applying the offsets
\item
  While tracking, take DIMM images with XXXs exposure time and inspect
  the quality.
\end{itemize}

}
\hdashrule[0.5ex]{\textwidth}{1pt}{3mm}
  Expected Result \\
{\footnotesize
\begin{itemize}
\tightlist
\item
  TMA reaches the DIMM position.
\item
  DIMM imaging quality is sufficient.
\end{itemize}

}

\begin{tabular}{p{2cm}}
\toprule
Step 791  \\ \hline
\end{tabular}
 Description \\
{\footnotesize
\textbf{Move TMA to the DIMM position and \textbf{Take DIMM images}}\\

\begin{itemize}
\tightlist
\item
  Command the TMA to the DIMM position by applying the offsets
\item
  While tracking, take DIMM images with XXXs exposure time and inspect
  the quality.
\end{itemize}

}
\hdashrule[0.5ex]{\textwidth}{1pt}{3mm}
  Expected Result \\
{\footnotesize
\begin{itemize}
\tightlist
\item
  TMA reaches the DIMM position.
\item
  DIMM imaging quality is sufficient.
\end{itemize}

}

\begin{tabular}{p{2cm}}
\toprule
Step 792  \\ \hline
\end{tabular}
 Description \\
{\footnotesize
\textbf{Move TMA to the DIMM position and \textbf{Take DIMM images}}\\

\begin{itemize}
\tightlist
\item
  Command the TMA to the DIMM position by applying the offsets
\item
  While tracking, take DIMM images with XXXs exposure time and inspect
  the quality.
\end{itemize}

}
\hdashrule[0.5ex]{\textwidth}{1pt}{3mm}
  Expected Result \\
{\footnotesize
\begin{itemize}
\tightlist
\item
  TMA reaches the DIMM position.
\item
  DIMM imaging quality is sufficient.
\end{itemize}

}

\begin{tabular}{p{2cm}}
\toprule
Step 793  \\ \hline
\end{tabular}
 Description \\
{\footnotesize
\textbf{Move TMA to the DIMM position and \textbf{Take DIMM images}}\\

\begin{itemize}
\tightlist
\item
  Command the TMA to the DIMM position by applying the offsets
\item
  While tracking, take DIMM images with XXXs exposure time and inspect
  the quality.
\end{itemize}

}
\hdashrule[0.5ex]{\textwidth}{1pt}{3mm}
  Expected Result \\
{\footnotesize
\begin{itemize}
\tightlist
\item
  TMA reaches the DIMM position.
\item
  DIMM imaging quality is sufficient.
\end{itemize}

}

\begin{tabular}{p{2cm}}
\toprule
Step 794  \\ \hline
\end{tabular}
 Description \\
{\footnotesize
\textbf{Move TMA to the DIMM position and \textbf{Take DIMM images}}\\

\begin{itemize}
\tightlist
\item
  Command the TMA to the DIMM position by applying the offsets
\item
  While tracking, take DIMM images with XXXs exposure time and inspect
  the quality.
\end{itemize}

}
\hdashrule[0.5ex]{\textwidth}{1pt}{3mm}
  Expected Result \\
{\footnotesize
\begin{itemize}
\tightlist
\item
  TMA reaches the DIMM position.
\item
  DIMM imaging quality is sufficient.
\end{itemize}

}

\begin{tabular}{p{2cm}}
\toprule
Step 795  \\ \hline
\end{tabular}
 Description \\
{\footnotesize
\textbf{Move TMA to the DIMM position and \textbf{Take DIMM images}}\\

\begin{itemize}
\tightlist
\item
  Command the TMA to the DIMM position by applying the offsets
\item
  While tracking, take DIMM images with XXXs exposure time and inspect
  the quality.
\end{itemize}

}
\hdashrule[0.5ex]{\textwidth}{1pt}{3mm}
  Expected Result \\
{\footnotesize
\begin{itemize}
\tightlist
\item
  TMA reaches the DIMM position.
\item
  DIMM imaging quality is sufficient.
\end{itemize}

}

\begin{tabular}{p{2cm}}
\toprule
Step 796  \\ \hline
\end{tabular}
 Description \\
{\footnotesize
\textbf{Move TMA to the DIMM position and \textbf{Take DIMM images}}\\

\begin{itemize}
\tightlist
\item
  Command the TMA to the DIMM position by applying the offsets
\item
  While tracking, take DIMM images with XXXs exposure time and inspect
  the quality.
\end{itemize}

}
\hdashrule[0.5ex]{\textwidth}{1pt}{3mm}
  Expected Result \\
{\footnotesize
\begin{itemize}
\tightlist
\item
  TMA reaches the DIMM position.
\item
  DIMM imaging quality is sufficient.
\end{itemize}

}

\begin{tabular}{p{2cm}}
\toprule
Step 797  \\ \hline
\end{tabular}
 Description \\
{\footnotesize
\textbf{Move TMA to the DIMM position and \textbf{Take DIMM images}}\\

\begin{itemize}
\tightlist
\item
  Command the TMA to the DIMM position by applying the offsets
\item
  While tracking, take DIMM images with XXXs exposure time and inspect
  the quality.
\end{itemize}

}
\hdashrule[0.5ex]{\textwidth}{1pt}{3mm}
  Expected Result \\
{\footnotesize
\begin{itemize}
\tightlist
\item
  TMA reaches the DIMM position.
\item
  DIMM imaging quality is sufficient.
\end{itemize}

}

\begin{tabular}{p{2cm}}
\toprule
Step 798  \\ \hline
\end{tabular}
 Description \\
{\footnotesize
\textbf{Move TMA to the DIMM position and \textbf{Take DIMM images}}\\

\begin{itemize}
\tightlist
\item
  Command the TMA to the DIMM position by applying the offsets
\item
  While tracking, take DIMM images with XXXs exposure time and inspect
  the quality.
\end{itemize}

}
\hdashrule[0.5ex]{\textwidth}{1pt}{3mm}
  Expected Result \\
{\footnotesize
\begin{itemize}
\tightlist
\item
  TMA reaches the DIMM position.
\item
  DIMM imaging quality is sufficient.
\end{itemize}

}

\begin{tabular}{p{2cm}}
\toprule
Step 799  \\ \hline
\end{tabular}
 Description \\
{\footnotesize
\textbf{Move TMA to the DIMM position and \textbf{Take DIMM images}}\\

\begin{itemize}
\tightlist
\item
  Command the TMA to the DIMM position by applying the offsets
\item
  While tracking, take DIMM images with XXXs exposure time and inspect
  the quality.
\end{itemize}

}
\hdashrule[0.5ex]{\textwidth}{1pt}{3mm}
  Expected Result \\
{\footnotesize
\begin{itemize}
\tightlist
\item
  TMA reaches the DIMM position.
\item
  DIMM imaging quality is sufficient.
\end{itemize}

}

\begin{tabular}{p{2cm}}
\toprule
Step 800  \\ \hline
\end{tabular}
 Description \\
{\footnotesize
\textbf{Move TMA to the DIMM position and \textbf{Take DIMM images}}\\

\begin{itemize}
\tightlist
\item
  Command the TMA to the DIMM position by applying the offsets
\item
  While tracking, take DIMM images with XXXs exposure time and inspect
  the quality.
\end{itemize}

}
\hdashrule[0.5ex]{\textwidth}{1pt}{3mm}
  Expected Result \\
{\footnotesize
\begin{itemize}
\tightlist
\item
  TMA reaches the DIMM position.
\item
  DIMM imaging quality is sufficient.
\end{itemize}

}

\begin{tabular}{p{2cm}}
\toprule
Step 801  \\ \hline
\end{tabular}
 Description \\
{\footnotesize
\textbf{Move TMA to the DIMM position and \textbf{Take DIMM images}}\\

\begin{itemize}
\tightlist
\item
  Command the TMA to the DIMM position by applying the offsets
\item
  While tracking, take DIMM images with XXXs exposure time and inspect
  the quality.
\end{itemize}

}
\hdashrule[0.5ex]{\textwidth}{1pt}{3mm}
  Expected Result \\
{\footnotesize
\begin{itemize}
\tightlist
\item
  TMA reaches the DIMM position.
\item
  DIMM imaging quality is sufficient.
\end{itemize}

}

\begin{tabular}{p{2cm}}
\toprule
Step 802  \\ \hline
\end{tabular}
 Description \\
{\footnotesize
\textbf{Point the TMA to (Az, El)-pattern position + DIMM pattern
offset\textbf{~and take DIMM images}\\
}

\begin{itemize}
\tightlist
\item
  Point the TMA back to {Pointing 1}⁠ at {-270}⁠ + DIMM offset, {15}⁠ +
  DIMM offset.
\item
  While tracking, take DIMM images with XXXs exposure time and inspect
  the quality.
\end{itemize}

}
\hdashrule[0.5ex]{\textwidth}{1pt}{3mm}
  Expected Result \\
{\footnotesize
\begin{itemize}
\tightlist
\item
  TMA reaches the position
\item
  DIMM image quality is sufficient
\end{itemize}

}

\begin{tabular}{p{2cm}}
\toprule
Step 803  \\ \hline
\end{tabular}
 Description \\
{\footnotesize
\textbf{Point the TMA to (Az, El)-pattern position + DIMM pattern
offset\textbf{~and take DIMM images}\\
}

\begin{itemize}
\tightlist
\item
  Point the TMA back to {Pointing 2}⁠ at {-270}⁠ + DIMM offset, {45}⁠ +
  DIMM offset.
\item
  While tracking, take DIMM images with XXXs exposure time and inspect
  the quality.
\end{itemize}

}
\hdashrule[0.5ex]{\textwidth}{1pt}{3mm}
  Expected Result \\
{\footnotesize
\begin{itemize}
\tightlist
\item
  TMA reaches the position
\item
  DIMM image quality is sufficient
\end{itemize}

}

\begin{tabular}{p{2cm}}
\toprule
Step 804  \\ \hline
\end{tabular}
 Description \\
{\footnotesize
\textbf{Point the TMA to (Az, El)-pattern position + DIMM pattern
offset\textbf{~and take DIMM images}\\
}

\begin{itemize}
\tightlist
\item
  Point the TMA back to {Pointing 3}⁠ at {-270}⁠ + DIMM offset, {75}⁠ +
  DIMM offset.
\item
  While tracking, take DIMM images with XXXs exposure time and inspect
  the quality.
\end{itemize}

}
\hdashrule[0.5ex]{\textwidth}{1pt}{3mm}
  Expected Result \\
{\footnotesize
\begin{itemize}
\tightlist
\item
  TMA reaches the position
\item
  DIMM image quality is sufficient
\end{itemize}

}

\begin{tabular}{p{2cm}}
\toprule
Step 805  \\ \hline
\end{tabular}
 Description \\
{\footnotesize
\textbf{Point the TMA to (Az, El)-pattern position + DIMM pattern
offset\textbf{~and take DIMM images}\\
}

\begin{itemize}
\tightlist
\item
  Point the TMA back to {Pointing 4}⁠ at {-270}⁠ + DIMM offset, {86.5}⁠
  + DIMM offset.
\item
  While tracking, take DIMM images with XXXs exposure time and inspect
  the quality.
\end{itemize}

}
\hdashrule[0.5ex]{\textwidth}{1pt}{3mm}
  Expected Result \\
{\footnotesize
\begin{itemize}
\tightlist
\item
  TMA reaches the position
\item
  DIMM image quality is sufficient
\end{itemize}

}

\begin{tabular}{p{2cm}}
\toprule
Step 806  \\ \hline
\end{tabular}
 Description \\
{\footnotesize
\textbf{Point the TMA to (Az, El)-pattern position + DIMM pattern
offset\textbf{~and take DIMM images}\\
}

\begin{itemize}
\tightlist
\item
  Point the TMA back to {Pointing 5}⁠ at {-180}⁠ + DIMM offset, {86.5}⁠
  + DIMM offset.
\item
  While tracking, take DIMM images with XXXs exposure time and inspect
  the quality.
\end{itemize}

}
\hdashrule[0.5ex]{\textwidth}{1pt}{3mm}
  Expected Result \\
{\footnotesize
\begin{itemize}
\tightlist
\item
  TMA reaches the position
\item
  DIMM image quality is sufficient
\end{itemize}

}

\begin{tabular}{p{2cm}}
\toprule
Step 807  \\ \hline
\end{tabular}
 Description \\
{\footnotesize
\textbf{Point the TMA to (Az, El)-pattern position + DIMM pattern
offset\textbf{~and take DIMM images}\\
}

\begin{itemize}
\tightlist
\item
  Point the TMA back to {Pointing 6}⁠ at {-180}⁠ + DIMM offset, {75}⁠ +
  DIMM offset.
\item
  While tracking, take DIMM images with XXXs exposure time and inspect
  the quality.
\end{itemize}

}
\hdashrule[0.5ex]{\textwidth}{1pt}{3mm}
  Expected Result \\
{\footnotesize
\begin{itemize}
\tightlist
\item
  TMA reaches the position
\item
  DIMM image quality is sufficient
\end{itemize}

}

\begin{tabular}{p{2cm}}
\toprule
Step 808  \\ \hline
\end{tabular}
 Description \\
{\footnotesize
\textbf{Point the TMA to (Az, El)-pattern position + DIMM pattern
offset\textbf{~and take DIMM images}\\
}

\begin{itemize}
\tightlist
\item
  Point the TMA back to {Pointing 7}⁠ at {-180}⁠ + DIMM offset, {45}⁠ +
  DIMM offset.
\item
  While tracking, take DIMM images with XXXs exposure time and inspect
  the quality.
\end{itemize}

}
\hdashrule[0.5ex]{\textwidth}{1pt}{3mm}
  Expected Result \\
{\footnotesize
\begin{itemize}
\tightlist
\item
  TMA reaches the position
\item
  DIMM image quality is sufficient
\end{itemize}

}

\begin{tabular}{p{2cm}}
\toprule
Step 809  \\ \hline
\end{tabular}
 Description \\
{\footnotesize
\textbf{Point the TMA to (Az, El)-pattern position + DIMM pattern
offset\textbf{~and take DIMM images}\\
}

\begin{itemize}
\tightlist
\item
  Point the TMA back to {Pointing 8}⁠ at {-180}⁠ + DIMM offset, {15}⁠ +
  DIMM offset.
\item
  While tracking, take DIMM images with XXXs exposure time and inspect
  the quality.
\end{itemize}

}
\hdashrule[0.5ex]{\textwidth}{1pt}{3mm}
  Expected Result \\
{\footnotesize
\begin{itemize}
\tightlist
\item
  TMA reaches the position
\item
  DIMM image quality is sufficient
\end{itemize}

}

\begin{tabular}{p{2cm}}
\toprule
Step 810  \\ \hline
\end{tabular}
 Description \\
{\footnotesize
\textbf{Point the TMA to (Az, El)-pattern position + DIMM pattern
offset\textbf{~and take DIMM images}\\
}

\begin{itemize}
\tightlist
\item
  Point the TMA back to {Pointing 9}⁠ at {-90}⁠ + DIMM offset, {15}⁠ +
  DIMM offset.
\item
  While tracking, take DIMM images with XXXs exposure time and inspect
  the quality.
\end{itemize}

}
\hdashrule[0.5ex]{\textwidth}{1pt}{3mm}
  Expected Result \\
{\footnotesize
\begin{itemize}
\tightlist
\item
  TMA reaches the position
\item
  DIMM image quality is sufficient
\end{itemize}

}

\begin{tabular}{p{2cm}}
\toprule
Step 811  \\ \hline
\end{tabular}
 Description \\
{\footnotesize
\textbf{Point the TMA to (Az, El)-pattern position + DIMM pattern
offset\textbf{~and take DIMM images}\\
}

\begin{itemize}
\tightlist
\item
  Point the TMA back to {Pointing 10}⁠ at {-90}⁠ + DIMM offset, {45}⁠ +
  DIMM offset.
\item
  While tracking, take DIMM images with XXXs exposure time and inspect
  the quality.
\end{itemize}

}
\hdashrule[0.5ex]{\textwidth}{1pt}{3mm}
  Expected Result \\
{\footnotesize
\begin{itemize}
\tightlist
\item
  TMA reaches the position
\item
  DIMM image quality is sufficient
\end{itemize}

}

\begin{tabular}{p{2cm}}
\toprule
Step 812  \\ \hline
\end{tabular}
 Description \\
{\footnotesize
\textbf{Point the TMA to (Az, El)-pattern position + DIMM pattern
offset\textbf{~and take DIMM images}\\
}

\begin{itemize}
\tightlist
\item
  Point the TMA back to {Pointing 11}⁠ at {-90}⁠ + DIMM offset, {75}⁠ +
  DIMM offset.
\item
  While tracking, take DIMM images with XXXs exposure time and inspect
  the quality.
\end{itemize}

}
\hdashrule[0.5ex]{\textwidth}{1pt}{3mm}
  Expected Result \\
{\footnotesize
\begin{itemize}
\tightlist
\item
  TMA reaches the position
\item
  DIMM image quality is sufficient
\end{itemize}

}

\begin{tabular}{p{2cm}}
\toprule
Step 813  \\ \hline
\end{tabular}
 Description \\
{\footnotesize
\textbf{Point the TMA to (Az, El)-pattern position + DIMM pattern
offset\textbf{~and take DIMM images}\\
}

\begin{itemize}
\tightlist
\item
  Point the TMA back to {Pointing 12}⁠ at {-90}⁠ + DIMM offset, {86.5}⁠
  + DIMM offset.
\item
  While tracking, take DIMM images with XXXs exposure time and inspect
  the quality.
\end{itemize}

}
\hdashrule[0.5ex]{\textwidth}{1pt}{3mm}
  Expected Result \\
{\footnotesize
\begin{itemize}
\tightlist
\item
  TMA reaches the position
\item
  DIMM image quality is sufficient
\end{itemize}

}

\begin{tabular}{p{2cm}}
\toprule
Step 814  \\ \hline
\end{tabular}
 Description \\
{\footnotesize
\textbf{Point the TMA to (Az, El)-pattern position + DIMM pattern
offset\textbf{~and take DIMM images}\\
}

\begin{itemize}
\tightlist
\item
  Point the TMA back to {Pointing 13}⁠ at {0}⁠ + DIMM offset, {86.5}⁠ +
  DIMM offset.
\item
  While tracking, take DIMM images with XXXs exposure time and inspect
  the quality.
\end{itemize}

}
\hdashrule[0.5ex]{\textwidth}{1pt}{3mm}
  Expected Result \\
{\footnotesize
\begin{itemize}
\tightlist
\item
  TMA reaches the position
\item
  DIMM image quality is sufficient
\end{itemize}

}

\begin{tabular}{p{2cm}}
\toprule
Step 815  \\ \hline
\end{tabular}
 Description \\
{\footnotesize
\textbf{Point the TMA to (Az, El)-pattern position + DIMM pattern
offset\textbf{~and take DIMM images}\\
}

\begin{itemize}
\tightlist
\item
  Point the TMA back to {Pointing 14}⁠ at {0}⁠ + DIMM offset, {75}⁠ +
  DIMM offset.
\item
  While tracking, take DIMM images with XXXs exposure time and inspect
  the quality.
\end{itemize}

}
\hdashrule[0.5ex]{\textwidth}{1pt}{3mm}
  Expected Result \\
{\footnotesize
\begin{itemize}
\tightlist
\item
  TMA reaches the position
\item
  DIMM image quality is sufficient
\end{itemize}

}

\begin{tabular}{p{2cm}}
\toprule
Step 816  \\ \hline
\end{tabular}
 Description \\
{\footnotesize
\textbf{Point the TMA to (Az, El)-pattern position + DIMM pattern
offset\textbf{~and take DIMM images}\\
}

\begin{itemize}
\tightlist
\item
  Point the TMA back to {Pointing 15}⁠ at {0}⁠ + DIMM offset, {45}⁠ +
  DIMM offset.
\item
  While tracking, take DIMM images with XXXs exposure time and inspect
  the quality.
\end{itemize}

}
\hdashrule[0.5ex]{\textwidth}{1pt}{3mm}
  Expected Result \\
{\footnotesize
\begin{itemize}
\tightlist
\item
  TMA reaches the position
\item
  DIMM image quality is sufficient
\end{itemize}

}

\begin{tabular}{p{2cm}}
\toprule
Step 817  \\ \hline
\end{tabular}
 Description \\
{\footnotesize
\textbf{Point the TMA to (Az, El)-pattern position + DIMM pattern
offset\textbf{~and take DIMM images}\\
}

\begin{itemize}
\tightlist
\item
  Point the TMA back to {Pointing 16}⁠ at {0}⁠ + DIMM offset, {15}⁠ +
  DIMM offset.
\item
  While tracking, take DIMM images with XXXs exposure time and inspect
  the quality.
\end{itemize}

}
\hdashrule[0.5ex]{\textwidth}{1pt}{3mm}
  Expected Result \\
{\footnotesize
\begin{itemize}
\tightlist
\item
  TMA reaches the position
\item
  DIMM image quality is sufficient
\end{itemize}

}

\begin{tabular}{p{2cm}}
\toprule
Step 818  \\ \hline
\end{tabular}
 Description \\
{\footnotesize
\textbf{Point the TMA to (Az, El)-pattern position + DIMM pattern
offset\textbf{~and take DIMM images}\\
}

\begin{itemize}
\tightlist
\item
  Point the TMA back to {Pointing 17}⁠ at {90}⁠ + DIMM offset, {15}⁠ +
  DIMM offset.
\item
  While tracking, take DIMM images with XXXs exposure time and inspect
  the quality.
\end{itemize}

}
\hdashrule[0.5ex]{\textwidth}{1pt}{3mm}
  Expected Result \\
{\footnotesize
\begin{itemize}
\tightlist
\item
  TMA reaches the position
\item
  DIMM image quality is sufficient
\end{itemize}

}

\begin{tabular}{p{2cm}}
\toprule
Step 819  \\ \hline
\end{tabular}
 Description \\
{\footnotesize
\textbf{Point the TMA to (Az, El)-pattern position + DIMM pattern
offset\textbf{~and take DIMM images}\\
}

\begin{itemize}
\tightlist
\item
  Point the TMA back to {Pointing 18}⁠ at {90}⁠ + DIMM offset, {45}⁠ +
  DIMM offset.
\item
  While tracking, take DIMM images with XXXs exposure time and inspect
  the quality.
\end{itemize}

}
\hdashrule[0.5ex]{\textwidth}{1pt}{3mm}
  Expected Result \\
{\footnotesize
\begin{itemize}
\tightlist
\item
  TMA reaches the position
\item
  DIMM image quality is sufficient
\end{itemize}

}

\begin{tabular}{p{2cm}}
\toprule
Step 820  \\ \hline
\end{tabular}
 Description \\
{\footnotesize
\textbf{Point the TMA to (Az, El)-pattern position + DIMM pattern
offset\textbf{~and take DIMM images}\\
}

\begin{itemize}
\tightlist
\item
  Point the TMA back to {Pointing 19}⁠ at {90}⁠ + DIMM offset, {75}⁠ +
  DIMM offset.
\item
  While tracking, take DIMM images with XXXs exposure time and inspect
  the quality.
\end{itemize}

}
\hdashrule[0.5ex]{\textwidth}{1pt}{3mm}
  Expected Result \\
{\footnotesize
\begin{itemize}
\tightlist
\item
  TMA reaches the position
\item
  DIMM image quality is sufficient
\end{itemize}

}

\begin{tabular}{p{2cm}}
\toprule
Step 821  \\ \hline
\end{tabular}
 Description \\
{\footnotesize
\textbf{Point the TMA to (Az, El)-pattern position + DIMM pattern
offset\textbf{~and take DIMM images}\\
}

\begin{itemize}
\tightlist
\item
  Point the TMA back to {Pointing 20}⁠ at {90}⁠ + DIMM offset, {86.5}⁠ +
  DIMM offset.
\item
  While tracking, take DIMM images with XXXs exposure time and inspect
  the quality.
\end{itemize}

}
\hdashrule[0.5ex]{\textwidth}{1pt}{3mm}
  Expected Result \\
{\footnotesize
\begin{itemize}
\tightlist
\item
  TMA reaches the position
\item
  DIMM image quality is sufficient
\end{itemize}

}

\begin{tabular}{p{2cm}}
\toprule
Step 822  \\ \hline
\end{tabular}
 Description \\
{\footnotesize
\textbf{Point the TMA to (Az, El)-pattern position + DIMM pattern
offset\textbf{~and take DIMM images}\\
}

\begin{itemize}
\tightlist
\item
  Point the TMA back to {Pointing 21}⁠ at {180}⁠ + DIMM offset, {86.5}⁠
  + DIMM offset.
\item
  While tracking, take DIMM images with XXXs exposure time and inspect
  the quality.
\end{itemize}

}
\hdashrule[0.5ex]{\textwidth}{1pt}{3mm}
  Expected Result \\
{\footnotesize
\begin{itemize}
\tightlist
\item
  TMA reaches the position
\item
  DIMM image quality is sufficient
\end{itemize}

}

\begin{tabular}{p{2cm}}
\toprule
Step 823  \\ \hline
\end{tabular}
 Description \\
{\footnotesize
\textbf{Point the TMA to (Az, El)-pattern position + DIMM pattern
offset\textbf{~and take DIMM images}\\
}

\begin{itemize}
\tightlist
\item
  Point the TMA back to {Pointing 22}⁠ at {180}⁠ + DIMM offset, {75}⁠ +
  DIMM offset.
\item
  While tracking, take DIMM images with XXXs exposure time and inspect
  the quality.
\end{itemize}

}
\hdashrule[0.5ex]{\textwidth}{1pt}{3mm}
  Expected Result \\
{\footnotesize
\begin{itemize}
\tightlist
\item
  TMA reaches the position
\item
  DIMM image quality is sufficient
\end{itemize}

}

\begin{tabular}{p{2cm}}
\toprule
Step 824  \\ \hline
\end{tabular}
 Description \\
{\footnotesize
\textbf{Point the TMA to (Az, El)-pattern position + DIMM pattern
offset\textbf{~and take DIMM images}\\
}

\begin{itemize}
\tightlist
\item
  Point the TMA back to {Pointing 23}⁠ at {180}⁠ + DIMM offset, {45}⁠ +
  DIMM offset.
\item
  While tracking, take DIMM images with XXXs exposure time and inspect
  the quality.
\end{itemize}

}
\hdashrule[0.5ex]{\textwidth}{1pt}{3mm}
  Expected Result \\
{\footnotesize
\begin{itemize}
\tightlist
\item
  TMA reaches the position
\item
  DIMM image quality is sufficient
\end{itemize}

}

\begin{tabular}{p{2cm}}
\toprule
Step 825  \\ \hline
\end{tabular}
 Description \\
{\footnotesize
\textbf{Point the TMA to (Az, El)-pattern position + DIMM pattern
offset\textbf{~and take DIMM images}\\
}

\begin{itemize}
\tightlist
\item
  Point the TMA back to {Pointing 24}⁠ at {180}⁠ + DIMM offset, {15}⁠ +
  DIMM offset.
\item
  While tracking, take DIMM images with XXXs exposure time and inspect
  the quality.
\end{itemize}

}
\hdashrule[0.5ex]{\textwidth}{1pt}{3mm}
  Expected Result \\
{\footnotesize
\begin{itemize}
\tightlist
\item
  TMA reaches the position
\item
  DIMM image quality is sufficient
\end{itemize}

}

\begin{tabular}{p{2cm}}
\toprule
Step 826  \\ \hline
\end{tabular}
 Description \\
{\footnotesize
\textbf{Point the TMA to (Az, El)-pattern position + DIMM pattern
offset\textbf{~and take DIMM images}\\
}

\begin{itemize}
\tightlist
\item
  Point the TMA back to {Pointing 25}⁠ at {270}⁠ + DIMM offset, {15}⁠ +
  DIMM offset.
\item
  While tracking, take DIMM images with XXXs exposure time and inspect
  the quality.
\end{itemize}

}
\hdashrule[0.5ex]{\textwidth}{1pt}{3mm}
  Expected Result \\
{\footnotesize
\begin{itemize}
\tightlist
\item
  TMA reaches the position
\item
  DIMM image quality is sufficient
\end{itemize}

}

\begin{tabular}{p{2cm}}
\toprule
Step 827  \\ \hline
\end{tabular}
 Description \\
{\footnotesize
\textbf{Point the TMA to (Az, El)-pattern position + DIMM pattern
offset\textbf{~and take DIMM images}\\
}

\begin{itemize}
\tightlist
\item
  Point the TMA back to {Pointing 26}⁠ at {270}⁠ + DIMM offset, {45}⁠ +
  DIMM offset.
\item
  While tracking, take DIMM images with XXXs exposure time and inspect
  the quality.
\end{itemize}

}
\hdashrule[0.5ex]{\textwidth}{1pt}{3mm}
  Expected Result \\
{\footnotesize
\begin{itemize}
\tightlist
\item
  TMA reaches the position
\item
  DIMM image quality is sufficient
\end{itemize}

}

\begin{tabular}{p{2cm}}
\toprule
Step 828  \\ \hline
\end{tabular}
 Description \\
{\footnotesize
\textbf{Point the TMA to (Az, El)-pattern position + DIMM pattern
offset\textbf{~and take DIMM images}\\
}

\begin{itemize}
\tightlist
\item
  Point the TMA back to {Pointing 12}⁠ at {-90}⁠ + DIMM offset, {86.5}⁠
  + DIMM offset.
\item
  While tracking, take DIMM images with XXXs exposure time and inspect
  the quality.
\end{itemize}

}
\hdashrule[0.5ex]{\textwidth}{1pt}{3mm}
  Expected Result \\
{\footnotesize
\begin{itemize}
\tightlist
\item
  TMA reaches the position
\item
  DIMM image quality is sufficient
\end{itemize}

}

\begin{tabular}{p{2cm}}
\toprule
Step 829  \\ \hline
\end{tabular}
 Description \\
{\footnotesize
\textbf{Point the TMA to (Az, El)-pattern position + DIMM pattern
offset\textbf{~and take DIMM images}\\
}

\begin{itemize}
\tightlist
\item
  Point the TMA back to {Pointing 27}⁠ at {270}⁠ + DIMM offset, {75}⁠ +
  DIMM offset.
\item
  While tracking, take DIMM images with XXXs exposure time and inspect
  the quality.
\end{itemize}

}
\hdashrule[0.5ex]{\textwidth}{1pt}{3mm}
  Expected Result \\
{\footnotesize
\begin{itemize}
\tightlist
\item
  TMA reaches the position
\item
  DIMM image quality is sufficient
\end{itemize}

}

\begin{tabular}{p{2cm}}
\toprule
Step 830  \\ \hline
\end{tabular}
 Description \\
{\footnotesize
\textbf{Point the TMA to (Az, El)-pattern position + DIMM pattern
offset\textbf{~and take DIMM images}\\
}

\begin{itemize}
\tightlist
\item
  Point the TMA back to {Pointing 13}⁠ at {0}⁠ + DIMM offset, {86.5}⁠ +
  DIMM offset.
\item
  While tracking, take DIMM images with XXXs exposure time and inspect
  the quality.
\end{itemize}

}
\hdashrule[0.5ex]{\textwidth}{1pt}{3mm}
  Expected Result \\
{\footnotesize
\begin{itemize}
\tightlist
\item
  TMA reaches the position
\item
  DIMM image quality is sufficient
\end{itemize}

}

\begin{tabular}{p{2cm}}
\toprule
Step 831  \\ \hline
\end{tabular}
 Description \\
{\footnotesize
\textbf{Point the TMA to (Az, El)-pattern position + DIMM pattern
offset\textbf{~and take DIMM images}\\
}

\begin{itemize}
\tightlist
\item
  Point the TMA back to {Pointing 28}⁠ at {270}⁠ + DIMM offset, {86.5}⁠
  + DIMM offset.
\item
  While tracking, take DIMM images with XXXs exposure time and inspect
  the quality.
\end{itemize}

}
\hdashrule[0.5ex]{\textwidth}{1pt}{3mm}
  Expected Result \\
{\footnotesize
\begin{itemize}
\tightlist
\item
  TMA reaches the position
\item
  DIMM image quality is sufficient
\end{itemize}

}

\begin{tabular}{p{2cm}}
\toprule
Step 832  \\ \hline
\end{tabular}
 Description \\
{\footnotesize
\textbf{Point the TMA to (Az, El)-pattern position + DIMM pattern
offset\textbf{~and take DIMM images}\\
}

\begin{itemize}
\tightlist
\item
  Point the TMA back to {Pointing 14}⁠ at {0}⁠ + DIMM offset, {75}⁠ +
  DIMM offset.
\item
  While tracking, take DIMM images with XXXs exposure time and inspect
  the quality.
\end{itemize}

}
\hdashrule[0.5ex]{\textwidth}{1pt}{3mm}
  Expected Result \\
{\footnotesize
\begin{itemize}
\tightlist
\item
  TMA reaches the position
\item
  DIMM image quality is sufficient
\end{itemize}

}

\begin{tabular}{p{2cm}}
\toprule
Step 833  \\ \hline
\end{tabular}
 Description \\
{\footnotesize
\textbf{Point the TMA to (Az, El)-pattern position + DIMM pattern
offset\textbf{~and take DIMM images}\\
}

\begin{itemize}
\tightlist
\item
  Point the TMA back to {Pointing 15}⁠ at {0}⁠ + DIMM offset, {45}⁠ +
  DIMM offset.
\item
  While tracking, take DIMM images with XXXs exposure time and inspect
  the quality.
\end{itemize}

}
\hdashrule[0.5ex]{\textwidth}{1pt}{3mm}
  Expected Result \\
{\footnotesize
\begin{itemize}
\tightlist
\item
  TMA reaches the position
\item
  DIMM image quality is sufficient
\end{itemize}

}

\begin{tabular}{p{2cm}}
\toprule
Step 834  \\ \hline
\end{tabular}
 Description \\
{\footnotesize
\textbf{Point the TMA to (Az, El)-pattern position + DIMM pattern
offset\textbf{~and take DIMM images}\\
}

\begin{itemize}
\tightlist
\item
  Point the TMA back to {Pointing 16}⁠ at {0}⁠ + DIMM offset, {15}⁠ +
  DIMM offset.
\item
  While tracking, take DIMM images with XXXs exposure time and inspect
  the quality.
\end{itemize}

}
\hdashrule[0.5ex]{\textwidth}{1pt}{3mm}
  Expected Result \\
{\footnotesize
\begin{itemize}
\tightlist
\item
  TMA reaches the position
\item
  DIMM image quality is sufficient
\end{itemize}

}

\begin{tabular}{p{2cm}}
\toprule
Step 835  \\ \hline
\end{tabular}
 Description \\
{\footnotesize
\textbf{Point the TMA to (Az, El)-pattern position + DIMM pattern
offset\textbf{~and take DIMM images}\\
}

\begin{itemize}
\tightlist
\item
  Point the TMA back to {Pointing 17}⁠ at {90}⁠ + DIMM offset, {15}⁠ +
  DIMM offset.
\item
  While tracking, take DIMM images with XXXs exposure time and inspect
  the quality.
\end{itemize}

}
\hdashrule[0.5ex]{\textwidth}{1pt}{3mm}
  Expected Result \\
{\footnotesize
\begin{itemize}
\tightlist
\item
  TMA reaches the position
\item
  DIMM image quality is sufficient
\end{itemize}

}

\begin{tabular}{p{2cm}}
\toprule
Step 836  \\ \hline
\end{tabular}
 Description \\
{\footnotesize
\textbf{Point the TMA to (Az, El)-pattern position + DIMM pattern
offset\textbf{~and take DIMM images}\\
}

\begin{itemize}
\tightlist
\item
  Point the TMA back to {Pointing 18}⁠ at {90}⁠ + DIMM offset, {45}⁠ +
  DIMM offset.
\item
  While tracking, take DIMM images with XXXs exposure time and inspect
  the quality.
\end{itemize}

}
\hdashrule[0.5ex]{\textwidth}{1pt}{3mm}
  Expected Result \\
{\footnotesize
\begin{itemize}
\tightlist
\item
  TMA reaches the position
\item
  DIMM image quality is sufficient
\end{itemize}

}

\begin{tabular}{p{2cm}}
\toprule
Step 837  \\ \hline
\end{tabular}
 Description \\
{\footnotesize
\textbf{Point the TMA to (Az, El)-pattern position + DIMM pattern
offset\textbf{~and take DIMM images}\\
}

\begin{itemize}
\tightlist
\item
  Point the TMA back to {Pointing 19}⁠ at {90}⁠ + DIMM offset, {75}⁠ +
  DIMM offset.
\item
  While tracking, take DIMM images with XXXs exposure time and inspect
  the quality.
\end{itemize}

}
\hdashrule[0.5ex]{\textwidth}{1pt}{3mm}
  Expected Result \\
{\footnotesize
\begin{itemize}
\tightlist
\item
  TMA reaches the position
\item
  DIMM image quality is sufficient
\end{itemize}

}

\begin{tabular}{p{2cm}}
\toprule
Step 838  \\ \hline
\end{tabular}
 Description \\
{\footnotesize
\textbf{Point the TMA to (Az, El)-pattern position + DIMM pattern
offset\textbf{~and take DIMM images}\\
}

\begin{itemize}
\tightlist
\item
  Point the TMA back to {Pointing 20}⁠ at {90}⁠ + DIMM offset, {86.5}⁠ +
  DIMM offset.
\item
  While tracking, take DIMM images with XXXs exposure time and inspect
  the quality.
\end{itemize}

}
\hdashrule[0.5ex]{\textwidth}{1pt}{3mm}
  Expected Result \\
{\footnotesize
\begin{itemize}
\tightlist
\item
  TMA reaches the position
\item
  DIMM image quality is sufficient
\end{itemize}

}

\begin{tabular}{p{2cm}}
\toprule
Step 839  \\ \hline
\end{tabular}
 Description \\
{\footnotesize
\textbf{Point the TMA to (Az, El)-pattern position + DIMM pattern
offset\textbf{~and take DIMM images}\\
}

\begin{itemize}
\tightlist
\item
  Point the TMA back to {Pointing 21}⁠ at {180}⁠ + DIMM offset, {86.5}⁠
  + DIMM offset.
\item
  While tracking, take DIMM images with XXXs exposure time and inspect
  the quality.
\end{itemize}

}
\hdashrule[0.5ex]{\textwidth}{1pt}{3mm}
  Expected Result \\
{\footnotesize
\begin{itemize}
\tightlist
\item
  TMA reaches the position
\item
  DIMM image quality is sufficient
\end{itemize}

}

\begin{tabular}{p{2cm}}
\toprule
Step 840  \\ \hline
\end{tabular}
 Description \\
{\footnotesize
\textbf{Point the TMA to (Az, El)-pattern position + DIMM pattern
offset\textbf{~and take DIMM images}\\
}

\begin{itemize}
\tightlist
\item
  Point the TMA back to {Pointing 22}⁠ at {180}⁠ + DIMM offset, {75}⁠ +
  DIMM offset.
\item
  While tracking, take DIMM images with XXXs exposure time and inspect
  the quality.
\end{itemize}

}
\hdashrule[0.5ex]{\textwidth}{1pt}{3mm}
  Expected Result \\
{\footnotesize
\begin{itemize}
\tightlist
\item
  TMA reaches the position
\item
  DIMM image quality is sufficient
\end{itemize}

}

\begin{tabular}{p{2cm}}
\toprule
Step 841  \\ \hline
\end{tabular}
 Description \\
{\footnotesize
\textbf{Point the TMA to (Az, El)-pattern position + DIMM pattern
offset\textbf{~and take DIMM images}\\
}

\begin{itemize}
\tightlist
\item
  Point the TMA back to {Pointing 23}⁠ at {180}⁠ + DIMM offset, {45}⁠ +
  DIMM offset.
\item
  While tracking, take DIMM images with XXXs exposure time and inspect
  the quality.
\end{itemize}

}
\hdashrule[0.5ex]{\textwidth}{1pt}{3mm}
  Expected Result \\
{\footnotesize
\begin{itemize}
\tightlist
\item
  TMA reaches the position
\item
  DIMM image quality is sufficient
\end{itemize}

}

\begin{tabular}{p{2cm}}
\toprule
Step 842  \\ \hline
\end{tabular}
 Description \\
{\footnotesize
\textbf{Point the TMA to (Az, El)-pattern position + DIMM pattern
offset\textbf{~and take DIMM images}\\
}

\begin{itemize}
\tightlist
\item
  Point the TMA back to {Pointing 24}⁠ at {180}⁠ + DIMM offset, {15}⁠ +
  DIMM offset.
\item
  While tracking, take DIMM images with XXXs exposure time and inspect
  the quality.
\end{itemize}

}
\hdashrule[0.5ex]{\textwidth}{1pt}{3mm}
  Expected Result \\
{\footnotesize
\begin{itemize}
\tightlist
\item
  TMA reaches the position
\item
  DIMM image quality is sufficient
\end{itemize}

}

\begin{tabular}{p{2cm}}
\toprule
Step 843  \\ \hline
\end{tabular}
 Description \\
{\footnotesize
\textbf{Point the TMA to (Az, El)-pattern position + DIMM pattern
offset\textbf{~and take DIMM images}\\
}

\begin{itemize}
\tightlist
\item
  Point the TMA back to {Pointing 25}⁠ at {270}⁠ + DIMM offset, {15}⁠ +
  DIMM offset.
\item
  While tracking, take DIMM images with XXXs exposure time and inspect
  the quality.
\end{itemize}

}
\hdashrule[0.5ex]{\textwidth}{1pt}{3mm}
  Expected Result \\
{\footnotesize
\begin{itemize}
\tightlist
\item
  TMA reaches the position
\item
  DIMM image quality is sufficient
\end{itemize}

}

\begin{tabular}{p{2cm}}
\toprule
Step 844  \\ \hline
\end{tabular}
 Description \\
{\footnotesize
\textbf{Point the TMA to (Az, El)-pattern position + DIMM pattern
offset\textbf{~and take DIMM images}\\
}

\begin{itemize}
\tightlist
\item
  Point the TMA back to {Pointing 26}⁠ at {270}⁠ + DIMM offset, {45}⁠ +
  DIMM offset.
\item
  While tracking, take DIMM images with XXXs exposure time and inspect
  the quality.
\end{itemize}

}
\hdashrule[0.5ex]{\textwidth}{1pt}{3mm}
  Expected Result \\
{\footnotesize
\begin{itemize}
\tightlist
\item
  TMA reaches the position
\item
  DIMM image quality is sufficient
\end{itemize}

}

\begin{tabular}{p{2cm}}
\toprule
Step 845  \\ \hline
\end{tabular}
 Description \\
{\footnotesize
\textbf{Point the TMA to (Az, El)-pattern position + DIMM pattern
offset\textbf{~and take DIMM images}\\
}

\begin{itemize}
\tightlist
\item
  Point the TMA back to {Pointing 27}⁠ at {270}⁠ + DIMM offset, {75}⁠ +
  DIMM offset.
\item
  While tracking, take DIMM images with XXXs exposure time and inspect
  the quality.
\end{itemize}

}
\hdashrule[0.5ex]{\textwidth}{1pt}{3mm}
  Expected Result \\
{\footnotesize
\begin{itemize}
\tightlist
\item
  TMA reaches the position
\item
  DIMM image quality is sufficient
\end{itemize}

}

\begin{tabular}{p{2cm}}
\toprule
Step 846  \\ \hline
\end{tabular}
 Description \\
{\footnotesize
\textbf{Point the TMA to (Az, El)-pattern position + DIMM pattern
offset\textbf{~and take DIMM images}\\
}

\begin{itemize}
\tightlist
\item
  Point the TMA back to {Pointing 28}⁠ at {270}⁠ + DIMM offset, {86.5}⁠
  + DIMM offset.
\item
  While tracking, take DIMM images with XXXs exposure time and inspect
  the quality.
\end{itemize}

}
\hdashrule[0.5ex]{\textwidth}{1pt}{3mm}
  Expected Result \\
{\footnotesize
\begin{itemize}
\tightlist
\item
  TMA reaches the position
\item
  DIMM image quality is sufficient
\end{itemize}

}

\begin{tabular}{p{2cm}}
\toprule
Step 847  \\ \hline
\end{tabular}
 Description \\
{\footnotesize
\textbf{TMA Settle Characterisation:}\\[2\baselineskip]Repeat the
previous step with the TMA damping turned off.

}
\hdashrule[0.5ex]{\textwidth}{1pt}{3mm}
  Expected Result \\
{\footnotesize
\begin{itemize}
\tightlist
\item
  The TMA takes longer to settle.
\item
  The inPosition event arrives later to the EFD.
\end{itemize}

}

\begin{tabular}{p{2cm}}
\toprule
Step 848  \\ \hline
\end{tabular}
 Description \\
{\footnotesize
\textbf{TMA Settle Characterisation:}\\[2\baselineskip]Repeat the
previous step with the TMA damping turned off.

}
\hdashrule[0.5ex]{\textwidth}{1pt}{3mm}
  Expected Result \\
{\footnotesize
\begin{itemize}
\tightlist
\item
  The TMA takes longer to settle.
\item
  The inPosition event arrives later to the EFD.
\end{itemize}

}

\begin{tabular}{p{2cm}}
\toprule
Step 849  \\ \hline
\end{tabular}
 Description \\
{\footnotesize
\textbf{TMA Settle Characterisation:}\\[2\baselineskip]Repeat the
previous step with the TMA damping turned off.

}
\hdashrule[0.5ex]{\textwidth}{1pt}{3mm}
  Expected Result \\
{\footnotesize
\begin{itemize}
\tightlist
\item
  The TMA takes longer to settle.
\item
  The inPosition event arrives later to the EFD.
\end{itemize}

}

\begin{tabular}{p{2cm}}
\toprule
Step 850  \\ \hline
\end{tabular}
 Description \\
{\footnotesize
\textbf{TMA Settle Characterisation:}\\[2\baselineskip]Repeat the
previous step with the TMA damping turned off.

}
\hdashrule[0.5ex]{\textwidth}{1pt}{3mm}
  Expected Result \\
{\footnotesize
\begin{itemize}
\tightlist
\item
  The TMA takes longer to settle.
\item
  The inPosition event arrives later to the EFD.
\end{itemize}

}

\begin{tabular}{p{2cm}}
\toprule
Step 851  \\ \hline
\end{tabular}
 Description \\
{\footnotesize
\textbf{TMA Settle Characterisation:}\\[2\baselineskip]Repeat the
previous step with the TMA damping turned off.

}
\hdashrule[0.5ex]{\textwidth}{1pt}{3mm}
  Expected Result \\
{\footnotesize
\begin{itemize}
\tightlist
\item
  The TMA takes longer to settle.
\item
  The inPosition event arrives later to the EFD.
\end{itemize}

}

\begin{tabular}{p{2cm}}
\toprule
Step 852  \\ \hline
\end{tabular}
 Description \\
{\footnotesize
\textbf{TMA Settle Characterisation:}\\[2\baselineskip]Repeat the
previous step with the TMA damping turned off.

}
\hdashrule[0.5ex]{\textwidth}{1pt}{3mm}
  Expected Result \\
{\footnotesize
\begin{itemize}
\tightlist
\item
  The TMA takes longer to settle.
\item
  The inPosition event arrives later to the EFD.
\end{itemize}

}

\begin{tabular}{p{2cm}}
\toprule
Step 853  \\ \hline
\end{tabular}
 Description \\
{\footnotesize
\textbf{TMA Settle Characterisation:}\\[2\baselineskip]Repeat the
previous step with the TMA damping turned off.

}
\hdashrule[0.5ex]{\textwidth}{1pt}{3mm}
  Expected Result \\
{\footnotesize
\begin{itemize}
\tightlist
\item
  The TMA takes longer to settle.
\item
  The inPosition event arrives later to the EFD.
\end{itemize}

}

\begin{tabular}{p{2cm}}
\toprule
Step 854  \\ \hline
\end{tabular}
 Description \\
{\footnotesize
\textbf{TMA Settle Characterisation:}\\[2\baselineskip]Repeat the
previous step with the TMA damping turned off.

}
\hdashrule[0.5ex]{\textwidth}{1pt}{3mm}
  Expected Result \\
{\footnotesize
\begin{itemize}
\tightlist
\item
  The TMA takes longer to settle.
\item
  The inPosition event arrives later to the EFD.
\end{itemize}

}

\begin{tabular}{p{2cm}}
\toprule
Step 855  \\ \hline
\end{tabular}
 Description \\
{\footnotesize
\textbf{TMA Settle Characterisation:}\\[2\baselineskip]Repeat the
previous step with the TMA damping turned off.

}
\hdashrule[0.5ex]{\textwidth}{1pt}{3mm}
  Expected Result \\
{\footnotesize
\begin{itemize}
\tightlist
\item
  The TMA takes longer to settle.
\item
  The inPosition event arrives later to the EFD.
\end{itemize}

}

\begin{tabular}{p{2cm}}
\toprule
Step 856  \\ \hline
\end{tabular}
 Description \\
{\footnotesize
\textbf{TMA Settle Characterisation:}\\[2\baselineskip]Repeat the
previous step with the TMA damping turned off.

}
\hdashrule[0.5ex]{\textwidth}{1pt}{3mm}
  Expected Result \\
{\footnotesize
\begin{itemize}
\tightlist
\item
  The TMA takes longer to settle.
\item
  The inPosition event arrives later to the EFD.
\end{itemize}

}

\begin{tabular}{p{2cm}}
\toprule
Step 857  \\ \hline
\end{tabular}
 Description \\
{\footnotesize
\textbf{TMA Settle Characterisation:}\\[2\baselineskip]Repeat the
previous step with the TMA damping turned off.

}
\hdashrule[0.5ex]{\textwidth}{1pt}{3mm}
  Expected Result \\
{\footnotesize
\begin{itemize}
\tightlist
\item
  The TMA takes longer to settle.
\item
  The inPosition event arrives later to the EFD.
\end{itemize}

}

\begin{tabular}{p{2cm}}
\toprule
Step 858  \\ \hline
\end{tabular}
 Description \\
{\footnotesize
\textbf{TMA Settle Characterisation:}\\[2\baselineskip]Repeat the
previous step with the TMA damping turned off.

}
\hdashrule[0.5ex]{\textwidth}{1pt}{3mm}
  Expected Result \\
{\footnotesize
\begin{itemize}
\tightlist
\item
  The TMA takes longer to settle.
\item
  The inPosition event arrives later to the EFD.
\end{itemize}

}

\begin{tabular}{p{2cm}}
\toprule
Step 859  \\ \hline
\end{tabular}
 Description \\
{\footnotesize
\textbf{TMA Settle Characterisation:}\\[2\baselineskip]Repeat the
previous step with the TMA damping turned off.

}
\hdashrule[0.5ex]{\textwidth}{1pt}{3mm}
  Expected Result \\
{\footnotesize
\begin{itemize}
\tightlist
\item
  The TMA takes longer to settle.
\item
  The inPosition event arrives later to the EFD.
\end{itemize}

}

\begin{tabular}{p{2cm}}
\toprule
Step 860  \\ \hline
\end{tabular}
 Description \\
{\footnotesize
\textbf{TMA Settle Characterisation:}\\[2\baselineskip]Repeat the
previous step with the TMA damping turned off.

}
\hdashrule[0.5ex]{\textwidth}{1pt}{3mm}
  Expected Result \\
{\footnotesize
\begin{itemize}
\tightlist
\item
  The TMA takes longer to settle.
\item
  The inPosition event arrives later to the EFD.
\end{itemize}

}

\begin{tabular}{p{2cm}}
\toprule
Step 861  \\ \hline
\end{tabular}
 Description \\
{\footnotesize
\textbf{TMA Settle Characterisation:}\\[2\baselineskip]Repeat the
previous step with the TMA damping turned off.

}
\hdashrule[0.5ex]{\textwidth}{1pt}{3mm}
  Expected Result \\
{\footnotesize
\begin{itemize}
\tightlist
\item
  The TMA takes longer to settle.
\item
  The inPosition event arrives later to the EFD.
\end{itemize}

}

\begin{tabular}{p{2cm}}
\toprule
Step 862  \\ \hline
\end{tabular}
 Description \\
{\footnotesize
\textbf{TMA Settle Characterisation:}\\[2\baselineskip]Repeat the
previous step with the TMA damping turned off.

}
\hdashrule[0.5ex]{\textwidth}{1pt}{3mm}
  Expected Result \\
{\footnotesize
\begin{itemize}
\tightlist
\item
  The TMA takes longer to settle.
\item
  The inPosition event arrives later to the EFD.
\end{itemize}

}

\begin{tabular}{p{2cm}}
\toprule
Step 863  \\ \hline
\end{tabular}
 Description \\
{\footnotesize
\textbf{TMA Settle Characterisation:}\\[2\baselineskip]Repeat the
previous step with the TMA damping turned off.

}
\hdashrule[0.5ex]{\textwidth}{1pt}{3mm}
  Expected Result \\
{\footnotesize
\begin{itemize}
\tightlist
\item
  The TMA takes longer to settle.
\item
  The inPosition event arrives later to the EFD.
\end{itemize}

}

\begin{tabular}{p{2cm}}
\toprule
Step 864  \\ \hline
\end{tabular}
 Description \\
{\footnotesize
\textbf{TMA Settle Characterisation:}\\[2\baselineskip]Repeat the
previous step with the TMA damping turned off.

}
\hdashrule[0.5ex]{\textwidth}{1pt}{3mm}
  Expected Result \\
{\footnotesize
\begin{itemize}
\tightlist
\item
  The TMA takes longer to settle.
\item
  The inPosition event arrives later to the EFD.
\end{itemize}

}

\begin{tabular}{p{2cm}}
\toprule
Step 865  \\ \hline
\end{tabular}
 Description \\
{\footnotesize
\textbf{TMA Settle Characterisation:}\\[2\baselineskip]Repeat the
previous step with the TMA damping turned off.

}
\hdashrule[0.5ex]{\textwidth}{1pt}{3mm}
  Expected Result \\
{\footnotesize
\begin{itemize}
\tightlist
\item
  The TMA takes longer to settle.
\item
  The inPosition event arrives later to the EFD.
\end{itemize}

}

\begin{tabular}{p{2cm}}
\toprule
Step 866  \\ \hline
\end{tabular}
 Description \\
{\footnotesize
\textbf{TMA Settle Characterisation:}\\[2\baselineskip]Repeat the
previous step with the TMA damping turned off.

}
\hdashrule[0.5ex]{\textwidth}{1pt}{3mm}
  Expected Result \\
{\footnotesize
\begin{itemize}
\tightlist
\item
  The TMA takes longer to settle.
\item
  The inPosition event arrives later to the EFD.
\end{itemize}

}

\begin{tabular}{p{2cm}}
\toprule
Step 867  \\ \hline
\end{tabular}
 Description \\
{\footnotesize
\textbf{TMA Settle Characterisation:}\\[2\baselineskip]Repeat the
previous step with the TMA damping turned off.

}
\hdashrule[0.5ex]{\textwidth}{1pt}{3mm}
  Expected Result \\
{\footnotesize
\begin{itemize}
\tightlist
\item
  The TMA takes longer to settle.
\item
  The inPosition event arrives later to the EFD.
\end{itemize}

}

\begin{tabular}{p{2cm}}
\toprule
Step 868  \\ \hline
\end{tabular}
 Description \\
{\footnotesize
\textbf{TMA Settle Characterisation:}\\[2\baselineskip]Repeat the
previous step with the TMA damping turned off.

}
\hdashrule[0.5ex]{\textwidth}{1pt}{3mm}
  Expected Result \\
{\footnotesize
\begin{itemize}
\tightlist
\item
  The TMA takes longer to settle.
\item
  The inPosition event arrives later to the EFD.
\end{itemize}

}

\begin{tabular}{p{2cm}}
\toprule
Step 869  \\ \hline
\end{tabular}
 Description \\
{\footnotesize
\textbf{TMA Settle Characterisation:}\\[2\baselineskip]Repeat the
previous step with the TMA damping turned off.

}
\hdashrule[0.5ex]{\textwidth}{1pt}{3mm}
  Expected Result \\
{\footnotesize
\begin{itemize}
\tightlist
\item
  The TMA takes longer to settle.
\item
  The inPosition event arrives later to the EFD.
\end{itemize}

}

\begin{tabular}{p{2cm}}
\toprule
Step 870  \\ \hline
\end{tabular}
 Description \\
{\footnotesize
\textbf{TMA Settle Characterisation:}\\[2\baselineskip]Repeat the
previous step with the TMA damping turned off.

}
\hdashrule[0.5ex]{\textwidth}{1pt}{3mm}
  Expected Result \\
{\footnotesize
\begin{itemize}
\tightlist
\item
  The TMA takes longer to settle.
\item
  The inPosition event arrives later to the EFD.
\end{itemize}

}

\begin{tabular}{p{2cm}}
\toprule
Step 871  \\ \hline
\end{tabular}
 Description \\
{\footnotesize
\textbf{TMA Settle Characterisation:}\\[2\baselineskip]Repeat the
previous step with the TMA damping turned off.

}
\hdashrule[0.5ex]{\textwidth}{1pt}{3mm}
  Expected Result \\
{\footnotesize
\begin{itemize}
\tightlist
\item
  The TMA takes longer to settle.
\item
  The inPosition event arrives later to the EFD.
\end{itemize}

}

\begin{tabular}{p{2cm}}
\toprule
Step 872  \\ \hline
\end{tabular}
 Description \\
{\footnotesize
\textbf{TMA Settle Characterisation:}\\[2\baselineskip]Repeat the
previous step with the TMA damping turned off.

}
\hdashrule[0.5ex]{\textwidth}{1pt}{3mm}
  Expected Result \\
{\footnotesize
\begin{itemize}
\tightlist
\item
  The TMA takes longer to settle.
\item
  The inPosition event arrives later to the EFD.
\end{itemize}

}

\begin{tabular}{p{2cm}}
\toprule
Step 873  \\ \hline
\end{tabular}
 Description \\
{\footnotesize
\textbf{TMA Settle Characterisation:}\\[2\baselineskip]Repeat the
previous step with the TMA damping turned off.

}
\hdashrule[0.5ex]{\textwidth}{1pt}{3mm}
  Expected Result \\
{\footnotesize
\begin{itemize}
\tightlist
\item
  The TMA takes longer to settle.
\item
  The inPosition event arrives later to the EFD.
\end{itemize}

}

\begin{tabular}{p{2cm}}
\toprule
Step 874  \\ \hline
\end{tabular}
 Description \\
{\footnotesize
\textbf{TMA Settle Characterisation:}\\[2\baselineskip]Repeat the
previous step with the TMA damping turned off.

}
\hdashrule[0.5ex]{\textwidth}{1pt}{3mm}
  Expected Result \\
{\footnotesize
\begin{itemize}
\tightlist
\item
  The TMA takes longer to settle.
\item
  The inPosition event arrives later to the EFD.
\end{itemize}

}

\begin{tabular}{p{2cm}}
\toprule
Step 875  \\ \hline
\end{tabular}
 Description \\
{\footnotesize
\textbf{TMA Settle Characterisation:}\\[2\baselineskip]Repeat the
previous step with the TMA damping turned off.

}
\hdashrule[0.5ex]{\textwidth}{1pt}{3mm}
  Expected Result \\
{\footnotesize
\begin{itemize}
\tightlist
\item
  The TMA takes longer to settle.
\item
  The inPosition event arrives later to the EFD.
\end{itemize}

}

\begin{tabular}{p{2cm}}
\toprule
Step 876  \\ \hline
\end{tabular}
 Description \\
{\footnotesize
\textbf{TMA Settle Characterisation:}\\[2\baselineskip]Repeat the
previous step with the TMA damping turned off.

}
\hdashrule[0.5ex]{\textwidth}{1pt}{3mm}
  Expected Result \\
{\footnotesize
\begin{itemize}
\tightlist
\item
  The TMA takes longer to settle.
\item
  The inPosition event arrives later to the EFD.
\end{itemize}

}

\begin{tabular}{p{2cm}}
\toprule
Step 877  \\ \hline
\end{tabular}
 Description \\
{\footnotesize
\textbf{TMA Settle Characterisation:}\\[2\baselineskip]Repeat the
previous step with the TMA damping turned off.

}
\hdashrule[0.5ex]{\textwidth}{1pt}{3mm}
  Expected Result \\
{\footnotesize
\begin{itemize}
\tightlist
\item
  The TMA takes longer to settle.
\item
  The inPosition event arrives later to the EFD.
\end{itemize}

}

\begin{tabular}{p{2cm}}
\toprule
Step 878  \\ \hline
\end{tabular}
 Description \\
{\footnotesize
\textbf{TMA Settle Characterisation:}\\[2\baselineskip]Repeat the
previous step with the TMA damping turned off.

}
\hdashrule[0.5ex]{\textwidth}{1pt}{3mm}
  Expected Result \\
{\footnotesize
\begin{itemize}
\tightlist
\item
  The TMA takes longer to settle.
\item
  The inPosition event arrives later to the EFD.
\end{itemize}

}

\begin{tabular}{p{2cm}}
\toprule
Step 879  \\ \hline
\end{tabular}
 Description \\
{\footnotesize
\textbf{TMA Settle Characterisation:}\\[2\baselineskip]Repeat the
previous step with the TMA damping turned off.

}
\hdashrule[0.5ex]{\textwidth}{1pt}{3mm}
  Expected Result \\
{\footnotesize
\begin{itemize}
\tightlist
\item
  The TMA takes longer to settle.
\item
  The inPosition event arrives later to the EFD.
\end{itemize}

}

\begin{tabular}{p{2cm}}
\toprule
Step 880  \\ \hline
\end{tabular}
 Description \\
{\footnotesize
\textbf{TMA Settle Characterisation:}\\[2\baselineskip]Repeat the
previous step with the TMA damping turned off.

}
\hdashrule[0.5ex]{\textwidth}{1pt}{3mm}
  Expected Result \\
{\footnotesize
\begin{itemize}
\tightlist
\item
  The TMA takes longer to settle.
\item
  The inPosition event arrives later to the EFD.
\end{itemize}

}

\begin{tabular}{p{2cm}}
\toprule
Step 881  \\ \hline
\end{tabular}
 Description \\
{\footnotesize
\textbf{TMA Settle Characterisation:}\\[2\baselineskip]Repeat the
previous step with the TMA damping turned off.

}
\hdashrule[0.5ex]{\textwidth}{1pt}{3mm}
  Expected Result \\
{\footnotesize
\begin{itemize}
\tightlist
\item
  The TMA takes longer to settle.
\item
  The inPosition event arrives later to the EFD.
\end{itemize}

}

\begin{tabular}{p{2cm}}
\toprule
Step 882  \\ \hline
\end{tabular}
 Description \\
{\footnotesize
\textbf{TMA Settle Characterisation:}\\[2\baselineskip]Repeat the
previous step with the TMA damping turned off.

}
\hdashrule[0.5ex]{\textwidth}{1pt}{3mm}
  Expected Result \\
{\footnotesize
\begin{itemize}
\tightlist
\item
  The TMA takes longer to settle.
\item
  The inPosition event arrives later to the EFD.
\end{itemize}

}

\begin{tabular}{p{2cm}}
\toprule
Step 883  \\ \hline
\end{tabular}
 Description \\
{\footnotesize
\textbf{TMA Settle Characterisation:}\\[2\baselineskip]Repeat the
previous step with the TMA damping turned off.

}
\hdashrule[0.5ex]{\textwidth}{1pt}{3mm}
  Expected Result \\
{\footnotesize
\begin{itemize}
\tightlist
\item
  The TMA takes longer to settle.
\item
  The inPosition event arrives later to the EFD.
\end{itemize}

}

\begin{tabular}{p{2cm}}
\toprule
Step 884  \\ \hline
\end{tabular}
 Description \\
{\footnotesize
\textbf{TMA Settle Characterisation:}\\[2\baselineskip]Repeat the
previous step with the TMA damping turned off.

}
\hdashrule[0.5ex]{\textwidth}{1pt}{3mm}
  Expected Result \\
{\footnotesize
\begin{itemize}
\tightlist
\item
  The TMA takes longer to settle.
\item
  The inPosition event arrives later to the EFD.
\end{itemize}

}

\begin{tabular}{p{2cm}}
\toprule
Step 885  \\ \hline
\end{tabular}
 Description \\
{\footnotesize
\textbf{TMA Settle Characterisation:}\\[2\baselineskip]Repeat the
previous step with the TMA damping turned off.

}
\hdashrule[0.5ex]{\textwidth}{1pt}{3mm}
  Expected Result \\
{\footnotesize
\begin{itemize}
\tightlist
\item
  The TMA takes longer to settle.
\item
  The inPosition event arrives later to the EFD.
\end{itemize}

}

\begin{tabular}{p{2cm}}
\toprule
Step 886  \\ \hline
\end{tabular}
 Description \\
{\footnotesize
\textbf{TMA Settle Characterisation:}\\[2\baselineskip]Repeat the
previous step with the TMA damping turned off.

}
\hdashrule[0.5ex]{\textwidth}{1pt}{3mm}
  Expected Result \\
{\footnotesize
\begin{itemize}
\tightlist
\item
  The TMA takes longer to settle.
\item
  The inPosition event arrives later to the EFD.
\end{itemize}

}

\begin{tabular}{p{2cm}}
\toprule
Step 887  \\ \hline
\end{tabular}
 Description \\
{\footnotesize
\textbf{TMA Settle Characterisation:}\\[2\baselineskip]Repeat the
previous step with the TMA damping turned off.

}
\hdashrule[0.5ex]{\textwidth}{1pt}{3mm}
  Expected Result \\
{\footnotesize
\begin{itemize}
\tightlist
\item
  The TMA takes longer to settle.
\item
  The inPosition event arrives later to the EFD.
\end{itemize}

}

\begin{tabular}{p{2cm}}
\toprule
Step 888  \\ \hline
\end{tabular}
 Description \\
{\footnotesize
\textbf{TMA Settle Characterisation:}\\[2\baselineskip]Repeat the
previous step with the TMA damping turned off.

}
\hdashrule[0.5ex]{\textwidth}{1pt}{3mm}
  Expected Result \\
{\footnotesize
\begin{itemize}
\tightlist
\item
  The TMA takes longer to settle.
\item
  The inPosition event arrives later to the EFD.
\end{itemize}

}

\begin{tabular}{p{2cm}}
\toprule
Step 889  \\ \hline
\end{tabular}
 Description \\
{\footnotesize
\textbf{TMA Settle Characterisation:}\\[2\baselineskip]Repeat the
previous step with the TMA damping turned off.

}
\hdashrule[0.5ex]{\textwidth}{1pt}{3mm}
  Expected Result \\
{\footnotesize
\begin{itemize}
\tightlist
\item
  The TMA takes longer to settle.
\item
  The inPosition event arrives later to the EFD.
\end{itemize}

}

\begin{tabular}{p{2cm}}
\toprule
Step 890  \\ \hline
\end{tabular}
 Description \\
{\footnotesize
\textbf{TMA Settle Characterisation:}\\[2\baselineskip]Repeat the
previous step with the TMA damping turned off.

}
\hdashrule[0.5ex]{\textwidth}{1pt}{3mm}
  Expected Result \\
{\footnotesize
\begin{itemize}
\tightlist
\item
  The TMA takes longer to settle.
\item
  The inPosition event arrives later to the EFD.
\end{itemize}

}

\begin{tabular}{p{2cm}}
\toprule
Step 891  \\ \hline
\end{tabular}
 Description \\
{\footnotesize
\textbf{TMA Settle Characterisation:}\\[2\baselineskip]Repeat the
previous step with the TMA damping turned off.

}
\hdashrule[0.5ex]{\textwidth}{1pt}{3mm}
  Expected Result \\
{\footnotesize
\begin{itemize}
\tightlist
\item
  The TMA takes longer to settle.
\item
  The inPosition event arrives later to the EFD.
\end{itemize}

}

\begin{tabular}{p{2cm}}
\toprule
Step 892  \\ \hline
\end{tabular}
 Description \\
{\footnotesize
\textbf{Point the TMA to (Az, El)-pattern position}\\
Point the TMA to {Pointing 28}⁠ at {270}⁠ , {86.5}⁠ .

}
\hdashrule[0.5ex]{\textwidth}{1pt}{3mm}
  Expected Result \\
{\footnotesize
The TMA starts moving.

}

\begin{tabular}{p{2cm}}
\toprule
Step 893  \\ \hline
\end{tabular}
 Description \\
{\footnotesize
Wait for the TMA to reach the commanded position.

}
\hdashrule[0.5ex]{\textwidth}{1pt}{3mm}
  Expected Result \\
{\footnotesize
The \emph{MTMount\_logevent\_azimuthInPosition} and
\emph{MTMount\_logevent\_elevationInPosition} inPosition parameter =
true.

}

\begin{tabular}{p{2cm}}
\toprule
Step 894  \\ \hline
\end{tabular}
 Description \\
{\footnotesize
\textbf{Find DIMM Object and DIMM Pattern Offset}\\

\begin{itemize}
\tightlist
\item
  While tracking, take a 10-sec exposure with the StarTracker.
\item
  Load the image into an image viewer.
\item
  Overlay the GAIA catalog.
\item
  Select a star brighter than XXX mag (bright enough for the DIMM).
\item
  Calculate the pixel offset between the StarTracker and the DIMM.
\item
  Transform the offset into AZ and EL offsets.
\end{itemize}

}
\hdashrule[0.5ex]{\textwidth}{1pt}{3mm}
  Expected Result \\
{\footnotesize
\begin{itemize}
\tightlist
\item
  An image was successfully taken with the StarTracker and is of
  sufficient quality.
\item
  AZ and EL offsets are available.
\end{itemize}

}

\begin{tabular}{p{2cm}}
\toprule
Step 895  \\ \hline
\end{tabular}
 Description \\
{\footnotesize
\textbf{Move TMA to the DIMM position and \textbf{Take DIMM images}}\\

\begin{itemize}
\tightlist
\item
  Command the TMA to the DIMM position by applying the offsets
\item
  While tracking, take DIMM images with XXXs exposure time and inspect
  the quality.
\end{itemize}

}
\hdashrule[0.5ex]{\textwidth}{1pt}{3mm}
  Expected Result \\
{\footnotesize
\begin{itemize}
\tightlist
\item
  TMA reaches the DIMM position.
\item
  DIMM imaging quality is sufficient.
\end{itemize}

}

\begin{tabular}{p{2cm}}
\toprule
Step 896  \\ \hline
\end{tabular}
 Description \\
{\footnotesize
\textbf{Move TMA to the DIMM position and \textbf{Take DIMM images}}\\

\begin{itemize}
\tightlist
\item
  Command the TMA to the DIMM position by applying the offsets
\item
  While tracking, take DIMM images with XXXs exposure time and inspect
  the quality.
\end{itemize}

}
\hdashrule[0.5ex]{\textwidth}{1pt}{3mm}
  Expected Result \\
{\footnotesize
\begin{itemize}
\tightlist
\item
  TMA reaches the DIMM position.
\item
  DIMM imaging quality is sufficient.
\end{itemize}

}

\begin{tabular}{p{2cm}}
\toprule
Step 897  \\ \hline
\end{tabular}
 Description \\
{\footnotesize
\textbf{Move TMA to the DIMM position and \textbf{Take DIMM images}}\\

\begin{itemize}
\tightlist
\item
  Command the TMA to the DIMM position by applying the offsets
\item
  While tracking, take DIMM images with XXXs exposure time and inspect
  the quality.
\end{itemize}

}
\hdashrule[0.5ex]{\textwidth}{1pt}{3mm}
  Expected Result \\
{\footnotesize
\begin{itemize}
\tightlist
\item
  TMA reaches the DIMM position.
\item
  DIMM imaging quality is sufficient.
\end{itemize}

}

\begin{tabular}{p{2cm}}
\toprule
Step 898  \\ \hline
\end{tabular}
 Description \\
{\footnotesize
\textbf{Move TMA to the DIMM position and \textbf{Take DIMM images}}\\

\begin{itemize}
\tightlist
\item
  Command the TMA to the DIMM position by applying the offsets
\item
  While tracking, take DIMM images with XXXs exposure time and inspect
  the quality.
\end{itemize}

}
\hdashrule[0.5ex]{\textwidth}{1pt}{3mm}
  Expected Result \\
{\footnotesize
\begin{itemize}
\tightlist
\item
  TMA reaches the DIMM position.
\item
  DIMM imaging quality is sufficient.
\end{itemize}

}

\begin{tabular}{p{2cm}}
\toprule
Step 899  \\ \hline
\end{tabular}
 Description \\
{\footnotesize
\textbf{Move TMA to the DIMM position and \textbf{Take DIMM images}}\\

\begin{itemize}
\tightlist
\item
  Command the TMA to the DIMM position by applying the offsets
\item
  While tracking, take DIMM images with XXXs exposure time and inspect
  the quality.
\end{itemize}

}
\hdashrule[0.5ex]{\textwidth}{1pt}{3mm}
  Expected Result \\
{\footnotesize
\begin{itemize}
\tightlist
\item
  TMA reaches the DIMM position.
\item
  DIMM imaging quality is sufficient.
\end{itemize}

}

\begin{tabular}{p{2cm}}
\toprule
Step 900  \\ \hline
\end{tabular}
 Description \\
{\footnotesize
\textbf{Move TMA to the DIMM position and \textbf{Take DIMM images}}\\

\begin{itemize}
\tightlist
\item
  Command the TMA to the DIMM position by applying the offsets
\item
  While tracking, take DIMM images with XXXs exposure time and inspect
  the quality.
\end{itemize}

}
\hdashrule[0.5ex]{\textwidth}{1pt}{3mm}
  Expected Result \\
{\footnotesize
\begin{itemize}
\tightlist
\item
  TMA reaches the DIMM position.
\item
  DIMM imaging quality is sufficient.
\end{itemize}

}

\begin{tabular}{p{2cm}}
\toprule
Step 901  \\ \hline
\end{tabular}
 Description \\
{\footnotesize
\textbf{Move TMA to the DIMM position and \textbf{Take DIMM images}}\\

\begin{itemize}
\tightlist
\item
  Command the TMA to the DIMM position by applying the offsets
\item
  While tracking, take DIMM images with XXXs exposure time and inspect
  the quality.
\end{itemize}

}
\hdashrule[0.5ex]{\textwidth}{1pt}{3mm}
  Expected Result \\
{\footnotesize
\begin{itemize}
\tightlist
\item
  TMA reaches the DIMM position.
\item
  DIMM imaging quality is sufficient.
\end{itemize}

}

\begin{tabular}{p{2cm}}
\toprule
Step 902  \\ \hline
\end{tabular}
 Description \\
{\footnotesize
\textbf{Move TMA to the DIMM position and \textbf{Take DIMM images}}\\

\begin{itemize}
\tightlist
\item
  Command the TMA to the DIMM position by applying the offsets
\item
  While tracking, take DIMM images with XXXs exposure time and inspect
  the quality.
\end{itemize}

}
\hdashrule[0.5ex]{\textwidth}{1pt}{3mm}
  Expected Result \\
{\footnotesize
\begin{itemize}
\tightlist
\item
  TMA reaches the DIMM position.
\item
  DIMM imaging quality is sufficient.
\end{itemize}

}

\begin{tabular}{p{2cm}}
\toprule
Step 903  \\ \hline
\end{tabular}
 Description \\
{\footnotesize
\textbf{Move TMA to the DIMM position and \textbf{Take DIMM images}}\\

\begin{itemize}
\tightlist
\item
  Command the TMA to the DIMM position by applying the offsets
\item
  While tracking, take DIMM images with XXXs exposure time and inspect
  the quality.
\end{itemize}

}
\hdashrule[0.5ex]{\textwidth}{1pt}{3mm}
  Expected Result \\
{\footnotesize
\begin{itemize}
\tightlist
\item
  TMA reaches the DIMM position.
\item
  DIMM imaging quality is sufficient.
\end{itemize}

}

\begin{tabular}{p{2cm}}
\toprule
Step 904  \\ \hline
\end{tabular}
 Description \\
{\footnotesize
\textbf{Move TMA to the DIMM position and \textbf{Take DIMM images}}\\

\begin{itemize}
\tightlist
\item
  Command the TMA to the DIMM position by applying the offsets
\item
  While tracking, take DIMM images with XXXs exposure time and inspect
  the quality.
\end{itemize}

}
\hdashrule[0.5ex]{\textwidth}{1pt}{3mm}
  Expected Result \\
{\footnotesize
\begin{itemize}
\tightlist
\item
  TMA reaches the DIMM position.
\item
  DIMM imaging quality is sufficient.
\end{itemize}

}

\begin{tabular}{p{2cm}}
\toprule
Step 905  \\ \hline
\end{tabular}
 Description \\
{\footnotesize
\textbf{Move TMA to the DIMM position and \textbf{Take DIMM images}}\\

\begin{itemize}
\tightlist
\item
  Command the TMA to the DIMM position by applying the offsets
\item
  While tracking, take DIMM images with XXXs exposure time and inspect
  the quality.
\end{itemize}

}
\hdashrule[0.5ex]{\textwidth}{1pt}{3mm}
  Expected Result \\
{\footnotesize
\begin{itemize}
\tightlist
\item
  TMA reaches the DIMM position.
\item
  DIMM imaging quality is sufficient.
\end{itemize}

}

\begin{tabular}{p{2cm}}
\toprule
Step 906  \\ \hline
\end{tabular}
 Description \\
{\footnotesize
\textbf{Move TMA to the DIMM position and \textbf{Take DIMM images}}\\

\begin{itemize}
\tightlist
\item
  Command the TMA to the DIMM position by applying the offsets
\item
  While tracking, take DIMM images with XXXs exposure time and inspect
  the quality.
\end{itemize}

}
\hdashrule[0.5ex]{\textwidth}{1pt}{3mm}
  Expected Result \\
{\footnotesize
\begin{itemize}
\tightlist
\item
  TMA reaches the DIMM position.
\item
  DIMM imaging quality is sufficient.
\end{itemize}

}

\begin{tabular}{p{2cm}}
\toprule
Step 907  \\ \hline
\end{tabular}
 Description \\
{\footnotesize
\textbf{Move TMA to the DIMM position and \textbf{Take DIMM images}}\\

\begin{itemize}
\tightlist
\item
  Command the TMA to the DIMM position by applying the offsets
\item
  While tracking, take DIMM images with XXXs exposure time and inspect
  the quality.
\end{itemize}

}
\hdashrule[0.5ex]{\textwidth}{1pt}{3mm}
  Expected Result \\
{\footnotesize
\begin{itemize}
\tightlist
\item
  TMA reaches the DIMM position.
\item
  DIMM imaging quality is sufficient.
\end{itemize}

}

\begin{tabular}{p{2cm}}
\toprule
Step 908  \\ \hline
\end{tabular}
 Description \\
{\footnotesize
\textbf{Move TMA to the DIMM position and \textbf{Take DIMM images}}\\

\begin{itemize}
\tightlist
\item
  Command the TMA to the DIMM position by applying the offsets
\item
  While tracking, take DIMM images with XXXs exposure time and inspect
  the quality.
\end{itemize}

}
\hdashrule[0.5ex]{\textwidth}{1pt}{3mm}
  Expected Result \\
{\footnotesize
\begin{itemize}
\tightlist
\item
  TMA reaches the DIMM position.
\item
  DIMM imaging quality is sufficient.
\end{itemize}

}

\begin{tabular}{p{2cm}}
\toprule
Step 909  \\ \hline
\end{tabular}
 Description \\
{\footnotesize
\textbf{Move TMA to the DIMM position and \textbf{Take DIMM images}}\\

\begin{itemize}
\tightlist
\item
  Command the TMA to the DIMM position by applying the offsets
\item
  While tracking, take DIMM images with XXXs exposure time and inspect
  the quality.
\end{itemize}

}
\hdashrule[0.5ex]{\textwidth}{1pt}{3mm}
  Expected Result \\
{\footnotesize
\begin{itemize}
\tightlist
\item
  TMA reaches the DIMM position.
\item
  DIMM imaging quality is sufficient.
\end{itemize}

}

\begin{tabular}{p{2cm}}
\toprule
Step 910  \\ \hline
\end{tabular}
 Description \\
{\footnotesize
\textbf{Move TMA to the DIMM position and \textbf{Take DIMM images}}\\

\begin{itemize}
\tightlist
\item
  Command the TMA to the DIMM position by applying the offsets
\item
  While tracking, take DIMM images with XXXs exposure time and inspect
  the quality.
\end{itemize}

}
\hdashrule[0.5ex]{\textwidth}{1pt}{3mm}
  Expected Result \\
{\footnotesize
\begin{itemize}
\tightlist
\item
  TMA reaches the DIMM position.
\item
  DIMM imaging quality is sufficient.
\end{itemize}

}

\begin{tabular}{p{2cm}}
\toprule
Step 911  \\ \hline
\end{tabular}
 Description \\
{\footnotesize
\textbf{Move TMA to the DIMM position and \textbf{Take DIMM images}}\\

\begin{itemize}
\tightlist
\item
  Command the TMA to the DIMM position by applying the offsets
\item
  While tracking, take DIMM images with XXXs exposure time and inspect
  the quality.
\end{itemize}

}
\hdashrule[0.5ex]{\textwidth}{1pt}{3mm}
  Expected Result \\
{\footnotesize
\begin{itemize}
\tightlist
\item
  TMA reaches the DIMM position.
\item
  DIMM imaging quality is sufficient.
\end{itemize}

}

\begin{tabular}{p{2cm}}
\toprule
Step 912  \\ \hline
\end{tabular}
 Description \\
{\footnotesize
\textbf{Move TMA to the DIMM position and \textbf{Take DIMM images}}\\

\begin{itemize}
\tightlist
\item
  Command the TMA to the DIMM position by applying the offsets
\item
  While tracking, take DIMM images with XXXs exposure time and inspect
  the quality.
\end{itemize}

}
\hdashrule[0.5ex]{\textwidth}{1pt}{3mm}
  Expected Result \\
{\footnotesize
\begin{itemize}
\tightlist
\item
  TMA reaches the DIMM position.
\item
  DIMM imaging quality is sufficient.
\end{itemize}

}

\begin{tabular}{p{2cm}}
\toprule
Step 913  \\ \hline
\end{tabular}
 Description \\
{\footnotesize
\textbf{Move TMA to the DIMM position and \textbf{Take DIMM images}}\\

\begin{itemize}
\tightlist
\item
  Command the TMA to the DIMM position by applying the offsets
\item
  While tracking, take DIMM images with XXXs exposure time and inspect
  the quality.
\end{itemize}

}
\hdashrule[0.5ex]{\textwidth}{1pt}{3mm}
  Expected Result \\
{\footnotesize
\begin{itemize}
\tightlist
\item
  TMA reaches the DIMM position.
\item
  DIMM imaging quality is sufficient.
\end{itemize}

}

\begin{tabular}{p{2cm}}
\toprule
Step 914  \\ \hline
\end{tabular}
 Description \\
{\footnotesize
\textbf{Move TMA to the DIMM position and \textbf{Take DIMM images}}\\

\begin{itemize}
\tightlist
\item
  Command the TMA to the DIMM position by applying the offsets
\item
  While tracking, take DIMM images with XXXs exposure time and inspect
  the quality.
\end{itemize}

}
\hdashrule[0.5ex]{\textwidth}{1pt}{3mm}
  Expected Result \\
{\footnotesize
\begin{itemize}
\tightlist
\item
  TMA reaches the DIMM position.
\item
  DIMM imaging quality is sufficient.
\end{itemize}

}

\begin{tabular}{p{2cm}}
\toprule
Step 915  \\ \hline
\end{tabular}
 Description \\
{\footnotesize
\textbf{Move TMA to the DIMM position and \textbf{Take DIMM images}}\\

\begin{itemize}
\tightlist
\item
  Command the TMA to the DIMM position by applying the offsets
\item
  While tracking, take DIMM images with XXXs exposure time and inspect
  the quality.
\end{itemize}

}
\hdashrule[0.5ex]{\textwidth}{1pt}{3mm}
  Expected Result \\
{\footnotesize
\begin{itemize}
\tightlist
\item
  TMA reaches the DIMM position.
\item
  DIMM imaging quality is sufficient.
\end{itemize}

}

\begin{tabular}{p{2cm}}
\toprule
Step 916  \\ \hline
\end{tabular}
 Description \\
{\footnotesize
\textbf{Move TMA to the DIMM position and \textbf{Take DIMM images}}\\

\begin{itemize}
\tightlist
\item
  Command the TMA to the DIMM position by applying the offsets
\item
  While tracking, take DIMM images with XXXs exposure time and inspect
  the quality.
\end{itemize}

}
\hdashrule[0.5ex]{\textwidth}{1pt}{3mm}
  Expected Result \\
{\footnotesize
\begin{itemize}
\tightlist
\item
  TMA reaches the DIMM position.
\item
  DIMM imaging quality is sufficient.
\end{itemize}

}

\begin{tabular}{p{2cm}}
\toprule
Step 917  \\ \hline
\end{tabular}
 Description \\
{\footnotesize
\textbf{Move TMA to the DIMM position and \textbf{Take DIMM images}}\\

\begin{itemize}
\tightlist
\item
  Command the TMA to the DIMM position by applying the offsets
\item
  While tracking, take DIMM images with XXXs exposure time and inspect
  the quality.
\end{itemize}

}
\hdashrule[0.5ex]{\textwidth}{1pt}{3mm}
  Expected Result \\
{\footnotesize
\begin{itemize}
\tightlist
\item
  TMA reaches the DIMM position.
\item
  DIMM imaging quality is sufficient.
\end{itemize}

}

\begin{tabular}{p{2cm}}
\toprule
Step 918  \\ \hline
\end{tabular}
 Description \\
{\footnotesize
\textbf{Move TMA to the DIMM position and \textbf{Take DIMM images}}\\

\begin{itemize}
\tightlist
\item
  Command the TMA to the DIMM position by applying the offsets
\item
  While tracking, take DIMM images with XXXs exposure time and inspect
  the quality.
\end{itemize}

}
\hdashrule[0.5ex]{\textwidth}{1pt}{3mm}
  Expected Result \\
{\footnotesize
\begin{itemize}
\tightlist
\item
  TMA reaches the DIMM position.
\item
  DIMM imaging quality is sufficient.
\end{itemize}

}

\begin{tabular}{p{2cm}}
\toprule
Step 919  \\ \hline
\end{tabular}
 Description \\
{\footnotesize
\textbf{Move TMA to the DIMM position and \textbf{Take DIMM images}}\\

\begin{itemize}
\tightlist
\item
  Command the TMA to the DIMM position by applying the offsets
\item
  While tracking, take DIMM images with XXXs exposure time and inspect
  the quality.
\end{itemize}

}
\hdashrule[0.5ex]{\textwidth}{1pt}{3mm}
  Expected Result \\
{\footnotesize
\begin{itemize}
\tightlist
\item
  TMA reaches the DIMM position.
\item
  DIMM imaging quality is sufficient.
\end{itemize}

}

\begin{tabular}{p{2cm}}
\toprule
Step 920  \\ \hline
\end{tabular}
 Description \\
{\footnotesize
\textbf{Move TMA to the DIMM position and \textbf{Take DIMM images}}\\

\begin{itemize}
\tightlist
\item
  Command the TMA to the DIMM position by applying the offsets
\item
  While tracking, take DIMM images with XXXs exposure time and inspect
  the quality.
\end{itemize}

}
\hdashrule[0.5ex]{\textwidth}{1pt}{3mm}
  Expected Result \\
{\footnotesize
\begin{itemize}
\tightlist
\item
  TMA reaches the DIMM position.
\item
  DIMM imaging quality is sufficient.
\end{itemize}

}

\begin{tabular}{p{2cm}}
\toprule
Step 921  \\ \hline
\end{tabular}
 Description \\
{\footnotesize
\textbf{Move TMA to the DIMM position and \textbf{Take DIMM images}}\\

\begin{itemize}
\tightlist
\item
  Command the TMA to the DIMM position by applying the offsets
\item
  While tracking, take DIMM images with XXXs exposure time and inspect
  the quality.
\end{itemize}

}
\hdashrule[0.5ex]{\textwidth}{1pt}{3mm}
  Expected Result \\
{\footnotesize
\begin{itemize}
\tightlist
\item
  TMA reaches the DIMM position.
\item
  DIMM imaging quality is sufficient.
\end{itemize}

}

\begin{tabular}{p{2cm}}
\toprule
Step 922  \\ \hline
\end{tabular}
 Description \\
{\footnotesize
\textbf{Move TMA to the DIMM position and \textbf{Take DIMM images}}\\

\begin{itemize}
\tightlist
\item
  Command the TMA to the DIMM position by applying the offsets
\item
  While tracking, take DIMM images with XXXs exposure time and inspect
  the quality.
\end{itemize}

}
\hdashrule[0.5ex]{\textwidth}{1pt}{3mm}
  Expected Result \\
{\footnotesize
\begin{itemize}
\tightlist
\item
  TMA reaches the DIMM position.
\item
  DIMM imaging quality is sufficient.
\end{itemize}

}

\begin{tabular}{p{2cm}}
\toprule
Step 923  \\ \hline
\end{tabular}
 Description \\
{\footnotesize
\textbf{Move TMA to the DIMM position and \textbf{Take DIMM images}}\\

\begin{itemize}
\tightlist
\item
  Command the TMA to the DIMM position by applying the offsets
\item
  While tracking, take DIMM images with XXXs exposure time and inspect
  the quality.
\end{itemize}

}
\hdashrule[0.5ex]{\textwidth}{1pt}{3mm}
  Expected Result \\
{\footnotesize
\begin{itemize}
\tightlist
\item
  TMA reaches the DIMM position.
\item
  DIMM imaging quality is sufficient.
\end{itemize}

}

\begin{tabular}{p{2cm}}
\toprule
Step 924  \\ \hline
\end{tabular}
 Description \\
{\footnotesize
\textbf{Move TMA to the DIMM position and \textbf{Take DIMM images}}\\

\begin{itemize}
\tightlist
\item
  Command the TMA to the DIMM position by applying the offsets
\item
  While tracking, take DIMM images with XXXs exposure time and inspect
  the quality.
\end{itemize}

}
\hdashrule[0.5ex]{\textwidth}{1pt}{3mm}
  Expected Result \\
{\footnotesize
\begin{itemize}
\tightlist
\item
  TMA reaches the DIMM position.
\item
  DIMM imaging quality is sufficient.
\end{itemize}

}

\begin{tabular}{p{2cm}}
\toprule
Step 925  \\ \hline
\end{tabular}
 Description \\
{\footnotesize
\textbf{Move TMA to the DIMM position and \textbf{Take DIMM images}}\\

\begin{itemize}
\tightlist
\item
  Command the TMA to the DIMM position by applying the offsets
\item
  While tracking, take DIMM images with XXXs exposure time and inspect
  the quality.
\end{itemize}

}
\hdashrule[0.5ex]{\textwidth}{1pt}{3mm}
  Expected Result \\
{\footnotesize
\begin{itemize}
\tightlist
\item
  TMA reaches the DIMM position.
\item
  DIMM imaging quality is sufficient.
\end{itemize}

}

\begin{tabular}{p{2cm}}
\toprule
Step 926  \\ \hline
\end{tabular}
 Description \\
{\footnotesize
\textbf{Move TMA to the DIMM position and \textbf{Take DIMM images}}\\

\begin{itemize}
\tightlist
\item
  Command the TMA to the DIMM position by applying the offsets
\item
  While tracking, take DIMM images with XXXs exposure time and inspect
  the quality.
\end{itemize}

}
\hdashrule[0.5ex]{\textwidth}{1pt}{3mm}
  Expected Result \\
{\footnotesize
\begin{itemize}
\tightlist
\item
  TMA reaches the DIMM position.
\item
  DIMM imaging quality is sufficient.
\end{itemize}

}

\begin{tabular}{p{2cm}}
\toprule
Step 927  \\ \hline
\end{tabular}
 Description \\
{\footnotesize
\textbf{Move TMA to the DIMM position and \textbf{Take DIMM images}}\\

\begin{itemize}
\tightlist
\item
  Command the TMA to the DIMM position by applying the offsets
\item
  While tracking, take DIMM images with XXXs exposure time and inspect
  the quality.
\end{itemize}

}
\hdashrule[0.5ex]{\textwidth}{1pt}{3mm}
  Expected Result \\
{\footnotesize
\begin{itemize}
\tightlist
\item
  TMA reaches the DIMM position.
\item
  DIMM imaging quality is sufficient.
\end{itemize}

}

\begin{tabular}{p{2cm}}
\toprule
Step 928  \\ \hline
\end{tabular}
 Description \\
{\footnotesize
\textbf{Move TMA to the DIMM position and \textbf{Take DIMM images}}\\

\begin{itemize}
\tightlist
\item
  Command the TMA to the DIMM position by applying the offsets
\item
  While tracking, take DIMM images with XXXs exposure time and inspect
  the quality.
\end{itemize}

}
\hdashrule[0.5ex]{\textwidth}{1pt}{3mm}
  Expected Result \\
{\footnotesize
\begin{itemize}
\tightlist
\item
  TMA reaches the DIMM position.
\item
  DIMM imaging quality is sufficient.
\end{itemize}

}

\begin{tabular}{p{2cm}}
\toprule
Step 929  \\ \hline
\end{tabular}
 Description \\
{\footnotesize
\textbf{Move TMA to the DIMM position and \textbf{Take DIMM images}}\\

\begin{itemize}
\tightlist
\item
  Command the TMA to the DIMM position by applying the offsets
\item
  While tracking, take DIMM images with XXXs exposure time and inspect
  the quality.
\end{itemize}

}
\hdashrule[0.5ex]{\textwidth}{1pt}{3mm}
  Expected Result \\
{\footnotesize
\begin{itemize}
\tightlist
\item
  TMA reaches the DIMM position.
\item
  DIMM imaging quality is sufficient.
\end{itemize}

}

\begin{tabular}{p{2cm}}
\toprule
Step 930  \\ \hline
\end{tabular}
 Description \\
{\footnotesize
\textbf{Move TMA to the DIMM position and \textbf{Take DIMM images}}\\

\begin{itemize}
\tightlist
\item
  Command the TMA to the DIMM position by applying the offsets
\item
  While tracking, take DIMM images with XXXs exposure time and inspect
  the quality.
\end{itemize}

}
\hdashrule[0.5ex]{\textwidth}{1pt}{3mm}
  Expected Result \\
{\footnotesize
\begin{itemize}
\tightlist
\item
  TMA reaches the DIMM position.
\item
  DIMM imaging quality is sufficient.
\end{itemize}

}

\begin{tabular}{p{2cm}}
\toprule
Step 931  \\ \hline
\end{tabular}
 Description \\
{\footnotesize
\textbf{Move TMA to the DIMM position and \textbf{Take DIMM images}}\\

\begin{itemize}
\tightlist
\item
  Command the TMA to the DIMM position by applying the offsets
\item
  While tracking, take DIMM images with XXXs exposure time and inspect
  the quality.
\end{itemize}

}
\hdashrule[0.5ex]{\textwidth}{1pt}{3mm}
  Expected Result \\
{\footnotesize
\begin{itemize}
\tightlist
\item
  TMA reaches the DIMM position.
\item
  DIMM imaging quality is sufficient.
\end{itemize}

}

\begin{tabular}{p{2cm}}
\toprule
Step 932  \\ \hline
\end{tabular}
 Description \\
{\footnotesize
\textbf{Move TMA to the DIMM position and \textbf{Take DIMM images}}\\

\begin{itemize}
\tightlist
\item
  Command the TMA to the DIMM position by applying the offsets
\item
  While tracking, take DIMM images with XXXs exposure time and inspect
  the quality.
\end{itemize}

}
\hdashrule[0.5ex]{\textwidth}{1pt}{3mm}
  Expected Result \\
{\footnotesize
\begin{itemize}
\tightlist
\item
  TMA reaches the DIMM position.
\item
  DIMM imaging quality is sufficient.
\end{itemize}

}

\begin{tabular}{p{2cm}}
\toprule
Step 933  \\ \hline
\end{tabular}
 Description \\
{\footnotesize
\textbf{Move TMA to the DIMM position and \textbf{Take DIMM images}}\\

\begin{itemize}
\tightlist
\item
  Command the TMA to the DIMM position by applying the offsets
\item
  While tracking, take DIMM images with XXXs exposure time and inspect
  the quality.
\end{itemize}

}
\hdashrule[0.5ex]{\textwidth}{1pt}{3mm}
  Expected Result \\
{\footnotesize
\begin{itemize}
\tightlist
\item
  TMA reaches the DIMM position.
\item
  DIMM imaging quality is sufficient.
\end{itemize}

}

\begin{tabular}{p{2cm}}
\toprule
Step 934  \\ \hline
\end{tabular}
 Description \\
{\footnotesize
\textbf{Move TMA to the DIMM position and \textbf{Take DIMM images}}\\

\begin{itemize}
\tightlist
\item
  Command the TMA to the DIMM position by applying the offsets
\item
  While tracking, take DIMM images with XXXs exposure time and inspect
  the quality.
\end{itemize}

}
\hdashrule[0.5ex]{\textwidth}{1pt}{3mm}
  Expected Result \\
{\footnotesize
\begin{itemize}
\tightlist
\item
  TMA reaches the DIMM position.
\item
  DIMM imaging quality is sufficient.
\end{itemize}

}

\begin{tabular}{p{2cm}}
\toprule
Step 935  \\ \hline
\end{tabular}
 Description \\
{\footnotesize
\textbf{Move TMA to the DIMM position and \textbf{Take DIMM images}}\\

\begin{itemize}
\tightlist
\item
  Command the TMA to the DIMM position by applying the offsets
\item
  While tracking, take DIMM images with XXXs exposure time and inspect
  the quality.
\end{itemize}

}
\hdashrule[0.5ex]{\textwidth}{1pt}{3mm}
  Expected Result \\
{\footnotesize
\begin{itemize}
\tightlist
\item
  TMA reaches the DIMM position.
\item
  DIMM imaging quality is sufficient.
\end{itemize}

}

\begin{tabular}{p{2cm}}
\toprule
Step 936  \\ \hline
\end{tabular}
 Description \\
{\footnotesize
\textbf{Move TMA to the DIMM position and \textbf{Take DIMM images}}\\

\begin{itemize}
\tightlist
\item
  Command the TMA to the DIMM position by applying the offsets
\item
  While tracking, take DIMM images with XXXs exposure time and inspect
  the quality.
\end{itemize}

}
\hdashrule[0.5ex]{\textwidth}{1pt}{3mm}
  Expected Result \\
{\footnotesize
\begin{itemize}
\tightlist
\item
  TMA reaches the DIMM position.
\item
  DIMM imaging quality is sufficient.
\end{itemize}

}

\begin{tabular}{p{2cm}}
\toprule
Step 937  \\ \hline
\end{tabular}
 Description \\
{\footnotesize
\textbf{Move TMA to the DIMM position and \textbf{Take DIMM images}}\\

\begin{itemize}
\tightlist
\item
  Command the TMA to the DIMM position by applying the offsets
\item
  While tracking, take DIMM images with XXXs exposure time and inspect
  the quality.
\end{itemize}

}
\hdashrule[0.5ex]{\textwidth}{1pt}{3mm}
  Expected Result \\
{\footnotesize
\begin{itemize}
\tightlist
\item
  TMA reaches the DIMM position.
\item
  DIMM imaging quality is sufficient.
\end{itemize}

}

\begin{tabular}{p{2cm}}
\toprule
Step 938  \\ \hline
\end{tabular}
 Description \\
{\footnotesize
\textbf{Move TMA to the DIMM position and \textbf{Take DIMM images}}\\

\begin{itemize}
\tightlist
\item
  Command the TMA to the DIMM position by applying the offsets
\item
  While tracking, take DIMM images with XXXs exposure time and inspect
  the quality.
\end{itemize}

}
\hdashrule[0.5ex]{\textwidth}{1pt}{3mm}
  Expected Result \\
{\footnotesize
\begin{itemize}
\tightlist
\item
  TMA reaches the DIMM position.
\item
  DIMM imaging quality is sufficient.
\end{itemize}

}

\begin{tabular}{p{2cm}}
\toprule
Step 939  \\ \hline
\end{tabular}
 Description \\
{\footnotesize
\textbf{Move TMA to the 1. random distance of 3.5deg}\\
Point the TMA to a random 3.5 deg combined offset in AZ and EL from
{Pointing 28}⁠ at {270}⁠, {86.5}⁠. Record the exact position of the
offset in AZ and El.

}
\hdashrule[0.5ex]{\textwidth}{1pt}{3mm}
  Expected Result \\
{\footnotesize
\begin{itemize}
\tightlist
\item
  The TMA reaches the commanded offset position.
\item
  The \emph{MTMount\_logevent\_azimuthInPosition} and
  \emph{MTMount\_logevent\_elevationInPosition}inPosition parameter =
  true.
\end{itemize}

}

\begin{tabular}{p{2cm}}
\toprule
Step 940  \\ \hline
\end{tabular}
 Description \\
{\footnotesize
\textbf{Find DIMM Object and DIMM Offset}\\

\begin{itemize}
\tightlist
\item
  While tracking, take a 10-sec exposure with the StarTracker.
\item
  Load the image into an image viewer.
\item
  Overlay the GAIA catalog.
\item
  Select a star brighter than XXX mag (bright enough for the DIMM).
\item
  Calculate the pixel offset between the StarTracker and the DIMM.
\item
  Transform the offset into AZ and EL offsets.
\end{itemize}

}
\hdashrule[0.5ex]{\textwidth}{1pt}{3mm}
  Expected Result \\
{\footnotesize
\begin{itemize}
\tightlist
\item
  An image was successfully taken with the StarTracker and is of
  sufficient quality.
\item
  AZ and EL offsets are available.
\end{itemize}

}

\begin{tabular}{p{2cm}}
\toprule
Step 941  \\ \hline
\end{tabular}
 Description \\
{\footnotesize
\textbf{Find DIMM Object and DIMM Offset}\\

\begin{itemize}
\tightlist
\item
  While tracking, take a 10-sec exposure with the StarTracker.
\item
  Load the image into an image viewer.
\item
  Overlay the GAIA catalog.
\item
  Select a star brighter than XXX mag (bright enough for the DIMM).
\item
  Calculate the pixel offset between the StarTracker and the DIMM.
\item
  Transform the offset into AZ and EL offsets.
\end{itemize}

}
\hdashrule[0.5ex]{\textwidth}{1pt}{3mm}
  Expected Result \\
{\footnotesize
\begin{itemize}
\tightlist
\item
  An image was successfully taken with the StarTracker and is of
  sufficient quality.
\item
  AZ and EL offsets are available.
\end{itemize}

}

\begin{tabular}{p{2cm}}
\toprule
Step 942  \\ \hline
\end{tabular}
 Description \\
{\footnotesize
\textbf{Find DIMM Object and DIMM Offset}\\

\begin{itemize}
\tightlist
\item
  While tracking, take a 10-sec exposure with the StarTracker.
\item
  Load the image into an image viewer.
\item
  Overlay the GAIA catalog.
\item
  Select a star brighter than XXX mag (bright enough for the DIMM).
\item
  Calculate the pixel offset between the StarTracker and the DIMM.
\item
  Transform the offset into AZ and EL offsets.
\end{itemize}

}
\hdashrule[0.5ex]{\textwidth}{1pt}{3mm}
  Expected Result \\
{\footnotesize
\begin{itemize}
\tightlist
\item
  An image was successfully taken with the StarTracker and is of
  sufficient quality.
\item
  AZ and EL offsets are available.
\end{itemize}

}

\begin{tabular}{p{2cm}}
\toprule
Step 943  \\ \hline
\end{tabular}
 Description \\
{\footnotesize
\textbf{Find DIMM Object and DIMM Offset}\\

\begin{itemize}
\tightlist
\item
  While tracking, take a 10-sec exposure with the StarTracker.
\item
  Load the image into an image viewer.
\item
  Overlay the GAIA catalog.
\item
  Select a star brighter than XXX mag (bright enough for the DIMM).
\item
  Calculate the pixel offset between the StarTracker and the DIMM.
\item
  Transform the offset into AZ and EL offsets.
\end{itemize}

}
\hdashrule[0.5ex]{\textwidth}{1pt}{3mm}
  Expected Result \\
{\footnotesize
\begin{itemize}
\tightlist
\item
  An image was successfully taken with the StarTracker and is of
  sufficient quality.
\item
  AZ and EL offsets are available.
\end{itemize}

}

\begin{tabular}{p{2cm}}
\toprule
Step 944  \\ \hline
\end{tabular}
 Description \\
{\footnotesize
\textbf{Find DIMM Object and DIMM Offset}\\

\begin{itemize}
\tightlist
\item
  While tracking, take a 10-sec exposure with the StarTracker.
\item
  Load the image into an image viewer.
\item
  Overlay the GAIA catalog.
\item
  Select a star brighter than XXX mag (bright enough for the DIMM).
\item
  Calculate the pixel offset between the StarTracker and the DIMM.
\item
  Transform the offset into AZ and EL offsets.
\end{itemize}

}
\hdashrule[0.5ex]{\textwidth}{1pt}{3mm}
  Expected Result \\
{\footnotesize
\begin{itemize}
\tightlist
\item
  An image was successfully taken with the StarTracker and is of
  sufficient quality.
\item
  AZ and EL offsets are available.
\end{itemize}

}

\begin{tabular}{p{2cm}}
\toprule
Step 945  \\ \hline
\end{tabular}
 Description \\
{\footnotesize
\textbf{Find DIMM Object and DIMM Offset}\\

\begin{itemize}
\tightlist
\item
  While tracking, take a 10-sec exposure with the StarTracker.
\item
  Load the image into an image viewer.
\item
  Overlay the GAIA catalog.
\item
  Select a star brighter than XXX mag (bright enough for the DIMM).
\item
  Calculate the pixel offset between the StarTracker and the DIMM.
\item
  Transform the offset into AZ and EL offsets.
\end{itemize}

}
\hdashrule[0.5ex]{\textwidth}{1pt}{3mm}
  Expected Result \\
{\footnotesize
\begin{itemize}
\tightlist
\item
  An image was successfully taken with the StarTracker and is of
  sufficient quality.
\item
  AZ and EL offsets are available.
\end{itemize}

}

\begin{tabular}{p{2cm}}
\toprule
Step 946  \\ \hline
\end{tabular}
 Description \\
{\footnotesize
\textbf{Find DIMM Object and DIMM Offset}\\

\begin{itemize}
\tightlist
\item
  While tracking, take a 10-sec exposure with the StarTracker.
\item
  Load the image into an image viewer.
\item
  Overlay the GAIA catalog.
\item
  Select a star brighter than XXX mag (bright enough for the DIMM).
\item
  Calculate the pixel offset between the StarTracker and the DIMM.
\item
  Transform the offset into AZ and EL offsets.
\end{itemize}

}
\hdashrule[0.5ex]{\textwidth}{1pt}{3mm}
  Expected Result \\
{\footnotesize
\begin{itemize}
\tightlist
\item
  An image was successfully taken with the StarTracker and is of
  sufficient quality.
\item
  AZ and EL offsets are available.
\end{itemize}

}

\begin{tabular}{p{2cm}}
\toprule
Step 947  \\ \hline
\end{tabular}
 Description \\
{\footnotesize
\textbf{Find DIMM Object and DIMM Offset}\\

\begin{itemize}
\tightlist
\item
  While tracking, take a 10-sec exposure with the StarTracker.
\item
  Load the image into an image viewer.
\item
  Overlay the GAIA catalog.
\item
  Select a star brighter than XXX mag (bright enough for the DIMM).
\item
  Calculate the pixel offset between the StarTracker and the DIMM.
\item
  Transform the offset into AZ and EL offsets.
\end{itemize}

}
\hdashrule[0.5ex]{\textwidth}{1pt}{3mm}
  Expected Result \\
{\footnotesize
\begin{itemize}
\tightlist
\item
  An image was successfully taken with the StarTracker and is of
  sufficient quality.
\item
  AZ and EL offsets are available.
\end{itemize}

}

\begin{tabular}{p{2cm}}
\toprule
Step 948  \\ \hline
\end{tabular}
 Description \\
{\footnotesize
\textbf{Find DIMM Object and DIMM Offset}\\

\begin{itemize}
\tightlist
\item
  While tracking, take a 10-sec exposure with the StarTracker.
\item
  Load the image into an image viewer.
\item
  Overlay the GAIA catalog.
\item
  Select a star brighter than XXX mag (bright enough for the DIMM).
\item
  Calculate the pixel offset between the StarTracker and the DIMM.
\item
  Transform the offset into AZ and EL offsets.
\end{itemize}

}
\hdashrule[0.5ex]{\textwidth}{1pt}{3mm}
  Expected Result \\
{\footnotesize
\begin{itemize}
\tightlist
\item
  An image was successfully taken with the StarTracker and is of
  sufficient quality.
\item
  AZ and EL offsets are available.
\end{itemize}

}

\begin{tabular}{p{2cm}}
\toprule
Step 949  \\ \hline
\end{tabular}
 Description \\
{\footnotesize
\textbf{Find DIMM Object and DIMM Offset}\\

\begin{itemize}
\tightlist
\item
  While tracking, take a 10-sec exposure with the StarTracker.
\item
  Load the image into an image viewer.
\item
  Overlay the GAIA catalog.
\item
  Select a star brighter than XXX mag (bright enough for the DIMM).
\item
  Calculate the pixel offset between the StarTracker and the DIMM.
\item
  Transform the offset into AZ and EL offsets.
\end{itemize}

}
\hdashrule[0.5ex]{\textwidth}{1pt}{3mm}
  Expected Result \\
{\footnotesize
\begin{itemize}
\tightlist
\item
  An image was successfully taken with the StarTracker and is of
  sufficient quality.
\item
  AZ and EL offsets are available.
\end{itemize}

}

\begin{tabular}{p{2cm}}
\toprule
Step 950  \\ \hline
\end{tabular}
 Description \\
{\footnotesize
\textbf{Find DIMM Object and DIMM Offset}\\

\begin{itemize}
\tightlist
\item
  While tracking, take a 10-sec exposure with the StarTracker.
\item
  Load the image into an image viewer.
\item
  Overlay the GAIA catalog.
\item
  Select a star brighter than XXX mag (bright enough for the DIMM).
\item
  Calculate the pixel offset between the StarTracker and the DIMM.
\item
  Transform the offset into AZ and EL offsets.
\end{itemize}

}
\hdashrule[0.5ex]{\textwidth}{1pt}{3mm}
  Expected Result \\
{\footnotesize
\begin{itemize}
\tightlist
\item
  An image was successfully taken with the StarTracker and is of
  sufficient quality.
\item
  AZ and EL offsets are available.
\end{itemize}

}

\begin{tabular}{p{2cm}}
\toprule
Step 951  \\ \hline
\end{tabular}
 Description \\
{\footnotesize
\textbf{Find DIMM Object and DIMM Offset}\\

\begin{itemize}
\tightlist
\item
  While tracking, take a 10-sec exposure with the StarTracker.
\item
  Load the image into an image viewer.
\item
  Overlay the GAIA catalog.
\item
  Select a star brighter than XXX mag (bright enough for the DIMM).
\item
  Calculate the pixel offset between the StarTracker and the DIMM.
\item
  Transform the offset into AZ and EL offsets.
\end{itemize}

}
\hdashrule[0.5ex]{\textwidth}{1pt}{3mm}
  Expected Result \\
{\footnotesize
\begin{itemize}
\tightlist
\item
  An image was successfully taken with the StarTracker and is of
  sufficient quality.
\item
  AZ and EL offsets are available.
\end{itemize}

}

\begin{tabular}{p{2cm}}
\toprule
Step 952  \\ \hline
\end{tabular}
 Description \\
{\footnotesize
\textbf{Find DIMM Object and DIMM Offset}\\

\begin{itemize}
\tightlist
\item
  While tracking, take a 10-sec exposure with the StarTracker.
\item
  Load the image into an image viewer.
\item
  Overlay the GAIA catalog.
\item
  Select a star brighter than XXX mag (bright enough for the DIMM).
\item
  Calculate the pixel offset between the StarTracker and the DIMM.
\item
  Transform the offset into AZ and EL offsets.
\end{itemize}

}
\hdashrule[0.5ex]{\textwidth}{1pt}{3mm}
  Expected Result \\
{\footnotesize
\begin{itemize}
\tightlist
\item
  An image was successfully taken with the StarTracker and is of
  sufficient quality.
\item
  AZ and EL offsets are available.
\end{itemize}

}

\begin{tabular}{p{2cm}}
\toprule
Step 953  \\ \hline
\end{tabular}
 Description \\
{\footnotesize
\textbf{Find DIMM Object and DIMM Offset}\\

\begin{itemize}
\tightlist
\item
  While tracking, take a 10-sec exposure with the StarTracker.
\item
  Load the image into an image viewer.
\item
  Overlay the GAIA catalog.
\item
  Select a star brighter than XXX mag (bright enough for the DIMM).
\item
  Calculate the pixel offset between the StarTracker and the DIMM.
\item
  Transform the offset into AZ and EL offsets.
\end{itemize}

}
\hdashrule[0.5ex]{\textwidth}{1pt}{3mm}
  Expected Result \\
{\footnotesize
\begin{itemize}
\tightlist
\item
  An image was successfully taken with the StarTracker and is of
  sufficient quality.
\item
  AZ and EL offsets are available.
\end{itemize}

}

\begin{tabular}{p{2cm}}
\toprule
Step 954  \\ \hline
\end{tabular}
 Description \\
{\footnotesize
\textbf{Find DIMM Object and DIMM Offset}\\

\begin{itemize}
\tightlist
\item
  While tracking, take a 10-sec exposure with the StarTracker.
\item
  Load the image into an image viewer.
\item
  Overlay the GAIA catalog.
\item
  Select a star brighter than XXX mag (bright enough for the DIMM).
\item
  Calculate the pixel offset between the StarTracker and the DIMM.
\item
  Transform the offset into AZ and EL offsets.
\end{itemize}

}
\hdashrule[0.5ex]{\textwidth}{1pt}{3mm}
  Expected Result \\
{\footnotesize
\begin{itemize}
\tightlist
\item
  An image was successfully taken with the StarTracker and is of
  sufficient quality.
\item
  AZ and EL offsets are available.
\end{itemize}

}

\begin{tabular}{p{2cm}}
\toprule
Step 955  \\ \hline
\end{tabular}
 Description \\
{\footnotesize
\textbf{Find DIMM Object and DIMM Offset}\\

\begin{itemize}
\tightlist
\item
  While tracking, take a 10-sec exposure with the StarTracker.
\item
  Load the image into an image viewer.
\item
  Overlay the GAIA catalog.
\item
  Select a star brighter than XXX mag (bright enough for the DIMM).
\item
  Calculate the pixel offset between the StarTracker and the DIMM.
\item
  Transform the offset into AZ and EL offsets.
\end{itemize}

}
\hdashrule[0.5ex]{\textwidth}{1pt}{3mm}
  Expected Result \\
{\footnotesize
\begin{itemize}
\tightlist
\item
  An image was successfully taken with the StarTracker and is of
  sufficient quality.
\item
  AZ and EL offsets are available.
\end{itemize}

}

\begin{tabular}{p{2cm}}
\toprule
Step 956  \\ \hline
\end{tabular}
 Description \\
{\footnotesize
\textbf{Find DIMM Object and DIMM Offset}\\

\begin{itemize}
\tightlist
\item
  While tracking, take a 10-sec exposure with the StarTracker.
\item
  Load the image into an image viewer.
\item
  Overlay the GAIA catalog.
\item
  Select a star brighter than XXX mag (bright enough for the DIMM).
\item
  Calculate the pixel offset between the StarTracker and the DIMM.
\item
  Transform the offset into AZ and EL offsets.
\end{itemize}

}
\hdashrule[0.5ex]{\textwidth}{1pt}{3mm}
  Expected Result \\
{\footnotesize
\begin{itemize}
\tightlist
\item
  An image was successfully taken with the StarTracker and is of
  sufficient quality.
\item
  AZ and EL offsets are available.
\end{itemize}

}

\begin{tabular}{p{2cm}}
\toprule
Step 957  \\ \hline
\end{tabular}
 Description \\
{\footnotesize
\textbf{Find DIMM Object and DIMM Offset}\\

\begin{itemize}
\tightlist
\item
  While tracking, take a 10-sec exposure with the StarTracker.
\item
  Load the image into an image viewer.
\item
  Overlay the GAIA catalog.
\item
  Select a star brighter than XXX mag (bright enough for the DIMM).
\item
  Calculate the pixel offset between the StarTracker and the DIMM.
\item
  Transform the offset into AZ and EL offsets.
\end{itemize}

}
\hdashrule[0.5ex]{\textwidth}{1pt}{3mm}
  Expected Result \\
{\footnotesize
\begin{itemize}
\tightlist
\item
  An image was successfully taken with the StarTracker and is of
  sufficient quality.
\item
  AZ and EL offsets are available.
\end{itemize}

}

\begin{tabular}{p{2cm}}
\toprule
Step 958  \\ \hline
\end{tabular}
 Description \\
{\footnotesize
\textbf{Find DIMM Object and DIMM Offset}\\

\begin{itemize}
\tightlist
\item
  While tracking, take a 10-sec exposure with the StarTracker.
\item
  Load the image into an image viewer.
\item
  Overlay the GAIA catalog.
\item
  Select a star brighter than XXX mag (bright enough for the DIMM).
\item
  Calculate the pixel offset between the StarTracker and the DIMM.
\item
  Transform the offset into AZ and EL offsets.
\end{itemize}

}
\hdashrule[0.5ex]{\textwidth}{1pt}{3mm}
  Expected Result \\
{\footnotesize
\begin{itemize}
\tightlist
\item
  An image was successfully taken with the StarTracker and is of
  sufficient quality.
\item
  AZ and EL offsets are available.
\end{itemize}

}

\begin{tabular}{p{2cm}}
\toprule
Step 959  \\ \hline
\end{tabular}
 Description \\
{\footnotesize
\textbf{Find DIMM Object and DIMM Offset}\\

\begin{itemize}
\tightlist
\item
  While tracking, take a 10-sec exposure with the StarTracker.
\item
  Load the image into an image viewer.
\item
  Overlay the GAIA catalog.
\item
  Select a star brighter than XXX mag (bright enough for the DIMM).
\item
  Calculate the pixel offset between the StarTracker and the DIMM.
\item
  Transform the offset into AZ and EL offsets.
\end{itemize}

}
\hdashrule[0.5ex]{\textwidth}{1pt}{3mm}
  Expected Result \\
{\footnotesize
\begin{itemize}
\tightlist
\item
  An image was successfully taken with the StarTracker and is of
  sufficient quality.
\item
  AZ and EL offsets are available.
\end{itemize}

}

\begin{tabular}{p{2cm}}
\toprule
Step 960  \\ \hline
\end{tabular}
 Description \\
{\footnotesize
\textbf{Find DIMM Object and DIMM Offset}\\

\begin{itemize}
\tightlist
\item
  While tracking, take a 10-sec exposure with the StarTracker.
\item
  Load the image into an image viewer.
\item
  Overlay the GAIA catalog.
\item
  Select a star brighter than XXX mag (bright enough for the DIMM).
\item
  Calculate the pixel offset between the StarTracker and the DIMM.
\item
  Transform the offset into AZ and EL offsets.
\end{itemize}

}
\hdashrule[0.5ex]{\textwidth}{1pt}{3mm}
  Expected Result \\
{\footnotesize
\begin{itemize}
\tightlist
\item
  An image was successfully taken with the StarTracker and is of
  sufficient quality.
\item
  AZ and EL offsets are available.
\end{itemize}

}

\begin{tabular}{p{2cm}}
\toprule
Step 961  \\ \hline
\end{tabular}
 Description \\
{\footnotesize
\textbf{Find DIMM Object and DIMM Offset}\\

\begin{itemize}
\tightlist
\item
  While tracking, take a 10-sec exposure with the StarTracker.
\item
  Load the image into an image viewer.
\item
  Overlay the GAIA catalog.
\item
  Select a star brighter than XXX mag (bright enough for the DIMM).
\item
  Calculate the pixel offset between the StarTracker and the DIMM.
\item
  Transform the offset into AZ and EL offsets.
\end{itemize}

}
\hdashrule[0.5ex]{\textwidth}{1pt}{3mm}
  Expected Result \\
{\footnotesize
\begin{itemize}
\tightlist
\item
  An image was successfully taken with the StarTracker and is of
  sufficient quality.
\item
  AZ and EL offsets are available.
\end{itemize}

}

\begin{tabular}{p{2cm}}
\toprule
Step 962  \\ \hline
\end{tabular}
 Description \\
{\footnotesize
\textbf{Find DIMM Object and DIMM Offset}\\

\begin{itemize}
\tightlist
\item
  While tracking, take a 10-sec exposure with the StarTracker.
\item
  Load the image into an image viewer.
\item
  Overlay the GAIA catalog.
\item
  Select a star brighter than XXX mag (bright enough for the DIMM).
\item
  Calculate the pixel offset between the StarTracker and the DIMM.
\item
  Transform the offset into AZ and EL offsets.
\end{itemize}

}
\hdashrule[0.5ex]{\textwidth}{1pt}{3mm}
  Expected Result \\
{\footnotesize
\begin{itemize}
\tightlist
\item
  An image was successfully taken with the StarTracker and is of
  sufficient quality.
\item
  AZ and EL offsets are available.
\end{itemize}

}

\begin{tabular}{p{2cm}}
\toprule
Step 963  \\ \hline
\end{tabular}
 Description \\
{\footnotesize
\textbf{Find DIMM Object and DIMM Offset}\\

\begin{itemize}
\tightlist
\item
  While tracking, take a 10-sec exposure with the StarTracker.
\item
  Load the image into an image viewer.
\item
  Overlay the GAIA catalog.
\item
  Select a star brighter than XXX mag (bright enough for the DIMM).
\item
  Calculate the pixel offset between the StarTracker and the DIMM.
\item
  Transform the offset into AZ and EL offsets.
\end{itemize}

}
\hdashrule[0.5ex]{\textwidth}{1pt}{3mm}
  Expected Result \\
{\footnotesize
\begin{itemize}
\tightlist
\item
  An image was successfully taken with the StarTracker and is of
  sufficient quality.
\item
  AZ and EL offsets are available.
\end{itemize}

}

\begin{tabular}{p{2cm}}
\toprule
Step 964  \\ \hline
\end{tabular}
 Description \\
{\footnotesize
\textbf{Find DIMM Object and DIMM Offset}\\

\begin{itemize}
\tightlist
\item
  While tracking, take a 10-sec exposure with the StarTracker.
\item
  Load the image into an image viewer.
\item
  Overlay the GAIA catalog.
\item
  Select a star brighter than XXX mag (bright enough for the DIMM).
\item
  Calculate the pixel offset between the StarTracker and the DIMM.
\item
  Transform the offset into AZ and EL offsets.
\end{itemize}

}
\hdashrule[0.5ex]{\textwidth}{1pt}{3mm}
  Expected Result \\
{\footnotesize
\begin{itemize}
\tightlist
\item
  An image was successfully taken with the StarTracker and is of
  sufficient quality.
\item
  AZ and EL offsets are available.
\end{itemize}

}

\begin{tabular}{p{2cm}}
\toprule
Step 965  \\ \hline
\end{tabular}
 Description \\
{\footnotesize
\textbf{Find DIMM Object and DIMM Offset}\\

\begin{itemize}
\tightlist
\item
  While tracking, take a 10-sec exposure with the StarTracker.
\item
  Load the image into an image viewer.
\item
  Overlay the GAIA catalog.
\item
  Select a star brighter than XXX mag (bright enough for the DIMM).
\item
  Calculate the pixel offset between the StarTracker and the DIMM.
\item
  Transform the offset into AZ and EL offsets.
\end{itemize}

}
\hdashrule[0.5ex]{\textwidth}{1pt}{3mm}
  Expected Result \\
{\footnotesize
\begin{itemize}
\tightlist
\item
  An image was successfully taken with the StarTracker and is of
  sufficient quality.
\item
  AZ and EL offsets are available.
\end{itemize}

}

\begin{tabular}{p{2cm}}
\toprule
Step 966  \\ \hline
\end{tabular}
 Description \\
{\footnotesize
\textbf{Find DIMM Object and DIMM Offset}\\

\begin{itemize}
\tightlist
\item
  While tracking, take a 10-sec exposure with the StarTracker.
\item
  Load the image into an image viewer.
\item
  Overlay the GAIA catalog.
\item
  Select a star brighter than XXX mag (bright enough for the DIMM).
\item
  Calculate the pixel offset between the StarTracker and the DIMM.
\item
  Transform the offset into AZ and EL offsets.
\end{itemize}

}
\hdashrule[0.5ex]{\textwidth}{1pt}{3mm}
  Expected Result \\
{\footnotesize
\begin{itemize}
\tightlist
\item
  An image was successfully taken with the StarTracker and is of
  sufficient quality.
\item
  AZ and EL offsets are available.
\end{itemize}

}

\begin{tabular}{p{2cm}}
\toprule
Step 967  \\ \hline
\end{tabular}
 Description \\
{\footnotesize
\textbf{Find DIMM Object and DIMM Offset}\\

\begin{itemize}
\tightlist
\item
  While tracking, take a 10-sec exposure with the StarTracker.
\item
  Load the image into an image viewer.
\item
  Overlay the GAIA catalog.
\item
  Select a star brighter than XXX mag (bright enough for the DIMM).
\item
  Calculate the pixel offset between the StarTracker and the DIMM.
\item
  Transform the offset into AZ and EL offsets.
\end{itemize}

}
\hdashrule[0.5ex]{\textwidth}{1pt}{3mm}
  Expected Result \\
{\footnotesize
\begin{itemize}
\tightlist
\item
  An image was successfully taken with the StarTracker and is of
  sufficient quality.
\item
  AZ and EL offsets are available.
\end{itemize}

}

\begin{tabular}{p{2cm}}
\toprule
Step 968  \\ \hline
\end{tabular}
 Description \\
{\footnotesize
\textbf{Find DIMM Object and DIMM Offset}\\

\begin{itemize}
\tightlist
\item
  While tracking, take a 10-sec exposure with the StarTracker.
\item
  Load the image into an image viewer.
\item
  Overlay the GAIA catalog.
\item
  Select a star brighter than XXX mag (bright enough for the DIMM).
\item
  Calculate the pixel offset between the StarTracker and the DIMM.
\item
  Transform the offset into AZ and EL offsets.
\end{itemize}

}
\hdashrule[0.5ex]{\textwidth}{1pt}{3mm}
  Expected Result \\
{\footnotesize
\begin{itemize}
\tightlist
\item
  An image was successfully taken with the StarTracker and is of
  sufficient quality.
\item
  AZ and EL offsets are available.
\end{itemize}

}

\begin{tabular}{p{2cm}}
\toprule
Step 969  \\ \hline
\end{tabular}
 Description \\
{\footnotesize
\textbf{Find DIMM Object and DIMM Offset}\\

\begin{itemize}
\tightlist
\item
  While tracking, take a 10-sec exposure with the StarTracker.
\item
  Load the image into an image viewer.
\item
  Overlay the GAIA catalog.
\item
  Select a star brighter than XXX mag (bright enough for the DIMM).
\item
  Calculate the pixel offset between the StarTracker and the DIMM.
\item
  Transform the offset into AZ and EL offsets.
\end{itemize}

}
\hdashrule[0.5ex]{\textwidth}{1pt}{3mm}
  Expected Result \\
{\footnotesize
\begin{itemize}
\tightlist
\item
  An image was successfully taken with the StarTracker and is of
  sufficient quality.
\item
  AZ and EL offsets are available.
\end{itemize}

}

\begin{tabular}{p{2cm}}
\toprule
Step 970  \\ \hline
\end{tabular}
 Description \\
{\footnotesize
\textbf{Find DIMM Object and DIMM Offset}\\

\begin{itemize}
\tightlist
\item
  While tracking, take a 10-sec exposure with the StarTracker.
\item
  Load the image into an image viewer.
\item
  Overlay the GAIA catalog.
\item
  Select a star brighter than XXX mag (bright enough for the DIMM).
\item
  Calculate the pixel offset between the StarTracker and the DIMM.
\item
  Transform the offset into AZ and EL offsets.
\end{itemize}

}
\hdashrule[0.5ex]{\textwidth}{1pt}{3mm}
  Expected Result \\
{\footnotesize
\begin{itemize}
\tightlist
\item
  An image was successfully taken with the StarTracker and is of
  sufficient quality.
\item
  AZ and EL offsets are available.
\end{itemize}

}

\begin{tabular}{p{2cm}}
\toprule
Step 971  \\ \hline
\end{tabular}
 Description \\
{\footnotesize
\textbf{Find DIMM Object and DIMM Offset}\\

\begin{itemize}
\tightlist
\item
  While tracking, take a 10-sec exposure with the StarTracker.
\item
  Load the image into an image viewer.
\item
  Overlay the GAIA catalog.
\item
  Select a star brighter than XXX mag (bright enough for the DIMM).
\item
  Calculate the pixel offset between the StarTracker and the DIMM.
\item
  Transform the offset into AZ and EL offsets.
\end{itemize}

}
\hdashrule[0.5ex]{\textwidth}{1pt}{3mm}
  Expected Result \\
{\footnotesize
\begin{itemize}
\tightlist
\item
  An image was successfully taken with the StarTracker and is of
  sufficient quality.
\item
  AZ and EL offsets are available.
\end{itemize}

}

\begin{tabular}{p{2cm}}
\toprule
Step 972  \\ \hline
\end{tabular}
 Description \\
{\footnotesize
\textbf{Find DIMM Object and DIMM Offset}\\

\begin{itemize}
\tightlist
\item
  While tracking, take a 10-sec exposure with the StarTracker.
\item
  Load the image into an image viewer.
\item
  Overlay the GAIA catalog.
\item
  Select a star brighter than XXX mag (bright enough for the DIMM).
\item
  Calculate the pixel offset between the StarTracker and the DIMM.
\item
  Transform the offset into AZ and EL offsets.
\end{itemize}

}
\hdashrule[0.5ex]{\textwidth}{1pt}{3mm}
  Expected Result \\
{\footnotesize
\begin{itemize}
\tightlist
\item
  An image was successfully taken with the StarTracker and is of
  sufficient quality.
\item
  AZ and EL offsets are available.
\end{itemize}

}

\begin{tabular}{p{2cm}}
\toprule
Step 973  \\ \hline
\end{tabular}
 Description \\
{\footnotesize
\textbf{Find DIMM Object and DIMM Offset}\\

\begin{itemize}
\tightlist
\item
  While tracking, take a 10-sec exposure with the StarTracker.
\item
  Load the image into an image viewer.
\item
  Overlay the GAIA catalog.
\item
  Select a star brighter than XXX mag (bright enough for the DIMM).
\item
  Calculate the pixel offset between the StarTracker and the DIMM.
\item
  Transform the offset into AZ and EL offsets.
\end{itemize}

}
\hdashrule[0.5ex]{\textwidth}{1pt}{3mm}
  Expected Result \\
{\footnotesize
\begin{itemize}
\tightlist
\item
  An image was successfully taken with the StarTracker and is of
  sufficient quality.
\item
  AZ and EL offsets are available.
\end{itemize}

}

\begin{tabular}{p{2cm}}
\toprule
Step 974  \\ \hline
\end{tabular}
 Description \\
{\footnotesize
\textbf{Find DIMM Object and DIMM Offset}\\

\begin{itemize}
\tightlist
\item
  While tracking, take a 10-sec exposure with the StarTracker.
\item
  Load the image into an image viewer.
\item
  Overlay the GAIA catalog.
\item
  Select a star brighter than XXX mag (bright enough for the DIMM).
\item
  Calculate the pixel offset between the StarTracker and the DIMM.
\item
  Transform the offset into AZ and EL offsets.
\end{itemize}

}
\hdashrule[0.5ex]{\textwidth}{1pt}{3mm}
  Expected Result \\
{\footnotesize
\begin{itemize}
\tightlist
\item
  An image was successfully taken with the StarTracker and is of
  sufficient quality.
\item
  AZ and EL offsets are available.
\end{itemize}

}

\begin{tabular}{p{2cm}}
\toprule
Step 975  \\ \hline
\end{tabular}
 Description \\
{\footnotesize
\textbf{Find DIMM Object and DIMM Offset}\\

\begin{itemize}
\tightlist
\item
  While tracking, take a 10-sec exposure with the StarTracker.
\item
  Load the image into an image viewer.
\item
  Overlay the GAIA catalog.
\item
  Select a star brighter than XXX mag (bright enough for the DIMM).
\item
  Calculate the pixel offset between the StarTracker and the DIMM.
\item
  Transform the offset into AZ and EL offsets.
\end{itemize}

}
\hdashrule[0.5ex]{\textwidth}{1pt}{3mm}
  Expected Result \\
{\footnotesize
\begin{itemize}
\tightlist
\item
  An image was successfully taken with the StarTracker and is of
  sufficient quality.
\item
  AZ and EL offsets are available.
\end{itemize}

}

\begin{tabular}{p{2cm}}
\toprule
Step 976  \\ \hline
\end{tabular}
 Description \\
{\footnotesize
\textbf{Find DIMM Object and DIMM Offset}\\

\begin{itemize}
\tightlist
\item
  While tracking, take a 10-sec exposure with the StarTracker.
\item
  Load the image into an image viewer.
\item
  Overlay the GAIA catalog.
\item
  Select a star brighter than XXX mag (bright enough for the DIMM).
\item
  Calculate the pixel offset between the StarTracker and the DIMM.
\item
  Transform the offset into AZ and EL offsets.
\end{itemize}

}
\hdashrule[0.5ex]{\textwidth}{1pt}{3mm}
  Expected Result \\
{\footnotesize
\begin{itemize}
\tightlist
\item
  An image was successfully taken with the StarTracker and is of
  sufficient quality.
\item
  AZ and EL offsets are available.
\end{itemize}

}

\begin{tabular}{p{2cm}}
\toprule
Step 977  \\ \hline
\end{tabular}
 Description \\
{\footnotesize
\textbf{Find DIMM Object and DIMM Offset}\\

\begin{itemize}
\tightlist
\item
  While tracking, take a 10-sec exposure with the StarTracker.
\item
  Load the image into an image viewer.
\item
  Overlay the GAIA catalog.
\item
  Select a star brighter than XXX mag (bright enough for the DIMM).
\item
  Calculate the pixel offset between the StarTracker and the DIMM.
\item
  Transform the offset into AZ and EL offsets.
\end{itemize}

}
\hdashrule[0.5ex]{\textwidth}{1pt}{3mm}
  Expected Result \\
{\footnotesize
\begin{itemize}
\tightlist
\item
  An image was successfully taken with the StarTracker and is of
  sufficient quality.
\item
  AZ and EL offsets are available.
\end{itemize}

}

\begin{tabular}{p{2cm}}
\toprule
Step 978  \\ \hline
\end{tabular}
 Description \\
{\footnotesize
\textbf{Find DIMM Object and DIMM Offset}\\

\begin{itemize}
\tightlist
\item
  While tracking, take a 10-sec exposure with the StarTracker.
\item
  Load the image into an image viewer.
\item
  Overlay the GAIA catalog.
\item
  Select a star brighter than XXX mag (bright enough for the DIMM).
\item
  Calculate the pixel offset between the StarTracker and the DIMM.
\item
  Transform the offset into AZ and EL offsets.
\end{itemize}

}
\hdashrule[0.5ex]{\textwidth}{1pt}{3mm}
  Expected Result \\
{\footnotesize
\begin{itemize}
\tightlist
\item
  An image was successfully taken with the StarTracker and is of
  sufficient quality.
\item
  AZ and EL offsets are available.
\end{itemize}

}

\begin{tabular}{p{2cm}}
\toprule
Step 979  \\ \hline
\end{tabular}
 Description \\
{\footnotesize
\textbf{Find DIMM Object and DIMM Offset}\\

\begin{itemize}
\tightlist
\item
  While tracking, take a 10-sec exposure with the StarTracker.
\item
  Load the image into an image viewer.
\item
  Overlay the GAIA catalog.
\item
  Select a star brighter than XXX mag (bright enough for the DIMM).
\item
  Calculate the pixel offset between the StarTracker and the DIMM.
\item
  Transform the offset into AZ and EL offsets.
\end{itemize}

}
\hdashrule[0.5ex]{\textwidth}{1pt}{3mm}
  Expected Result \\
{\footnotesize
\begin{itemize}
\tightlist
\item
  An image was successfully taken with the StarTracker and is of
  sufficient quality.
\item
  AZ and EL offsets are available.
\end{itemize}

}

\begin{tabular}{p{2cm}}
\toprule
Step 980  \\ \hline
\end{tabular}
 Description \\
{\footnotesize
\textbf{Find DIMM Object and DIMM Offset}\\

\begin{itemize}
\tightlist
\item
  While tracking, take a 10-sec exposure with the StarTracker.
\item
  Load the image into an image viewer.
\item
  Overlay the GAIA catalog.
\item
  Select a star brighter than XXX mag (bright enough for the DIMM).
\item
  Calculate the pixel offset between the StarTracker and the DIMM.
\item
  Transform the offset into AZ and EL offsets.
\end{itemize}

}
\hdashrule[0.5ex]{\textwidth}{1pt}{3mm}
  Expected Result \\
{\footnotesize
\begin{itemize}
\tightlist
\item
  An image was successfully taken with the StarTracker and is of
  sufficient quality.
\item
  AZ and EL offsets are available.
\end{itemize}

}

\begin{tabular}{p{2cm}}
\toprule
Step 981  \\ \hline
\end{tabular}
 Description \\
{\footnotesize
\textbf{Find DIMM Object and DIMM Offset}\\

\begin{itemize}
\tightlist
\item
  While tracking, take a 10-sec exposure with the StarTracker.
\item
  Load the image into an image viewer.
\item
  Overlay the GAIA catalog.
\item
  Select a star brighter than XXX mag (bright enough for the DIMM).
\item
  Calculate the pixel offset between the StarTracker and the DIMM.
\item
  Transform the offset into AZ and EL offsets.
\end{itemize}

}
\hdashrule[0.5ex]{\textwidth}{1pt}{3mm}
  Expected Result \\
{\footnotesize
\begin{itemize}
\tightlist
\item
  An image was successfully taken with the StarTracker and is of
  sufficient quality.
\item
  AZ and EL offsets are available.
\end{itemize}

}

\begin{tabular}{p{2cm}}
\toprule
Step 982  \\ \hline
\end{tabular}
 Description \\
{\footnotesize
\textbf{Find DIMM Object and DIMM Offset}\\

\begin{itemize}
\tightlist
\item
  While tracking, take a 10-sec exposure with the StarTracker.
\item
  Load the image into an image viewer.
\item
  Overlay the GAIA catalog.
\item
  Select a star brighter than XXX mag (bright enough for the DIMM).
\item
  Calculate the pixel offset between the StarTracker and the DIMM.
\item
  Transform the offset into AZ and EL offsets.
\end{itemize}

}
\hdashrule[0.5ex]{\textwidth}{1pt}{3mm}
  Expected Result \\
{\footnotesize
\begin{itemize}
\tightlist
\item
  An image was successfully taken with the StarTracker and is of
  sufficient quality.
\item
  AZ and EL offsets are available.
\end{itemize}

}

\begin{tabular}{p{2cm}}
\toprule
Step 983  \\ \hline
\end{tabular}
 Description \\
{\footnotesize
\textbf{Find DIMM Object and DIMM Offset}\\

\begin{itemize}
\tightlist
\item
  While tracking, take a 10-sec exposure with the StarTracker.
\item
  Load the image into an image viewer.
\item
  Overlay the GAIA catalog.
\item
  Select a star brighter than XXX mag (bright enough for the DIMM).
\item
  Calculate the pixel offset between the StarTracker and the DIMM.
\item
  Transform the offset into AZ and EL offsets.
\end{itemize}

}
\hdashrule[0.5ex]{\textwidth}{1pt}{3mm}
  Expected Result \\
{\footnotesize
\begin{itemize}
\tightlist
\item
  An image was successfully taken with the StarTracker and is of
  sufficient quality.
\item
  AZ and EL offsets are available.
\end{itemize}

}

\begin{tabular}{p{2cm}}
\toprule
Step 984  \\ \hline
\end{tabular}
 Description \\
{\footnotesize
\textbf{Move TMA to the DIMM position and \textbf{Take DIMM images}}\\

\begin{itemize}
\tightlist
\item
  Command the TMA to the DIMM position by applying the offsets
\item
  While tracking, take DIMM images with XXXs exposure time and inspect
  the quality.
\end{itemize}

}
\hdashrule[0.5ex]{\textwidth}{1pt}{3mm}
  Expected Result \\
{\footnotesize
\begin{itemize}
\tightlist
\item
  TMA reaches the DIMM position.
\item
  DIMM imaging quality is sufficient.
\end{itemize}

}

\begin{tabular}{p{2cm}}
\toprule
Step 985  \\ \hline
\end{tabular}
 Description \\
{\footnotesize
\textbf{Move TMA to the DIMM position and \textbf{Take DIMM images}}\\

\begin{itemize}
\tightlist
\item
  Command the TMA to the DIMM position by applying the offsets
\item
  While tracking, take DIMM images with XXXs exposure time and inspect
  the quality.
\end{itemize}

}
\hdashrule[0.5ex]{\textwidth}{1pt}{3mm}
  Expected Result \\
{\footnotesize
\begin{itemize}
\tightlist
\item
  TMA reaches the DIMM position.
\item
  DIMM imaging quality is sufficient.
\end{itemize}

}

\begin{tabular}{p{2cm}}
\toprule
Step 986  \\ \hline
\end{tabular}
 Description \\
{\footnotesize
\textbf{Move TMA to the DIMM position and \textbf{Take DIMM images}}\\

\begin{itemize}
\tightlist
\item
  Command the TMA to the DIMM position by applying the offsets
\item
  While tracking, take DIMM images with XXXs exposure time and inspect
  the quality.
\end{itemize}

}
\hdashrule[0.5ex]{\textwidth}{1pt}{3mm}
  Expected Result \\
{\footnotesize
\begin{itemize}
\tightlist
\item
  TMA reaches the DIMM position.
\item
  DIMM imaging quality is sufficient.
\end{itemize}

}

\begin{tabular}{p{2cm}}
\toprule
Step 987  \\ \hline
\end{tabular}
 Description \\
{\footnotesize
\textbf{Move TMA to the DIMM position and \textbf{Take DIMM images}}\\

\begin{itemize}
\tightlist
\item
  Command the TMA to the DIMM position by applying the offsets
\item
  While tracking, take DIMM images with XXXs exposure time and inspect
  the quality.
\end{itemize}

}
\hdashrule[0.5ex]{\textwidth}{1pt}{3mm}
  Expected Result \\
{\footnotesize
\begin{itemize}
\tightlist
\item
  TMA reaches the DIMM position.
\item
  DIMM imaging quality is sufficient.
\end{itemize}

}

\begin{tabular}{p{2cm}}
\toprule
Step 988  \\ \hline
\end{tabular}
 Description \\
{\footnotesize
\textbf{Move TMA to the DIMM position and \textbf{Take DIMM images}}\\

\begin{itemize}
\tightlist
\item
  Command the TMA to the DIMM position by applying the offsets
\item
  While tracking, take DIMM images with XXXs exposure time and inspect
  the quality.
\end{itemize}

}
\hdashrule[0.5ex]{\textwidth}{1pt}{3mm}
  Expected Result \\
{\footnotesize
\begin{itemize}
\tightlist
\item
  TMA reaches the DIMM position.
\item
  DIMM imaging quality is sufficient.
\end{itemize}

}

\begin{tabular}{p{2cm}}
\toprule
Step 989  \\ \hline
\end{tabular}
 Description \\
{\footnotesize
\textbf{Move TMA to the DIMM position and \textbf{Take DIMM images}}\\

\begin{itemize}
\tightlist
\item
  Command the TMA to the DIMM position by applying the offsets
\item
  While tracking, take DIMM images with XXXs exposure time and inspect
  the quality.
\end{itemize}

}
\hdashrule[0.5ex]{\textwidth}{1pt}{3mm}
  Expected Result \\
{\footnotesize
\begin{itemize}
\tightlist
\item
  TMA reaches the DIMM position.
\item
  DIMM imaging quality is sufficient.
\end{itemize}

}

\begin{tabular}{p{2cm}}
\toprule
Step 990  \\ \hline
\end{tabular}
 Description \\
{\footnotesize
\textbf{Move TMA to the DIMM position and \textbf{Take DIMM images}}\\

\begin{itemize}
\tightlist
\item
  Command the TMA to the DIMM position by applying the offsets
\item
  While tracking, take DIMM images with XXXs exposure time and inspect
  the quality.
\end{itemize}

}
\hdashrule[0.5ex]{\textwidth}{1pt}{3mm}
  Expected Result \\
{\footnotesize
\begin{itemize}
\tightlist
\item
  TMA reaches the DIMM position.
\item
  DIMM imaging quality is sufficient.
\end{itemize}

}

\begin{tabular}{p{2cm}}
\toprule
Step 991  \\ \hline
\end{tabular}
 Description \\
{\footnotesize
\textbf{Move TMA to the DIMM position and \textbf{Take DIMM images}}\\

\begin{itemize}
\tightlist
\item
  Command the TMA to the DIMM position by applying the offsets
\item
  While tracking, take DIMM images with XXXs exposure time and inspect
  the quality.
\end{itemize}

}
\hdashrule[0.5ex]{\textwidth}{1pt}{3mm}
  Expected Result \\
{\footnotesize
\begin{itemize}
\tightlist
\item
  TMA reaches the DIMM position.
\item
  DIMM imaging quality is sufficient.
\end{itemize}

}

\begin{tabular}{p{2cm}}
\toprule
Step 992  \\ \hline
\end{tabular}
 Description \\
{\footnotesize
\textbf{Move TMA to the DIMM position and \textbf{Take DIMM images}}\\

\begin{itemize}
\tightlist
\item
  Command the TMA to the DIMM position by applying the offsets
\item
  While tracking, take DIMM images with XXXs exposure time and inspect
  the quality.
\end{itemize}

}
\hdashrule[0.5ex]{\textwidth}{1pt}{3mm}
  Expected Result \\
{\footnotesize
\begin{itemize}
\tightlist
\item
  TMA reaches the DIMM position.
\item
  DIMM imaging quality is sufficient.
\end{itemize}

}

\begin{tabular}{p{2cm}}
\toprule
Step 993  \\ \hline
\end{tabular}
 Description \\
{\footnotesize
\textbf{Move TMA to the DIMM position and \textbf{Take DIMM images}}\\

\begin{itemize}
\tightlist
\item
  Command the TMA to the DIMM position by applying the offsets
\item
  While tracking, take DIMM images with XXXs exposure time and inspect
  the quality.
\end{itemize}

}
\hdashrule[0.5ex]{\textwidth}{1pt}{3mm}
  Expected Result \\
{\footnotesize
\begin{itemize}
\tightlist
\item
  TMA reaches the DIMM position.
\item
  DIMM imaging quality is sufficient.
\end{itemize}

}

\begin{tabular}{p{2cm}}
\toprule
Step 994  \\ \hline
\end{tabular}
 Description \\
{\footnotesize
\textbf{Move TMA to the DIMM position and \textbf{Take DIMM images}}\\

\begin{itemize}
\tightlist
\item
  Command the TMA to the DIMM position by applying the offsets
\item
  While tracking, take DIMM images with XXXs exposure time and inspect
  the quality.
\end{itemize}

}
\hdashrule[0.5ex]{\textwidth}{1pt}{3mm}
  Expected Result \\
{\footnotesize
\begin{itemize}
\tightlist
\item
  TMA reaches the DIMM position.
\item
  DIMM imaging quality is sufficient.
\end{itemize}

}

\begin{tabular}{p{2cm}}
\toprule
Step 995  \\ \hline
\end{tabular}
 Description \\
{\footnotesize
\textbf{Move TMA to the DIMM position and \textbf{Take DIMM images}}\\

\begin{itemize}
\tightlist
\item
  Command the TMA to the DIMM position by applying the offsets
\item
  While tracking, take DIMM images with XXXs exposure time and inspect
  the quality.
\end{itemize}

}
\hdashrule[0.5ex]{\textwidth}{1pt}{3mm}
  Expected Result \\
{\footnotesize
\begin{itemize}
\tightlist
\item
  TMA reaches the DIMM position.
\item
  DIMM imaging quality is sufficient.
\end{itemize}

}

\begin{tabular}{p{2cm}}
\toprule
Step 996  \\ \hline
\end{tabular}
 Description \\
{\footnotesize
\textbf{Move TMA to the DIMM position and \textbf{Take DIMM images}}\\

\begin{itemize}
\tightlist
\item
  Command the TMA to the DIMM position by applying the offsets
\item
  While tracking, take DIMM images with XXXs exposure time and inspect
  the quality.
\end{itemize}

}
\hdashrule[0.5ex]{\textwidth}{1pt}{3mm}
  Expected Result \\
{\footnotesize
\begin{itemize}
\tightlist
\item
  TMA reaches the DIMM position.
\item
  DIMM imaging quality is sufficient.
\end{itemize}

}

\begin{tabular}{p{2cm}}
\toprule
Step 997  \\ \hline
\end{tabular}
 Description \\
{\footnotesize
\textbf{Move TMA to the DIMM position and \textbf{Take DIMM images}}\\

\begin{itemize}
\tightlist
\item
  Command the TMA to the DIMM position by applying the offsets
\item
  While tracking, take DIMM images with XXXs exposure time and inspect
  the quality.
\end{itemize}

}
\hdashrule[0.5ex]{\textwidth}{1pt}{3mm}
  Expected Result \\
{\footnotesize
\begin{itemize}
\tightlist
\item
  TMA reaches the DIMM position.
\item
  DIMM imaging quality is sufficient.
\end{itemize}

}

\begin{tabular}{p{2cm}}
\toprule
Step 998  \\ \hline
\end{tabular}
 Description \\
{\footnotesize
\textbf{Move TMA to the DIMM position and \textbf{Take DIMM images}}\\

\begin{itemize}
\tightlist
\item
  Command the TMA to the DIMM position by applying the offsets
\item
  While tracking, take DIMM images with XXXs exposure time and inspect
  the quality.
\end{itemize}

}
\hdashrule[0.5ex]{\textwidth}{1pt}{3mm}
  Expected Result \\
{\footnotesize
\begin{itemize}
\tightlist
\item
  TMA reaches the DIMM position.
\item
  DIMM imaging quality is sufficient.
\end{itemize}

}

\begin{tabular}{p{2cm}}
\toprule
Step 999  \\ \hline
\end{tabular}
 Description \\
{\footnotesize
\textbf{Move TMA to the DIMM position and \textbf{Take DIMM images}}\\

\begin{itemize}
\tightlist
\item
  Command the TMA to the DIMM position by applying the offsets
\item
  While tracking, take DIMM images with XXXs exposure time and inspect
  the quality.
\end{itemize}

}
\hdashrule[0.5ex]{\textwidth}{1pt}{3mm}
  Expected Result \\
{\footnotesize
\begin{itemize}
\tightlist
\item
  TMA reaches the DIMM position.
\item
  DIMM imaging quality is sufficient.
\end{itemize}

}

\begin{tabular}{p{2cm}}
\toprule
Step 1000  \\ \hline
\end{tabular}
 Description \\
{\footnotesize
\textbf{Move TMA to the DIMM position and \textbf{Take DIMM images}}\\

\begin{itemize}
\tightlist
\item
  Command the TMA to the DIMM position by applying the offsets
\item
  While tracking, take DIMM images with XXXs exposure time and inspect
  the quality.
\end{itemize}

}
\hdashrule[0.5ex]{\textwidth}{1pt}{3mm}
  Expected Result \\
{\footnotesize
\begin{itemize}
\tightlist
\item
  TMA reaches the DIMM position.
\item
  DIMM imaging quality is sufficient.
\end{itemize}

}

\begin{tabular}{p{2cm}}
\toprule
Step 1001  \\ \hline
\end{tabular}
 Description \\
{\footnotesize
\textbf{Move TMA to the DIMM position and \textbf{Take DIMM images}}\\

\begin{itemize}
\tightlist
\item
  Command the TMA to the DIMM position by applying the offsets
\item
  While tracking, take DIMM images with XXXs exposure time and inspect
  the quality.
\end{itemize}

}
\hdashrule[0.5ex]{\textwidth}{1pt}{3mm}
  Expected Result \\
{\footnotesize
\begin{itemize}
\tightlist
\item
  TMA reaches the DIMM position.
\item
  DIMM imaging quality is sufficient.
\end{itemize}

}

\begin{tabular}{p{2cm}}
\toprule
Step 1002  \\ \hline
\end{tabular}
 Description \\
{\footnotesize
\textbf{Move TMA to the DIMM position and \textbf{Take DIMM images}}\\

\begin{itemize}
\tightlist
\item
  Command the TMA to the DIMM position by applying the offsets
\item
  While tracking, take DIMM images with XXXs exposure time and inspect
  the quality.
\end{itemize}

}
\hdashrule[0.5ex]{\textwidth}{1pt}{3mm}
  Expected Result \\
{\footnotesize
\begin{itemize}
\tightlist
\item
  TMA reaches the DIMM position.
\item
  DIMM imaging quality is sufficient.
\end{itemize}

}

\begin{tabular}{p{2cm}}
\toprule
Step 1003  \\ \hline
\end{tabular}
 Description \\
{\footnotesize
\textbf{Move TMA to the DIMM position and \textbf{Take DIMM images}}\\

\begin{itemize}
\tightlist
\item
  Command the TMA to the DIMM position by applying the offsets
\item
  While tracking, take DIMM images with XXXs exposure time and inspect
  the quality.
\end{itemize}

}
\hdashrule[0.5ex]{\textwidth}{1pt}{3mm}
  Expected Result \\
{\footnotesize
\begin{itemize}
\tightlist
\item
  TMA reaches the DIMM position.
\item
  DIMM imaging quality is sufficient.
\end{itemize}

}

\begin{tabular}{p{2cm}}
\toprule
Step 1004  \\ \hline
\end{tabular}
 Description \\
{\footnotesize
\textbf{Move TMA to the DIMM position and \textbf{Take DIMM images}}\\

\begin{itemize}
\tightlist
\item
  Command the TMA to the DIMM position by applying the offsets
\item
  While tracking, take DIMM images with XXXs exposure time and inspect
  the quality.
\end{itemize}

}
\hdashrule[0.5ex]{\textwidth}{1pt}{3mm}
  Expected Result \\
{\footnotesize
\begin{itemize}
\tightlist
\item
  TMA reaches the DIMM position.
\item
  DIMM imaging quality is sufficient.
\end{itemize}

}

\begin{tabular}{p{2cm}}
\toprule
Step 1005  \\ \hline
\end{tabular}
 Description \\
{\footnotesize
\textbf{Move TMA to the DIMM position and \textbf{Take DIMM images}}\\

\begin{itemize}
\tightlist
\item
  Command the TMA to the DIMM position by applying the offsets
\item
  While tracking, take DIMM images with XXXs exposure time and inspect
  the quality.
\end{itemize}

}
\hdashrule[0.5ex]{\textwidth}{1pt}{3mm}
  Expected Result \\
{\footnotesize
\begin{itemize}
\tightlist
\item
  TMA reaches the DIMM position.
\item
  DIMM imaging quality is sufficient.
\end{itemize}

}

\begin{tabular}{p{2cm}}
\toprule
Step 1006  \\ \hline
\end{tabular}
 Description \\
{\footnotesize
\textbf{Move TMA to the DIMM position and \textbf{Take DIMM images}}\\

\begin{itemize}
\tightlist
\item
  Command the TMA to the DIMM position by applying the offsets
\item
  While tracking, take DIMM images with XXXs exposure time and inspect
  the quality.
\end{itemize}

}
\hdashrule[0.5ex]{\textwidth}{1pt}{3mm}
  Expected Result \\
{\footnotesize
\begin{itemize}
\tightlist
\item
  TMA reaches the DIMM position.
\item
  DIMM imaging quality is sufficient.
\end{itemize}

}

\begin{tabular}{p{2cm}}
\toprule
Step 1007  \\ \hline
\end{tabular}
 Description \\
{\footnotesize
\textbf{Move TMA to the DIMM position and \textbf{Take DIMM images}}\\

\begin{itemize}
\tightlist
\item
  Command the TMA to the DIMM position by applying the offsets
\item
  While tracking, take DIMM images with XXXs exposure time and inspect
  the quality.
\end{itemize}

}
\hdashrule[0.5ex]{\textwidth}{1pt}{3mm}
  Expected Result \\
{\footnotesize
\begin{itemize}
\tightlist
\item
  TMA reaches the DIMM position.
\item
  DIMM imaging quality is sufficient.
\end{itemize}

}

\begin{tabular}{p{2cm}}
\toprule
Step 1008  \\ \hline
\end{tabular}
 Description \\
{\footnotesize
\textbf{Move TMA to the DIMM position and \textbf{Take DIMM images}}\\

\begin{itemize}
\tightlist
\item
  Command the TMA to the DIMM position by applying the offsets
\item
  While tracking, take DIMM images with XXXs exposure time and inspect
  the quality.
\end{itemize}

}
\hdashrule[0.5ex]{\textwidth}{1pt}{3mm}
  Expected Result \\
{\footnotesize
\begin{itemize}
\tightlist
\item
  TMA reaches the DIMM position.
\item
  DIMM imaging quality is sufficient.
\end{itemize}

}

\begin{tabular}{p{2cm}}
\toprule
Step 1009  \\ \hline
\end{tabular}
 Description \\
{\footnotesize
\textbf{Move TMA to the DIMM position and \textbf{Take DIMM images}}\\

\begin{itemize}
\tightlist
\item
  Command the TMA to the DIMM position by applying the offsets
\item
  While tracking, take DIMM images with XXXs exposure time and inspect
  the quality.
\end{itemize}

}
\hdashrule[0.5ex]{\textwidth}{1pt}{3mm}
  Expected Result \\
{\footnotesize
\begin{itemize}
\tightlist
\item
  TMA reaches the DIMM position.
\item
  DIMM imaging quality is sufficient.
\end{itemize}

}

\begin{tabular}{p{2cm}}
\toprule
Step 1010  \\ \hline
\end{tabular}
 Description \\
{\footnotesize
\textbf{Move TMA to the DIMM position and \textbf{Take DIMM images}}\\

\begin{itemize}
\tightlist
\item
  Command the TMA to the DIMM position by applying the offsets
\item
  While tracking, take DIMM images with XXXs exposure time and inspect
  the quality.
\end{itemize}

}
\hdashrule[0.5ex]{\textwidth}{1pt}{3mm}
  Expected Result \\
{\footnotesize
\begin{itemize}
\tightlist
\item
  TMA reaches the DIMM position.
\item
  DIMM imaging quality is sufficient.
\end{itemize}

}

\begin{tabular}{p{2cm}}
\toprule
Step 1011  \\ \hline
\end{tabular}
 Description \\
{\footnotesize
\textbf{Move TMA to the DIMM position and \textbf{Take DIMM images}}\\

\begin{itemize}
\tightlist
\item
  Command the TMA to the DIMM position by applying the offsets
\item
  While tracking, take DIMM images with XXXs exposure time and inspect
  the quality.
\end{itemize}

}
\hdashrule[0.5ex]{\textwidth}{1pt}{3mm}
  Expected Result \\
{\footnotesize
\begin{itemize}
\tightlist
\item
  TMA reaches the DIMM position.
\item
  DIMM imaging quality is sufficient.
\end{itemize}

}

\begin{tabular}{p{2cm}}
\toprule
Step 1012  \\ \hline
\end{tabular}
 Description \\
{\footnotesize
\textbf{Move TMA to the DIMM position and \textbf{Take DIMM images}}\\

\begin{itemize}
\tightlist
\item
  Command the TMA to the DIMM position by applying the offsets
\item
  While tracking, take DIMM images with XXXs exposure time and inspect
  the quality.
\end{itemize}

}
\hdashrule[0.5ex]{\textwidth}{1pt}{3mm}
  Expected Result \\
{\footnotesize
\begin{itemize}
\tightlist
\item
  TMA reaches the DIMM position.
\item
  DIMM imaging quality is sufficient.
\end{itemize}

}

\begin{tabular}{p{2cm}}
\toprule
Step 1013  \\ \hline
\end{tabular}
 Description \\
{\footnotesize
\textbf{Move TMA to the DIMM position and \textbf{Take DIMM images}}\\

\begin{itemize}
\tightlist
\item
  Command the TMA to the DIMM position by applying the offsets
\item
  While tracking, take DIMM images with XXXs exposure time and inspect
  the quality.
\end{itemize}

}
\hdashrule[0.5ex]{\textwidth}{1pt}{3mm}
  Expected Result \\
{\footnotesize
\begin{itemize}
\tightlist
\item
  TMA reaches the DIMM position.
\item
  DIMM imaging quality is sufficient.
\end{itemize}

}

\begin{tabular}{p{2cm}}
\toprule
Step 1014  \\ \hline
\end{tabular}
 Description \\
{\footnotesize
\textbf{Move TMA to the DIMM position and \textbf{Take DIMM images}}\\

\begin{itemize}
\tightlist
\item
  Command the TMA to the DIMM position by applying the offsets
\item
  While tracking, take DIMM images with XXXs exposure time and inspect
  the quality.
\end{itemize}

}
\hdashrule[0.5ex]{\textwidth}{1pt}{3mm}
  Expected Result \\
{\footnotesize
\begin{itemize}
\tightlist
\item
  TMA reaches the DIMM position.
\item
  DIMM imaging quality is sufficient.
\end{itemize}

}

\begin{tabular}{p{2cm}}
\toprule
Step 1015  \\ \hline
\end{tabular}
 Description \\
{\footnotesize
\textbf{Move TMA to the DIMM position and \textbf{Take DIMM images}}\\

\begin{itemize}
\tightlist
\item
  Command the TMA to the DIMM position by applying the offsets
\item
  While tracking, take DIMM images with XXXs exposure time and inspect
  the quality.
\end{itemize}

}
\hdashrule[0.5ex]{\textwidth}{1pt}{3mm}
  Expected Result \\
{\footnotesize
\begin{itemize}
\tightlist
\item
  TMA reaches the DIMM position.
\item
  DIMM imaging quality is sufficient.
\end{itemize}

}

\begin{tabular}{p{2cm}}
\toprule
Step 1016  \\ \hline
\end{tabular}
 Description \\
{\footnotesize
\textbf{Move TMA to the DIMM position and \textbf{Take DIMM images}}\\

\begin{itemize}
\tightlist
\item
  Command the TMA to the DIMM position by applying the offsets
\item
  While tracking, take DIMM images with XXXs exposure time and inspect
  the quality.
\end{itemize}

}
\hdashrule[0.5ex]{\textwidth}{1pt}{3mm}
  Expected Result \\
{\footnotesize
\begin{itemize}
\tightlist
\item
  TMA reaches the DIMM position.
\item
  DIMM imaging quality is sufficient.
\end{itemize}

}

\begin{tabular}{p{2cm}}
\toprule
Step 1017  \\ \hline
\end{tabular}
 Description \\
{\footnotesize
\textbf{Move TMA to the DIMM position and \textbf{Take DIMM images}}\\

\begin{itemize}
\tightlist
\item
  Command the TMA to the DIMM position by applying the offsets
\item
  While tracking, take DIMM images with XXXs exposure time and inspect
  the quality.
\end{itemize}

}
\hdashrule[0.5ex]{\textwidth}{1pt}{3mm}
  Expected Result \\
{\footnotesize
\begin{itemize}
\tightlist
\item
  TMA reaches the DIMM position.
\item
  DIMM imaging quality is sufficient.
\end{itemize}

}

\begin{tabular}{p{2cm}}
\toprule
Step 1018  \\ \hline
\end{tabular}
 Description \\
{\footnotesize
\textbf{Move TMA to the DIMM position and \textbf{Take DIMM images}}\\

\begin{itemize}
\tightlist
\item
  Command the TMA to the DIMM position by applying the offsets
\item
  While tracking, take DIMM images with XXXs exposure time and inspect
  the quality.
\end{itemize}

}
\hdashrule[0.5ex]{\textwidth}{1pt}{3mm}
  Expected Result \\
{\footnotesize
\begin{itemize}
\tightlist
\item
  TMA reaches the DIMM position.
\item
  DIMM imaging quality is sufficient.
\end{itemize}

}

\begin{tabular}{p{2cm}}
\toprule
Step 1019  \\ \hline
\end{tabular}
 Description \\
{\footnotesize
\textbf{Move TMA to the DIMM position and \textbf{Take DIMM images}}\\

\begin{itemize}
\tightlist
\item
  Command the TMA to the DIMM position by applying the offsets
\item
  While tracking, take DIMM images with XXXs exposure time and inspect
  the quality.
\end{itemize}

}
\hdashrule[0.5ex]{\textwidth}{1pt}{3mm}
  Expected Result \\
{\footnotesize
\begin{itemize}
\tightlist
\item
  TMA reaches the DIMM position.
\item
  DIMM imaging quality is sufficient.
\end{itemize}

}

\begin{tabular}{p{2cm}}
\toprule
Step 1020  \\ \hline
\end{tabular}
 Description \\
{\footnotesize
\textbf{Move TMA to the DIMM position and \textbf{Take DIMM images}}\\

\begin{itemize}
\tightlist
\item
  Command the TMA to the DIMM position by applying the offsets
\item
  While tracking, take DIMM images with XXXs exposure time and inspect
  the quality.
\end{itemize}

}
\hdashrule[0.5ex]{\textwidth}{1pt}{3mm}
  Expected Result \\
{\footnotesize
\begin{itemize}
\tightlist
\item
  TMA reaches the DIMM position.
\item
  DIMM imaging quality is sufficient.
\end{itemize}

}

\begin{tabular}{p{2cm}}
\toprule
Step 1021  \\ \hline
\end{tabular}
 Description \\
{\footnotesize
\textbf{Move TMA to the DIMM position and \textbf{Take DIMM images}}\\

\begin{itemize}
\tightlist
\item
  Command the TMA to the DIMM position by applying the offsets
\item
  While tracking, take DIMM images with XXXs exposure time and inspect
  the quality.
\end{itemize}

}
\hdashrule[0.5ex]{\textwidth}{1pt}{3mm}
  Expected Result \\
{\footnotesize
\begin{itemize}
\tightlist
\item
  TMA reaches the DIMM position.
\item
  DIMM imaging quality is sufficient.
\end{itemize}

}

\begin{tabular}{p{2cm}}
\toprule
Step 1022  \\ \hline
\end{tabular}
 Description \\
{\footnotesize
\textbf{Move TMA to the DIMM position and \textbf{Take DIMM images}}\\

\begin{itemize}
\tightlist
\item
  Command the TMA to the DIMM position by applying the offsets
\item
  While tracking, take DIMM images with XXXs exposure time and inspect
  the quality.
\end{itemize}

}
\hdashrule[0.5ex]{\textwidth}{1pt}{3mm}
  Expected Result \\
{\footnotesize
\begin{itemize}
\tightlist
\item
  TMA reaches the DIMM position.
\item
  DIMM imaging quality is sufficient.
\end{itemize}

}

\begin{tabular}{p{2cm}}
\toprule
Step 1023  \\ \hline
\end{tabular}
 Description \\
{\footnotesize
\textbf{Move TMA to the DIMM position and \textbf{Take DIMM images}}\\

\begin{itemize}
\tightlist
\item
  Command the TMA to the DIMM position by applying the offsets
\item
  While tracking, take DIMM images with XXXs exposure time and inspect
  the quality.
\end{itemize}

}
\hdashrule[0.5ex]{\textwidth}{1pt}{3mm}
  Expected Result \\
{\footnotesize
\begin{itemize}
\tightlist
\item
  TMA reaches the DIMM position.
\item
  DIMM imaging quality is sufficient.
\end{itemize}

}

\begin{tabular}{p{2cm}}
\toprule
Step 1024  \\ \hline
\end{tabular}
 Description \\
{\footnotesize
\textbf{Move TMA to the DIMM position and \textbf{Take DIMM images}}\\

\begin{itemize}
\tightlist
\item
  Command the TMA to the DIMM position by applying the offsets
\item
  While tracking, take DIMM images with XXXs exposure time and inspect
  the quality.
\end{itemize}

}
\hdashrule[0.5ex]{\textwidth}{1pt}{3mm}
  Expected Result \\
{\footnotesize
\begin{itemize}
\tightlist
\item
  TMA reaches the DIMM position.
\item
  DIMM imaging quality is sufficient.
\end{itemize}

}

\begin{tabular}{p{2cm}}
\toprule
Step 1025  \\ \hline
\end{tabular}
 Description \\
{\footnotesize
\textbf{Move TMA to the DIMM position and \textbf{Take DIMM images}}\\

\begin{itemize}
\tightlist
\item
  Command the TMA to the DIMM position by applying the offsets
\item
  While tracking, take DIMM images with XXXs exposure time and inspect
  the quality.
\end{itemize}

}
\hdashrule[0.5ex]{\textwidth}{1pt}{3mm}
  Expected Result \\
{\footnotesize
\begin{itemize}
\tightlist
\item
  TMA reaches the DIMM position.
\item
  DIMM imaging quality is sufficient.
\end{itemize}

}

\begin{tabular}{p{2cm}}
\toprule
Step 1026  \\ \hline
\end{tabular}
 Description \\
{\footnotesize
\textbf{Move TMA to the DIMM position and \textbf{Take DIMM images}}\\

\begin{itemize}
\tightlist
\item
  Command the TMA to the DIMM position by applying the offsets
\item
  While tracking, take DIMM images with XXXs exposure time and inspect
  the quality.
\end{itemize}

}
\hdashrule[0.5ex]{\textwidth}{1pt}{3mm}
  Expected Result \\
{\footnotesize
\begin{itemize}
\tightlist
\item
  TMA reaches the DIMM position.
\item
  DIMM imaging quality is sufficient.
\end{itemize}

}

\begin{tabular}{p{2cm}}
\toprule
Step 1027  \\ \hline
\end{tabular}
 Description \\
{\footnotesize
\textbf{Move TMA to the DIMM position and \textbf{Take DIMM images}}\\

\begin{itemize}
\tightlist
\item
  Command the TMA to the DIMM position by applying the offsets
\item
  While tracking, take DIMM images with XXXs exposure time and inspect
  the quality.
\end{itemize}

}
\hdashrule[0.5ex]{\textwidth}{1pt}{3mm}
  Expected Result \\
{\footnotesize
\begin{itemize}
\tightlist
\item
  TMA reaches the DIMM position.
\item
  DIMM imaging quality is sufficient.
\end{itemize}

}

\begin{tabular}{p{2cm}}
\toprule
Step 1028  \\ \hline
\end{tabular}
 Description \\
{\footnotesize
\textbf{Point the TMA to (Az, El)-pattern position}\\
Point the TMA to {Pointing 1}⁠ at {-270}⁠ , {15}⁠ .

}
\hdashrule[0.5ex]{\textwidth}{1pt}{3mm}
  Expected Result \\
{\footnotesize
The TMA starts moving.

}

\begin{tabular}{p{2cm}}
\toprule
Step 1029  \\ \hline
\end{tabular}
 Description \\
{\footnotesize
\textbf{Position the dome}\\
Command the Dome to {Pointing 3}⁠ to {\{Dome AZ\}}⁠

}
\hdashrule[0.5ex]{\textwidth}{1pt}{3mm}
  Expected Result \\
{\footnotesize
The Dome starts moving.

}

\begin{tabular}{p{2cm}}
\toprule
Step 1030  \\ \hline
\end{tabular}
 Description \\
{\footnotesize
\textbf{Point the TMA to (Az, El)-pattern position + DIMM pattern offset
\textbf{\textbf{and take DIMM images}}\\
}

\begin{itemize}
\tightlist
\item
  Point the TMA back to {Pointing 3}⁠ at {-270}⁠ + DIMM offset, {75}⁠ +
  DIMM offset.
\item
  While tracking, take DIMM images with XXXs exposure time and inspect
  the quality.
\end{itemize}

}
\hdashrule[0.5ex]{\textwidth}{1pt}{3mm}
  Expected Result \\
{\footnotesize
\begin{itemize}
\tightlist
\item
  TMA reaches the position.
\item
  DIMM image quality is sufficient
\end{itemize}

}

\begin{tabular}{p{2cm}}
\toprule
Step 1031  \\ \hline
\end{tabular}
 Description \\
{\footnotesize
Wait for the Dome to reach the commanded position.

}
\hdashrule[0.5ex]{\textwidth}{1pt}{3mm}
  Expected Result \\
{\footnotesize
The \emph{MTDome\_logevent\_azMotion} and
\emph{MTDome\_logevent\_elMotion} inPosition parameter = true.

}

\begin{tabular}{p{2cm}}
\toprule
Step 1032  \\ \hline
\end{tabular}
 Description \\
{\footnotesize
\textbf{Point the TMA to (Az, El)-pattern position}\\
Point the TMA to {Pointing 3}⁠ at {-270}⁠ , {75}⁠ .

}
\hdashrule[0.5ex]{\textwidth}{1pt}{3mm}
  Expected Result \\
{\footnotesize
The TMA starts moving.

}

\begin{tabular}{p{2cm}}
\toprule
Step 1033  \\ \hline
\end{tabular}
 Description \\
{\footnotesize
Wait for the TMA to reach the commanded position.

}
\hdashrule[0.5ex]{\textwidth}{1pt}{3mm}
  Expected Result \\
{\footnotesize
The \emph{MTMount\_logevent\_azimuthInPosition} and
\emph{MTMount\_logevent\_elevationInPosition} inPosition parameter =
true.

}

\begin{tabular}{p{2cm}}
\toprule
Step 1034  \\ \hline
\end{tabular}
 Description \\
{\footnotesize
\textbf{Find DIMM Object and DIMM Pattern Offset}\\

\begin{itemize}
\tightlist
\item
  While tracking, take a 10-sec exposure with the StarTracker.
\item
  Load the image into an image viewer.
\item
  Overlay the GAIA catalog.
\item
  Select a star brighter than XXX mag (bright enough for the DIMM).
\item
  Calculate the pixel offset between the StarTracker and the DIMM.
\item
  Transform the offset into AZ and EL offsets.
\end{itemize}

}
\hdashrule[0.5ex]{\textwidth}{1pt}{3mm}
  Expected Result \\
{\footnotesize
\begin{itemize}
\tightlist
\item
  An image was successfully taken with the StarTracker and is of
  sufficient quality.
\item
  AZ and EL offsets are available.
\end{itemize}

}

\begin{tabular}{p{2cm}}
\toprule
Step 1035  \\ \hline
\end{tabular}
 Description \\
{\footnotesize
\textbf{Move TMA to the 1. random distance of 3.5deg}\\
Point the TMA to a random 3.5 deg combined offset in AZ and EL from
{Pointing 3}⁠ at {-270}⁠, {75}⁠. Record the exact position of the offset
in AZ and El.

}
\hdashrule[0.5ex]{\textwidth}{1pt}{3mm}
  Expected Result \\
{\footnotesize
\begin{itemize}
\tightlist
\item
  The TMA reaches the commanded offset position.
\item
  The \emph{MTMount\_logevent\_azimuthInPosition} and
  \emph{MTMount\_logevent\_elevationInPosition}inPosition parameter =
  true.
\end{itemize}

}

\begin{tabular}{p{2cm}}
\toprule
Step 1036  \\ \hline
\end{tabular}
 Description \\
{\footnotesize
Wait for the TMA to reach the commanded position.

}
\hdashrule[0.5ex]{\textwidth}{1pt}{3mm}
  Expected Result \\
{\footnotesize
The \emph{MTMount\_logevent\_azimuthInPosition} and
\emph{MTMount\_logevent\_elevationInPosition} inPosition parameter =
true.

}

\begin{tabular}{p{2cm}}
\toprule
Step 1037  \\ \hline
\end{tabular}
 Description \\
{\footnotesize
\textbf{Position the dome}\\
Command the Dome to {Pointing 4}⁠ to {\{Dome AZ\}}⁠

}
\hdashrule[0.5ex]{\textwidth}{1pt}{3mm}
  Expected Result \\
{\footnotesize
The Dome starts moving.

}

\begin{tabular}{p{2cm}}
\toprule
Step 1038  \\ \hline
\end{tabular}
 Description \\
{\footnotesize
\textbf{Point the TMA to (Az, El)-pattern position + DIMM pattern offset
\textbf{\textbf{and take DIMM images}}\\
}

\begin{itemize}
\tightlist
\item
  Point the TMA back to {Pointing 4}⁠ at {-270}⁠ + DIMM offset, {86.5}⁠
  + DIMM offset.
\item
  While tracking, take DIMM images with XXXs exposure time and inspect
  the quality.
\end{itemize}

}
\hdashrule[0.5ex]{\textwidth}{1pt}{3mm}
  Expected Result \\
{\footnotesize
\begin{itemize}
\tightlist
\item
  TMA reaches the position.
\item
  DIMM image quality is sufficient
\end{itemize}

}

\begin{tabular}{p{2cm}}
\toprule
Step 1039  \\ \hline
\end{tabular}
 Description \\
{\footnotesize
Wait for the Dome to reach the commanded position.

}
\hdashrule[0.5ex]{\textwidth}{1pt}{3mm}
  Expected Result \\
{\footnotesize
The \emph{MTDome\_logevent\_azMotion} and
\emph{MTDome\_logevent\_elMotion} inPosition parameter = true.

}

\begin{tabular}{p{2cm}}
\toprule
Step 1040  \\ \hline
\end{tabular}
 Description \\
{\footnotesize
\textbf{Point the TMA to (Az, El)-pattern position}\\
Point the TMA to {Pointing 4}⁠ at {-270}⁠ , {86.5}⁠ .

}
\hdashrule[0.5ex]{\textwidth}{1pt}{3mm}
  Expected Result \\
{\footnotesize
The TMA starts moving.

}

\begin{tabular}{p{2cm}}
\toprule
Step 1041  \\ \hline
\end{tabular}
 Description \\
{\footnotesize
Wait for the TMA to reach the commanded position.

}
\hdashrule[0.5ex]{\textwidth}{1pt}{3mm}
  Expected Result \\
{\footnotesize
The \emph{MTMount\_logevent\_azimuthInPosition} and
\emph{MTMount\_logevent\_elevationInPosition} inPosition parameter =
true.

}

\begin{tabular}{p{2cm}}
\toprule
Step 1042  \\ \hline
\end{tabular}
 Description \\
{\footnotesize
\textbf{Find DIMM Object and DIMM Pattern Offset}\\

\begin{itemize}
\tightlist
\item
  While tracking, take a 10-sec exposure with the StarTracker.
\item
  Load the image into an image viewer.
\item
  Overlay the GAIA catalog.
\item
  Select a star brighter than XXX mag (bright enough for the DIMM).
\item
  Calculate the pixel offset between the StarTracker and the DIMM.
\item
  Transform the offset into AZ and EL offsets.
\end{itemize}

}
\hdashrule[0.5ex]{\textwidth}{1pt}{3mm}
  Expected Result \\
{\footnotesize
\begin{itemize}
\tightlist
\item
  An image was successfully taken with the StarTracker and is of
  sufficient quality.
\item
  AZ and EL offsets are available.
\end{itemize}

}

\begin{tabular}{p{2cm}}
\toprule
Step 1043  \\ \hline
\end{tabular}
 Description \\
{\footnotesize
\textbf{Move TMA to the 1. random distance of 3.5deg}\\
Point the TMA to a random 3.5 deg combined offset in AZ and EL from
{Pointing 4}⁠ at {-270}⁠, {86.5}⁠. Record the exact position of the
offset in AZ and El.

}
\hdashrule[0.5ex]{\textwidth}{1pt}{3mm}
  Expected Result \\
{\footnotesize
\begin{itemize}
\tightlist
\item
  The TMA reaches the commanded offset position.
\item
  The \emph{MTMount\_logevent\_azimuthInPosition} and
  \emph{MTMount\_logevent\_elevationInPosition}inPosition parameter =
  true.
\end{itemize}

}

\begin{tabular}{p{2cm}}
\toprule
Step 1044  \\ \hline
\end{tabular}
 Description \\
{\footnotesize
\textbf{Find DIMM Object and DIMM Pattern Offset}\\

\begin{itemize}
\tightlist
\item
  While tracking, take a 10-sec exposure with the StarTracker.
\item
  Load the image into an image viewer.
\item
  Overlay the GAIA catalog.
\item
  Select a star brighter than XXX mag (bright enough for the DIMM).
\item
  Calculate the pixel offset between the StarTracker and the DIMM.
\item
  Transform the offset into AZ and EL offsets.
\end{itemize}

}
\hdashrule[0.5ex]{\textwidth}{1pt}{3mm}
  Expected Result \\
{\footnotesize
\begin{itemize}
\tightlist
\item
  An image was successfully taken with the StarTracker and is of
  sufficient quality.
\item
  AZ and EL offsets are available.
\end{itemize}

}

\begin{tabular}{p{2cm}}
\toprule
Step 1045  \\ \hline
\end{tabular}
 Description \\
{\footnotesize
\textbf{Position the dome}\\
Command the Dome to {Pointing 5}⁠ to {\{Dome AZ\}}⁠

}
\hdashrule[0.5ex]{\textwidth}{1pt}{3mm}
  Expected Result \\
{\footnotesize
The Dome starts moving.

}

\begin{tabular}{p{2cm}}
\toprule
Step 1046  \\ \hline
\end{tabular}
 Description \\
{\footnotesize
\textbf{Point the TMA to (Az, El)-pattern position + DIMM pattern offset
\textbf{\textbf{and take DIMM images}}\\
}

\begin{itemize}
\tightlist
\item
  Point the TMA back to {Pointing 5}⁠ at {-180}⁠ + DIMM offset, {86.5}⁠
  + DIMM offset.
\item
  While tracking, take DIMM images with XXXs exposure time and inspect
  the quality.
\end{itemize}

}
\hdashrule[0.5ex]{\textwidth}{1pt}{3mm}
  Expected Result \\
{\footnotesize
\begin{itemize}
\tightlist
\item
  TMA reaches the position.
\item
  DIMM image quality is sufficient
\end{itemize}

}

\begin{tabular}{p{2cm}}
\toprule
Step 1047  \\ \hline
\end{tabular}
 Description \\
{\footnotesize
Wait for the Dome to reach the commanded position.

}
\hdashrule[0.5ex]{\textwidth}{1pt}{3mm}
  Expected Result \\
{\footnotesize
The \emph{MTDome\_logevent\_azMotion} and
\emph{MTDome\_logevent\_elMotion} inPosition parameter = true.

}

\begin{tabular}{p{2cm}}
\toprule
Step 1048  \\ \hline
\end{tabular}
 Description \\
{\footnotesize
\textbf{Point the TMA to (Az, El)-pattern position}\\
Point the TMA to {Pointing 5}⁠ at {-180}⁠ , {86.5}⁠ .

}
\hdashrule[0.5ex]{\textwidth}{1pt}{3mm}
  Expected Result \\
{\footnotesize
The TMA starts moving.

}

\begin{tabular}{p{2cm}}
\toprule
Step 1049  \\ \hline
\end{tabular}
 Description \\
{\footnotesize
Wait for the TMA to reach the commanded position.

}
\hdashrule[0.5ex]{\textwidth}{1pt}{3mm}
  Expected Result \\
{\footnotesize
The \emph{MTMount\_logevent\_azimuthInPosition} and
\emph{MTMount\_logevent\_elevationInPosition} inPosition parameter =
true.

}

\begin{tabular}{p{2cm}}
\toprule
Step 1050  \\ \hline
\end{tabular}
 Description \\
{\footnotesize
\textbf{Find DIMM Object and DIMM Pattern Offset}\\

\begin{itemize}
\tightlist
\item
  While tracking, take a 10-sec exposure with the StarTracker.
\item
  Load the image into an image viewer.
\item
  Overlay the GAIA catalog.
\item
  Select a star brighter than XXX mag (bright enough for the DIMM).
\item
  Calculate the pixel offset between the StarTracker and the DIMM.
\item
  Transform the offset into AZ and EL offsets.
\end{itemize}

}
\hdashrule[0.5ex]{\textwidth}{1pt}{3mm}
  Expected Result \\
{\footnotesize
\begin{itemize}
\tightlist
\item
  An image was successfully taken with the StarTracker and is of
  sufficient quality.
\item
  AZ and EL offsets are available.
\end{itemize}

}

\begin{tabular}{p{2cm}}
\toprule
Step 1051  \\ \hline
\end{tabular}
 Description \\
{\footnotesize
\textbf{Move TMA to the 1. random distance of 3.5deg}\\
Point the TMA to a random 3.5 deg combined offset in AZ and EL from
{Pointing 5}⁠ at {-180}⁠, {86.5}⁠. Record the exact position of the
offset in AZ and El.

}
\hdashrule[0.5ex]{\textwidth}{1pt}{3mm}
  Expected Result \\
{\footnotesize
\begin{itemize}
\tightlist
\item
  The TMA reaches the commanded offset position.
\item
  The \emph{MTMount\_logevent\_azimuthInPosition} and
  \emph{MTMount\_logevent\_elevationInPosition}inPosition parameter =
  true.
\end{itemize}

}

\begin{tabular}{p{2cm}}
\toprule
Step 1052  \\ \hline
\end{tabular}
 Description \\
{\footnotesize
\textbf{Position the dome}\\
Command the Dome to {Pointing 6}⁠ to {\{Dome AZ\}}⁠

}
\hdashrule[0.5ex]{\textwidth}{1pt}{3mm}
  Expected Result \\
{\footnotesize
The Dome starts moving.

}

\begin{tabular}{p{2cm}}
\toprule
Step 1053  \\ \hline
\end{tabular}
 Description \\
{\footnotesize
\textbf{Point the TMA to (Az, El)-pattern position + DIMM pattern offset
\textbf{\textbf{and take DIMM images}}\\
}

\begin{itemize}
\tightlist
\item
  Point the TMA back to {Pointing 6}⁠ at {-180}⁠ + DIMM offset, {75}⁠ +
  DIMM offset.
\item
  While tracking, take DIMM images with XXXs exposure time and inspect
  the quality.
\end{itemize}

}
\hdashrule[0.5ex]{\textwidth}{1pt}{3mm}
  Expected Result \\
{\footnotesize
\begin{itemize}
\tightlist
\item
  TMA reaches the position.
\item
  DIMM image quality is sufficient
\end{itemize}

}

\begin{tabular}{p{2cm}}
\toprule
Step 1054  \\ \hline
\end{tabular}
 Description \\
{\footnotesize
Wait for the Dome to reach the commanded position.

}
\hdashrule[0.5ex]{\textwidth}{1pt}{3mm}
  Expected Result \\
{\footnotesize
The \emph{MTDome\_logevent\_azMotion} and
\emph{MTDome\_logevent\_elMotion} inPosition parameter = true.

}

\begin{tabular}{p{2cm}}
\toprule
Step 1055  \\ \hline
\end{tabular}
 Description \\
{\footnotesize
\textbf{Point the TMA to (Az, El)-pattern position}\\
Point the TMA to {Pointing 6}⁠ at {-180}⁠ , {75}⁠ .

}
\hdashrule[0.5ex]{\textwidth}{1pt}{3mm}
  Expected Result \\
{\footnotesize
The TMA starts moving.

}

\begin{tabular}{p{2cm}}
\toprule
Step 1056  \\ \hline
\end{tabular}
 Description \\
{\footnotesize
Wait for the TMA to reach the commanded position.

}
\hdashrule[0.5ex]{\textwidth}{1pt}{3mm}
  Expected Result \\
{\footnotesize
The \emph{MTMount\_logevent\_azimuthInPosition} and
\emph{MTMount\_logevent\_elevationInPosition} inPosition parameter =
true.

}

\begin{tabular}{p{2cm}}
\toprule
Step 1057  \\ \hline
\end{tabular}
 Description \\
{\footnotesize
\textbf{Find DIMM Object and DIMM Pattern Offset}\\

\begin{itemize}
\tightlist
\item
  While tracking, take a 10-sec exposure with the StarTracker.
\item
  Load the image into an image viewer.
\item
  Overlay the GAIA catalog.
\item
  Select a star brighter than XXX mag (bright enough for the DIMM).
\item
  Calculate the pixel offset between the StarTracker and the DIMM.
\item
  Transform the offset into AZ and EL offsets.
\end{itemize}

}
\hdashrule[0.5ex]{\textwidth}{1pt}{3mm}
  Expected Result \\
{\footnotesize
\begin{itemize}
\tightlist
\item
  An image was successfully taken with the StarTracker and is of
  sufficient quality.
\item
  AZ and EL offsets are available.
\end{itemize}

}

\begin{tabular}{p{2cm}}
\toprule
Step 1058  \\ \hline
\end{tabular}
 Description \\
{\footnotesize
\textbf{Move TMA to the 1. random distance of 3.5deg}\\
Point the TMA to a random 3.5 deg combined offset in AZ and EL from
{Pointing 6}⁠ at {-180}⁠, {75}⁠. Record the exact position of the offset
in AZ and El.

}
\hdashrule[0.5ex]{\textwidth}{1pt}{3mm}
  Expected Result \\
{\footnotesize
\begin{itemize}
\tightlist
\item
  The TMA reaches the commanded offset position.
\item
  The \emph{MTMount\_logevent\_azimuthInPosition} and
  \emph{MTMount\_logevent\_elevationInPosition}inPosition parameter =
  true.
\end{itemize}

}

\begin{tabular}{p{2cm}}
\toprule
Step 1059  \\ \hline
\end{tabular}
 Description \\
{\footnotesize
\textbf{Move TMA to the 1. random distance of 3.5deg}\\
Point the TMA to a random 3.5 deg combined offset in AZ and EL from
{Pointing 1}⁠ at {-270}⁠, {15}⁠. Record the exact position of the offset
in AZ and El.

}
\hdashrule[0.5ex]{\textwidth}{1pt}{3mm}
  Expected Result \\
{\footnotesize
\begin{itemize}
\tightlist
\item
  The TMA reaches the commanded offset position.
\item
  The \emph{MTMount\_logevent\_azimuthInPosition} and
  \emph{MTMount\_logevent\_elevationInPosition}inPosition parameter =
  true.
\end{itemize}

}

\begin{tabular}{p{2cm}}
\toprule
Step 1060  \\ \hline
\end{tabular}
 Description \\
{\footnotesize
\textbf{Position the dome}\\
Command the Dome to {Pointing 7}⁠ to {\{Dome AZ\}}⁠

}
\hdashrule[0.5ex]{\textwidth}{1pt}{3mm}
  Expected Result \\
{\footnotesize
The Dome starts moving.

}

\begin{tabular}{p{2cm}}
\toprule
Step 1061  \\ \hline
\end{tabular}
 Description \\
{\footnotesize
\textbf{Point the TMA to (Az, El)-pattern position + DIMM pattern offset
\textbf{\textbf{and take DIMM images}}\\
}

\begin{itemize}
\tightlist
\item
  Point the TMA back to {Pointing 7}⁠ at {-180}⁠ + DIMM offset, {45}⁠ +
  DIMM offset.
\item
  While tracking, take DIMM images with XXXs exposure time and inspect
  the quality.
\end{itemize}

}
\hdashrule[0.5ex]{\textwidth}{1pt}{3mm}
  Expected Result \\
{\footnotesize
\begin{itemize}
\tightlist
\item
  TMA reaches the position.
\item
  DIMM image quality is sufficient
\end{itemize}

}

\begin{tabular}{p{2cm}}
\toprule
Step 1062  \\ \hline
\end{tabular}
 Description \\
{\footnotesize
Wait for the Dome to reach the commanded position.

}
\hdashrule[0.5ex]{\textwidth}{1pt}{3mm}
  Expected Result \\
{\footnotesize
The \emph{MTDome\_logevent\_azMotion} and
\emph{MTDome\_logevent\_elMotion} inPosition parameter = true.

}

\begin{tabular}{p{2cm}}
\toprule
Step 1063  \\ \hline
\end{tabular}
 Description \\
{\footnotesize
\textbf{Point the TMA to (Az, El)-pattern position}\\
Point the TMA to {Pointing 7}⁠ at {-180}⁠ , {45}⁠ .

}
\hdashrule[0.5ex]{\textwidth}{1pt}{3mm}
  Expected Result \\
{\footnotesize
The TMA starts moving.

}

\begin{tabular}{p{2cm}}
\toprule
Step 1064  \\ \hline
\end{tabular}
 Description \\
{\footnotesize
Wait for the TMA to reach the commanded position.

}
\hdashrule[0.5ex]{\textwidth}{1pt}{3mm}
  Expected Result \\
{\footnotesize
The \emph{MTMount\_logevent\_azimuthInPosition} and
\emph{MTMount\_logevent\_elevationInPosition} inPosition parameter =
true.

}

\begin{tabular}{p{2cm}}
\toprule
Step 1065  \\ \hline
\end{tabular}
 Description \\
{\footnotesize
\textbf{Find DIMM Object and DIMM Pattern Offset}\\

\begin{itemize}
\tightlist
\item
  While tracking, take a 10-sec exposure with the StarTracker.
\item
  Load the image into an image viewer.
\item
  Overlay the GAIA catalog.
\item
  Select a star brighter than XXX mag (bright enough for the DIMM).
\item
  Calculate the pixel offset between the StarTracker and the DIMM.
\item
  Transform the offset into AZ and EL offsets.
\end{itemize}

}
\hdashrule[0.5ex]{\textwidth}{1pt}{3mm}
  Expected Result \\
{\footnotesize
\begin{itemize}
\tightlist
\item
  An image was successfully taken with the StarTracker and is of
  sufficient quality.
\item
  AZ and EL offsets are available.
\end{itemize}

}

\begin{tabular}{p{2cm}}
\toprule
Step 1066  \\ \hline
\end{tabular}
 Description \\
{\footnotesize
\textbf{Move TMA to the 1. random distance of 3.5deg}\\
Point the TMA to a random 3.5 deg combined offset in AZ and EL from
{Pointing 7}⁠ at {-180}⁠, {45}⁠. Record the exact position of the offset
in AZ and El.

}
\hdashrule[0.5ex]{\textwidth}{1pt}{3mm}
  Expected Result \\
{\footnotesize
\begin{itemize}
\tightlist
\item
  The TMA reaches the commanded offset position.
\item
  The \emph{MTMount\_logevent\_azimuthInPosition} and
  \emph{MTMount\_logevent\_elevationInPosition}inPosition parameter =
  true.
\end{itemize}

}

\begin{tabular}{p{2cm}}
\toprule
Step 1067  \\ \hline
\end{tabular}
 Description \\
{\footnotesize
\textbf{Position the dome}\\
Command the Dome to {Pointing 8}⁠ to {\{Dome AZ\}}⁠

}
\hdashrule[0.5ex]{\textwidth}{1pt}{3mm}
  Expected Result \\
{\footnotesize
The Dome starts moving.

}

\begin{tabular}{p{2cm}}
\toprule
Step 1068  \\ \hline
\end{tabular}
 Description \\
{\footnotesize
\textbf{Point the TMA to (Az, El)-pattern position + DIMM pattern offset
\textbf{\textbf{and take DIMM images}}\\
}

\begin{itemize}
\tightlist
\item
  Point the TMA back to {Pointing 8}⁠ at {-180}⁠ + DIMM offset, {15}⁠ +
  DIMM offset.
\item
  While tracking, take DIMM images with XXXs exposure time and inspect
  the quality.
\end{itemize}

}
\hdashrule[0.5ex]{\textwidth}{1pt}{3mm}
  Expected Result \\
{\footnotesize
\begin{itemize}
\tightlist
\item
  TMA reaches the position.
\item
  DIMM image quality is sufficient
\end{itemize}

}

\begin{tabular}{p{2cm}}
\toprule
Step 1069  \\ \hline
\end{tabular}
 Description \\
{\footnotesize
Wait for the Dome to reach the commanded position.

}
\hdashrule[0.5ex]{\textwidth}{1pt}{3mm}
  Expected Result \\
{\footnotesize
The \emph{MTDome\_logevent\_azMotion} and
\emph{MTDome\_logevent\_elMotion} inPosition parameter = true.

}

\begin{tabular}{p{2cm}}
\toprule
Step 1070  \\ \hline
\end{tabular}
 Description \\
{\footnotesize
\textbf{Point the TMA to (Az, El)-pattern position}\\
Point the TMA to {Pointing 8}⁠ at {-180}⁠ , {15}⁠ .

}
\hdashrule[0.5ex]{\textwidth}{1pt}{3mm}
  Expected Result \\
{\footnotesize
The TMA starts moving.

}

\begin{tabular}{p{2cm}}
\toprule
Step 1071  \\ \hline
\end{tabular}
 Description \\
{\footnotesize
Wait for the TMA to reach the commanded position.

}
\hdashrule[0.5ex]{\textwidth}{1pt}{3mm}
  Expected Result \\
{\footnotesize
The \emph{MTMount\_logevent\_azimuthInPosition} and
\emph{MTMount\_logevent\_elevationInPosition} inPosition parameter =
true.

}

\begin{tabular}{p{2cm}}
\toprule
Step 1072  \\ \hline
\end{tabular}
 Description \\
{\footnotesize
\textbf{Find DIMM Object and DIMM Pattern Offset}\\

\begin{itemize}
\tightlist
\item
  While tracking, take a 10-sec exposure with the StarTracker.
\item
  Load the image into an image viewer.
\item
  Overlay the GAIA catalog.
\item
  Select a star brighter than XXX mag (bright enough for the DIMM).
\item
  Calculate the pixel offset between the StarTracker and the DIMM.
\item
  Transform the offset into AZ and EL offsets.
\end{itemize}

}
\hdashrule[0.5ex]{\textwidth}{1pt}{3mm}
  Expected Result \\
{\footnotesize
\begin{itemize}
\tightlist
\item
  An image was successfully taken with the StarTracker and is of
  sufficient quality.
\item
  AZ and EL offsets are available.
\end{itemize}

}

\begin{tabular}{p{2cm}}
\toprule
Step 1073  \\ \hline
\end{tabular}
 Description \\
{\footnotesize
\textbf{Move TMA to the 1. random distance of 3.5deg}\\
Point the TMA to a random 3.5 deg combined offset in AZ and EL from
{Pointing 8}⁠ at {-180}⁠, {15}⁠. Record the exact position of the offset
in AZ and El.

}
\hdashrule[0.5ex]{\textwidth}{1pt}{3mm}
  Expected Result \\
{\footnotesize
\begin{itemize}
\tightlist
\item
  The TMA reaches the commanded offset position.
\item
  The \emph{MTMount\_logevent\_azimuthInPosition} and
  \emph{MTMount\_logevent\_elevationInPosition}inPosition parameter =
  true.
\end{itemize}

}

\begin{tabular}{p{2cm}}
\toprule
Step 1074  \\ \hline
\end{tabular}
 Description \\
{\footnotesize
\textbf{Position the dome}\\
Command the Dome to {Pointing 9}⁠ to {\{Dome AZ\}}⁠

}
\hdashrule[0.5ex]{\textwidth}{1pt}{3mm}
  Expected Result \\
{\footnotesize
The Dome starts moving.

}

\begin{tabular}{p{2cm}}
\toprule
Step 1075  \\ \hline
\end{tabular}
 Description \\
{\footnotesize
\textbf{Point the TMA to (Az, El)-pattern position + DIMM pattern offset
\textbf{\textbf{and take DIMM images}}\\
}

\begin{itemize}
\tightlist
\item
  Point the TMA back to {Pointing 9}⁠ at {-90}⁠ + DIMM offset, {15}⁠ +
  DIMM offset.
\item
  While tracking, take DIMM images with XXXs exposure time and inspect
  the quality.
\end{itemize}

}
\hdashrule[0.5ex]{\textwidth}{1pt}{3mm}
  Expected Result \\
{\footnotesize
\begin{itemize}
\tightlist
\item
  TMA reaches the position.
\item
  DIMM image quality is sufficient
\end{itemize}

}

\begin{tabular}{p{2cm}}
\toprule
Step 1076  \\ \hline
\end{tabular}
 Description \\
{\footnotesize
Wait for the Dome to reach the commanded position.

}
\hdashrule[0.5ex]{\textwidth}{1pt}{3mm}
  Expected Result \\
{\footnotesize
The \emph{MTDome\_logevent\_azMotion} and
\emph{MTDome\_logevent\_elMotion} inPosition parameter = true.

}

\begin{tabular}{p{2cm}}
\toprule
Step 1077  \\ \hline
\end{tabular}
 Description \\
{\footnotesize
\textbf{Point the TMA to (Az, El)-pattern position}\\
Point the TMA to {Pointing 9}⁠ at {-90}⁠ , {15}⁠ .

}
\hdashrule[0.5ex]{\textwidth}{1pt}{3mm}
  Expected Result \\
{\footnotesize
The TMA starts moving.

}

\begin{tabular}{p{2cm}}
\toprule
Step 1078  \\ \hline
\end{tabular}
 Description \\
{\footnotesize
Wait for the TMA to reach the commanded position.

}
\hdashrule[0.5ex]{\textwidth}{1pt}{3mm}
  Expected Result \\
{\footnotesize
The \emph{MTMount\_logevent\_azimuthInPosition} and
\emph{MTMount\_logevent\_elevationInPosition} inPosition parameter =
true.

}

\begin{tabular}{p{2cm}}
\toprule
Step 1079  \\ \hline
\end{tabular}
 Description \\
{\footnotesize
\textbf{Find DIMM Object and DIMM Pattern Offset}\\

\begin{itemize}
\tightlist
\item
  While tracking, take a 10-sec exposure with the StarTracker.
\item
  Load the image into an image viewer.
\item
  Overlay the GAIA catalog.
\item
  Select a star brighter than XXX mag (bright enough for the DIMM).
\item
  Calculate the pixel offset between the StarTracker and the DIMM.
\item
  Transform the offset into AZ and EL offsets.
\end{itemize}

}
\hdashrule[0.5ex]{\textwidth}{1pt}{3mm}
  Expected Result \\
{\footnotesize
\begin{itemize}
\tightlist
\item
  An image was successfully taken with the StarTracker and is of
  sufficient quality.
\item
  AZ and EL offsets are available.
\end{itemize}

}

\begin{tabular}{p{2cm}}
\toprule
Step 1080  \\ \hline
\end{tabular}
 Description \\
{\footnotesize
\textbf{Move TMA to the 1. random distance of 3.5deg}\\
Point the TMA to a random 3.5 deg combined offset in AZ and EL from
{Pointing 9}⁠ at {-90}⁠, {15}⁠. Record the exact position of the offset
in AZ and El.

}
\hdashrule[0.5ex]{\textwidth}{1pt}{3mm}
  Expected Result \\
{\footnotesize
\begin{itemize}
\tightlist
\item
  The TMA reaches the commanded offset position.
\item
  The \emph{MTMount\_logevent\_azimuthInPosition} and
  \emph{MTMount\_logevent\_elevationInPosition}inPosition parameter =
  true.
\end{itemize}

}

\paragraph{ LVV-T2715 - Configure Observatory Environment for Daytime Operations }\mbox{}\\

Version \textbf{1}.
Open  \href{https://jira.lsstcorp.org/secure/Tests.jspa#/testCase/LVV-T2715}{\textit{ LVV-T2715 } }
test case in Jira.

After using the observatory during the nighttime, prepare the
observatory for daytime operations.

\textbf{ Preconditions}:\\
The observatory was used during nighttime.

Final comment:\\


Detailed steps :

\begin{tabular}{p{2cm}}
\toprule
Step 1  \\ \hline
\end{tabular}
 Description \\
{\footnotesize
\textbf{CSCs}\\

\begin{itemize}
\tightlist
\item
  Transition the CSCs into STANDBY state
\end{itemize}

}
\hdashrule[0.5ex]{\textwidth}{1pt}{3mm}
  Expected Result \\
{\footnotesize
All CSCs are in their standbyState.

}

\begin{tabular}{p{2cm}}
\toprule
Step 2  \\ \hline
\end{tabular}
 Description \\
{\footnotesize
\textbf{Telescope daytime preparations:}

\begin{itemize}
\tightlist
\item
  Switch off or bring into standby the StarTracker and DIMM instruments
\item
  Install the caps on top of the StarTracker telescopes and the DIMM
\end{itemize}

}
\hdashrule[0.5ex]{\textwidth}{1pt}{3mm}
  Expected Result \\
{\footnotesize
The caps are installed.

}

\begin{tabular}{p{2cm}}
\toprule
Step 3  \\ \hline
\end{tabular}
 Description \\
{\footnotesize
\textbf{Dome:}\\

\begin{itemize}
\tightlist
\item
  Bring the dome into the park position
\end{itemize}

Until the dome shutter is motorized:\textbf{\\
}

\begin{itemize}
\tightlist
\item
  Send a message to the site manager :

  \begin{itemize}
  \tightlist
  \item
    confirming that nightly operations have finished~
  \item
    asking for a dome closer before the sun starts to shine on the
    StarTracker and the DIMM.
  \end{itemize}
\end{itemize}

}
\hdashrule[0.5ex]{\textwidth}{1pt}{3mm}
  Expected Result \\
{\footnotesize
Dome closure is organized.

}

\begin{tabular}{p{2cm}}
\toprule
Step 4  \\ \hline
\end{tabular}
 Description \\
{\footnotesize
\textbf{Auxillary systems~daytime preparations:}\\
If needed for daytime operations:\textbf{\\
}

\begin{itemize}
\tightlist
\item
  Switch on the UMA in the morning.
\item
  When available and need to be modified for the day:

  \begin{itemize}
  \tightlist
  \item
    Oil supply system on standby?
  \item
    Dynalyne into standby?
  \end{itemize}
\end{itemize}

}
\hdashrule[0.5ex]{\textwidth}{1pt}{3mm}
  Expected Result \\
{\footnotesize
All auxiliary systems are in the states suitable for daytime operations.

\begin{itemize}
\tightlist
\item
  The UMA is switched on.
\end{itemize}

}

\begin{tabular}{p{2cm}}
\toprule
Step 5  \\ \hline
\end{tabular}
 Description \\
{\footnotesize
\textbf{TMA position in the morning}\\

\begin{itemize}
\tightlist
\item
  Park the TMA in the position needed for the next day.
\end{itemize}

}
\hdashrule[0.5ex]{\textwidth}{1pt}{3mm}
  Expected Result \\
{\footnotesize
TMA parked in the corresponding position.

}

\begin{tabular}{p{2cm}}
\toprule
Step 6  \\ \hline
\end{tabular}
 Description \\
{\footnotesize

}
\hdashrule[0.5ex]{\textwidth}{1pt}{3mm}
  Expected Result \\
{\footnotesize

}

\begin{tabular}{p{2cm}}
\toprule
Step 7  \\ \hline
\end{tabular}
 Description \\
{\footnotesize
\textbf{Night log}\\

\begin{itemize}
\tightlist
\item
  Close the night log by writing a summary of the nightly events
\item
  Send a link with the summary to the site manager.
\end{itemize}

}
\hdashrule[0.5ex]{\textwidth}{1pt}{3mm}
  Expected Result \\
{\footnotesize
The night log is closed.

}

\subsection{Test Cycle LVV-C232 }

Open test cycle {\it \href{https://jira.lsstcorp.org/secure/Tests.jspa#/testrun/LVV-C232}{TMA Pointing and Tracking - Analysis - In Depth}} in Jira.

Test Cycle name: TMA Pointing and Tracking - Analysis - In Depth\\
Status: In Progress

Requirements verification for the pointing and tracking using the Star
Tracker and the DIMM on a dedicated mounting plate connector to the top
end of the TMA.

\subsubsection{Software Version/Baseline}
Star Tracker software version:\\
Dimm software version:\\
CSC software version:\\
Analysis software repository:

\subsubsection{Configuration}
Not provided.

\subsubsection{Test Cases in LVV-C232 Test Cycle}

\paragraph{ LVV-T2738 - StarTracker Pointing and Tracking Test In-Depth Analysis, Pointing,
Offset, Tracking Drift }\mbox{}\\

Version \textbf{1}.
Open  \href{https://jira.lsstcorp.org/secure/Tests.jspa#/testCase/LVV-T2738}{\textit{ LVV-T2738 } }
test case in Jira.

The objective of this test case is the {\textbf{analysis of the data}
taken to verify that the\\
}

\begin{itemize}
\tightlist
\item
  {TMA achieves a pointing accuracy of 50 arcsec RMS relative to its own
  reference system}{~for any motion within the pointing range.~}
\item
  TMA achieves pointing repeatability within the value of 1arcsec RMS
  for any motion within the pointing range. (LTS-103-REQ-0011-V-01:
  2.1.5\_1)
\end{itemize}

\textbf{NOTE:} The Dome needs an offset of a min of 0.6 deg for the TMA
motion during tracking, or else the Dome would need to be moved.

\textbf{ Preconditions}:\\
At least one dataset for the forward or reverse, or random pointing
pattern must be available.

Final comment:\\


Detailed steps :

\begin{tabular}{p{2cm}}
\toprule
Step 1  \\ \hline
\end{tabular}
 Description \\
{\footnotesize
Use the existing data to plot the tracking drift for the DIMM data in Az
and El and calculate the resulting RMS.\\
mcs = Mount Control System\\
Plot:

\begin{itemize}
\tightlist
\item
  AZ\_sky versus time
\item
  EL\_sky versus time
\item
  AZ\_mcs versus time
\item
  EL\_mcs versus time
\item
  (AZ\_sky - AZ\_mcs) versus time
\item
  (EL\_sky - EL\_mcs) versus time
\end{itemize}

Include the following information:\\

\begin{itemize}
\tightlist
\item
  Start and end AZ/EL as appropriate
\item
  AZ/EL midpoint of time series as appropriate
\item
  Start and end times
\end{itemize}

}
\hdashrule[0.5ex]{\textwidth}{1pt}{3mm}
  Expected Result \\
{\footnotesize
The RMS of the tracking drift is less than or equal to 1 arcsec.

}

\begin{tabular}{p{2cm}}
\toprule
Step 2  \\ \hline
\end{tabular}
 Description \\
{\footnotesize
If the plots show differences in Az and El, the following additional
analysis shall be done:

\begin{enumerate}
\tightlist
\item
  Conduct a least-squares linear fit versus time.
\item
  Include uncertainty analysis of slope and y-intercept: Best-fit slope
  and y-intercept and their 1-sigma deviations
\item
  De-trend the difference data by the slope of the most likely linear
  fit. ~
\item
  Plot the marginal histogram of the de-trended difference data for AZ
  and EL.
\item
  Generate histogram plots for de-trended AZ\_sky - AZ\_mcs and EL\_sky
  - EL\_mcs.
\item
  Fit a model to the histogram (expectation is a simple gaussian or
  perhaps a double gaussian to capture wings of the distribution
\item
  Calculate histogram statistics - mean, mode STDev
\end{enumerate}

\textbf{Note:}\\
MCS = Mount control system = measured AZ\\
Sky= calculated values taken from the astrometry.net
solution\\[2\baselineskip]

}
\hdashrule[0.5ex]{\textwidth}{1pt}{3mm}
  Expected Result \\
{\footnotesize
\begin{enumerate}
\tightlist
\item
  It is expected the tracking drift will be linear over the time
  baseline of the individual time series for each (Az, El) in the
  pointing grid.\\
  A higher-order model can also be used, but this will require follow-up
  investigations to understand why the drift is not linear.
\end{enumerate}

}

\begin{tabular}{p{2cm}}
\toprule
Step 3  \\ \hline
\end{tabular}
 Description \\
{\footnotesize
Use the existing data to calculate the RMS for each measured position
for each step, do the RMS for all steps, and calculate the error.

}
\hdashrule[0.5ex]{\textwidth}{1pt}{3mm}
  Expected Result \\
{\footnotesize
\begin{itemize}
\tightlist
\item
  The RMS of the position was calculated.
\item
  The plots show the repeatability per pointing location. Each point was
  reached at least 5 times from a random 3.5 deg offset, and the
  corresponding plot shows a 5-point delta AZ vs. delta EL. Plot a
  circle for the expected value and one for the reached RMS.
\item
  Choose different colors for going forward and backward through the
  pointings.
\end{itemize}

}

\begin{tabular}{p{2cm}}
\toprule
Step 4  \\ \hline
\end{tabular}
 Description \\
{\footnotesize
After all of the pointing locations have been reached 5 times:\\

\begin{itemize}
\tightlist
\item
  Plot the difference between the demand and measured position for El
  anf Az. (Delta EL vs Delta Az)
\item
  Plot the in different colors the forward and backward moving.
\item
  Plot the ring for expected RMS and measured RMS.
\item
  Histogramm for delta Az, delta EL.
\item
  fit the historgram with a gausian (nominal)

  \begin{itemize}
  \tightlist
  \item
    Fitted mean measures the accuracy
  \item
    Fitted width to meet repeatability requirement.
  \end{itemize}
\end{itemize}

}
\hdashrule[0.5ex]{\textwidth}{1pt}{3mm}
  Expected Result \\
{\footnotesize
All values are within 50 arcsec.\\[2\baselineskip]

}

\begin{tabular}{p{2cm}}
\toprule
Step 5  \\ \hline
\end{tabular}
 Description \\
{\footnotesize
Verify the TMA north is in the general north direction.

}
\hdashrule[0.5ex]{\textwidth}{1pt}{3mm}
  Expected Result \\
{\footnotesize
The TMA's north is about 0deg in azimuth.

}

\begin{tabular}{p{2cm}}
\toprule
Step 6  \\ \hline
\end{tabular}
 Description \\
{\footnotesize
Take the data from the accelerometers and derive the acceleration and
jerk values during the tests.

}
\hdashrule[0.5ex]{\textwidth}{1pt}{3mm}
  Test Data \\
 {\footnotesize
\textbf{Note:} It is expected that the TMA operates outside of the
maximum slewing rates. A deviation request may be required from UTE.
However, the data from this test should show that the values are safe
for the TMA and the requirement can be accepted as is.

}
\hdashrule[0.5ex]{\textwidth}{1pt}{3mm}
  Expected Result \\
{\footnotesize
The analysis shows the acceleration and jerk values are within the
maximum slewing rates as defined in LTS-103 Section 2.2.2.1.

}

\paragraph{ LVV-T2703 - TMA Tracking Jitter Validation - Data Analysis }\mbox{}\\

Version \textbf{1}.
Open  \href{https://jira.lsstcorp.org/secure/Tests.jspa#/testCase/LVV-T2703}{\textit{ LVV-T2703 } }
test case in Jira.

Analyze the data taken with the DIMM to characterize the TMA tracking
Jitter.\\
\textbf{\\
Notes from the data acquisition:}

\begin{itemize}
\tightlist
\item
  Using the DIMM. Tracking siderially is obligatory.
\item
  Encoder stream at 50 Hz, Nyquist sampling at 100 -150 Hz.
\item
  The standard frequency is 75Hz smaller frames allow for higher.
\item
  StarTracker and DIMM are mounted and working at the same time.
\item
  Observing strategy:

  \begin{itemize}
  \tightlist
  \item
    Same observing pattern as the forward grid.
  \item
    The object is selected on the StarTracker.
  \item
    The offset to the DIMM is applied.
  \item
    DIMM streaming is used to confirm that the star is centered on the
    DIMM.
  \item
    This is done in parallel with the Slew and Settle data acquisition.
  \end{itemize}
\end{itemize}

\textbf{ Preconditions}:\\
Data for the Tracking Jitter are taken.

Final comment:\\


Detailed steps :

\begin{tabular}{p{2cm}}
\toprule
Step 1  \\ \hline
\end{tabular}
 Description \\
{\footnotesize
\begin{itemize}
\tightlist
\item
  Plot the Az and El vs time while the TMA was tracking.
\item
  Calculate the RMS of the Tracking jitter.
\end{itemize}

}
\hdashrule[0.5ex]{\textwidth}{1pt}{3mm}
  Expected Result \\
{\footnotesize
The RMS of the tracking jitter in Az and El are within the allowed
range.

}

\paragraph{ LVV-T2749 - Slew and Settle analysis }\mbox{}\\

Version \textbf{1}.
Open  \href{https://jira.lsstcorp.org/secure/Tests.jspa#/testCase/LVV-T2749}{\textit{ LVV-T2749 } }
test case in Jira.

Analyze data to characterize the settling after a slew.

\textbf{ Preconditions}:\\
Data were taken during observations for:
\href{https://jira.lsstcorp.org/secure/Tests.jspa\#/testCase/LVV-T2732}{LVV-T2732}

Final comment:\\


Detailed steps :

\begin{tabular}{p{2cm}}
\toprule
Step 1  \\ \hline
\end{tabular}
 Description \\
{\footnotesize
Settling characterization:\\

\begin{enumerate}
\tightlist
\item
  Compute the star centroid from the Fast Camera data
\item
  Plot the star centroid vs. time
\item
  Fit a sin wave to the data
\item
  Determine the damping time
\end{enumerate}

}
\hdashrule[0.5ex]{\textwidth}{1pt}{3mm}
  Test Data \\
 {\footnotesize
Notes for the implementation:\\

\begin{itemize}
\tightlist
\item
  Generic Camera CSC does not do high-speed streaming
\item
  Dave has a version of AVT streaming code that can stream to FITS files
\item
  Dome-seeing monitor code does do the streaming and centroiding
\item
  There maybe an issue with seeing injecting noise to the measurement
\item
  DIMM can in principle save a stream as through VimbaView
\end{itemize}

}
\hdashrule[0.5ex]{\textwidth}{1pt}{3mm}
  Expected Result \\
{\footnotesize
\begin{enumerate}
\tightlist
\item
  Ringing during settlement should be at the 1st fundamental frequency
  of the mount at XXX Hz.
\item
  ~A decaying sin wave is observed.
\item
  The damp time is below XXX sec.
\end{enumerate}

}

\begin{tabular}{p{2cm}}
\toprule
Step 2  \\ \hline
\end{tabular}
 Description \\
{\footnotesize
Accelerometer data analysis:\\[2\baselineskip]Analyze the accelerometer
data:\\
Plot accelerations at EL close to the zenith and close to the Horizon.\\
AZ should be large.\\[2\baselineskip]Note:\\
Main concern is for a 3.5 degree offset.\\[5\baselineskip]

}
\hdashrule[0.5ex]{\textwidth}{1pt}{3mm}
  Expected Result \\
{\footnotesize
The maximum acceleration and jerk at slew are within the
requirement.\\[2\baselineskip]Slew and settle should be within the
requirements.

}

\subsection{Test Cycle LVV-C233 }

Open test cycle {\it \href{https://jira.lsstcorp.org/secure/Tests.jspa#/testrun/LVV-C233}{TMA Pointing and Tracking - Part 3 - Tracking at Random Positions - 50"
- StarTracker}} in Jira.

Test Cycle name: TMA Pointing and Tracking - Part 3 - Tracking at Random Positions - 50"
- StarTracker\\
Status: Not Executed

Requirements verification for the pointing and tracking using the Star
Tracker and the DIMM on a dedicated mounting plate connector to the top
end of the TMA.

\subsubsection{Software Version/Baseline}
Star Tracker software version:\\
Dimm software version:\\
CSC software version:\\
Analysis software repository:

\subsubsection{Configuration}
Not provided.

\subsubsection{Test Cases in LVV-C233 Test Cycle}

\paragraph{ LVV-T2707 - Evening Summit Tailgate Meeting - TMA and Dome Testing Safety Assurance }\mbox{}\\

Version \textbf{2}.
Open  \href{https://jira.lsstcorp.org/secure/Tests.jspa#/testCase/LVV-T2707}{\textit{ LVV-T2707 } }
test case in Jira.

Ensure the safety of observation with the main telescope during
nighttime operations.\\
\textbf{Tailgate Meeting:} Hold a tailgate for the upcoming task with
personnel on the summit working during the night. Go over any relevant
procedures, roles, and
responsibilities.\\[2\baselineskip]\textbf{Note:~}Version two is for
tests that do not involve moving or opening the dome.

\textbf{ Preconditions}:\\
All nonessential personnel has vacated the area.

Final comment:\\


Detailed steps :

\begin{tabular}{p{2cm}}
\toprule
Step 1  \\ \hline
\end{tabular}
 Description \\
{\footnotesize
Verify that there is unrestricted space for the TMA movement.

}
\hdashrule[0.5ex]{\textwidth}{1pt}{3mm}
  Expected Result \\
{\footnotesize

}

\begin{tabular}{p{2cm}}
\toprule
Step 2  \\ \hline
\end{tabular}
 Description \\
{\footnotesize

}
\hdashrule[0.5ex]{\textwidth}{1pt}{3mm}
  Expected Result \\
{\footnotesize

}

\paragraph{ LVV-T2714 - Configure Observatory Environment for Nighttime Operations }\mbox{}\\

Version \textbf{1}.
Open  \href{https://jira.lsstcorp.org/secure/Tests.jspa#/testCase/LVV-T2714}{\textit{ LVV-T2714 } }
test case in Jira.

At the beginning of the night, prepare the observatory for nightly
operations.

\textbf{ Preconditions}:\\
Dome and the TMA, or at least the TMA must available for observations.

Final comment:\\


Detailed steps :

\begin{tabular}{p{2cm}}
\toprule
Step 1  \\ \hline
\end{tabular}
 Description \\
{\footnotesize
\textbf{Telescope preparation:}

\begin{itemize}
\tightlist
\item
  Remove the caps on top of the StarTracker telescopes and the DIMM.
\item
  Check the instrument's health status by taking a test image.
\end{itemize}

}
\hdashrule[0.5ex]{\textwidth}{1pt}{3mm}
  Expected Result \\
{\footnotesize
The caps are removed.\\
The test image is taken and stored correspondingly.

}

\begin{tabular}{p{2cm}}
\toprule
Step 2  \\ \hline
\end{tabular}
 Description \\
{\footnotesize
\textbf{Calibration images:}\\
Not needed at the moment, to be included if data analysis reveals the
need.

\begin{itemize}
\tightlist
\item
  Take 10 ``darks'' with the StarTracker and the DIMM instruments. Use
  the same exposure time as for the images.
\item
  Take 10 ``sky flats'' with the StarTracker and the DIMM instruments.
\end{itemize}

}
\hdashrule[0.5ex]{\textwidth}{1pt}{3mm}
  Expected Result \\
{\footnotesize
The Darks and flats are stored in the expected location.

}

\begin{tabular}{p{2cm}}
\toprule
Step 3  \\ \hline
\end{tabular}
 Description \\
{\footnotesize
\textbf{Auxillary systems~nighttime~preparations:}

\begin{itemize}
\tightlist
\item
  Switch off the UMA in the afternoon.
\end{itemize}

}
\hdashrule[0.5ex]{\textwidth}{1pt}{3mm}
  Expected Result \\
{\footnotesize
The UMA is switched off.

}

\begin{tabular}{p{2cm}}
\toprule
Step 4  \\ \hline
\end{tabular}
 Description \\
{\footnotesize
\textbf{Night logging page:}\\
Start the night log similar to the AuxTel night
log:\\[2\baselineskip]https://confluence.lsstcorp.org/display/LSSTCOM/Night+Logs\\[2\baselineskip]

}
\hdashrule[0.5ex]{\textwidth}{1pt}{3mm}
  Expected Result \\
{\footnotesize
Page created with template information.

}

\begin{tabular}{p{2cm}}
\toprule
Step 5  \\ \hline
\end{tabular}
 Description \\
{\footnotesize
\textbf{TMA preparation}

\begin{itemize}
\tightlist
\item
  Check the Oil Supply System (OSS) on the EUI
\item
  Follow the attached manual to startup the TMA.
\end{itemize}

}
\hdashrule[0.5ex]{\textwidth}{1pt}{3mm}
  Expected Result \\
{\footnotesize
The OSS is operational:

}

\begin{tabular}{p{2cm}}
\toprule
Step 6  \\ \hline
\end{tabular}
 Description \\
{\footnotesize
\textbf{CSC activation:}\\[2\baselineskip]Use L.O.V.E to bring the CSC
to the enabled state.\\[2\baselineskip]

}
\hdashrule[0.5ex]{\textwidth}{1pt}{3mm}
  Expected Result \\
{\footnotesize
All needed CSCs are in the enabled state.

}

\paragraph{ LVV-T2740 - StarTracker Pointing and Tracking Test - Tracking at Random Positions }\mbox{}\\

Version \textbf{1}.
Open  \href{https://jira.lsstcorp.org/secure/Tests.jspa#/testCase/LVV-T2740}{\textit{ LVV-T2740 } }
test case in Jira.

Collect data with the StarTracker following the reverse azimuth pattern
with respect to the forward Azimuth pattern
(https://jira.lsstcorp.org/secure/Tests.jspa\#/testCase/LVV-T2731)\\
The azimuth and elevation pattern here is randomized. The values used
are coming from the pool of 270, 180, 90, 0, -90, -180, -270 deg Azimuth
angles and 15, 45, 86,5 deg elevation angles. Optionally Azimuth angels
at Elevation 75 can be added.\\[2\baselineskip]This test

\begin{itemize}
\tightlist
\item
  is foreseen the third of four tests
\item
  takes about one summer night (7 hours) in the full version and a bit
  more than 5hours in the shortened version. ~
\end{itemize}

\textbf{ Preconditions}:\\
TMA and Dome are controlable from the CSC.

Final comment:\\


Detailed steps :

\begin{tabular}{p{2cm}}
\toprule
Step 1  \\ \hline
\end{tabular}
 Description \\
{\footnotesize
\textbf{Point} \textbf{the Dome:}\\
Command the Dome to {Pointing 1}⁠ to {270}⁠

}
\hdashrule[0.5ex]{\textwidth}{1pt}{3mm}
  Expected Result \\
{\footnotesize
The Dome starts moving.

}

\begin{tabular}{p{2cm}}
\toprule
Step 2  \\ \hline
\end{tabular}
 Description \\
{\footnotesize
\textbf{Point} \textbf{the Dome:}\\
Command the Dome to {Pointing 13}⁠ to {90}⁠

}
\hdashrule[0.5ex]{\textwidth}{1pt}{3mm}
  Expected Result \\
{\footnotesize
The Dome starts moving.

}

\begin{tabular}{p{2cm}}
\toprule
Step 3  \\ \hline
\end{tabular}
 Description \\
{\footnotesize
Wait for the Dome to reach the commanded position.

}
\hdashrule[0.5ex]{\textwidth}{1pt}{3mm}
  Expected Result \\
{\footnotesize
The \emph{MTDome\_logevent\_azMotion} and
\emph{MTDome\_logevent\_elMotion} inPosition parameter = true.

}

\begin{tabular}{p{2cm}}
\toprule
Step 4  \\ \hline
\end{tabular}
 Description \\
{\footnotesize
\textbf{Point the TMA}\\
Command the TMA to {Pointing 13}⁠ at {90}⁠ , {45}⁠ .

}
\hdashrule[0.5ex]{\textwidth}{1pt}{3mm}
  Expected Result \\
{\footnotesize
The TMA starts moving

}

\begin{tabular}{p{2cm}}
\toprule
Step 5  \\ \hline
\end{tabular}
 Description \\
{\footnotesize
Wait for the TMA to reach the commanded position.

}
\hdashrule[0.5ex]{\textwidth}{1pt}{3mm}
  Expected Result \\
{\footnotesize
The \emph{MTMount\_logevent\_azimuthInPosition} and
\emph{MTMount\_logevent\_elevationInPosition} inPosition parameter =
true.

}

\begin{tabular}{p{2cm}}
\toprule
Step 6  \\ \hline
\end{tabular}
 Description \\
{\footnotesize
\textbf{Image preparation}\\
If the preparation to take images takes longer than 10sec, do
repositioning to target {{{90}⁠}}, {{{45}⁠~}}.

}
\hdashrule[0.5ex]{\textwidth}{1pt}{3mm}
  Expected Result \\
{\footnotesize
TMA reaches the commanded position.

}

\begin{tabular}{p{2cm}}
\toprule
Step 7  \\ \hline
\end{tabular}
 Description \\
{\footnotesize
\textbf{Track position and take images}\\[2\baselineskip]Take a
StarTracker image with 10s exposure time.\\[2\baselineskip]If the time
the available:

\begin{itemize}
\tightlist
\item
  Track a position for 10 min and take StarTracker images.
\end{itemize}

}
\hdashrule[0.5ex]{\textwidth}{1pt}{3mm}
  Expected Result \\
{\footnotesize
\begin{itemize}
\tightlist
\item
  If time is available: The TMA is tracking a given position for 10 min
  and taking images.
\item
  At least one image is successfully taken with the StarTracker.
\end{itemize}

}

\begin{tabular}{p{2cm}}
\toprule
Step 8  \\ \hline
\end{tabular}
 Description \\
{\footnotesize
\textbf{On-the-fly Image Quality Check}\\
While tracking and taking images, check the images on RubinTV for an
astrometric solution.

}
\hdashrule[0.5ex]{\textwidth}{1pt}{3mm}
  Expected Result \\
{\footnotesize
RubinTV is showing an astrometric solution.

}

\begin{tabular}{p{2cm}}
\toprule
Step 9  \\ \hline
\end{tabular}
 Description \\
{\footnotesize
\textbf{Offline analysis results}\\
Offline analysis in Test case
\href{https://jira.lsstcorp.org/secure/Tests.jspa\#/testCase/LVV-T2739}{LVV-T2739}
Says that we do not have sufficient image quality.

}
\hdashrule[0.5ex]{\textwidth}{1pt}{3mm}
  Expected Result \\
{\footnotesize
Image quality is sufficient.

}

\begin{tabular}{p{2cm}}
\toprule
Step 10  \\ \hline
\end{tabular}
 Description \\
{\footnotesize
\textbf{Point} \textbf{the Dome:}\\
Command the Dome to {Pointing 15}⁠ to {-270}⁠

}
\hdashrule[0.5ex]{\textwidth}{1pt}{3mm}
  Expected Result \\
{\footnotesize
The Dome starts moving.

}

\begin{tabular}{p{2cm}}
\toprule
Step 11  \\ \hline
\end{tabular}
 Description \\
{\footnotesize
Wait for the Dome to reach the commanded position.

}
\hdashrule[0.5ex]{\textwidth}{1pt}{3mm}
  Expected Result \\
{\footnotesize
The \emph{MTDome\_logevent\_azMotion} and
\emph{MTDome\_logevent\_elMotion} inPosition parameter = true.

}

\begin{tabular}{p{2cm}}
\toprule
Step 12  \\ \hline
\end{tabular}
 Description \\
{\footnotesize
\textbf{Point the TMA}\\
Command the TMA to {Pointing 15}⁠ at {-270}⁠ , {86.5}⁠ .

}
\hdashrule[0.5ex]{\textwidth}{1pt}{3mm}
  Expected Result \\
{\footnotesize
The TMA starts moving

}

\begin{tabular}{p{2cm}}
\toprule
Step 13  \\ \hline
\end{tabular}
 Description \\
{\footnotesize
Wait for the TMA to reach the commanded position.

}
\hdashrule[0.5ex]{\textwidth}{1pt}{3mm}
  Expected Result \\
{\footnotesize
The \emph{MTMount\_logevent\_azimuthInPosition} and
\emph{MTMount\_logevent\_elevationInPosition} inPosition parameter =
true.

}

\begin{tabular}{p{2cm}}
\toprule
Step 14  \\ \hline
\end{tabular}
 Description \\
{\footnotesize
\textbf{Image preparation}\\
If the preparation to take images takes longer than 10sec, do
repositioning to target {{{-270}⁠}}, {{{86.5}⁠~}}.

}
\hdashrule[0.5ex]{\textwidth}{1pt}{3mm}
  Expected Result \\
{\footnotesize
TMA reaches the commanded position.

}

\begin{tabular}{p{2cm}}
\toprule
Step 15  \\ \hline
\end{tabular}
 Description \\
{\footnotesize
\textbf{Track position and take images}\\[2\baselineskip]Take a
StarTracker image with 10s exposure time.\\[2\baselineskip]If the time
the available:

\begin{itemize}
\tightlist
\item
  Track a position for 10 min and take StarTracker images.
\end{itemize}

}
\hdashrule[0.5ex]{\textwidth}{1pt}{3mm}
  Expected Result \\
{\footnotesize
\begin{itemize}
\tightlist
\item
  If time is available: The TMA is tracking a given position for 10 min
  and taking images.
\item
  At least one image is successfully taken with the StarTracker.
\end{itemize}

}

\begin{tabular}{p{2cm}}
\toprule
Step 16  \\ \hline
\end{tabular}
 Description \\
{\footnotesize
\textbf{On-the-fly Image Quality Check}\\
While tracking and taking images, check the images on RubinTV for an
astrometric solution.

}
\hdashrule[0.5ex]{\textwidth}{1pt}{3mm}
  Expected Result \\
{\footnotesize
RubinTV is showing an astrometric solution.

}

\begin{tabular}{p{2cm}}
\toprule
Step 17  \\ \hline
\end{tabular}
 Description \\
{\footnotesize
\textbf{Offline analysis results}\\
Offline analysis in Test case
\href{https://jira.lsstcorp.org/secure/Tests.jspa\#/testCase/LVV-T2739}{LVV-T2739}
Says that we do not have sufficient image quality.

}
\hdashrule[0.5ex]{\textwidth}{1pt}{3mm}
  Expected Result \\
{\footnotesize
Image quality is sufficient.

}

\begin{tabular}{p{2cm}}
\toprule
Step 18  \\ \hline
\end{tabular}
 Description \\
{\footnotesize
\textbf{Point} \textbf{the Dome:}\\
Command the Dome to {Pointing 16}⁠ to {180}⁠

}
\hdashrule[0.5ex]{\textwidth}{1pt}{3mm}
  Expected Result \\
{\footnotesize
The Dome starts moving.

}

\begin{tabular}{p{2cm}}
\toprule
Step 19  \\ \hline
\end{tabular}
 Description \\
{\footnotesize
Wait for the Dome to reach the commanded position.

}
\hdashrule[0.5ex]{\textwidth}{1pt}{3mm}
  Expected Result \\
{\footnotesize
The \emph{MTDome\_logevent\_azMotion} and
\emph{MTDome\_logevent\_elMotion} inPosition parameter = true.

}

\begin{tabular}{p{2cm}}
\toprule
Step 20  \\ \hline
\end{tabular}
 Description \\
{\footnotesize
\textbf{Point the TMA}\\
Command the TMA to {Pointing 16}⁠ at {180}⁠ , {45}⁠ .

}
\hdashrule[0.5ex]{\textwidth}{1pt}{3mm}
  Expected Result \\
{\footnotesize
The TMA starts moving

}

\begin{tabular}{p{2cm}}
\toprule
Step 21  \\ \hline
\end{tabular}
 Description \\
{\footnotesize
Wait for the TMA to reach the commanded position.

}
\hdashrule[0.5ex]{\textwidth}{1pt}{3mm}
  Expected Result \\
{\footnotesize
The \emph{MTMount\_logevent\_azimuthInPosition} and
\emph{MTMount\_logevent\_elevationInPosition} inPosition parameter =
true.

}

\begin{tabular}{p{2cm}}
\toprule
Step 22  \\ \hline
\end{tabular}
 Description \\
{\footnotesize
\textbf{Image preparation}\\
If the preparation to take images takes longer than 10sec, do
repositioning to target {{{180}⁠}}, {{{45}⁠~}}.

}
\hdashrule[0.5ex]{\textwidth}{1pt}{3mm}
  Expected Result \\
{\footnotesize
TMA reaches the commanded position.

}

\begin{tabular}{p{2cm}}
\toprule
Step 23  \\ \hline
\end{tabular}
 Description \\
{\footnotesize
\textbf{Track position and take images}\\[2\baselineskip]Take a
StarTracker image with 10s exposure time.\\[2\baselineskip]If the time
the available:

\begin{itemize}
\tightlist
\item
  Track a position for 10 min and take StarTracker images.
\end{itemize}

}
\hdashrule[0.5ex]{\textwidth}{1pt}{3mm}
  Expected Result \\
{\footnotesize
\begin{itemize}
\tightlist
\item
  If time is available: The TMA is tracking a given position for 10 min
  and taking images.
\item
  At least one image is successfully taken with the StarTracker.
\end{itemize}

}

\begin{tabular}{p{2cm}}
\toprule
Step 24  \\ \hline
\end{tabular}
 Description \\
{\footnotesize
\textbf{On-the-fly Image Quality Check}\\
While tracking and taking images, check the images on RubinTV for an
astrometric solution.

}
\hdashrule[0.5ex]{\textwidth}{1pt}{3mm}
  Expected Result \\
{\footnotesize
RubinTV is showing an astrometric solution.

}

\begin{tabular}{p{2cm}}
\toprule
Step 25  \\ \hline
\end{tabular}
 Description \\
{\footnotesize
\textbf{Offline analysis results}\\
Offline analysis in Test case
\href{https://jira.lsstcorp.org/secure/Tests.jspa\#/testCase/LVV-T2739}{LVV-T2739}
Says that we do not have sufficient image quality.

}
\hdashrule[0.5ex]{\textwidth}{1pt}{3mm}
  Expected Result \\
{\footnotesize
Image quality is sufficient.

}

\begin{tabular}{p{2cm}}
\toprule
Step 26  \\ \hline
\end{tabular}
 Description \\
{\footnotesize
\textbf{Point} \textbf{the Dome:}\\
Command the Dome to {Pointing 17}⁠ to {0}⁠

}
\hdashrule[0.5ex]{\textwidth}{1pt}{3mm}
  Expected Result \\
{\footnotesize
The Dome starts moving.

}

\begin{tabular}{p{2cm}}
\toprule
Step 27  \\ \hline
\end{tabular}
 Description \\
{\footnotesize
Wait for the Dome to reach the commanded position.

}
\hdashrule[0.5ex]{\textwidth}{1pt}{3mm}
  Expected Result \\
{\footnotesize
The \emph{MTDome\_logevent\_azMotion} and
\emph{MTDome\_logevent\_elMotion} inPosition parameter = true.

}

\begin{tabular}{p{2cm}}
\toprule
Step 28  \\ \hline
\end{tabular}
 Description \\
{\footnotesize
\textbf{Point the TMA}\\
Command the TMA to {Pointing 17}⁠ at {0}⁠ , {15}⁠ .

}
\hdashrule[0.5ex]{\textwidth}{1pt}{3mm}
  Expected Result \\
{\footnotesize
The TMA starts moving

}

\begin{tabular}{p{2cm}}
\toprule
Step 29  \\ \hline
\end{tabular}
 Description \\
{\footnotesize
Wait for the TMA to reach the commanded position.

}
\hdashrule[0.5ex]{\textwidth}{1pt}{3mm}
  Expected Result \\
{\footnotesize
The \emph{MTMount\_logevent\_azimuthInPosition} and
\emph{MTMount\_logevent\_elevationInPosition} inPosition parameter =
true.

}

\begin{tabular}{p{2cm}}
\toprule
Step 30  \\ \hline
\end{tabular}
 Description \\
{\footnotesize
\textbf{Image preparation}\\
If the preparation to take images takes longer than 10sec, do
repositioning to target {{{0}⁠}}, {{{15}⁠~}}.

}
\hdashrule[0.5ex]{\textwidth}{1pt}{3mm}
  Expected Result \\
{\footnotesize
TMA reaches the commanded position.

}

\begin{tabular}{p{2cm}}
\toprule
Step 31  \\ \hline
\end{tabular}
 Description \\
{\footnotesize
\textbf{Track position and take images}\\[2\baselineskip]Take a
StarTracker image with 10s exposure time.\\[2\baselineskip]If the time
the available:

\begin{itemize}
\tightlist
\item
  Track a position for 10 min and take StarTracker images.
\end{itemize}

}
\hdashrule[0.5ex]{\textwidth}{1pt}{3mm}
  Expected Result \\
{\footnotesize
\begin{itemize}
\tightlist
\item
  If time is available: The TMA is tracking a given position for 10 min
  and taking images.
\item
  At least one image is successfully taken with the StarTracker.
\end{itemize}

}

\begin{tabular}{p{2cm}}
\toprule
Step 32  \\ \hline
\end{tabular}
 Description \\
{\footnotesize
\textbf{On-the-fly Image Quality Check}\\
While tracking and taking images, check the images on RubinTV for an
astrometric solution.

}
\hdashrule[0.5ex]{\textwidth}{1pt}{3mm}
  Expected Result \\
{\footnotesize
RubinTV is showing an astrometric solution.

}

\begin{tabular}{p{2cm}}
\toprule
Step 33  \\ \hline
\end{tabular}
 Description \\
{\footnotesize
\textbf{Offline analysis results}\\
Offline analysis in Test case
\href{https://jira.lsstcorp.org/secure/Tests.jspa\#/testCase/LVV-T2739}{LVV-T2739}
Says that we do not have sufficient image quality.

}
\hdashrule[0.5ex]{\textwidth}{1pt}{3mm}
  Expected Result \\
{\footnotesize
Image quality is sufficient.

}

\begin{tabular}{p{2cm}}
\toprule
Step 34  \\ \hline
\end{tabular}
 Description \\
{\footnotesize
\textbf{Point} \textbf{the Dome:}\\
Command the Dome to {Pointing 18}⁠ to {-90}⁠

}
\hdashrule[0.5ex]{\textwidth}{1pt}{3mm}
  Expected Result \\
{\footnotesize
The Dome starts moving.

}

\begin{tabular}{p{2cm}}
\toprule
Step 35  \\ \hline
\end{tabular}
 Description \\
{\footnotesize
Wait for the Dome to reach the commanded position.

}
\hdashrule[0.5ex]{\textwidth}{1pt}{3mm}
  Expected Result \\
{\footnotesize
The \emph{MTDome\_logevent\_azMotion} and
\emph{MTDome\_logevent\_elMotion} inPosition parameter = true.

}

\begin{tabular}{p{2cm}}
\toprule
Step 36  \\ \hline
\end{tabular}
 Description \\
{\footnotesize
\textbf{Point the TMA}\\
Command the TMA to {Pointing 18}⁠ at {-90}⁠ , {86.5}⁠ .

}
\hdashrule[0.5ex]{\textwidth}{1pt}{3mm}
  Expected Result \\
{\footnotesize
The TMA starts moving

}

\begin{tabular}{p{2cm}}
\toprule
Step 37  \\ \hline
\end{tabular}
 Description \\
{\footnotesize
Wait for the TMA to reach the commanded position.

}
\hdashrule[0.5ex]{\textwidth}{1pt}{3mm}
  Expected Result \\
{\footnotesize
The \emph{MTMount\_logevent\_azimuthInPosition} and
\emph{MTMount\_logevent\_elevationInPosition} inPosition parameter =
true.

}

\begin{tabular}{p{2cm}}
\toprule
Step 38  \\ \hline
\end{tabular}
 Description \\
{\footnotesize
\textbf{Image preparation}\\
If the preparation to take images takes longer than 10sec, do
repositioning to target {{{-90}⁠}}, {{{86.5}⁠~}}.

}
\hdashrule[0.5ex]{\textwidth}{1pt}{3mm}
  Expected Result \\
{\footnotesize
TMA reaches the commanded position.

}

\begin{tabular}{p{2cm}}
\toprule
Step 39  \\ \hline
\end{tabular}
 Description \\
{\footnotesize
\textbf{Track position and take images}\\[2\baselineskip]Take a
StarTracker image with 10s exposure time.\\[2\baselineskip]If the time
the available:

\begin{itemize}
\tightlist
\item
  Track a position for 10 min and take StarTracker images.
\end{itemize}

}
\hdashrule[0.5ex]{\textwidth}{1pt}{3mm}
  Expected Result \\
{\footnotesize
\begin{itemize}
\tightlist
\item
  If time is available: The TMA is tracking a given position for 10 min
  and taking images.
\item
  At least one image is successfully taken with the StarTracker.
\end{itemize}

}

\begin{tabular}{p{2cm}}
\toprule
Step 40  \\ \hline
\end{tabular}
 Description \\
{\footnotesize
\textbf{On-the-fly Image Quality Check}\\
While tracking and taking images, check the images on RubinTV for an
astrometric solution.

}
\hdashrule[0.5ex]{\textwidth}{1pt}{3mm}
  Expected Result \\
{\footnotesize
RubinTV is showing an astrometric solution.

}

\begin{tabular}{p{2cm}}
\toprule
Step 41  \\ \hline
\end{tabular}
 Description \\
{\footnotesize
\textbf{Offline analysis results}\\
Offline analysis in Test case
\href{https://jira.lsstcorp.org/secure/Tests.jspa\#/testCase/LVV-T2739}{LVV-T2739}
Says that we do not have sufficient image quality.

}
\hdashrule[0.5ex]{\textwidth}{1pt}{3mm}
  Expected Result \\
{\footnotesize
Image quality is sufficient.

}

\begin{tabular}{p{2cm}}
\toprule
Step 42  \\ \hline
\end{tabular}
 Description \\
{\footnotesize
\textbf{Point} \textbf{the Dome:}\\
Command the Dome to {Pointing 20}⁠ to {270}⁠

}
\hdashrule[0.5ex]{\textwidth}{1pt}{3mm}
  Expected Result \\
{\footnotesize
The Dome starts moving.

}

\begin{tabular}{p{2cm}}
\toprule
Step 43  \\ \hline
\end{tabular}
 Description \\
{\footnotesize
Wait for the Dome to reach the commanded position.

}
\hdashrule[0.5ex]{\textwidth}{1pt}{3mm}
  Expected Result \\
{\footnotesize
The \emph{MTDome\_logevent\_azMotion} and
\emph{MTDome\_logevent\_elMotion} inPosition parameter = true.

}

\begin{tabular}{p{2cm}}
\toprule
Step 44  \\ \hline
\end{tabular}
 Description \\
{\footnotesize
\textbf{Point the TMA}\\
Command the TMA to {Pointing 20}⁠ at {270}⁠ , {45}⁠ .

}
\hdashrule[0.5ex]{\textwidth}{1pt}{3mm}
  Expected Result \\
{\footnotesize
The TMA starts moving

}

\begin{tabular}{p{2cm}}
\toprule
Step 45  \\ \hline
\end{tabular}
 Description \\
{\footnotesize
Wait for the TMA to reach the commanded position.

}
\hdashrule[0.5ex]{\textwidth}{1pt}{3mm}
  Expected Result \\
{\footnotesize
The \emph{MTMount\_logevent\_azimuthInPosition} and
\emph{MTMount\_logevent\_elevationInPosition} inPosition parameter =
true.

}

\begin{tabular}{p{2cm}}
\toprule
Step 46  \\ \hline
\end{tabular}
 Description \\
{\footnotesize
\textbf{Image preparation}\\
If the preparation to take images takes longer than 10sec, do
repositioning to target {{{270}⁠}}, {{{45}⁠~}}.

}
\hdashrule[0.5ex]{\textwidth}{1pt}{3mm}
  Expected Result \\
{\footnotesize
TMA reaches the commanded position.

}

\begin{tabular}{p{2cm}}
\toprule
Step 47  \\ \hline
\end{tabular}
 Description \\
{\footnotesize
\textbf{Track position and take images}\\[2\baselineskip]Take a
StarTracker image with 10s exposure time.\\[2\baselineskip]If the time
the available:

\begin{itemize}
\tightlist
\item
  Track a position for 10 min and take StarTracker images.
\end{itemize}

}
\hdashrule[0.5ex]{\textwidth}{1pt}{3mm}
  Expected Result \\
{\footnotesize
\begin{itemize}
\tightlist
\item
  If time is available: The TMA is tracking a given position for 10 min
  and taking images.
\item
  At least one image is successfully taken with the StarTracker.
\end{itemize}

}

\begin{tabular}{p{2cm}}
\toprule
Step 48  \\ \hline
\end{tabular}
 Description \\
{\footnotesize
\textbf{On-the-fly Image Quality Check}\\
While tracking and taking images, check the images on RubinTV for an
astrometric solution.

}
\hdashrule[0.5ex]{\textwidth}{1pt}{3mm}
  Expected Result \\
{\footnotesize
RubinTV is showing an astrometric solution.

}

\begin{tabular}{p{2cm}}
\toprule
Step 49  \\ \hline
\end{tabular}
 Description \\
{\footnotesize
\textbf{Offline analysis results}\\
Offline analysis in Test case
\href{https://jira.lsstcorp.org/secure/Tests.jspa\#/testCase/LVV-T2739}{LVV-T2739}
Says that we do not have sufficient image quality.

}
\hdashrule[0.5ex]{\textwidth}{1pt}{3mm}
  Expected Result \\
{\footnotesize
Image quality is sufficient.

}

\begin{tabular}{p{2cm}}
\toprule
Step 50  \\ \hline
\end{tabular}
 Description \\
{\footnotesize
\textbf{Point} \textbf{the Dome:}\\
Command the Dome to {Pointing 21}⁠ to {-180}⁠

}
\hdashrule[0.5ex]{\textwidth}{1pt}{3mm}
  Expected Result \\
{\footnotesize
The Dome starts moving.

}

\begin{tabular}{p{2cm}}
\toprule
Step 51  \\ \hline
\end{tabular}
 Description \\
{\footnotesize
Wait for the Dome to reach the commanded position.

}
\hdashrule[0.5ex]{\textwidth}{1pt}{3mm}
  Expected Result \\
{\footnotesize
The \emph{MTDome\_logevent\_azMotion} and
\emph{MTDome\_logevent\_elMotion} inPosition parameter = true.

}

\begin{tabular}{p{2cm}}
\toprule
Step 52  \\ \hline
\end{tabular}
 Description \\
{\footnotesize
\textbf{Point the TMA}\\
Command the TMA to {Pointing 21}⁠ at {-180}⁠ , {15}⁠ .

}
\hdashrule[0.5ex]{\textwidth}{1pt}{3mm}
  Expected Result \\
{\footnotesize
The TMA starts moving

}

\begin{tabular}{p{2cm}}
\toprule
Step 53  \\ \hline
\end{tabular}
 Description \\
{\footnotesize
Wait for the TMA to reach the commanded position.

}
\hdashrule[0.5ex]{\textwidth}{1pt}{3mm}
  Expected Result \\
{\footnotesize
The \emph{MTMount\_logevent\_azimuthInPosition} and
\emph{MTMount\_logevent\_elevationInPosition} inPosition parameter =
true.

}

\begin{tabular}{p{2cm}}
\toprule
Step 54  \\ \hline
\end{tabular}
 Description \\
{\footnotesize
\textbf{Image preparation}\\
If the preparation to take images takes longer than 10sec, do
repositioning to target {{{-180}⁠}}, {{{15}⁠~}}.

}
\hdashrule[0.5ex]{\textwidth}{1pt}{3mm}
  Expected Result \\
{\footnotesize
TMA reaches the commanded position.

}

\begin{tabular}{p{2cm}}
\toprule
Step 55  \\ \hline
\end{tabular}
 Description \\
{\footnotesize
\textbf{Track position and take images}\\[2\baselineskip]Take a
StarTracker image with 10s exposure time.\\[2\baselineskip]If the time
the available:

\begin{itemize}
\tightlist
\item
  Track a position for 10 min and take StarTracker images.
\end{itemize}

}
\hdashrule[0.5ex]{\textwidth}{1pt}{3mm}
  Expected Result \\
{\footnotesize
\begin{itemize}
\tightlist
\item
  If time is available: The TMA is tracking a given position for 10 min
  and taking images.
\item
  At least one image is successfully taken with the StarTracker.
\end{itemize}

}

\begin{tabular}{p{2cm}}
\toprule
Step 56  \\ \hline
\end{tabular}
 Description \\
{\footnotesize
\textbf{On-the-fly Image Quality Check}\\
While tracking and taking images, check the images on RubinTV for an
astrometric solution.

}
\hdashrule[0.5ex]{\textwidth}{1pt}{3mm}
  Expected Result \\
{\footnotesize
RubinTV is showing an astrometric solution.

}

\begin{tabular}{p{2cm}}
\toprule
Step 57  \\ \hline
\end{tabular}
 Description \\
{\footnotesize
\textbf{Offline analysis results}\\
Offline analysis in Test case
\href{https://jira.lsstcorp.org/secure/Tests.jspa\#/testCase/LVV-T2739}{LVV-T2739}
Says that we do not have sufficient image quality.

}
\hdashrule[0.5ex]{\textwidth}{1pt}{3mm}
  Expected Result \\
{\footnotesize
Image quality is sufficient.

}

\begin{tabular}{p{2cm}}
\toprule
Step 58  \\ \hline
\end{tabular}
 Description \\
{\footnotesize
\textbf{Point} \textbf{the Dome:}\\
Command the Dome to {Pointing 23}⁠ to {90}⁠

}
\hdashrule[0.5ex]{\textwidth}{1pt}{3mm}
  Expected Result \\
{\footnotesize
The Dome starts moving.

}

\begin{tabular}{p{2cm}}
\toprule
Step 59  \\ \hline
\end{tabular}
 Description \\
{\footnotesize
Wait for the Dome to reach the commanded position.

}
\hdashrule[0.5ex]{\textwidth}{1pt}{3mm}
  Expected Result \\
{\footnotesize
The \emph{MTDome\_logevent\_azMotion} and
\emph{MTDome\_logevent\_elMotion} inPosition parameter = true.

}

\begin{tabular}{p{2cm}}
\toprule
Step 60  \\ \hline
\end{tabular}
 Description \\
{\footnotesize
\textbf{Point the TMA}\\
Command the TMA to {Pointing 23}⁠ at {90}⁠ , {15}⁠ .

}
\hdashrule[0.5ex]{\textwidth}{1pt}{3mm}
  Expected Result \\
{\footnotesize
The TMA starts moving

}

\begin{tabular}{p{2cm}}
\toprule
Step 61  \\ \hline
\end{tabular}
 Description \\
{\footnotesize
Wait for the TMA to reach the commanded position.

}
\hdashrule[0.5ex]{\textwidth}{1pt}{3mm}
  Expected Result \\
{\footnotesize
The \emph{MTMount\_logevent\_azimuthInPosition} and
\emph{MTMount\_logevent\_elevationInPosition} inPosition parameter =
true.

}

\begin{tabular}{p{2cm}}
\toprule
Step 62  \\ \hline
\end{tabular}
 Description \\
{\footnotesize
\textbf{Image preparation}\\
If the preparation to take images takes longer than 10sec, do
repositioning to target {{{90}⁠}}, {{{15}⁠~}}.

}
\hdashrule[0.5ex]{\textwidth}{1pt}{3mm}
  Expected Result \\
{\footnotesize
TMA reaches the commanded position.

}

\begin{tabular}{p{2cm}}
\toprule
Step 63  \\ \hline
\end{tabular}
 Description \\
{\footnotesize
\textbf{Track position and take images}\\[2\baselineskip]Take a
StarTracker image with 10s exposure time.\\[2\baselineskip]If the time
the available:

\begin{itemize}
\tightlist
\item
  Track a position for 10 min and take StarTracker images.
\end{itemize}

}
\hdashrule[0.5ex]{\textwidth}{1pt}{3mm}
  Expected Result \\
{\footnotesize
\begin{itemize}
\tightlist
\item
  If time is available: The TMA is tracking a given position for 10 min
  and taking images.
\item
  At least one image is successfully taken with the StarTracker.
\end{itemize}

}

\begin{tabular}{p{2cm}}
\toprule
Step 64  \\ \hline
\end{tabular}
 Description \\
{\footnotesize
\textbf{On-the-fly Image Quality Check}\\
While tracking and taking images, check the images on RubinTV for an
astrometric solution.

}
\hdashrule[0.5ex]{\textwidth}{1pt}{3mm}
  Expected Result \\
{\footnotesize
RubinTV is showing an astrometric solution.

}

\begin{tabular}{p{2cm}}
\toprule
Step 65  \\ \hline
\end{tabular}
 Description \\
{\footnotesize
\textbf{Offline analysis results}\\
Offline analysis in Test case
\href{https://jira.lsstcorp.org/secure/Tests.jspa\#/testCase/LVV-T2739}{LVV-T2739}
Says that we do not have sufficient image quality.

}
\hdashrule[0.5ex]{\textwidth}{1pt}{3mm}
  Expected Result \\
{\footnotesize
Image quality is sufficient.

}

\begin{tabular}{p{2cm}}
\toprule
Step 66  \\ \hline
\end{tabular}
 Description \\
{\footnotesize
\textbf{Point} \textbf{the Dome:}\\
Command the Dome to {Pointing 24}⁠ to {-270}⁠

}
\hdashrule[0.5ex]{\textwidth}{1pt}{3mm}
  Expected Result \\
{\footnotesize
The Dome starts moving.

}

\begin{tabular}{p{2cm}}
\toprule
Step 67  \\ \hline
\end{tabular}
 Description \\
{\footnotesize
Wait for the Dome to reach the commanded position.

}
\hdashrule[0.5ex]{\textwidth}{1pt}{3mm}
  Expected Result \\
{\footnotesize
The \emph{MTDome\_logevent\_azMotion} and
\emph{MTDome\_logevent\_elMotion} inPosition parameter = true.

}

\begin{tabular}{p{2cm}}
\toprule
Step 68  \\ \hline
\end{tabular}
 Description \\
{\footnotesize
\textbf{Point the TMA}\\
Command the TMA to {Pointing 24}⁠ at {-270}⁠ , {45}⁠ .

}
\hdashrule[0.5ex]{\textwidth}{1pt}{3mm}
  Expected Result \\
{\footnotesize
The TMA starts moving

}

\begin{tabular}{p{2cm}}
\toprule
Step 69  \\ \hline
\end{tabular}
 Description \\
{\footnotesize
Wait for the TMA to reach the commanded position.

}
\hdashrule[0.5ex]{\textwidth}{1pt}{3mm}
  Expected Result \\
{\footnotesize
The \emph{MTMount\_logevent\_azimuthInPosition} and
\emph{MTMount\_logevent\_elevationInPosition} inPosition parameter =
true.

}

\begin{tabular}{p{2cm}}
\toprule
Step 70  \\ \hline
\end{tabular}
 Description \\
{\footnotesize
\textbf{Image preparation}\\
If the preparation to take images takes longer than 10sec, do
repositioning to target {{{-270}⁠}}, {{{45}⁠~}}.

}
\hdashrule[0.5ex]{\textwidth}{1pt}{3mm}
  Expected Result \\
{\footnotesize
TMA reaches the commanded position.

}

\begin{tabular}{p{2cm}}
\toprule
Step 71  \\ \hline
\end{tabular}
 Description \\
{\footnotesize
\textbf{Track position and take images}\\[2\baselineskip]Take a
StarTracker image with 10s exposure time.\\[2\baselineskip]If the time
the available:

\begin{itemize}
\tightlist
\item
  Track a position for 10 min and take StarTracker images.
\end{itemize}

}
\hdashrule[0.5ex]{\textwidth}{1pt}{3mm}
  Expected Result \\
{\footnotesize
\begin{itemize}
\tightlist
\item
  If time is available: The TMA is tracking a given position for 10 min
  and taking images.
\item
  At least one image is successfully taken with the StarTracker.
\end{itemize}

}

\begin{tabular}{p{2cm}}
\toprule
Step 72  \\ \hline
\end{tabular}
 Description \\
{\footnotesize
\textbf{On-the-fly Image Quality Check}\\
While tracking and taking images, check the images on RubinTV for an
astrometric solution.

}
\hdashrule[0.5ex]{\textwidth}{1pt}{3mm}
  Expected Result \\
{\footnotesize
RubinTV is showing an astrometric solution.

}

\begin{tabular}{p{2cm}}
\toprule
Step 73  \\ \hline
\end{tabular}
 Description \\
{\footnotesize
\textbf{Offline analysis results}\\
Offline analysis in Test case
\href{https://jira.lsstcorp.org/secure/Tests.jspa\#/testCase/LVV-T2739}{LVV-T2739}
Says that we do not have sufficient image quality.

}
\hdashrule[0.5ex]{\textwidth}{1pt}{3mm}
  Expected Result \\
{\footnotesize
Image quality is sufficient.

}

\begin{tabular}{p{2cm}}
\toprule
Step 74  \\ \hline
\end{tabular}
 Description \\
{\footnotesize
\textbf{Point} \textbf{the Dome:}\\
Command the Dome to {Pointing 25}⁠ to {180}⁠

}
\hdashrule[0.5ex]{\textwidth}{1pt}{3mm}
  Expected Result \\
{\footnotesize
The Dome starts moving.

}

\begin{tabular}{p{2cm}}
\toprule
Step 75  \\ \hline
\end{tabular}
 Description \\
{\footnotesize
Wait for the Dome to reach the commanded position.

}
\hdashrule[0.5ex]{\textwidth}{1pt}{3mm}
  Expected Result \\
{\footnotesize
The \emph{MTDome\_logevent\_azMotion} and
\emph{MTDome\_logevent\_elMotion} inPosition parameter = true.

}

\begin{tabular}{p{2cm}}
\toprule
Step 76  \\ \hline
\end{tabular}
 Description \\
{\footnotesize
\textbf{Point the TMA}\\
Command the TMA to {Pointing 25}⁠ at {180}⁠ , {86.5}⁠ .

}
\hdashrule[0.5ex]{\textwidth}{1pt}{3mm}
  Expected Result \\
{\footnotesize
The TMA starts moving

}

\begin{tabular}{p{2cm}}
\toprule
Step 77  \\ \hline
\end{tabular}
 Description \\
{\footnotesize
Wait for the TMA to reach the commanded position.

}
\hdashrule[0.5ex]{\textwidth}{1pt}{3mm}
  Expected Result \\
{\footnotesize
The \emph{MTMount\_logevent\_azimuthInPosition} and
\emph{MTMount\_logevent\_elevationInPosition} inPosition parameter =
true.

}

\begin{tabular}{p{2cm}}
\toprule
Step 78  \\ \hline
\end{tabular}
 Description \\
{\footnotesize
\textbf{Image preparation}\\
If the preparation to take images takes longer than 10sec, do
repositioning to target {{{180}⁠}}, {{{86.5}⁠~}}.

}
\hdashrule[0.5ex]{\textwidth}{1pt}{3mm}
  Expected Result \\
{\footnotesize
TMA reaches the commanded position.

}

\begin{tabular}{p{2cm}}
\toprule
Step 79  \\ \hline
\end{tabular}
 Description \\
{\footnotesize
\textbf{Track position and take images}\\[2\baselineskip]Take a
StarTracker image with 10s exposure time.\\[2\baselineskip]If the time
the available:

\begin{itemize}
\tightlist
\item
  Track a position for 10 min and take StarTracker images.
\end{itemize}

}
\hdashrule[0.5ex]{\textwidth}{1pt}{3mm}
  Expected Result \\
{\footnotesize
\begin{itemize}
\tightlist
\item
  If time is available: The TMA is tracking a given position for 10 min
  and taking images.
\item
  At least one image is successfully taken with the StarTracker.
\end{itemize}

}

\begin{tabular}{p{2cm}}
\toprule
Step 80  \\ \hline
\end{tabular}
 Description \\
{\footnotesize
\textbf{On-the-fly Image Quality Check}\\
While tracking and taking images, check the images on RubinTV for an
astrometric solution.

}
\hdashrule[0.5ex]{\textwidth}{1pt}{3mm}
  Expected Result \\
{\footnotesize
RubinTV is showing an astrometric solution.

}

\begin{tabular}{p{2cm}}
\toprule
Step 81  \\ \hline
\end{tabular}
 Description \\
{\footnotesize
\textbf{Offline analysis results}\\
Offline analysis in Test case
\href{https://jira.lsstcorp.org/secure/Tests.jspa\#/testCase/LVV-T2739}{LVV-T2739}
Says that we do not have sufficient image quality.

}
\hdashrule[0.5ex]{\textwidth}{1pt}{3mm}
  Expected Result \\
{\footnotesize
Image quality is sufficient.

}

\begin{tabular}{p{2cm}}
\toprule
Step 82  \\ \hline
\end{tabular}
 Description \\
{\footnotesize
Wait for the Dome to reach the commanded position.

}
\hdashrule[0.5ex]{\textwidth}{1pt}{3mm}
  Expected Result \\
{\footnotesize
The \emph{MTDome\_logevent\_azMotion} and
\emph{MTDome\_logevent\_elMotion} inPosition parameter = true.

}

\begin{tabular}{p{2cm}}
\toprule
Step 83  \\ \hline
\end{tabular}
 Description \\
{\footnotesize
\textbf{Point} \textbf{the Dome:}\\
Command the Dome to {Pointing 2}⁠ to {-180}⁠

}
\hdashrule[0.5ex]{\textwidth}{1pt}{3mm}
  Expected Result \\
{\footnotesize
The Dome starts moving.

}

\begin{tabular}{p{2cm}}
\toprule
Step 84  \\ \hline
\end{tabular}
 Description \\
{\footnotesize
Wait for the Dome to reach the commanded position.

}
\hdashrule[0.5ex]{\textwidth}{1pt}{3mm}
  Expected Result \\
{\footnotesize
The \emph{MTDome\_logevent\_azMotion} and
\emph{MTDome\_logevent\_elMotion} inPosition parameter = true.

}

\begin{tabular}{p{2cm}}
\toprule
Step 85  \\ \hline
\end{tabular}
 Description \\
{\footnotesize
\textbf{Point the TMA}\\
Command the TMA to {Pointing 2}⁠ at {-180}⁠ , {45}⁠ .

}
\hdashrule[0.5ex]{\textwidth}{1pt}{3mm}
  Expected Result \\
{\footnotesize
The TMA starts moving

}

\begin{tabular}{p{2cm}}
\toprule
Step 86  \\ \hline
\end{tabular}
 Description \\
{\footnotesize
Wait for the TMA to reach the commanded position.

}
\hdashrule[0.5ex]{\textwidth}{1pt}{3mm}
  Expected Result \\
{\footnotesize
The \emph{MTMount\_logevent\_azimuthInPosition} and
\emph{MTMount\_logevent\_elevationInPosition} inPosition parameter =
true.

}

\begin{tabular}{p{2cm}}
\toprule
Step 87  \\ \hline
\end{tabular}
 Description \\
{\footnotesize
\textbf{Image preparation}\\
If the preparation to take images takes longer than 10sec, do
repositioning to target {{{-180}⁠}}, {{{45}⁠~}}.

}
\hdashrule[0.5ex]{\textwidth}{1pt}{3mm}
  Expected Result \\
{\footnotesize
TMA reaches the commanded position.

}

\begin{tabular}{p{2cm}}
\toprule
Step 88  \\ \hline
\end{tabular}
 Description \\
{\footnotesize
\textbf{Track position and take images}\\[2\baselineskip]Take a
StarTracker image with 10s exposure time.\\[2\baselineskip]If the time
the available:

\begin{itemize}
\tightlist
\item
  Track a position for 10 min and take StarTracker images.
\end{itemize}

}
\hdashrule[0.5ex]{\textwidth}{1pt}{3mm}
  Expected Result \\
{\footnotesize
\begin{itemize}
\tightlist
\item
  If time is available: The TMA is tracking a given position for 10 min
  and taking images.
\item
  At least one image is successfully taken with the StarTracker.
\end{itemize}

}

\begin{tabular}{p{2cm}}
\toprule
Step 89  \\ \hline
\end{tabular}
 Description \\
{\footnotesize
\textbf{On-the-fly Image Quality Check}\\
While tracking and taking images, check the images on RubinTV for an
astrometric solution.

}
\hdashrule[0.5ex]{\textwidth}{1pt}{3mm}
  Expected Result \\
{\footnotesize
RubinTV is showing an astrometric solution.

}

\begin{tabular}{p{2cm}}
\toprule
Step 90  \\ \hline
\end{tabular}
 Description \\
{\footnotesize
\textbf{Offline analysis results}\\
Offline analysis in Test case
\href{https://jira.lsstcorp.org/secure/Tests.jspa\#/testCase/LVV-T2739}{LVV-T2739}
Says that we do not have sufficient image quality.

}
\hdashrule[0.5ex]{\textwidth}{1pt}{3mm}
  Expected Result \\
{\footnotesize
Image quality is sufficient.

}

\begin{tabular}{p{2cm}}
\toprule
Step 91  \\ \hline
\end{tabular}
 Description \\
{\footnotesize
\textbf{Point} \textbf{the Dome:}\\
Command the Dome to {Pointing 26}⁠ to {0}⁠

}
\hdashrule[0.5ex]{\textwidth}{1pt}{3mm}
  Expected Result \\
{\footnotesize
The Dome starts moving.

}

\begin{tabular}{p{2cm}}
\toprule
Step 92  \\ \hline
\end{tabular}
 Description \\
{\footnotesize
Wait for the Dome to reach the commanded position.

}
\hdashrule[0.5ex]{\textwidth}{1pt}{3mm}
  Expected Result \\
{\footnotesize
The \emph{MTDome\_logevent\_azMotion} and
\emph{MTDome\_logevent\_elMotion} inPosition parameter = true.

}

\begin{tabular}{p{2cm}}
\toprule
Step 93  \\ \hline
\end{tabular}
 Description \\
{\footnotesize
\textbf{Point the TMA}\\
Command the TMA to {Pointing 26}⁠ at {0}⁠ , {45}⁠ .

}
\hdashrule[0.5ex]{\textwidth}{1pt}{3mm}
  Expected Result \\
{\footnotesize
The TMA starts moving

}

\begin{tabular}{p{2cm}}
\toprule
Step 94  \\ \hline
\end{tabular}
 Description \\
{\footnotesize
Wait for the TMA to reach the commanded position.

}
\hdashrule[0.5ex]{\textwidth}{1pt}{3mm}
  Expected Result \\
{\footnotesize
The \emph{MTMount\_logevent\_azimuthInPosition} and
\emph{MTMount\_logevent\_elevationInPosition} inPosition parameter =
true.

}

\begin{tabular}{p{2cm}}
\toprule
Step 95  \\ \hline
\end{tabular}
 Description \\
{\footnotesize
\textbf{Image preparation}\\
If the preparation to take images takes longer than 10sec, do
repositioning to target {{{0}⁠}}, {{{45}⁠~}}.

}
\hdashrule[0.5ex]{\textwidth}{1pt}{3mm}
  Expected Result \\
{\footnotesize
TMA reaches the commanded position.

}

\begin{tabular}{p{2cm}}
\toprule
Step 96  \\ \hline
\end{tabular}
 Description \\
{\footnotesize
\textbf{Track position and take images}\\[2\baselineskip]Take a
StarTracker image with 10s exposure time.\\[2\baselineskip]If the time
the available:

\begin{itemize}
\tightlist
\item
  Track a position for 10 min and take StarTracker images.
\end{itemize}

}
\hdashrule[0.5ex]{\textwidth}{1pt}{3mm}
  Expected Result \\
{\footnotesize
\begin{itemize}
\tightlist
\item
  If time is available: The TMA is tracking a given position for 10 min
  and taking images.
\item
  At least one image is successfully taken with the StarTracker.
\end{itemize}

}

\begin{tabular}{p{2cm}}
\toprule
Step 97  \\ \hline
\end{tabular}
 Description \\
{\footnotesize
\textbf{On-the-fly Image Quality Check}\\
While tracking and taking images, check the images on RubinTV for an
astrometric solution.

}
\hdashrule[0.5ex]{\textwidth}{1pt}{3mm}
  Expected Result \\
{\footnotesize
RubinTV is showing an astrometric solution.

}

\begin{tabular}{p{2cm}}
\toprule
Step 98  \\ \hline
\end{tabular}
 Description \\
{\footnotesize
\textbf{Offline analysis results}\\
Offline analysis in Test case
\href{https://jira.lsstcorp.org/secure/Tests.jspa\#/testCase/LVV-T2739}{LVV-T2739}
Says that we do not have sufficient image quality.

}
\hdashrule[0.5ex]{\textwidth}{1pt}{3mm}
  Expected Result \\
{\footnotesize
Image quality is sufficient.

}

\begin{tabular}{p{2cm}}
\toprule
Step 99  \\ \hline
\end{tabular}
 Description \\
{\footnotesize
\textbf{Point} \textbf{the Dome:}\\
Command the Dome to {Pointing 28}⁠ to {-90}⁠

}
\hdashrule[0.5ex]{\textwidth}{1pt}{3mm}
  Expected Result \\
{\footnotesize
The Dome starts moving.

}

\begin{tabular}{p{2cm}}
\toprule
Step 100  \\ \hline
\end{tabular}
 Description \\
{\footnotesize
Wait for the Dome to reach the commanded position.

}
\hdashrule[0.5ex]{\textwidth}{1pt}{3mm}
  Expected Result \\
{\footnotesize
The \emph{MTDome\_logevent\_azMotion} and
\emph{MTDome\_logevent\_elMotion} inPosition parameter = true.

}

\begin{tabular}{p{2cm}}
\toprule
Step 101  \\ \hline
\end{tabular}
 Description \\
{\footnotesize
\textbf{Point the TMA}\\
Command the TMA to {Pointing 28}⁠ at {-90}⁠ , {45}⁠ .

}
\hdashrule[0.5ex]{\textwidth}{1pt}{3mm}
  Expected Result \\
{\footnotesize
The TMA starts moving

}

\begin{tabular}{p{2cm}}
\toprule
Step 102  \\ \hline
\end{tabular}
 Description \\
{\footnotesize
Wait for the TMA to reach the commanded position.

}
\hdashrule[0.5ex]{\textwidth}{1pt}{3mm}
  Expected Result \\
{\footnotesize
The \emph{MTMount\_logevent\_azimuthInPosition} and
\emph{MTMount\_logevent\_elevationInPosition} inPosition parameter =
true.

}

\begin{tabular}{p{2cm}}
\toprule
Step 103  \\ \hline
\end{tabular}
 Description \\
{\footnotesize
\textbf{Image preparation}\\
If the preparation to take images takes longer than 10sec, do
repositioning to target {{{-90}⁠}}, {{{45}⁠~}}.

}
\hdashrule[0.5ex]{\textwidth}{1pt}{3mm}
  Expected Result \\
{\footnotesize
TMA reaches the commanded position.

}

\begin{tabular}{p{2cm}}
\toprule
Step 104  \\ \hline
\end{tabular}
 Description \\
{\footnotesize
\textbf{Track position and take images}\\[2\baselineskip]Take a
StarTracker image with 10s exposure time.\\[2\baselineskip]If the time
the available:

\begin{itemize}
\tightlist
\item
  Track a position for 10 min and take StarTracker images.
\end{itemize}

}
\hdashrule[0.5ex]{\textwidth}{1pt}{3mm}
  Expected Result \\
{\footnotesize
\begin{itemize}
\tightlist
\item
  If time is available: The TMA is tracking a given position for 10 min
  and taking images.
\item
  At least one image is successfully taken with the StarTracker.
\end{itemize}

}

\begin{tabular}{p{2cm}}
\toprule
Step 105  \\ \hline
\end{tabular}
 Description \\
{\footnotesize
\textbf{On-the-fly Image Quality Check}\\
While tracking and taking images, check the images on RubinTV for an
astrometric solution.

}
\hdashrule[0.5ex]{\textwidth}{1pt}{3mm}
  Expected Result \\
{\footnotesize
RubinTV is showing an astrometric solution.

}

\begin{tabular}{p{2cm}}
\toprule
Step 106  \\ \hline
\end{tabular}
 Description \\
{\footnotesize
\textbf{Offline analysis results}\\
Offline analysis in Test case
\href{https://jira.lsstcorp.org/secure/Tests.jspa\#/testCase/LVV-T2739}{LVV-T2739}
Says that we do not have sufficient image quality.

}
\hdashrule[0.5ex]{\textwidth}{1pt}{3mm}
  Expected Result \\
{\footnotesize
Image quality is sufficient.

}

\begin{tabular}{p{2cm}}
\toprule
Step 107  \\ \hline
\end{tabular}
 Description \\
{\footnotesize
\textbf{Point} \textbf{the Dome:}\\
Command the Dome to {Pointing 22}⁠ to {270}⁠

}
\hdashrule[0.5ex]{\textwidth}{1pt}{3mm}
  Expected Result \\
{\footnotesize
The Dome starts moving.

}

\begin{tabular}{p{2cm}}
\toprule
Step 108  \\ \hline
\end{tabular}
 Description \\
{\footnotesize
Wait for the Dome to reach the commanded position.

}
\hdashrule[0.5ex]{\textwidth}{1pt}{3mm}
  Expected Result \\
{\footnotesize
The \emph{MTDome\_logevent\_azMotion} and
\emph{MTDome\_logevent\_elMotion} inPosition parameter = true.

}

\begin{tabular}{p{2cm}}
\toprule
Step 109  \\ \hline
\end{tabular}
 Description \\
{\footnotesize
\textbf{Point the TMA}\\
Command the TMA to {Pointing 22}⁠ at {270}⁠ , {75}⁠ .

}
\hdashrule[0.5ex]{\textwidth}{1pt}{3mm}
  Expected Result \\
{\footnotesize
The TMA starts moving

}

\begin{tabular}{p{2cm}}
\toprule
Step 110  \\ \hline
\end{tabular}
 Description \\
{\footnotesize
Wait for the TMA to reach the commanded position.

}
\hdashrule[0.5ex]{\textwidth}{1pt}{3mm}
  Expected Result \\
{\footnotesize
The \emph{MTMount\_logevent\_azimuthInPosition} and
\emph{MTMount\_logevent\_elevationInPosition} inPosition parameter =
true.

}

\begin{tabular}{p{2cm}}
\toprule
Step 111  \\ \hline
\end{tabular}
 Description \\
{\footnotesize
\textbf{Image preparation}\\
If the preparation to take images takes longer than 10sec, do
repositioning to target {{{270}⁠}}, {{{75}⁠~}}.

}
\hdashrule[0.5ex]{\textwidth}{1pt}{3mm}
  Expected Result \\
{\footnotesize
TMA reaches the commanded position.

}

\begin{tabular}{p{2cm}}
\toprule
Step 112  \\ \hline
\end{tabular}
 Description \\
{\footnotesize
\textbf{Track position and take images}\\[2\baselineskip]Take a
StarTracker image with 10s exposure time.\\[2\baselineskip]If the time
the available:

\begin{itemize}
\tightlist
\item
  Track a position for 10 min and take StarTracker images.
\end{itemize}

}
\hdashrule[0.5ex]{\textwidth}{1pt}{3mm}
  Expected Result \\
{\footnotesize
\begin{itemize}
\tightlist
\item
  If time is available: The TMA is tracking a given position for 10 min
  and taking images.
\item
  At least one image is successfully taken with the StarTracker.
\end{itemize}

}

\begin{tabular}{p{2cm}}
\toprule
Step 113  \\ \hline
\end{tabular}
 Description \\
{\footnotesize
\textbf{On-the-fly Image Quality Check}\\
While tracking and taking images, check the images on RubinTV for an
astrometric solution.

}
\hdashrule[0.5ex]{\textwidth}{1pt}{3mm}
  Expected Result \\
{\footnotesize
RubinTV is showing an astrometric solution.

}

\begin{tabular}{p{2cm}}
\toprule
Step 114  \\ \hline
\end{tabular}
 Description \\
{\footnotesize
\textbf{Offline analysis results}\\
Offline analysis in Test case
\href{https://jira.lsstcorp.org/secure/Tests.jspa\#/testCase/LVV-T2739}{LVV-T2739}
Says that we do not have sufficient image quality.

}
\hdashrule[0.5ex]{\textwidth}{1pt}{3mm}
  Expected Result \\
{\footnotesize
Image quality is sufficient.

}

\begin{tabular}{p{2cm}}
\toprule
Step 115  \\ \hline
\end{tabular}
 Description \\
{\footnotesize
\textbf{Point} \textbf{the Dome:}\\
Command the Dome to {Pointing 23}⁠ to {-180}⁠

}
\hdashrule[0.5ex]{\textwidth}{1pt}{3mm}
  Expected Result \\
{\footnotesize
The Dome starts moving.

}

\begin{tabular}{p{2cm}}
\toprule
Step 116  \\ \hline
\end{tabular}
 Description \\
{\footnotesize
Wait for the Dome to reach the commanded position.

}
\hdashrule[0.5ex]{\textwidth}{1pt}{3mm}
  Expected Result \\
{\footnotesize
The \emph{MTDome\_logevent\_azMotion} and
\emph{MTDome\_logevent\_elMotion} inPosition parameter = true.

}

\begin{tabular}{p{2cm}}
\toprule
Step 117  \\ \hline
\end{tabular}
 Description \\
{\footnotesize
\textbf{Point the TMA}\\
Command the TMA to {Pointing 23}⁠ at {-180}⁠ , {75}⁠ .

}
\hdashrule[0.5ex]{\textwidth}{1pt}{3mm}
  Expected Result \\
{\footnotesize
The TMA starts moving

}

\begin{tabular}{p{2cm}}
\toprule
Step 118  \\ \hline
\end{tabular}
 Description \\
{\footnotesize
Wait for the TMA to reach the commanded position.

}
\hdashrule[0.5ex]{\textwidth}{1pt}{3mm}
  Expected Result \\
{\footnotesize
The \emph{MTMount\_logevent\_azimuthInPosition} and
\emph{MTMount\_logevent\_elevationInPosition} inPosition parameter =
true.

}

\begin{tabular}{p{2cm}}
\toprule
Step 119  \\ \hline
\end{tabular}
 Description \\
{\footnotesize
\textbf{Image preparation}\\
If the preparation to take images takes longer than 10sec, do
repositioning to target {{{-180}⁠}}, {{{75}⁠~}}.

}
\hdashrule[0.5ex]{\textwidth}{1pt}{3mm}
  Expected Result \\
{\footnotesize
TMA reaches the commanded position.

}

\begin{tabular}{p{2cm}}
\toprule
Step 120  \\ \hline
\end{tabular}
 Description \\
{\footnotesize
\textbf{Track position and take images}\\[2\baselineskip]Take a
StarTracker image with 10s exposure time.\\[2\baselineskip]If the time
the available:

\begin{itemize}
\tightlist
\item
  Track a position for 10 min and take StarTracker images.
\end{itemize}

}
\hdashrule[0.5ex]{\textwidth}{1pt}{3mm}
  Expected Result \\
{\footnotesize
\begin{itemize}
\tightlist
\item
  If time is available: The TMA is tracking a given position for 10 min
  and taking images.
\item
  At least one image is successfully taken with the StarTracker.
\end{itemize}

}

\begin{tabular}{p{2cm}}
\toprule
Step 121  \\ \hline
\end{tabular}
 Description \\
{\footnotesize
\textbf{On-the-fly Image Quality Check}\\
While tracking and taking images, check the images on RubinTV for an
astrometric solution.

}
\hdashrule[0.5ex]{\textwidth}{1pt}{3mm}
  Expected Result \\
{\footnotesize
RubinTV is showing an astrometric solution.

}

\begin{tabular}{p{2cm}}
\toprule
Step 122  \\ \hline
\end{tabular}
 Description \\
{\footnotesize
\textbf{Offline analysis results}\\
Offline analysis in Test case
\href{https://jira.lsstcorp.org/secure/Tests.jspa\#/testCase/LVV-T2739}{LVV-T2739}
Says that we do not have sufficient image quality.

}
\hdashrule[0.5ex]{\textwidth}{1pt}{3mm}
  Expected Result \\
{\footnotesize
Image quality is sufficient.

}

\begin{tabular}{p{2cm}}
\toprule
Step 123  \\ \hline
\end{tabular}
 Description \\
{\footnotesize
\textbf{Point} \textbf{the Dome:}\\
Command the Dome to {Pointing 24}⁠ to {90}⁠

}
\hdashrule[0.5ex]{\textwidth}{1pt}{3mm}
  Expected Result \\
{\footnotesize
The Dome starts moving.

}

\begin{tabular}{p{2cm}}
\toprule
Step 124  \\ \hline
\end{tabular}
 Description \\
{\footnotesize
Wait for the Dome to reach the commanded position.

}
\hdashrule[0.5ex]{\textwidth}{1pt}{3mm}
  Expected Result \\
{\footnotesize
The \emph{MTDome\_logevent\_azMotion} and
\emph{MTDome\_logevent\_elMotion} inPosition parameter = true.

}

\begin{tabular}{p{2cm}}
\toprule
Step 125  \\ \hline
\end{tabular}
 Description \\
{\footnotesize
\textbf{Point the TMA}\\
Command the TMA to {Pointing 24}⁠ at {90}⁠ , {75}⁠ .

}
\hdashrule[0.5ex]{\textwidth}{1pt}{3mm}
  Expected Result \\
{\footnotesize
The TMA starts moving

}

\begin{tabular}{p{2cm}}
\toprule
Step 126  \\ \hline
\end{tabular}
 Description \\
{\footnotesize
Wait for the TMA to reach the commanded position.

}
\hdashrule[0.5ex]{\textwidth}{1pt}{3mm}
  Expected Result \\
{\footnotesize
The \emph{MTMount\_logevent\_azimuthInPosition} and
\emph{MTMount\_logevent\_elevationInPosition} inPosition parameter =
true.

}

\begin{tabular}{p{2cm}}
\toprule
Step 127  \\ \hline
\end{tabular}
 Description \\
{\footnotesize
\textbf{Image preparation}\\
If the preparation to take images takes longer than 10sec, do
repositioning to target {{{90}⁠}}, {{{75}⁠~}}.

}
\hdashrule[0.5ex]{\textwidth}{1pt}{3mm}
  Expected Result \\
{\footnotesize
TMA reaches the commanded position.

}

\begin{tabular}{p{2cm}}
\toprule
Step 128  \\ \hline
\end{tabular}
 Description \\
{\footnotesize
\textbf{Track position and take images}\\[2\baselineskip]Take a
StarTracker image with 10s exposure time.\\[2\baselineskip]If the time
the available:

\begin{itemize}
\tightlist
\item
  Track a position for 10 min and take StarTracker images.
\end{itemize}

}
\hdashrule[0.5ex]{\textwidth}{1pt}{3mm}
  Expected Result \\
{\footnotesize
\begin{itemize}
\tightlist
\item
  If time is available: The TMA is tracking a given position for 10 min
  and taking images.
\item
  At least one image is successfully taken with the StarTracker.
\end{itemize}

}

\begin{tabular}{p{2cm}}
\toprule
Step 129  \\ \hline
\end{tabular}
 Description \\
{\footnotesize
\textbf{On-the-fly Image Quality Check}\\
While tracking and taking images, check the images on RubinTV for an
astrometric solution.

}
\hdashrule[0.5ex]{\textwidth}{1pt}{3mm}
  Expected Result \\
{\footnotesize
RubinTV is showing an astrometric solution.

}

\begin{tabular}{p{2cm}}
\toprule
Step 130  \\ \hline
\end{tabular}
 Description \\
{\footnotesize
\textbf{Offline analysis results}\\
Offline analysis in Test case
\href{https://jira.lsstcorp.org/secure/Tests.jspa\#/testCase/LVV-T2739}{LVV-T2739}
Says that we do not have sufficient image quality.

}
\hdashrule[0.5ex]{\textwidth}{1pt}{3mm}
  Expected Result \\
{\footnotesize
Image quality is sufficient.

}

\begin{tabular}{p{2cm}}
\toprule
Step 131  \\ \hline
\end{tabular}
 Description \\
{\footnotesize
\textbf{Point} \textbf{the Dome:}\\
Command the Dome to {Pointing 25}⁠ to {-270}⁠

}
\hdashrule[0.5ex]{\textwidth}{1pt}{3mm}
  Expected Result \\
{\footnotesize
The Dome starts moving.

}

\begin{tabular}{p{2cm}}
\toprule
Step 132  \\ \hline
\end{tabular}
 Description \\
{\footnotesize
Wait for the Dome to reach the commanded position.

}
\hdashrule[0.5ex]{\textwidth}{1pt}{3mm}
  Expected Result \\
{\footnotesize
The \emph{MTDome\_logevent\_azMotion} and
\emph{MTDome\_logevent\_elMotion} inPosition parameter = true.

}

\begin{tabular}{p{2cm}}
\toprule
Step 133  \\ \hline
\end{tabular}
 Description \\
{\footnotesize
\textbf{Point the TMA}\\
Command the TMA to {Pointing 25}⁠ at {-270}⁠ , {75}⁠ .

}
\hdashrule[0.5ex]{\textwidth}{1pt}{3mm}
  Expected Result \\
{\footnotesize
The TMA starts moving

}

\begin{tabular}{p{2cm}}
\toprule
Step 134  \\ \hline
\end{tabular}
 Description \\
{\footnotesize
Wait for the TMA to reach the commanded position.

}
\hdashrule[0.5ex]{\textwidth}{1pt}{3mm}
  Expected Result \\
{\footnotesize
The \emph{MTMount\_logevent\_azimuthInPosition} and
\emph{MTMount\_logevent\_elevationInPosition} inPosition parameter =
true.

}

\begin{tabular}{p{2cm}}
\toprule
Step 135  \\ \hline
\end{tabular}
 Description \\
{\footnotesize
\textbf{Image preparation}\\
If the preparation to take images takes longer than 10sec, do
repositioning to target {{{-270}⁠}}, {{{75}⁠~}}.

}
\hdashrule[0.5ex]{\textwidth}{1pt}{3mm}
  Expected Result \\
{\footnotesize
TMA reaches the commanded position.

}

\begin{tabular}{p{2cm}}
\toprule
Step 136  \\ \hline
\end{tabular}
 Description \\
{\footnotesize
\textbf{Track position and take images}\\[2\baselineskip]Take a
StarTracker image with 10s exposure time.\\[2\baselineskip]If the time
the available:

\begin{itemize}
\tightlist
\item
  Track a position for 10 min and take StarTracker images.
\end{itemize}

}
\hdashrule[0.5ex]{\textwidth}{1pt}{3mm}
  Expected Result \\
{\footnotesize
\begin{itemize}
\tightlist
\item
  If time is available: The TMA is tracking a given position for 10 min
  and taking images.
\item
  At least one image is successfully taken with the StarTracker.
\end{itemize}

}

\begin{tabular}{p{2cm}}
\toprule
Step 137  \\ \hline
\end{tabular}
 Description \\
{\footnotesize
\textbf{On-the-fly Image Quality Check}\\
While tracking and taking images, check the images on RubinTV for an
astrometric solution.

}
\hdashrule[0.5ex]{\textwidth}{1pt}{3mm}
  Expected Result \\
{\footnotesize
RubinTV is showing an astrometric solution.

}

\begin{tabular}{p{2cm}}
\toprule
Step 138  \\ \hline
\end{tabular}
 Description \\
{\footnotesize
\textbf{Offline analysis results}\\
Offline analysis in Test case
\href{https://jira.lsstcorp.org/secure/Tests.jspa\#/testCase/LVV-T2739}{LVV-T2739}
Says that we do not have sufficient image quality.

}
\hdashrule[0.5ex]{\textwidth}{1pt}{3mm}
  Expected Result \\
{\footnotesize
Image quality is sufficient.

}

\begin{tabular}{p{2cm}}
\toprule
Step 139  \\ \hline
\end{tabular}
 Description \\
{\footnotesize
\textbf{Point} \textbf{the Dome:}\\
Command the Dome to {Pointing 26}⁠ to {180}⁠

}
\hdashrule[0.5ex]{\textwidth}{1pt}{3mm}
  Expected Result \\
{\footnotesize
The Dome starts moving.

}

\begin{tabular}{p{2cm}}
\toprule
Step 140  \\ \hline
\end{tabular}
 Description \\
{\footnotesize
Wait for the Dome to reach the commanded position.

}
\hdashrule[0.5ex]{\textwidth}{1pt}{3mm}
  Expected Result \\
{\footnotesize
The \emph{MTDome\_logevent\_azMotion} and
\emph{MTDome\_logevent\_elMotion} inPosition parameter = true.

}

\begin{tabular}{p{2cm}}
\toprule
Step 141  \\ \hline
\end{tabular}
 Description \\
{\footnotesize
\textbf{Point the TMA}\\
Command the TMA to {Pointing 26}⁠ at {180}⁠ , {75}⁠ .

}
\hdashrule[0.5ex]{\textwidth}{1pt}{3mm}
  Expected Result \\
{\footnotesize
The TMA starts moving

}

\begin{tabular}{p{2cm}}
\toprule
Step 142  \\ \hline
\end{tabular}
 Description \\
{\footnotesize
Wait for the TMA to reach the commanded position.

}
\hdashrule[0.5ex]{\textwidth}{1pt}{3mm}
  Expected Result \\
{\footnotesize
The \emph{MTMount\_logevent\_azimuthInPosition} and
\emph{MTMount\_logevent\_elevationInPosition} inPosition parameter =
true.

}

\begin{tabular}{p{2cm}}
\toprule
Step 143  \\ \hline
\end{tabular}
 Description \\
{\footnotesize
\textbf{Image preparation}\\
If the preparation to take images takes longer than 10sec, do
repositioning to target {{{180}⁠}}, {{{75}⁠~}}.

}
\hdashrule[0.5ex]{\textwidth}{1pt}{3mm}
  Expected Result \\
{\footnotesize
TMA reaches the commanded position.

}

\begin{tabular}{p{2cm}}
\toprule
Step 144  \\ \hline
\end{tabular}
 Description \\
{\footnotesize
\textbf{Track position and take images}\\[2\baselineskip]Take a
StarTracker image with 10s exposure time.\\[2\baselineskip]If the time
the available:

\begin{itemize}
\tightlist
\item
  Track a position for 10 min and take StarTracker images.
\end{itemize}

}
\hdashrule[0.5ex]{\textwidth}{1pt}{3mm}
  Expected Result \\
{\footnotesize
\begin{itemize}
\tightlist
\item
  If time is available: The TMA is tracking a given position for 10 min
  and taking images.
\item
  At least one image is successfully taken with the StarTracker.
\end{itemize}

}

\begin{tabular}{p{2cm}}
\toprule
Step 145  \\ \hline
\end{tabular}
 Description \\
{\footnotesize
\textbf{On-the-fly Image Quality Check}\\
While tracking and taking images, check the images on RubinTV for an
astrometric solution.

}
\hdashrule[0.5ex]{\textwidth}{1pt}{3mm}
  Expected Result \\
{\footnotesize
RubinTV is showing an astrometric solution.

}

\begin{tabular}{p{2cm}}
\toprule
Step 146  \\ \hline
\end{tabular}
 Description \\
{\footnotesize
\textbf{Offline analysis results}\\
Offline analysis in Test case
\href{https://jira.lsstcorp.org/secure/Tests.jspa\#/testCase/LVV-T2739}{LVV-T2739}
Says that we do not have sufficient image quality.

}
\hdashrule[0.5ex]{\textwidth}{1pt}{3mm}
  Expected Result \\
{\footnotesize
Image quality is sufficient.

}

\begin{tabular}{p{2cm}}
\toprule
Step 147  \\ \hline
\end{tabular}
 Description \\
{\footnotesize
\textbf{Point} \textbf{the Dome:}\\
Command the Dome to {Pointing 27}⁠ to {0}⁠

}
\hdashrule[0.5ex]{\textwidth}{1pt}{3mm}
  Expected Result \\
{\footnotesize
The Dome starts moving.

}

\begin{tabular}{p{2cm}}
\toprule
Step 148  \\ \hline
\end{tabular}
 Description \\
{\footnotesize
Wait for the Dome to reach the commanded position.

}
\hdashrule[0.5ex]{\textwidth}{1pt}{3mm}
  Expected Result \\
{\footnotesize
The \emph{MTDome\_logevent\_azMotion} and
\emph{MTDome\_logevent\_elMotion} inPosition parameter = true.

}

\begin{tabular}{p{2cm}}
\toprule
Step 149  \\ \hline
\end{tabular}
 Description \\
{\footnotesize
\textbf{Point the TMA}\\
Command the TMA to {Pointing 27}⁠ at {0}⁠ , {75}⁠ .

}
\hdashrule[0.5ex]{\textwidth}{1pt}{3mm}
  Expected Result \\
{\footnotesize
The TMA starts moving

}

\begin{tabular}{p{2cm}}
\toprule
Step 150  \\ \hline
\end{tabular}
 Description \\
{\footnotesize
Wait for the TMA to reach the commanded position.

}
\hdashrule[0.5ex]{\textwidth}{1pt}{3mm}
  Expected Result \\
{\footnotesize
The \emph{MTMount\_logevent\_azimuthInPosition} and
\emph{MTMount\_logevent\_elevationInPosition} inPosition parameter =
true.

}

\begin{tabular}{p{2cm}}
\toprule
Step 151  \\ \hline
\end{tabular}
 Description \\
{\footnotesize
\textbf{Image preparation}\\
If the preparation to take images takes longer than 10sec, do
repositioning to target {{{0}⁠}}, {{{75}⁠~}}.

}
\hdashrule[0.5ex]{\textwidth}{1pt}{3mm}
  Expected Result \\
{\footnotesize
TMA reaches the commanded position.

}

\begin{tabular}{p{2cm}}
\toprule
Step 152  \\ \hline
\end{tabular}
 Description \\
{\footnotesize
\textbf{Track position and take images}\\[2\baselineskip]Take a
StarTracker image with 10s exposure time.\\[2\baselineskip]If the time
the available:

\begin{itemize}
\tightlist
\item
  Track a position for 10 min and take StarTracker images.
\end{itemize}

}
\hdashrule[0.5ex]{\textwidth}{1pt}{3mm}
  Expected Result \\
{\footnotesize
\begin{itemize}
\tightlist
\item
  If time is available: The TMA is tracking a given position for 10 min
  and taking images.
\item
  At least one image is successfully taken with the StarTracker.
\end{itemize}

}

\begin{tabular}{p{2cm}}
\toprule
Step 153  \\ \hline
\end{tabular}
 Description \\
{\footnotesize
\textbf{On-the-fly Image Quality Check}\\
While tracking and taking images, check the images on RubinTV for an
astrometric solution.

}
\hdashrule[0.5ex]{\textwidth}{1pt}{3mm}
  Expected Result \\
{\footnotesize
RubinTV is showing an astrometric solution.

}

\begin{tabular}{p{2cm}}
\toprule
Step 154  \\ \hline
\end{tabular}
 Description \\
{\footnotesize
\textbf{Offline analysis results}\\
Offline analysis in Test case
\href{https://jira.lsstcorp.org/secure/Tests.jspa\#/testCase/LVV-T2739}{LVV-T2739}
Says that we do not have sufficient image quality.

}
\hdashrule[0.5ex]{\textwidth}{1pt}{3mm}
  Expected Result \\
{\footnotesize
Image quality is sufficient.

}

\begin{tabular}{p{2cm}}
\toprule
Step 155  \\ \hline
\end{tabular}
 Description \\
{\footnotesize
\textbf{Point} \textbf{the Dome:}\\
Command the Dome to {Pointing 28}⁠ to {-90}⁠

}
\hdashrule[0.5ex]{\textwidth}{1pt}{3mm}
  Expected Result \\
{\footnotesize
The Dome starts moving.

}

\begin{tabular}{p{2cm}}
\toprule
Step 156  \\ \hline
\end{tabular}
 Description \\
{\footnotesize
Wait for the Dome to reach the commanded position.

}
\hdashrule[0.5ex]{\textwidth}{1pt}{3mm}
  Expected Result \\
{\footnotesize
The \emph{MTDome\_logevent\_azMotion} and
\emph{MTDome\_logevent\_elMotion} inPosition parameter = true.

}

\begin{tabular}{p{2cm}}
\toprule
Step 157  \\ \hline
\end{tabular}
 Description \\
{\footnotesize
\textbf{Point the TMA}\\
Command the TMA to {Pointing 28}⁠ at {-90}⁠ , {75}⁠ .

}
\hdashrule[0.5ex]{\textwidth}{1pt}{3mm}
  Expected Result \\
{\footnotesize
The TMA starts moving

}

\begin{tabular}{p{2cm}}
\toprule
Step 158  \\ \hline
\end{tabular}
 Description \\
{\footnotesize
Wait for the TMA to reach the commanded position.

}
\hdashrule[0.5ex]{\textwidth}{1pt}{3mm}
  Expected Result \\
{\footnotesize
The \emph{MTMount\_logevent\_azimuthInPosition} and
\emph{MTMount\_logevent\_elevationInPosition} inPosition parameter =
true.

}

\begin{tabular}{p{2cm}}
\toprule
Step 159  \\ \hline
\end{tabular}
 Description \\
{\footnotesize
\textbf{Image preparation}\\
If the preparation to take images takes longer than 10sec, do
repositioning to target {{{-90}⁠}}, {{{75}⁠~}}.

}
\hdashrule[0.5ex]{\textwidth}{1pt}{3mm}
  Expected Result \\
{\footnotesize
TMA reaches the commanded position.

}

\begin{tabular}{p{2cm}}
\toprule
Step 160  \\ \hline
\end{tabular}
 Description \\
{\footnotesize
\textbf{Track position and take images}\\[2\baselineskip]Take a
StarTracker image with 10s exposure time.\\[2\baselineskip]If the time
the available:

\begin{itemize}
\tightlist
\item
  Track a position for 10 min and take StarTracker images.
\end{itemize}

}
\hdashrule[0.5ex]{\textwidth}{1pt}{3mm}
  Expected Result \\
{\footnotesize
\begin{itemize}
\tightlist
\item
  If time is available: The TMA is tracking a given position for 10 min
  and taking images.
\item
  At least one image is successfully taken with the StarTracker.
\end{itemize}

}

\begin{tabular}{p{2cm}}
\toprule
Step 161  \\ \hline
\end{tabular}
 Description \\
{\footnotesize
\textbf{On-the-fly Image Quality Check}\\
While tracking and taking images, check the images on RubinTV for an
astrometric solution.

}
\hdashrule[0.5ex]{\textwidth}{1pt}{3mm}
  Expected Result \\
{\footnotesize
RubinTV is showing an astrometric solution.

}

\begin{tabular}{p{2cm}}
\toprule
Step 162  \\ \hline
\end{tabular}
 Description \\
{\footnotesize
\textbf{Offline analysis results}\\
Offline analysis in Test case
\href{https://jira.lsstcorp.org/secure/Tests.jspa\#/testCase/LVV-T2739}{LVV-T2739}
Says that we do not have sufficient image quality.

}
\hdashrule[0.5ex]{\textwidth}{1pt}{3mm}
  Expected Result \\
{\footnotesize
Image quality is sufficient.

}

\begin{tabular}{p{2cm}}
\toprule
Step 163  \\ \hline
\end{tabular}
 Description \\
{\footnotesize
\textbf{Point the TMA}\\
Command the TMA to {Pointing 1}⁠ at {270}⁠ , {15}⁠ .

}
\hdashrule[0.5ex]{\textwidth}{1pt}{3mm}
  Expected Result \\
{\footnotesize
The TMA starts moving

}

\begin{tabular}{p{2cm}}
\toprule
Step 164  \\ \hline
\end{tabular}
 Description \\
{\footnotesize
\textbf{Point} \textbf{the Dome:}\\
Command the Dome to {Pointing 4}⁠ to {90}⁠

}
\hdashrule[0.5ex]{\textwidth}{1pt}{3mm}
  Expected Result \\
{\footnotesize
The Dome starts moving.

}

\begin{tabular}{p{2cm}}
\toprule
Step 165  \\ \hline
\end{tabular}
 Description \\
{\footnotesize
Wait for the Dome to reach the commanded position.

}
\hdashrule[0.5ex]{\textwidth}{1pt}{3mm}
  Expected Result \\
{\footnotesize
The \emph{MTDome\_logevent\_azMotion} and
\emph{MTDome\_logevent\_elMotion} inPosition parameter = true.

}

\begin{tabular}{p{2cm}}
\toprule
Step 166  \\ \hline
\end{tabular}
 Description \\
{\footnotesize
\textbf{Point the TMA}\\
Command the TMA to {Pointing 4}⁠ at {90}⁠ , {86.5}⁠ .

}
\hdashrule[0.5ex]{\textwidth}{1pt}{3mm}
  Expected Result \\
{\footnotesize
The TMA starts moving

}

\begin{tabular}{p{2cm}}
\toprule
Step 167  \\ \hline
\end{tabular}
 Description \\
{\footnotesize
Wait for the TMA to reach the commanded position.

}
\hdashrule[0.5ex]{\textwidth}{1pt}{3mm}
  Expected Result \\
{\footnotesize
The \emph{MTMount\_logevent\_azimuthInPosition} and
\emph{MTMount\_logevent\_elevationInPosition} inPosition parameter =
true.

}

\begin{tabular}{p{2cm}}
\toprule
Step 168  \\ \hline
\end{tabular}
 Description \\
{\footnotesize
\textbf{Image preparation}\\
If the preparation to take images takes longer than 10sec, do
repositioning to target {{{90}⁠}}, {{{86.5}⁠~}}.

}
\hdashrule[0.5ex]{\textwidth}{1pt}{3mm}
  Expected Result \\
{\footnotesize
TMA reaches the commanded position.

}

\begin{tabular}{p{2cm}}
\toprule
Step 169  \\ \hline
\end{tabular}
 Description \\
{\footnotesize
\textbf{Track position and take images}\\[2\baselineskip]Take a
StarTracker image with 10s exposure time.\\[2\baselineskip]If the time
the available:

\begin{itemize}
\tightlist
\item
  Track a position for 10 min and take StarTracker images.
\end{itemize}

}
\hdashrule[0.5ex]{\textwidth}{1pt}{3mm}
  Expected Result \\
{\footnotesize
\begin{itemize}
\tightlist
\item
  If time is available: The TMA is tracking a given position for 10 min
  and taking images.
\item
  At least one image is successfully taken with the StarTracker.
\end{itemize}

}

\begin{tabular}{p{2cm}}
\toprule
Step 170  \\ \hline
\end{tabular}
 Description \\
{\footnotesize
\textbf{On-the-fly Image Quality Check}\\
While tracking and taking images, check the images on RubinTV for an
astrometric solution.

}
\hdashrule[0.5ex]{\textwidth}{1pt}{3mm}
  Expected Result \\
{\footnotesize
RubinTV is showing an astrometric solution.

}

\begin{tabular}{p{2cm}}
\toprule
Step 171  \\ \hline
\end{tabular}
 Description \\
{\footnotesize
\textbf{Offline analysis results}\\
Offline analysis in Test case
\href{https://jira.lsstcorp.org/secure/Tests.jspa\#/testCase/LVV-T2739}{LVV-T2739}
Says that we do not have sufficient image quality.

}
\hdashrule[0.5ex]{\textwidth}{1pt}{3mm}
  Expected Result \\
{\footnotesize
Image quality is sufficient.

}

\begin{tabular}{p{2cm}}
\toprule
Step 172  \\ \hline
\end{tabular}
 Description \\
{\footnotesize
Wait for the TMA to reach the commanded position.

}
\hdashrule[0.5ex]{\textwidth}{1pt}{3mm}
  Expected Result \\
{\footnotesize
The \emph{MTMount\_logevent\_azimuthInPosition} and
\emph{MTMount\_logevent\_elevationInPosition} inPosition parameter =
true.

}

\begin{tabular}{p{2cm}}
\toprule
Step 173  \\ \hline
\end{tabular}
 Description \\
{\footnotesize
\textbf{Point} \textbf{the Dome:}\\
Command the Dome to {Pointing 5}⁠ to {-270}⁠

}
\hdashrule[0.5ex]{\textwidth}{1pt}{3mm}
  Expected Result \\
{\footnotesize
The Dome starts moving.

}

\begin{tabular}{p{2cm}}
\toprule
Step 174  \\ \hline
\end{tabular}
 Description \\
{\footnotesize
Wait for the Dome to reach the commanded position.

}
\hdashrule[0.5ex]{\textwidth}{1pt}{3mm}
  Expected Result \\
{\footnotesize
The \emph{MTDome\_logevent\_azMotion} and
\emph{MTDome\_logevent\_elMotion} inPosition parameter = true.

}

\begin{tabular}{p{2cm}}
\toprule
Step 175  \\ \hline
\end{tabular}
 Description \\
{\footnotesize
\textbf{Point the TMA}\\
Command the TMA to {Pointing 5}⁠ at {-270}⁠ , {15}⁠ .

}
\hdashrule[0.5ex]{\textwidth}{1pt}{3mm}
  Expected Result \\
{\footnotesize
The TMA starts moving

}

\begin{tabular}{p{2cm}}
\toprule
Step 176  \\ \hline
\end{tabular}
 Description \\
{\footnotesize
Wait for the TMA to reach the commanded position.

}
\hdashrule[0.5ex]{\textwidth}{1pt}{3mm}
  Expected Result \\
{\footnotesize
The \emph{MTMount\_logevent\_azimuthInPosition} and
\emph{MTMount\_logevent\_elevationInPosition} inPosition parameter =
true.

}

\begin{tabular}{p{2cm}}
\toprule
Step 177  \\ \hline
\end{tabular}
 Description \\
{\footnotesize
\textbf{Image preparation}\\
If the preparation to take images takes longer than 10sec, do
repositioning to target {{{-270}⁠}}, {{{15}⁠~}}.

}
\hdashrule[0.5ex]{\textwidth}{1pt}{3mm}
  Expected Result \\
{\footnotesize
TMA reaches the commanded position.

}

\begin{tabular}{p{2cm}}
\toprule
Step 178  \\ \hline
\end{tabular}
 Description \\
{\footnotesize
\textbf{Track position and take images}\\[2\baselineskip]Take a
StarTracker image with 10s exposure time.\\[2\baselineskip]If the time
the available:

\begin{itemize}
\tightlist
\item
  Track a position for 10 min and take StarTracker images.
\end{itemize}

}
\hdashrule[0.5ex]{\textwidth}{1pt}{3mm}
  Expected Result \\
{\footnotesize
\begin{itemize}
\tightlist
\item
  If time is available: The TMA is tracking a given position for 10 min
  and taking images.
\item
  At least one image is successfully taken with the StarTracker.
\end{itemize}

}

\begin{tabular}{p{2cm}}
\toprule
Step 179  \\ \hline
\end{tabular}
 Description \\
{\footnotesize
\textbf{On-the-fly Image Quality Check}\\
While tracking and taking images, check the images on RubinTV for an
astrometric solution.

}
\hdashrule[0.5ex]{\textwidth}{1pt}{3mm}
  Expected Result \\
{\footnotesize
RubinTV is showing an astrometric solution.

}

\begin{tabular}{p{2cm}}
\toprule
Step 180  \\ \hline
\end{tabular}
 Description \\
{\footnotesize
\textbf{Offline analysis results}\\
Offline analysis in Test case
\href{https://jira.lsstcorp.org/secure/Tests.jspa\#/testCase/LVV-T2739}{LVV-T2739}
Says that we do not have sufficient image quality.

}
\hdashrule[0.5ex]{\textwidth}{1pt}{3mm}
  Expected Result \\
{\footnotesize
Image quality is sufficient.

}

\begin{tabular}{p{2cm}}
\toprule
Step 181  \\ \hline
\end{tabular}
 Description \\
{\footnotesize
\textbf{Image preparation}\\
If the preparation to take images takes longer than 10sec, do
repositioning to target {{{270}⁠}}, {{{15}⁠~}}.

}
\hdashrule[0.5ex]{\textwidth}{1pt}{3mm}
  Expected Result \\
{\footnotesize
TMA reaches the commanded position.

}

\begin{tabular}{p{2cm}}
\toprule
Step 182  \\ \hline
\end{tabular}
 Description \\
{\footnotesize
\textbf{Point} \textbf{the Dome:}\\
Command the Dome to {Pointing 7}⁠ to {180}⁠

}
\hdashrule[0.5ex]{\textwidth}{1pt}{3mm}
  Expected Result \\
{\footnotesize
The Dome starts moving.

}

\begin{tabular}{p{2cm}}
\toprule
Step 183  \\ \hline
\end{tabular}
 Description \\
{\footnotesize
Wait for the Dome to reach the commanded position.

}
\hdashrule[0.5ex]{\textwidth}{1pt}{3mm}
  Expected Result \\
{\footnotesize
The \emph{MTDome\_logevent\_azMotion} and
\emph{MTDome\_logevent\_elMotion} inPosition parameter = true.

}

\begin{tabular}{p{2cm}}
\toprule
Step 184  \\ \hline
\end{tabular}
 Description \\
{\footnotesize
\textbf{Point the TMA}\\
Command the TMA to {Pointing 7}⁠ at {180}⁠ , {45}⁠ .

}
\hdashrule[0.5ex]{\textwidth}{1pt}{3mm}
  Expected Result \\
{\footnotesize
The TMA starts moving

}

\begin{tabular}{p{2cm}}
\toprule
Step 185  \\ \hline
\end{tabular}
 Description \\
{\footnotesize
Wait for the TMA to reach the commanded position.

}
\hdashrule[0.5ex]{\textwidth}{1pt}{3mm}
  Expected Result \\
{\footnotesize
The \emph{MTMount\_logevent\_azimuthInPosition} and
\emph{MTMount\_logevent\_elevationInPosition} inPosition parameter =
true.

}

\begin{tabular}{p{2cm}}
\toprule
Step 186  \\ \hline
\end{tabular}
 Description \\
{\footnotesize
\textbf{Image preparation}\\
If the preparation to take images takes longer than 10sec, do
repositioning to target {{{180}⁠}}, {{{45}⁠~}}.

}
\hdashrule[0.5ex]{\textwidth}{1pt}{3mm}
  Expected Result \\
{\footnotesize
TMA reaches the commanded position.

}

\begin{tabular}{p{2cm}}
\toprule
Step 187  \\ \hline
\end{tabular}
 Description \\
{\footnotesize
\textbf{Track position and take images}\\[2\baselineskip]Take a
StarTracker image with 10s exposure time.\\[2\baselineskip]If the time
the available:

\begin{itemize}
\tightlist
\item
  Track a position for 10 min and take StarTracker images.
\end{itemize}

}
\hdashrule[0.5ex]{\textwidth}{1pt}{3mm}
  Expected Result \\
{\footnotesize
\begin{itemize}
\tightlist
\item
  If time is available: The TMA is tracking a given position for 10 min
  and taking images.
\item
  At least one image is successfully taken with the StarTracker.
\end{itemize}

}

\begin{tabular}{p{2cm}}
\toprule
Step 188  \\ \hline
\end{tabular}
 Description \\
{\footnotesize
\textbf{On-the-fly Image Quality Check}\\
While tracking and taking images, check the images on RubinTV for an
astrometric solution.

}
\hdashrule[0.5ex]{\textwidth}{1pt}{3mm}
  Expected Result \\
{\footnotesize
RubinTV is showing an astrometric solution.

}

\begin{tabular}{p{2cm}}
\toprule
Step 189  \\ \hline
\end{tabular}
 Description \\
{\footnotesize
\textbf{Offline analysis results}\\
Offline analysis in Test case
\href{https://jira.lsstcorp.org/secure/Tests.jspa\#/testCase/LVV-T2739}{LVV-T2739}
Says that we do not have sufficient image quality.

}
\hdashrule[0.5ex]{\textwidth}{1pt}{3mm}
  Expected Result \\
{\footnotesize
Image quality is sufficient.

}

\begin{tabular}{p{2cm}}
\toprule
Step 190  \\ \hline
\end{tabular}
 Description \\
{\footnotesize
\textbf{Track position and take images}\\[2\baselineskip]Take a
StarTracker image with 10s exposure time.\\[2\baselineskip]If the time
the available:

\begin{itemize}
\tightlist
\item
  Track a position for 10 min and take StarTracker images.
\end{itemize}

}
\hdashrule[0.5ex]{\textwidth}{1pt}{3mm}
  Expected Result \\
{\footnotesize
\begin{itemize}
\tightlist
\item
  If time is available: The TMA is tracking a given position for 10 min
  and taking images.
\item
  At least one image is successfully taken with the StarTracker.
\end{itemize}

}

\begin{tabular}{p{2cm}}
\toprule
Step 191  \\ \hline
\end{tabular}
 Description \\
{\footnotesize
\textbf{Point} \textbf{the Dome:}\\
Command the Dome to {Pointing 8}⁠ to {0}⁠

}
\hdashrule[0.5ex]{\textwidth}{1pt}{3mm}
  Expected Result \\
{\footnotesize
The Dome starts moving.

}

\begin{tabular}{p{2cm}}
\toprule
Step 192  \\ \hline
\end{tabular}
 Description \\
{\footnotesize
Wait for the Dome to reach the commanded position.

}
\hdashrule[0.5ex]{\textwidth}{1pt}{3mm}
  Expected Result \\
{\footnotesize
The \emph{MTDome\_logevent\_azMotion} and
\emph{MTDome\_logevent\_elMotion} inPosition parameter = true.

}

\begin{tabular}{p{2cm}}
\toprule
Step 193  \\ \hline
\end{tabular}
 Description \\
{\footnotesize
\textbf{Point the TMA}\\
Command the TMA to {Pointing 8}⁠ at {0}⁠ , {86.5}⁠ .

}
\hdashrule[0.5ex]{\textwidth}{1pt}{3mm}
  Expected Result \\
{\footnotesize
The TMA starts moving

}

\begin{tabular}{p{2cm}}
\toprule
Step 194  \\ \hline
\end{tabular}
 Description \\
{\footnotesize
Wait for the TMA to reach the commanded position.

}
\hdashrule[0.5ex]{\textwidth}{1pt}{3mm}
  Expected Result \\
{\footnotesize
The \emph{MTMount\_logevent\_azimuthInPosition} and
\emph{MTMount\_logevent\_elevationInPosition} inPosition parameter =
true.

}

\begin{tabular}{p{2cm}}
\toprule
Step 195  \\ \hline
\end{tabular}
 Description \\
{\footnotesize
\textbf{Image preparation}\\
If the preparation to take images takes longer than 10sec, do
repositioning to target {{{0}⁠}}, {{{86.5}⁠~}}.

}
\hdashrule[0.5ex]{\textwidth}{1pt}{3mm}
  Expected Result \\
{\footnotesize
TMA reaches the commanded position.

}

\begin{tabular}{p{2cm}}
\toprule
Step 196  \\ \hline
\end{tabular}
 Description \\
{\footnotesize
\textbf{Track position and take images}\\[2\baselineskip]Take a
StarTracker image with 10s exposure time.\\[2\baselineskip]If the time
the available:

\begin{itemize}
\tightlist
\item
  Track a position for 10 min and take StarTracker images.
\end{itemize}

}
\hdashrule[0.5ex]{\textwidth}{1pt}{3mm}
  Expected Result \\
{\footnotesize
\begin{itemize}
\tightlist
\item
  If time is available: The TMA is tracking a given position for 10 min
  and taking images.
\item
  At least one image is successfully taken with the StarTracker.
\end{itemize}

}

\begin{tabular}{p{2cm}}
\toprule
Step 197  \\ \hline
\end{tabular}
 Description \\
{\footnotesize
\textbf{On-the-fly Image Quality Check}\\
While tracking and taking images, check the images on RubinTV for an
astrometric solution.

}
\hdashrule[0.5ex]{\textwidth}{1pt}{3mm}
  Expected Result \\
{\footnotesize
RubinTV is showing an astrometric solution.

}

\begin{tabular}{p{2cm}}
\toprule
Step 198  \\ \hline
\end{tabular}
 Description \\
{\footnotesize
\textbf{Offline analysis results}\\
Offline analysis in Test case
\href{https://jira.lsstcorp.org/secure/Tests.jspa\#/testCase/LVV-T2739}{LVV-T2739}
Says that we do not have sufficient image quality.

}
\hdashrule[0.5ex]{\textwidth}{1pt}{3mm}
  Expected Result \\
{\footnotesize
Image quality is sufficient.

}

\begin{tabular}{p{2cm}}
\toprule
Step 199  \\ \hline
\end{tabular}
 Description \\
{\footnotesize
\textbf{On-the-fly Image Quality Check}\\
While tracking and taking images, check the images on RubinTV for an
astrometric solution.

}
\hdashrule[0.5ex]{\textwidth}{1pt}{3mm}
  Expected Result \\
{\footnotesize
RubinTV is showing an astrometric solution.

}

\begin{tabular}{p{2cm}}
\toprule
Step 200  \\ \hline
\end{tabular}
 Description \\
{\footnotesize
\textbf{Point} \textbf{the Dome:}\\
Command the Dome to {Pointing 9}⁠ to {-90}⁠

}
\hdashrule[0.5ex]{\textwidth}{1pt}{3mm}
  Expected Result \\
{\footnotesize
The Dome starts moving.

}

\begin{tabular}{p{2cm}}
\toprule
Step 201  \\ \hline
\end{tabular}
 Description \\
{\footnotesize
Wait for the Dome to reach the commanded position.

}
\hdashrule[0.5ex]{\textwidth}{1pt}{3mm}
  Expected Result \\
{\footnotesize
The \emph{MTDome\_logevent\_azMotion} and
\emph{MTDome\_logevent\_elMotion} inPosition parameter = true.

}

\begin{tabular}{p{2cm}}
\toprule
Step 202  \\ \hline
\end{tabular}
 Description \\
{\footnotesize
\textbf{Point the TMA}\\
Command the TMA to {Pointing 9}⁠ at {-90}⁠ , {15}⁠ .

}
\hdashrule[0.5ex]{\textwidth}{1pt}{3mm}
  Expected Result \\
{\footnotesize
The TMA starts moving

}

\begin{tabular}{p{2cm}}
\toprule
Step 203  \\ \hline
\end{tabular}
 Description \\
{\footnotesize
Wait for the TMA to reach the commanded position.

}
\hdashrule[0.5ex]{\textwidth}{1pt}{3mm}
  Expected Result \\
{\footnotesize
The \emph{MTMount\_logevent\_azimuthInPosition} and
\emph{MTMount\_logevent\_elevationInPosition} inPosition parameter =
true.

}

\begin{tabular}{p{2cm}}
\toprule
Step 204  \\ \hline
\end{tabular}
 Description \\
{\footnotesize
\textbf{Image preparation}\\
If the preparation to take images takes longer than 10sec, do
repositioning to target {{{-90}⁠}}, {{{15}⁠~}}.

}
\hdashrule[0.5ex]{\textwidth}{1pt}{3mm}
  Expected Result \\
{\footnotesize
TMA reaches the commanded position.

}

\begin{tabular}{p{2cm}}
\toprule
Step 205  \\ \hline
\end{tabular}
 Description \\
{\footnotesize
\textbf{Track position and take images}\\[2\baselineskip]Take a
StarTracker image with 10s exposure time.\\[2\baselineskip]If the time
the available:

\begin{itemize}
\tightlist
\item
  Track a position for 10 min and take StarTracker images.
\end{itemize}

}
\hdashrule[0.5ex]{\textwidth}{1pt}{3mm}
  Expected Result \\
{\footnotesize
\begin{itemize}
\tightlist
\item
  If time is available: The TMA is tracking a given position for 10 min
  and taking images.
\item
  At least one image is successfully taken with the StarTracker.
\end{itemize}

}

\begin{tabular}{p{2cm}}
\toprule
Step 206  \\ \hline
\end{tabular}
 Description \\
{\footnotesize
\textbf{On-the-fly Image Quality Check}\\
While tracking and taking images, check the images on RubinTV for an
astrometric solution.

}
\hdashrule[0.5ex]{\textwidth}{1pt}{3mm}
  Expected Result \\
{\footnotesize
RubinTV is showing an astrometric solution.

}

\begin{tabular}{p{2cm}}
\toprule
Step 207  \\ \hline
\end{tabular}
 Description \\
{\footnotesize
\textbf{Offline analysis results}\\
Offline analysis in Test case
\href{https://jira.lsstcorp.org/secure/Tests.jspa\#/testCase/LVV-T2739}{LVV-T2739}
Says that we do not have sufficient image quality.

}
\hdashrule[0.5ex]{\textwidth}{1pt}{3mm}
  Expected Result \\
{\footnotesize
Image quality is sufficient.

}

\begin{tabular}{p{2cm}}
\toprule
Step 208  \\ \hline
\end{tabular}
 Description \\
{\footnotesize
\textbf{Offline analysis results}\\
Offline analysis in Test case
\href{https://jira.lsstcorp.org/secure/Tests.jspa\#/testCase/LVV-T2739}{LVV-T2739}
Says that we do not have sufficient image quality.

}
\hdashrule[0.5ex]{\textwidth}{1pt}{3mm}
  Expected Result \\
{\footnotesize
Image quality is sufficient.

}

\begin{tabular}{p{2cm}}
\toprule
Step 209  \\ \hline
\end{tabular}
 Description \\
{\footnotesize
\textbf{Point} \textbf{the Dome:}\\
Command the Dome to {Pointing 10}⁠ to {270}⁠

}
\hdashrule[0.5ex]{\textwidth}{1pt}{3mm}
  Expected Result \\
{\footnotesize
The Dome starts moving.

}

\begin{tabular}{p{2cm}}
\toprule
Step 210  \\ \hline
\end{tabular}
 Description \\
{\footnotesize
Wait for the Dome to reach the commanded position.

}
\hdashrule[0.5ex]{\textwidth}{1pt}{3mm}
  Expected Result \\
{\footnotesize
The \emph{MTDome\_logevent\_azMotion} and
\emph{MTDome\_logevent\_elMotion} inPosition parameter = true.

}

\begin{tabular}{p{2cm}}
\toprule
Step 211  \\ \hline
\end{tabular}
 Description \\
{\footnotesize
\textbf{Point the TMA}\\
Command the TMA to {Pointing 10}⁠ at {270}⁠ , {45}⁠ .

}
\hdashrule[0.5ex]{\textwidth}{1pt}{3mm}
  Expected Result \\
{\footnotesize
The TMA starts moving

}

\begin{tabular}{p{2cm}}
\toprule
Step 212  \\ \hline
\end{tabular}
 Description \\
{\footnotesize
Wait for the TMA to reach the commanded position.

}
\hdashrule[0.5ex]{\textwidth}{1pt}{3mm}
  Expected Result \\
{\footnotesize
The \emph{MTMount\_logevent\_azimuthInPosition} and
\emph{MTMount\_logevent\_elevationInPosition} inPosition parameter =
true.

}

\begin{tabular}{p{2cm}}
\toprule
Step 213  \\ \hline
\end{tabular}
 Description \\
{\footnotesize
\textbf{Image preparation}\\
If the preparation to take images takes longer than 10sec, do
repositioning to target {{{270}⁠}}, {{{45}⁠~}}.

}
\hdashrule[0.5ex]{\textwidth}{1pt}{3mm}
  Expected Result \\
{\footnotesize
TMA reaches the commanded position.

}

\begin{tabular}{p{2cm}}
\toprule
Step 214  \\ \hline
\end{tabular}
 Description \\
{\footnotesize
\textbf{Track position and take images}\\[2\baselineskip]Take a
StarTracker image with 10s exposure time.\\[2\baselineskip]If the time
the available:

\begin{itemize}
\tightlist
\item
  Track a position for 10 min and take StarTracker images.
\end{itemize}

}
\hdashrule[0.5ex]{\textwidth}{1pt}{3mm}
  Expected Result \\
{\footnotesize
\begin{itemize}
\tightlist
\item
  If time is available: The TMA is tracking a given position for 10 min
  and taking images.
\item
  At least one image is successfully taken with the StarTracker.
\end{itemize}

}

\begin{tabular}{p{2cm}}
\toprule
Step 215  \\ \hline
\end{tabular}
 Description \\
{\footnotesize
\textbf{On-the-fly Image Quality Check}\\
While tracking and taking images, check the images on RubinTV for an
astrometric solution.

}
\hdashrule[0.5ex]{\textwidth}{1pt}{3mm}
  Expected Result \\
{\footnotesize
RubinTV is showing an astrometric solution.

}

\begin{tabular}{p{2cm}}
\toprule
Step 216  \\ \hline
\end{tabular}
 Description \\
{\footnotesize
\textbf{Offline analysis results}\\
Offline analysis in Test case
\href{https://jira.lsstcorp.org/secure/Tests.jspa\#/testCase/LVV-T2739}{LVV-T2739}
Says that we do not have sufficient image quality.

}
\hdashrule[0.5ex]{\textwidth}{1pt}{3mm}
  Expected Result \\
{\footnotesize
Image quality is sufficient.

}

\begin{tabular}{p{2cm}}
\toprule
Step 217  \\ \hline
\end{tabular}
 Description \\
{\footnotesize
\textbf{Point} \textbf{the Dome:}\\
Command the Dome to {Pointing 12}⁠ to {-180}⁠

}
\hdashrule[0.5ex]{\textwidth}{1pt}{3mm}
  Expected Result \\
{\footnotesize
The Dome starts moving.

}

\begin{tabular}{p{2cm}}
\toprule
Step 218  \\ \hline
\end{tabular}
 Description \\
{\footnotesize
Wait for the Dome to reach the commanded position.

}
\hdashrule[0.5ex]{\textwidth}{1pt}{3mm}
  Expected Result \\
{\footnotesize
The \emph{MTDome\_logevent\_azMotion} and
\emph{MTDome\_logevent\_elMotion} inPosition parameter = true.

}

\begin{tabular}{p{2cm}}
\toprule
Step 219  \\ \hline
\end{tabular}
 Description \\
{\footnotesize
\textbf{Point the TMA}\\
Command the TMA to {Pointing 12}⁠ at {-180}⁠ , {86.5}⁠ .

}
\hdashrule[0.5ex]{\textwidth}{1pt}{3mm}
  Expected Result \\
{\footnotesize
The TMA starts moving

}

\begin{tabular}{p{2cm}}
\toprule
Step 220  \\ \hline
\end{tabular}
 Description \\
{\footnotesize
Wait for the TMA to reach the commanded position.

}
\hdashrule[0.5ex]{\textwidth}{1pt}{3mm}
  Expected Result \\
{\footnotesize
The \emph{MTMount\_logevent\_azimuthInPosition} and
\emph{MTMount\_logevent\_elevationInPosition} inPosition parameter =
true.

}

\begin{tabular}{p{2cm}}
\toprule
Step 221  \\ \hline
\end{tabular}
 Description \\
{\footnotesize
\textbf{Image preparation}\\
If the preparation to take images takes longer than 10sec, do
repositioning to target {{{-180}⁠}}, {{{86.5}⁠~}}.

}
\hdashrule[0.5ex]{\textwidth}{1pt}{3mm}
  Expected Result \\
{\footnotesize
TMA reaches the commanded position.

}

\begin{tabular}{p{2cm}}
\toprule
Step 222  \\ \hline
\end{tabular}
 Description \\
{\footnotesize
\textbf{Track position and take images}\\[2\baselineskip]Take a
StarTracker image with 10s exposure time.\\[2\baselineskip]If the time
the available:

\begin{itemize}
\tightlist
\item
  Track a position for 10 min and take StarTracker images.
\end{itemize}

}
\hdashrule[0.5ex]{\textwidth}{1pt}{3mm}
  Expected Result \\
{\footnotesize
\begin{itemize}
\tightlist
\item
  If time is available: The TMA is tracking a given position for 10 min
  and taking images.
\item
  At least one image is successfully taken with the StarTracker.
\end{itemize}

}

\begin{tabular}{p{2cm}}
\toprule
Step 223  \\ \hline
\end{tabular}
 Description \\
{\footnotesize
\textbf{On-the-fly Image Quality Check}\\
While tracking and taking images, check the images on RubinTV for an
astrometric solution.

}
\hdashrule[0.5ex]{\textwidth}{1pt}{3mm}
  Expected Result \\
{\footnotesize
RubinTV is showing an astrometric solution.

}

\begin{tabular}{p{2cm}}
\toprule
Step 224  \\ \hline
\end{tabular}
 Description \\
{\footnotesize
\textbf{Offline analysis results}\\
Offline analysis in Test case
\href{https://jira.lsstcorp.org/secure/Tests.jspa\#/testCase/LVV-T2739}{LVV-T2739}
Says that we do not have sufficient image quality.

}
\hdashrule[0.5ex]{\textwidth}{1pt}{3mm}
  Expected Result \\
{\footnotesize
Image quality is sufficient.

}

\paragraph{ LVV-T2715 - Configure Observatory Environment for Daytime Operations }\mbox{}\\

Version \textbf{1}.
Open  \href{https://jira.lsstcorp.org/secure/Tests.jspa#/testCase/LVV-T2715}{\textit{ LVV-T2715 } }
test case in Jira.

After using the observatory during the nighttime, prepare the
observatory for daytime operations.

\textbf{ Preconditions}:\\
The observatory was used during nighttime.

Final comment:\\


Detailed steps :

\begin{tabular}{p{2cm}}
\toprule
Step 1  \\ \hline
\end{tabular}
 Description \\
{\footnotesize
\textbf{CSCs}\\

\begin{itemize}
\tightlist
\item
  Transition the CSCs into STANDBY state
\end{itemize}

}
\hdashrule[0.5ex]{\textwidth}{1pt}{3mm}
  Expected Result \\
{\footnotesize
All CSCs are in their standbyState.

}

\begin{tabular}{p{2cm}}
\toprule
Step 2  \\ \hline
\end{tabular}
 Description \\
{\footnotesize
\textbf{Telescope daytime preparations:}

\begin{itemize}
\tightlist
\item
  Switch off or bring into standby the StarTracker and DIMM instruments
\item
  Install the caps on top of the StarTracker telescopes and the DIMM
\end{itemize}

}
\hdashrule[0.5ex]{\textwidth}{1pt}{3mm}
  Expected Result \\
{\footnotesize
The caps are installed.

}

\begin{tabular}{p{2cm}}
\toprule
Step 3  \\ \hline
\end{tabular}
 Description \\
{\footnotesize
\textbf{Dome:}\\

\begin{itemize}
\tightlist
\item
  Bring the dome into the park position
\end{itemize}

Until the dome shutter is motorized:\textbf{\\
}

\begin{itemize}
\tightlist
\item
  Send a message to the site manager :

  \begin{itemize}
  \tightlist
  \item
    confirming that nightly operations have finished~
  \item
    asking for a dome closer before the sun starts to shine on the
    StarTracker and the DIMM.
  \end{itemize}
\end{itemize}

}
\hdashrule[0.5ex]{\textwidth}{1pt}{3mm}
  Expected Result \\
{\footnotesize
Dome closure is organized.

}

\begin{tabular}{p{2cm}}
\toprule
Step 4  \\ \hline
\end{tabular}
 Description \\
{\footnotesize
\textbf{Auxillary systems~daytime preparations:}\\
If needed for daytime operations:\textbf{\\
}

\begin{itemize}
\tightlist
\item
  Switch on the UMA in the morning.
\item
  When available and need to be modified for the day:

  \begin{itemize}
  \tightlist
  \item
    Oil supply system on standby?
  \item
    Dynalyne into standby?
  \end{itemize}
\end{itemize}

}
\hdashrule[0.5ex]{\textwidth}{1pt}{3mm}
  Expected Result \\
{\footnotesize
All auxiliary systems are in the states suitable for daytime operations.

\begin{itemize}
\tightlist
\item
  The UMA is switched on.
\end{itemize}

}

\begin{tabular}{p{2cm}}
\toprule
Step 5  \\ \hline
\end{tabular}
 Description \\
{\footnotesize
\textbf{TMA position in the morning}\\

\begin{itemize}
\tightlist
\item
  Park the TMA in the position needed for the next day.
\end{itemize}

}
\hdashrule[0.5ex]{\textwidth}{1pt}{3mm}
  Expected Result \\
{\footnotesize
TMA parked in the corresponding position.

}

\begin{tabular}{p{2cm}}
\toprule
Step 6  \\ \hline
\end{tabular}
 Description \\
{\footnotesize

}
\hdashrule[0.5ex]{\textwidth}{1pt}{3mm}
  Expected Result \\
{\footnotesize

}

\begin{tabular}{p{2cm}}
\toprule
Step 7  \\ \hline
\end{tabular}
 Description \\
{\footnotesize
\textbf{Night log}\\

\begin{itemize}
\tightlist
\item
  Close the night log by writing a summary of the nightly events
\item
  Send a link with the summary to the site manager.
\end{itemize}

}
\hdashrule[0.5ex]{\textwidth}{1pt}{3mm}
  Expected Result \\
{\footnotesize
The night log is closed.

}




% This appendix is put in as part of the template. You may edit and add to it.
% It is not overwritten by Docsteady.

\newpage
\appendix
\section{Documentation}
The verification process is defined in \citeds{LSE-160}.
The use of Docsteady to format Jira information in various test and planing documents is
described in \citeds{DMTN-140} and practical commands are given in \citeds{DMTN-178}.

\section{Acronyms used in this document}\label{sec:acronyms}
\input{acronyms.tex}

\newpage

% Uncomment this if Docsteady makes you additional appendix
%\input{SCTR-81.appendix.tex}

\end{document}
